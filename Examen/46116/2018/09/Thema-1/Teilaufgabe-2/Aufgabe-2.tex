\documentclass{bschlangaul-aufgabe}
\bLadePakete{er}
\begin{document}
\bAufgabenMetadaten{
  Titel = {Aufgabe 2: ER-Diagramm},
  Thematik = {Freizeitparks},
  Referenz = 46116-2018-H.T1-TA2-A2,
  RelativerPfad = Staatsexamen/46116/2018/09/Thema-1/Teilaufgabe-2/Aufgabe-2.tex,
  ZitatSchluessel = db:pu:1,
  ZitatBeschreibung = {Seite 2},
  BearbeitungsStand = mit Lösung,
  Korrektheit = unbekannt,
  Ueberprueft = {unbekannt},
  Stichwoerter = {Entity-Relation-Modell},
  EinzelpruefungsNr = 46116,
  Jahr = 2018,
  Monat = 09,
  ThemaNr = 1,
  TeilaufgabeNr = 2,
  AufgabeNr = 2,
}

\let\a=\bErMpAttribute
\let\d=\bErDatenbankName
\let\e=\bErMpEntity
\let\r=\bErMpRelationship

\noindent
Im\index{Entity-Relation-Modell} \footcite[Seite 2]{db:pu:1} Folgenden
finden Sie die Beschreibung eines Systems zur Verwaltung von
Freizeitparks. Erstellen Sie zu dieser Beschreibung ein erweitertes
ER-Diagramm. Kennzeichnen Sie die Primär\-schlüssel durch passendes
Unterstreichen und geben Sie die Kardinalitäten in Chen-Notation (=
Funktionalitäten) an. Kennzeichnen Sie auch die totale Teilnahme (=
Existenzabhängigkeit, Partizipität) von Entitytypen.
\footcite[DB/ST - Herbst 2018 (46116, nicht vertieft), Thema 1 Teilaufgabe 2 Aufgabe 2]{examen:46116:2018:09}

\begin{itemize}
\item Der \e{Freizeitpark} ist in mehrere Gebiete \r{eingeteilt}.

\item Ein \e{Gebiet} hat einen eindeutigen \a{Namen} und eine
\a{Beschreibung}.

\item In jedem Gebiet \r{gibt} es eine oder mehrere \e{Attraktionen}.
Diese verfügen über eine innerhalb ihres Gebiets eindeutige \a{Nummer}.
Außerdem gibt es zu jeder Attraktion einen \a{Namen}, eine
\a{Beschreibung} und ein oder mehrere Fotos.

\item Der Freizeitpark \r{hat} \e{Mitarbeiter}. Zu diesen werden jeweils
eine eindeutige \a{ID}, der \a{Vorname} und der \a{Nachname}
gespeichert. Weiterhin hat jeder Mitarbeiter ein \a{Geburtsdatum}, das
sich aus \a{Tag}, \a{Monat} und \a{Jahr} zusammensetzt.

\item Die Arbeit im Freizeitpark ist in \e{Schichten} organisiert. Eine
Schicht kann eindeutig durch das \a{Datum} und die \a{Startzeit}
identifiziert werden. Jede Schicht hat weiterhin eine \a{Dauer}.

\item Mitarbeiter können in Schichten an Attraktionen \r{arbeiten}.
Dabei wird die \a{Aufgabe} gespeichert, die der Mitarbeiter übernimmt.
Pro Schicht kann der selbe Mitarbeiter nur an maximal einer Attraktion
arbeiten.
\end{itemize}

%\includegraphics[width=\linewidth]{Freizeitpark.eps}

\begin{tikzpicture}[er2]
\node[entity] (Freizeitpark) {Freizeitpark};

\node[weak entity,right=4cm of Freizeitpark] (Gebiet) {Gebiet};

\node[entity,below=3cm of Freizeitpark] (Mitarbeiter) {Mitarbeiter};

\node[entity,below=3cm of Gebiet] (Schicht) {Schicht};

\node[weak entity,below right=3cm of Mitarbeiter] (Attraktion) {Attraktion};

\end{tikzpicture}
\end{document}
