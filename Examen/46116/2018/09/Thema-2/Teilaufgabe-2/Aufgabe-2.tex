\documentclass{bschlangaul-aufgabe}
\bLadePakete{mathe}
\begin{document}
\bAufgabenMetadaten{
  Titel = {Aufgabe 2},
  Thematik = {Relationen R, S und T},
  Referenz = 46116-2018-H.T2-TA2-A2,
  RelativerPfad = Staatsexamen/46116/2018/09/Thema-2/Teilaufgabe-2/Aufgabe-2.tex,
  ZitatSchluessel = examen:46116:2018:09,
  BearbeitungsStand = mit Lösung,
  Korrektheit = unbekannt,
  Ueberprueft = {unbekannt},
  Stichwoerter = {Relationale Algebra},
  EinzelpruefungsNr = 46116,
  Jahr = 2018,
  Monat = 09,
  ThemaNr = 2,
  TeilaufgabeNr = 2,
  AufgabeNr = 2,
}

Geben\index{Relationale Algebra} \footcite{examen:46116:2018:09} Sie die
Ergebnisrelation folgender Ausdrücke der relationalen Algebra als
Tabellen an. Begründen Sie Ihr Ergebnis, gegebenenfalls durch
Zwischenschritte. Gegeben seien folgende Relationen:
\footcite[Seite 1, Aufgabe 1]{db:ab:3}

\bigskip

\begin{minipage}[t]{5cm}
\subsection*{R}
\begin{tabular}{llllll}
A & B & C & D & E & F \\\hline
6 & 8 & 1 & 7 & 3 & 7 \\
5 & 3 & 4 & 4 & 5 & 7 \\
0 & 6 & 3 & 0 & 1 & 7
\end{tabular}
\end{minipage}
%
\begin{minipage}[t]{3.8cm}
\subsection*{S}
\begin{tabular}{llll}
A & C & X & Z \\\hline
7 & 8 & 6 & 1 \\
0 & 3 & 0 & 0 \\
2 & 3 & 0 & 5 \\
0 & 6 & 1 & 6 \\
6 & 7 & 1 & 7 \\
7 & 1 & 2 & 2 \\
1 & 8 & 8 & 0 \\
5 & 1 & 5 & 5 \\
7 & 3 & 0 & 2 \\
4 & 8 & 2 & 7 \\
\end{tabular}
\end{minipage}
%
\begin{minipage}[t]{2cm}
\subsection*{T}
\begin{tabular}{ll}
X & Y \\\hline
5 & 3 \\
0 & 5 \\
8 & 6 \\
3 & 6 \\
5 & 7 \\
2 & 8 \\
\end{tabular}
\end{minipage}

\begin{enumerate}

%%
% (a)
%%

\item $\sigma_{A>6}(S) \bowtie_{S.X=T.Y} \pi_Y(T)$

\begin{bAntwort}
\begin{tabular}{lllll}
A & C & X & Z & Y \\\hline
7 & 8 & 6 & 1 & 6 \\
\end{tabular}
\end{bAntwort}

%%
% (b)
%%

\item $\pi_{A,C}(S) - (\pi_A(R) \times \pi_C(\sigma_{x=1}(S)))$

\begin{bAntwort}
\begin{minipage}[t]{4cm}
$\sigma_{x=1}(S)$:

\bigskip
\begin{tabular}{llll}
A & C & X & Z \\\hline
0 & 6 & 1 & 6 \\
6 & 7 & 1 & 7 \\
\end{tabular}
\end{minipage}
%
\begin{minipage}[t]{3cm}
$\pi_C(\sigma_{x=1}(S))$:

\bigskip
\begin{tabular}{l}
C \\\hline
6 \\
7 \\
\end{tabular}
\end{minipage}
%
\begin{minipage}[t]{3cm}
$\pi_A(R)$:

\bigskip
\begin{tabular}{l}
A \\\hline
6 \\
5 \\
0 \\
\end{tabular}
\end{minipage}

\begin{minipage}[t]{5cm}
$(\pi_A(R) \times \pi_C(\sigma_{x=1}(S)))$

\bigskip
\begin{tabular}{ll}
A & C \\\hline
6 & 6 \\
5 & 6 \\
0 & 6 \\
6 & 7 \\
5 & 7 \\
0 & 7 \\
\end{tabular}
\end{minipage}
%
\begin{minipage}[t]{4cm}
$\pi_{A,C}(S)$

\bigskip
\begin{tabular}{llll}
A & C \\\hline
7 & 8 \\
0 & 3 \\
2 & 3 \\
0 & 6 \\
6 & 7 \\
7 & 1 \\
1 & 8 \\
5 & 1 \\
7 & 3 \\
4 & 8 \\
\end{tabular}
\end{minipage}

\begin{tabular}{llll}
A & C \\\hline
7 & 8 \\
0 & 3 \\
2 & 3 \\
7 & 1 \\
1 & 8 \\
5 & 1 \\
7 & 3 \\
4 & 8 \\
\end{tabular}
\end{bAntwort}

%%
% (c)
%%

\item $(\pi_D(R) \times \pi_E(R)) \div \pi_E(R)$

\begin{bAntwort}
\begin{minipage}[t]{3cm}
$\pi_D(R) \times \pi_E(R)$

\bigskip
\begin{tabular}{ll}
A & E \\\hline
7 & 3 \\
4 & 3 \\
0 & 3 \\
7 & 5 \\
4 & 5 \\
0 & 5 \\
7 & 1 \\
4 & 1 \\
0 & 1 \\
\end{tabular}
\end{minipage}
%
\begin{minipage}[t]{2cm}
$\pi_E(R)$

\bigskip
\begin{tabular}{l}
E \\\hline
3 \\
5 \\
1 \\
\end{tabular}
\end{minipage}
%
\begin{minipage}[t]{6cm}
$(\pi_D(R) \times \pi_E(R)) \div \pi_E(R)$

\bigskip
\begin{tabular}{l}
D \\\hline
7 \\
4 \\
0 \\
\end{tabular}
\end{minipage}
\end{bAntwort}
\end{enumerate}
\end{document}
