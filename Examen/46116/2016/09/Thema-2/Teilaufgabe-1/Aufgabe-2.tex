\documentclass{bschlangaul-aufgabe}
\bLadePakete{uml}
\begin{document}
\bAufgabenMetadaten{
  Titel = {Aufgabe 2: Modellierung von Interaktionen durch Sequenzdiagramme},
  Thematik = {Spielzeugeisenbahnen},
  Referenz = 46116-2016-H.T2-TA1-A2,
  RelativerPfad = Staatsexamen/46116/2016/09/Thema-2/Teilaufgabe-1/Aufgabe-2.tex,
  ZitatSchluessel = examen:46116:2016:09,
  ZitatBeschreibung = {Thema 2 Teilaufgabe 1 Aufgabe 2},
  BearbeitungsStand = TeX-Fehler,
  Korrektheit = unbekannt,
  Ueberprueft = {unbekannt},
  Stichwoerter = {Sequenzdiagramm},
  EinzelpruefungsNr = 46116,
  Jahr = 2016,
  Monat = 09,
  ThemaNr = 2,
  TeilaufgabeNr = 1,
  AufgabeNr = 2,
}

Die Fahrtrichtung der ersten elektrischen Spielzeugeisenbahnen wurde
häufig durch Stromunterbrechung gesteuert. Dazu betrachten wir das
Anwendungsfall-Diagramm.\index{Sequenzdiagramm}
\footcite[Thema 2 Teilaufgabe 1 Aufgabe 2]{examen:46116:2016:09}

\begin{center}
\begin{tikzpicture}
\umlactor{Spieler}
\begin{umlsystem}[x=5]{}
\umlusecase[name={Steuere Lokomotive}]{Steuere Lokomotive}
\end{umlsystem}
\umlassoc{Spieler}{Steuere Lokomotive}
\end{tikzpicture}
\end{center}

\noindent
An dem Anwendungsfall \emph{„Steuere Lokomotive“} sind ein Spieler, als
Aktor außerhalb des Systems, und jeweils ein Objekt der Klassen
\emph{Stromschalter}, \emph{Lokomotive}, \emph{Scheinwerfer} und
\emph{Rad}, als Objekte innerhalb des Systems, beteiligt. Zur
Vereinfachung wird nur ein Objekt der Klasse \emph{Rad} stellvertretend
für alle vier Räder modelliert. Der Anwendungsfall zum Steuern einer
Lokomotive wird durch folgendes Szenario beschrieben.

\begin{enumerate}

\item Der Spieler schaltet den Stromschalter ein, woraufhin der Schalter
der Lokomotive \emph{Strom zuführt}.

\item Die Lokomotive schickt nun den Rädern ein Signal um
\emph{vorwärts} zu fahren.

\item Dann schaltet der Spieler den Stromschalter aus, woraufhin der
Schalter die Stromzufuhr bei der Lokomotive \emph{abstellt}.

\item Daraufhin schickt die Lokomotive das Signal \emph{stop} an die
Räder.

\item Der Spieler schaltet jetzt den Stromschalter wieder ein, woraufhin
der Schalter der Lokomotive \emph{Strom zuführt}.

\item Die Lokomotive schickt den Rädern ein Signal um \emph{rückwärts}
zu fahren.

\item Nun schaltet der Spieler den Stromschalter wieder aus, woraufhin
der Schalter die Stromzufuhr bei die Lokomotive \emph{abstellt}.

\item Daraufhin schickt die Lokomotive wieder das Signal \emph{stop} an
die Räder.
\end{enumerate}

\noindent
Geben Sie ein Sequenzdiagramm an, das die oben beschriebenen
Interaktionen zwischen Spieler, Stromschalter, Lokomotive und Rädern
beschreibt.

\begin{tikzpicture}
\begin{umlseqdiag}
\umlactor{Spieler}
\umlobject[class=Stromschalter]{schalter}
\umlobject[class=Lokomotive]{lok}
\umlobject[class=Rad]{rad}

\begin{umlcall}[op=einschalten()]{Spieler}{schalter}
\begin{umlcall}[op=stromZufuehren()]{schalter}{lok}
\begin{umlcall}[op=vorwaertsFahren()]{lok}{rad}
\end{umlcall}
\end{umlcall}
\end{umlcall}

\begin{umlcall}[op=ausschalten(),with return]{Spieler}{schalter}
\begin{umlcall}[op=stromAbstellen(),with return]{schalter}{lok}
\begin{umlcall}[op=stop(),with return]{lok}{rad}
\end{umlcall}
\end{umlcall}
\end{umlcall}

\begin{umlcall}[op=einschalten()]{Spieler}{schalter}
\begin{umlcall}[op=stromZufuehren()]{schalter}{lok}
\begin{umlcall}[op=rueckwaertsFahren()]{lok}{rad}
\end{umlcall}
\end{umlcall}
\end{umlcall}

\begin{umlcall}[op=ausschalten(),with return]{Spieler}{schalter}
\begin{umlcall}[op=stromAbstellen(),with return]{schalter}{lok}
\begin{umlcall}[op=stop(),with return]{lok}{rad}
\end{umlcall}
\end{umlcall}
\end{umlcall}
\end{umlseqdiag}
\end{tikzpicture}
\end{document}
