\documentclass{bschlangaul-aufgabe}
\bLadePakete{rmodell,sql,mathe}
\begin{document}
\bAufgabenMetadaten{
  Titel = {Aufgabe 4},
  Thematik = {Versicherungsgesellschaft „Insurance Pro“},
  Referenz = 46116-2015-H.T1-TA2-A4,
  RelativerPfad = Staatsexamen/46116/2015/09/Thema-1/Teilaufgabe-2/Aufgabe-4.tex,
  ZitatSchluessel = db:ab:4,
  ZitatBeschreibung = {Aufgabe 4: SQL
(Check-Up)},
  BearbeitungsStand = mit Lösung,
  Korrektheit = unbekannt,
  Ueberprueft = {unbekannt},
  EinzelpruefungsNr = 46116,
  Jahr = 2015,
  Monat = 09,
  ThemaNr = 1,
  TeilaufgabeNr = 2,
  AufgabeNr = 4,
}

Das Datenbankschema der Versicherungsgesellschaft „Insurance Pro“ enthält
sowohl Kunden- als auch Mitarbeiterdaten, sowie Informationen über
Schadensfälle. Das Schema ist von folgender Gestalt, wobei FS für
Fremdschlüssel und PS für Primärschlüssel steht:\footcite[Aufgabe 4: SQL
(Check-Up)]{db:ab:4}

\bigskip

{
\ttfamily
\noindent
Versicherungsnehmer (\bPrimaer{KNr}, Name, Anschrift, Geburtsdatum)\\

\noindent
Fahrzeug (\bPrimaer{Kennzeichen}, Farbe, Fahrzeugtyp)\\

\noindent
Mitarbeiter (\bPrimaer{MNr}, Name, Kontaktdaten, \bFremd{Abteilung},
Leiter)\\

\noindent
Abteilung (\bPrimaer{ANr}, Bezeichnung, Ort)\\

\noindent
Versicherungsvertrag (\bPrimaer{VNr}, Abschlussdatum, Art,
\bFremd{Versicherungsnehmer}, \bFremd{Fahrzeug},
\bFremd{Mitarbeiter})\\

\noindent
Schadensfall (\bPrimaer{SNr}, Datum, Gesamtschaden, Beschreibung,
\bFremd{Mitarbeiter})\\

\noindent
Beteiligung (\bFremd{Schadensfall}, \bFremd{Fahrzeug},
Fahrzeugschaden)}

\bigskip

\begin{itemize}
\item \emph{Abteilung} ist FS und bezieht sich auf den PS von
\emph{Abteilung}.

\item \emph{Versicherungsnehmer} ist FS und bezieht sich auf den PS von
\emph{Versicherungsnehmer}.

\item \emph{Fahrzeug} ist FS und bezieht sich auf den PS von
\emph{Fahrzeug}.

\item \emph{Mitarbeiter} ist FS und bezieht sich auf den PS von
\emph{Mitarbeiter}.

\item Attribut \emph{Art} kann die Werte \emph{HP} (Haftpflicht),
\emph{TK} (Teilkasko) und \emph{VK} (Vollkasko) annehmen.

\item \emph{Mitarbeiter} ist FS und bezieht sich auf den PS von
\emph{Mitarbeiter}.

\item \emph{Schadensfall} ist FS und bezieht sich auf den PS von
\emph{Schadensfall}.

\item \emph{Fahrzeug} ist FS und bezieht sich auf den PS von
\emph{Fahrzeug}.
\end{itemize}

\begin{enumerate}

%%
% (a)
%%

\item Herr Meier schließt eine Teilkaskoversicherung bei „Insurance Pro“
am 11.11.2011 für seinen neuen Wagen, einen roten VW Golf VII mit
amtlichem Kennzeichen BT-BT 2011, ab. Der Vertrag erhält die laufende
Nummer 1631 und wird von Frau Schmied mit der Personalnummer 27
bearbeitet. Herr Meier erhält die Kundennummer 588, da er bisher noch
kein Kunde war.

Geben Sie die SQL-Befehle an, um die neue Versicherung in die Datenbank
von „Insurance Pro“ einzutragen. Werte, die hierbei nicht bekannt sind,
sollen weggelassen werden, wobei angenommen werden darf, dass die
entsprechenden Attribute nicht mit der Bedingung NOT NULL versehen sind.

\begin{mdframed}
\begin{minted}{sql}
INSERT INTO Fahrzeug
  (Kennzeichen, Farbe, Fahrzeugtyp) VALUES
  ('BT-BT 2011', 'rot', 'VW Golf VII');

INSERT INTO Versicherungsnehmer
  (KNr, Name)  VALUES
  (588, 'Meier');

INSERT INTO Versicherungsvertrag
  (VNr, Abschlussdatum, Art, Versicherungsnehmer, Fahrzeug, Mitarbeiter) VALUES
  (1631, 11.11.2011, 'TK', 588, 'BT-BT 2011', 27);
\end{minted}
\end{mdframed}

%%
% (b)
%%

\item Beschreiben Sie umgangssprachlich, wonach mit folgendem Ausdruck
gesucht wird:

$\pi_{\text{Versicherungsnehmer}} (
  \sigma_{\text{Fahrzeugtyp} = \mlq \text{Fiat500} \mrq} (
    \text{FZ} \bowtie_{\rho_{\text{Kennzeichen} \leftarrow \text{Fahrzeug}}} (\text{VV})
  )
)$

\textbf{Anmerkung:} Die Abkürzung \emph{FZ} steht für die Tabelle
\emph{Fahrzeug} und \emph{VV} für die Tabelle
\emph{Versicherungsvertrag}.

\begin{mdframed}
Es wird nach der Kundennummer aller Versicherungsnehmer gesucht, die
einen Versicherungsvertrag für einen „Fiat 500“ abgeschlossen haben.
\end{mdframed}

%%
% (c)
%%

\item Die Angaben der Mitarbeiter (Nr., Name und Kontakt), deren
Abteilung ihren Sitz in München oder Stuttgart hat, sollen explizit
alphabetisch nach Namen sortiert ausgegeben werden. Geben Sie einen
SQL-Befehl hierfür an.

\begin{mdframed}
\begin{minted}{sql}
SELECT m.MNr AS Nr, m.Name, m.Kontaktdaten AS Kontakt
FROM Mitarbeiter m, Abteilung a
WHERE
  a.ANr = m.Abteilung AND
  a.Ort IN ('München', 'Stuttgart')
ORDER BY m.Name;
\end{minted}
\end{mdframed}

%%
% (d)
%%

\item Es soll ausgegeben werden, wie oft jeder Versicherungsnehmer (KNr,
Name, Anschrift) an einem Schadensfall beteiligt war. Hierbei sollen
nur die Kunden, die an mindestens drei Schadensfällen beteiligt waren,
in absteigender Reihenfolge aufgelistet werden.

\begin{mdframed}
\begin{minted}{sql}
SELECT vn.KNr, vn.Name, vn.Anschrift, COUNT(*) AS Schadensfaelle
FROM
  Versicherungsnehmer vn,
  Schadensfall s,
  Beteilung b,
  Versicherungsvertrag vv
WHERE
  vn.KNr = vv.Versicherungsnehmer AND
  s.SNr = b.Schadensfall AND
  b.Fahrzeug = vv.Fahrzeug
GROUP BY vn.KNr, vn.Name, vn.Anschrift
HAVING COUNT(*) >= 3
ORDER BY Schadensfaelle DESC;
\end{minted}

Es ist nicht genau angegeben, nach was sortiert werden soll. Ich
sortiere nach Anzahl an Schadensfällen, weil dass meiner Meinung nach am
meisten Sinn macht.
\end{mdframed}
\end{enumerate}
\end{document}
