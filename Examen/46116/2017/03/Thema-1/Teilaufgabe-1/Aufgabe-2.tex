\documentclass{bschlangaul-aufgabe}
\bLadePakete{uml}
\begin{document}
\bAufgabenMetadaten{
  Titel = {Aufgabe 2},
  Thematik = {Geldautomat},
  Referenz = 46116-2017-F.T1-TA1-A2,
  RelativerPfad = Staatsexamen/46116/2017/03/Thema-1/Teilaufgabe-1/Aufgabe-2.tex,
  ZitatSchluessel = examen:46116:2017:03,
  BearbeitungsStand = mit Lösung,
  Korrektheit = unbekannt,
  Ueberprueft = {unbekannt},
  Stichwoerter = {Anwendungsfalldiagramm},
  EinzelpruefungsNr = 46116,
  Jahr = 2017,
  Monat = 03,
  ThemaNr = 1,
  TeilaufgabeNr = 1,
  AufgabeNr = 2,
}

Im Folgenden sehen Sie ein fehlerhaftes Use-Case-Diagramm für das System
„Geldautomat”. Geben Sie ein korrektes Use-Case-Diagramm (Beziehungen
und Beschriftungen) entsprechend der folgenden Beschreibung an. Sollte
es mehrere Möglichkeiten geben, begründen Sie Ihre Entscheidung kurz.
\index{Anwendungsfalldiagramm}
\footcite{examen:46116:2017:03}

Kunden können am Geldautomat verschiedene Services nutzen, es kann Geld
abgehoben und eingezahlt sowie der Kontostand abgefragt werden. Für
jeden dieser Services ist eine Authentifizierung notwendig. Diese
besteht aus der Prüfung der Bankkarte und des eingegebenen PINs. Manche
Bankautomaten können Karten bei zu vielen Fehlversuchen (3) bei der
Anmeldung sperren, andere geben Fehlermeldungen aus, falls die Karte
gesperrt wurde oder nicht mehr genügend Geld auf dem Konto oder im
Bankautomaten vorhanden ist. Mitarbeiter bzw. Mitarbeiterinnen der Bank
können sich ebenfalls über PIN und Karte authentifizieren, um dann den
Bankautomaten neu zu starten oder mit Geld aufzufüllen. Bevor Geld
abgehoben werden kann, ist sicherzustellen, dass auf dem Konto und im
Bankautomaten genügend Geld vorhanden ist.
\footcite[Aufgabe 3]{sosy:ab:3}

\begin{bAntwort}
\begin{tikzpicture}[scale=0.7,transform shape]
\begin{umlsystem}{Geldautomat}

\umlusecase[x=2,y=0,name={Service nutzen}]{Service nutzen}
\umlusecase[x=8,y=0,name=Authentifizierung]{Authentifizierung}
\umlinclude{Service nutzen}{Authentifizierung}

\umlusecase[x=13,y=2,name={Karte prüfen}]{Karte prüfen}
\umlusecase[x=13,y=0,name={PIN prüfen}]{PIN prüfen}
\umlusecase[x=13,y=-2,name={Karte sperren}]{Karte sperren}

\umlinclude{Authentifizierung}{Karte prüfen}
\umlinclude{Authentifizierung}{PIN prüfen}
\umlextend[name=extend-sperren]{Authentifizierung}{Karte sperren}
\umlnote[x=13,y=-4]{extend-sperren-1}{Karte und PIN konten 3x nicht erfolgreich geprüft werden.}

\umlusecase[x=9,y=-4,name={Bankautomat neustarten},text width=2cm]{Bankautomat neustarten}
\umlusecase[x=5,y=-4,name={Bankautomat auffüllen},text width=2cm]{Bankautomat auffüllen}
\umlinclude{Bankautomat neustarten}{Authentifizierung}
\umlinclude{Bankautomat auffüllen}{Authentifizierung}
\end{umlsystem}

\umlactor[x=7,y=-6.5,name=Mitarbeiter]{Mitarbeiter}
\umlVHVassoc{Mitarbeiter}{Bankautomat neustarten}
\umlVHVassoc{Mitarbeiter}{Bankautomat auffüllen}

\umlactor[x=-1,y=0,name=Kunde]{Kunde}
\umlassoc{Kunde}{Service nutzen}

\umlusecase[x=0,y=2,name={Geld abheben},text width=1.5cm]{Geld abheben}
\umlusecase[x=3,y=2,name={Geld einzahlen},text width=1.5cm]{Geld einzahlen}
\umlusecase[x=6,y=2,name={Kontostand abfragen},text width=2cm]{Kontostand abfragen}
\umlVHVinherit{Geld abheben}{Service nutzen}
\umlVHVinherit{Geld einzahlen}{Service nutzen}
\umlVHVinherit{Kontostand abfragen}{Service nutzen}

\umlusecase[x=0,y=4,name={Kontostand / Automat prüfen},text width=1.5cm]{Kontostand / Automat prüfen}
\umlusecase[x=2,y=-2,name={Fehlermeldung},text width=1.5cm]{Fehlermeldung}
\end{tikzpicture}
\end{bAntwort}
\end{document}
