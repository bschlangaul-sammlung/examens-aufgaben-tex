\documentclass{bschlangaul-aufgabe}

\bLadePakete{java}

\begin{document}
\bAufgabenMetadaten{
  Titel = {Aufgabe 4},
  Thematik = {Abenteuerspiel},
  Referenz = 46116-2017-F.T1-TA1-A4,
  RelativerPfad = Examen/46116/2017/03/Thema-1/Teilaufgabe-1/Aufgabe-4.tex,
  ZitatSchluessel = examen:46116:2017:03,
  BearbeitungsStand = unbekannt,
  Korrektheit = unbekannt,
  Ueberprueft = {unbekannt},
  Stichwoerter = {Entwurfsmuster},
  EinzelpruefungsNr = 46116,
  Jahr = 2017,
  Monat = 03,
  ThemaNr = 1,
  TeilaufgabeNr = 1,
  AufgabeNr = 4,
}

In einem Abenteuerspiel soll ein Held Monster bekämpfen. Zu Beginn ist
die Spielfigur schwach und kann als Attacke nur mit den Fäusten boxen.
Im Laufe des Spiels erhält der Charakter diverse Items und wird dadurch
zu einer stärkeren Figur. Bei jeder Attacke auf ein Monster werden alle
Items, die er erworben hat, der Reihe nach eingesetzt.
\index{Entwurfsmuster}
\footcite{examen:46116:2017:03}

Der folgende Code zeigt einen Ausschnitt aus der Implementierung.

Welches Designpattern wurde hier eingesetzt? Zeichnen Sie das zugehörige
UML-Diagramm, das genau die im Code beschriebene Situation wiedergibt.

\bJavaExamen{46116}{2017}{03}{abendteuerspiel/Spielfigur.java}
\bJavaExamen{46116}{2017}{03}{abendteuerspiel/EinfacheFigur.java}
\bJavaExamen{46116}{2017}{03}{abendteuerspiel/BewaffneteFigur.java}
\bJavaExamen{46116}{2017}{03}{abendteuerspiel/FigurMitHammer.java}
\bJavaExamen{46116}{2017}{03}{abendteuerspiel/FigurMitTraenengas.java}
\bJavaExamen{46116}{2017}{03}{abendteuerspiel/FigurMitWasserpistole.java}

\end{document}
