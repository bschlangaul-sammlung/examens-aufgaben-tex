\documentclass{bschlangaul-aufgabe}
\bLadePakete{gantt}
\begin{document}
\bAufgabenMetadaten{
  Titel = {Aufgabe 5: Projektmanagement},
  Thematik = {Gantt und PERT},
  Referenz = 46116-2017-F.T1-TA1-A5,
  RelativerPfad = Staatsexamen/46116/2017/03/Thema-1/Teilaufgabe-1/Aufgabe-5.tex,
  ZitatSchluessel = examen:46116:2017:03,
  BearbeitungsStand = mit Lösung,
  Korrektheit = unbekannt,
  Ueberprueft = {unbekannt},
  Stichwoerter = {Gantt-Diagramm},
  EinzelpruefungsNr = 46116,
  Jahr = 2017,
  Monat = 03,
  ThemaNr = 1,
  TeilaufgabeNr = 1,
  AufgabeNr = 5,
}

Betrachten Sie die folgende Tabelle zum Projektmanagement:
\index{Gantt-Diagramm}
\footcite{examen:46116:2017:03}

\begin{center}
\begin{tabular}{|l|l|l|}
\hline
Name & Dauer (Tage) & Abhängig von\\\hline\hline
A1 & 5 & \\\hline
A2 & 10 & \\\hline
A3 & 5  & A1 \\\hline
AA & 15 & A2, A3\\\hline
AS & 10 & A1 \\\hline
A6 & 10 & A1, A2\\\hline
A7 & 10 & A2, A4\\\hline
A8 & 15 & A4, A5\\\hline
\end{tabular}
\end{center}

Tabelle 1: Übersicht Arbeitspakete

\begin{enumerate}

%%
% a)
%%

\item Erstellen Sie ein Gantt-Diagramm, das die in der Tabelle
angegebenen Abhängigkeiten berücksichtigt.

\begin{bAntwort}
\begin{center}
\begin{ganttchart}[
  x unit=0.24cm,
  y unit chart=0.8cm,
  vgrid
]{1}{40}
\ganttbar[name=1]{A1}{1}{5} \\
\ganttbar[name=2]{A2}{1}{10} \\
\ganttbar[name=3]{A3}{6}{10} \\
\ganttbar[name=4]{A4}{11}{25} \\
\ganttbar[name=5]{A5}{6}{15} \\
\ganttbar[name=6]{A6}{11}{20} \\
\ganttbar[name=7]{A7}{26}{35} \\
\ganttbar[name=8]{A8}{26}{40}
%
\ganttlink{1}{3}
%
\ganttlink{2}{4}
\ganttlink{3}{4}
%
\ganttlink{1}{5}
%
\ganttlink[link type=f-s]{1}{6}
\ganttlink[link type=f-s]{2}{6}
%
\ganttlink[link type=f-s]{2}{7}
\ganttlink[link type=f-s]{4}{7}
%
\ganttlink{4}{8}
\ganttlink[link type=f-s]{5}{8}
\gantttitlelist[
  title list options={var=\y, evaluate={} as \x}
]{1,...,40}{1}\\
\gantttitlelist[
  title list options={var=\i, evaluate={int(\i * 5)} as \x}
]{1,...,8}{5}\\
\end{ganttchart}
\end{center}
\end{bAntwort}

%%
% b)
%%

\item Wie lange dauert das Projekt mindestens?

\begin{bAntwort}
40 Tage
\end{bAntwort}

%%
% c)
%%

\item Geben Sie den oder die kritischen Pfad(e) an.

\begin{bAntwort}
A2 A4 A8

A1 A3 A4 A8
\end{bAntwort}

%%
% d)
%%

\item Konstruieren Sie ein PERT-Chart zum obigen Problem.

\begin{bAntwort}
\bMetaNochKeineLoesung
\end{bAntwort}

\end{enumerate}
\end{document}
