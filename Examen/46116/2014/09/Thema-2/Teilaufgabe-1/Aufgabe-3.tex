\documentclass{bschlangaul-aufgabe}
\bLadePakete{uml}
\begin{document}
\bAufgabenMetadaten{
  Titel = {Aufgabe 3: UML Diagramme in der Anwendung},
  Thematik = {Bestellsystem},
  Referenz = 46116-2014-H.T2-TA1-A3,
  RelativerPfad = Staatsexamen/46116/2014/09/Thema-2/Teilaufgabe-1/Aufgabe-3.tex,
  ZitatSchluessel = examen:46116:2014:09,
  BearbeitungsStand = mit Lösung,
  Korrektheit = unbekannt,
  Ueberprueft = {unbekannt},
  Stichwoerter = {UML-Diagramme, Anwendungsfalldiagramm, Klassendiagramm, Objektdiagramm, Zustandsdiagramm zeichnen},
  EinzelpruefungsNr = 46116,
  Jahr = 2014,
  Monat = 09,
  ThemaNr = 2,
  TeilaufgabeNr = 1,
  AufgabeNr = 3,
}

Gegeben\index{UML-Diagramme} \footcite{examen:46116:2014:09} sei
folgender Sachverhalt:\footcite{sosy:pu:2}

Für eine Verwaltungssoftware einer Behörde soll ein Bestellsystem
entwickelt werden. Dabei sollen die Nutzer ihre Raummaße eingeben
können. Anschließend können die Nutzer über ein Web-Interface das Büro
gestalten und Möbel (wie zum Beispiel Wandschränke) und andere
Einrichtungsgegenstände in einem virtuellen Büro platzieren. Aus dem
Web-Interface kann die Einrichtung dann direkt bestellt werden. Dazu
müssen die Nutzer ihre Büro-Nummer und den Namen und die Adresse der
Behörde eingeben und die Bestellung bestätigen.

Weiterhin können Nutzer auch Büromaterialien über das Web-Interface
bestellen. Dazu ist anstatt der Eingabe der Raummaße nur das Eingeben
von Büro-Nummer und des Namens und der Adresse der Behörde erforderlich.

Zusätzlich zum Standard-Nutzer können sich auch Administratoren im
System anmelden und Möbel zur Kollektion hinzufügen und aus der
Kollektion entfernen. Die Möbel können eindeutig durch ihre
Inventurnummer identifiziert werden.

Um jegliche Veränderungen im System protokollieren zu können müssen
Nutzer und Administratoren zur Bestätigung eingeloggt sein.

\begin{enumerate}

%%
% a)
%%

\item Erfassen Sie die drei Systemfunktionen \emph{Möbel bestellen},
\bEmph{Login} und \bEmph{Kollektion verwalten} in einem UML-konformen
Use Case Diagramm\index{Anwendungsfalldiagramm}.

\begin{bAntwort}
% Nach Video gezeichnet

%%
%
%%

\begin{tikzpicture}
\begin{umlsystem}{Möbel bestellen}
\umlusecase[x=1,y=0,name=login]{Login}
\umlusecase[x=5,y=-1,text width=1.5cm,name=bestellung]{Bestellung bestätigen}
\umlusecase[x=4,y=-3,text width=1.5cm,name=buero]{Büro gestalten}
\umlusecase[x=1,y=-5,text width=1.5cm,name=raummass]{Raummaß eingeben}
\end{umlsystem}

\umlactor[x=-2,y=-2]{Nutzer}
\umlassoc{Nutzer}{login}
\umlassoc{Nutzer}{bestellung}
\umlassoc{Nutzer}{buero}
\umlassoc{Nutzer}{raummass}

\umlinclude{buero}{raummass}
\umlinclude{bestellung}{buero}
\umlinclude{bestellung}{login}
\end{tikzpicture}

%%
%
%%

\begin{tikzpicture}
\begin{umlsystem}{Login}
\umlusecase[x=1,y=-1,text width=1.5cm,name=bueronummer]{Büronummer eingeben}
\umlusecase[x=4,y=-3,name=name]{Name eingeben}
\umlusecase[x=1,y=-4,text width=1.5cm,name=adresse]{Adresse Behörde eingeben}
\end{umlsystem}

\umlactor[x=-2,y=-2]{Nutzer}
\umlassoc{Nutzer}{bueronummer}
\umlassoc{Nutzer}{name}
\umlassoc{Nutzer}{adresse}
\end{tikzpicture}

%%
%
%%

\begin{tikzpicture}
\begin{umlsystem}{Kollektion verwalten}
\umlusecase[x=1,y=-1,text width=1.5cm,name=hinzufuegen]{Möbel hinzufügen}
\umlusecase[x=4,y=-3,name=login]{Login}
\umlusecase[x=1,y=-4,text width=1.5cm,name=entfernen]{Möbel entfernen}
\end{umlsystem}

\umlactor[x=-2,y=-2]{Administrator}
\umlassoc{Administrator}{login}
\umlassoc{Administrator}{hinzufuegen}
\umlassoc{Administrator}{entfernen}

\umlinclude{hinzufuegen}{login}
\umlinclude{entfernen}{login}
\end{tikzpicture}
\end{bAntwort}

%%
% b)
%%

\item Erstellen Sie ein UML-Klassendiagramm,\index{Klassendiagramm}
welches die Beziehungen und sinnvolle Attribute der Klassen „Nutzer,
Büro, Möbelstück und Wandschrank“ darstellt.

%%
% c)
%%

\item Instanziieren Sie das Klassendiagramm in einem
Objektdiagramm\index{Objektdiagramm} mit den zwei Nutzern mit Namen
Ernie und Bernd, einem Büro mit der Nummer CAPITOL2 und zwei Schränken
mit den Inventurnummern S1.88 und S1.77.

%%
% d)
%%

\item Geben Sie für ein Büromöbelstück ein
Zustandsdiagramm\index{Zustandsdiagramm zeichnen} an. Überlegen Sie dazu, welche
möglichen Zustände ein Möbelstück während des Bestellvorgangs haben kann
und finden Sie geeignete Zustandsübergänge.
\end{enumerate}
\end{document}
