\documentclass{bschlangaul-aufgabe}
\bLadePakete{mathe,normalformen,spalten}
\begin{document}
\bAufgabenMetadaten{
  Titel = {Aufgabe 2},
  Thematik = {Funktionale Abhängigkeiten, Normalisierung},
  Referenz = 66113-2002-H.T2-A2,
  RelativerPfad = Staatsexamen/66113/2002/09/Thema-2/Aufgabe-2.tex,
  ZitatSchluessel = 66113:2002:09,
  BearbeitungsStand = mit Lösung,
  Korrektheit = unbekannt,
  Ueberprueft = {unbekannt},
  Stichwoerter = {Normalformen},
  EinzelpruefungsNr = 66113,
  Jahr = 2002,
  Monat = 09,
  ThemaNr = 2,
  AufgabeNr = 2,
}

\let\ah=\bAttributHuelle
\let\m=\bAttributMenge
\let\FA=\bFunktionaleAbhaengigkeiten

Gegeben\index{Normalformen} \footcite{66113:2002:09} sei ein
Relationenschema $R$ mit Attributen $A, B, C, D$. Für dieses
Relationenschema seien die folgenden Mengen an funktionalen
Abhängigkeiten (FDs) gegeben:
\footcite[Seite 12]{db:fs:4}

\begin{multicols}{2}
\begin{enumerate}
\item \FA{
  A -> B;
  B -> C;
  A -> D;
}

\item \FA{
  A -> B;
  B -> C;
  C -> D;
  C -> A;
}

\columnbreak

\item \FA{
  A, B -> C;
  B -> D;
}

\item \FA{
  A, B -> C;
  A, C -> D;
  A, D -> B;
}

\item \FA{
  A, B -> C;
  A -> D;
  C, D -> A;
}
\end{enumerate}
\end{multicols}
\begin{enumerate}

%%
%
%%

\item Bestimmen Sie für das Relationschema R für jede der angegebenen
Mengen an funktionalen Abhängigkeiten jeweils alle möglichen
Schlüssel(-kandidaten)'

\begin{bAntwort}
\bPseudoUeberschrift{Abkürzung}

$A$ kommt auf keiner rechten Seite der FD’s vor.
Man kann es über FD's nicht erreichen. $A$ muss also Teil des
Schlüsselkandidaten sein.

\begin{displaymath}
\ah{F, \m{A}} = R \rightarrow \textit{Superschlüssel}
\end{displaymath}

$A$ ist minimal, deshalb handelt es bei $A$ um einen Schlüsselkandidat.
Jeder weitere Schlüsselkandidat muss ebenfalls minimal sein und zudem
$A$ enthalten. Daraus folgt, dass $A$ der einzige Schlüsselkandidat ist.

%%
%
%%

\bPseudoUeberschrift{Mit Hilfe des Algorithmus:}

$Test = \m{\m{A, B, C, D}}$ $Erg = \m{}$

\begin{enumerate}

%%
% 1.
%%

\item $K = \m{A, B, C, D}$

$K \SLASH A: \ah{F, \m{B, C, D}} = \m{B, C, D}$ !\\

$K \SLASH B: \ah{F, \m{A, C, D}} = R$\\
$\rightarrow Test = \m{\m{A, C, D}}$ \\

$K \SLASH C: \ah{F, \m{A, B, D}} = R$\\
$\rightarrow Test = \m{\m{A, C, D}, \m{A, B, D}}$ \\

$K \SLASH D: \ah{F, \m{A, B, C}} = R$\\
$\rightarrow Test = \m{\m{A, C, D}, \m{A, B, D}, \m{A, B, C}}$ \\

%%
% 2.
%%

\item $K = \m{A, C, D}$

$K \SLASH A: \ah{F, \m{C, D}} = \m{C, D}$ !\\

$K \SLASH C: \ah{F, \m{A, D}} = R$\\
$\rightarrow Test = \m{\m{A, D}, \m{A, C, D}, \m{A, B, D}, \m{A, B, C}}$ \\

$K \SLASH C: \ah{F, \m{A, C}} = R$\\
$\rightarrow Test = \m{\m{A, C}, \m{A, D}, \m{A, C, D}, \m{A, B, D}, \m{A, B, C}}$ \\

%%
% 3.
%%

\item $K = \m{A, C}$

$K \SLASH A: \ah{F, \m{C}} = \m{C}$ !\\

$K \SLASH C: \ah{F, \m{A}} = R$\\
$\rightarrow Test = \m{\m{A}, \m{A, D}, \m{A, C, D}, \m{A, B, D}, \m{A, B, C}}$ \\

%%
% 4.
%%

\item $K = \m{A}$

$K \SLASH A$: ! $\rightarrow$ kein Superschlüssel ohne A mehr möglich\\
$\rightarrow$ dieses K wandert in Ergebnis und wird in Test gelöscht

$\rightarrow Test = \m{\m{A, D}, \m{A, C, D}, \m{A, B, D}, \m{A, B, C}}$ \\
$\rightarrow Erg = \m{\m{A}}$ \\
\end{enumerate}

analog verfahren wir mit den übrigen Mengen in Test, wie man bereits
sieht bleibt $\m{A}$ einziger Schlüsselkandidat.

\end{bAntwort}

\item Geben Sie für jede der Mengen an funktionalen Abhängigkeiten an,
ob das Relationenschema R in 2. Normalform (2NF) und ob es in 3.
Normalform (3NF) ist. Begründen Sie dies jeweils kurz!

\item Für die Fälle, in denen R nicht in 2NF bzw. 3NF ist, geben Sie
bitte neue Relationenschemata in 3NF an! Erläutern Sie die dazu
durchzuführenden Schritte jeweils kurz!

\item Untersuchen Sie für die Fälle d) und e), ob das Relationenschema
in Boyce-Codd-Normalform (BCNEF) ist! Geben Sie jeweils eine kurze
Begründung an! Wenn das Relationenschema nicht in BCNF ist, erläutern
Sie, ob eine Zerlegung in eine seman- tisch äquivalente Menge an
Relationenschemata in BCNF möglich ist.
\end{enumerate}
\end{document}
