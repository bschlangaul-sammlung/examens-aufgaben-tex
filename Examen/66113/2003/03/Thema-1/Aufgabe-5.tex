\documentclass{bschlangaul-aufgabe}
\bLadePakete{syntax}
\begin{document}
\bAufgabenMetadaten{
  Titel = {Aufgabe 5: SQL},
  Thematik = {Mitarbeiterverwaltung},
  Referenz = 66113-2003-F.T1-A5,
  RelativerPfad = Staatsexamen/66113/2003/03/Thema-1/Aufgabe-5.tex,
  ZitatSchluessel = 66113:2003:03,
  ZitatBeschreibung = {Thema 1 Aufgabe 5},
  BearbeitungsStand = mit Lösung,
  Korrektheit = unbekannt,
  Ueberprueft = {unbekannt},
  Stichwoerter = {SQL, SQL mit Übungsdatenbank},
  EinzelpruefungsNr = 66113,
  Jahr = 2003,
  Monat = 03,
  ThemaNr = 1,
  AufgabeNr = 5,
}

Gegeben\index{SQL} \footcite[Thema 1 Aufgabe 5]{66113:2003:03} seien die
folgenden drei Relationen. Diese Relationen erfassen die
Mitarbeiterverwaltung eines Unternehmens. Schlüssel sind fett
dargestellt und Fremdschlüssel sind kursiv dargestellt. So werden
Mitarbeiter, Abteilungen und Unternehmen jeweils durch ihre Nummer
identifiziert. AbtNr ist die Nummer der Abteilung, in der ein
Mitarbeiter arbeitet. Manager ist die Nummer des Mitarbeiters, der die
Abteilung leitet. \verb|UntNr| ist die Nummer des Unternehmens, dem eine
Abteilung zugeordnet ist.
\footcite[Aufgabe 2: SQL]{db:pu:2}

\begin{minted}{md}
Mitarbeiter(Nummer, Name, Alter, Gehalt, AbtNr)
Abteilung(Nummer, Name, Budget, Manager, UntNr)
Unternehmen(Nummer, Name, Adresse)
\end{minted}

% Datenbankname: mitarbeiterverwaltung
\begin{minted}{sql}
CREATE TABLE unternehmen (
  Nummer integer NOT NULL PRIMARY KEY,
  Name VARCHAR(20) DEFAULT NULL,
  Adresse VARCHAR(50) DEFAULT NULL
);

CREATE TABLE abteilung (
  Nummer integer NOT NULL PRIMARY KEY,
  Name VARCHAR(20) DEFAULT NULL,
  Budget float DEFAULT NULL,
  Manager VARCHAR(20) NOT NULL,
  UntNr integer DEFAULT NULL REFERENCES unternehmen (Nummer)
);

CREATE TABLE mitarbeiter (
  Nummer integer NOT NULL PRIMARY KEY,
  Name VARCHAR(20) NOT NULL,
  Alter integer NOT NULL,
  Gehalt float NOT NULL,
  AbtNr integer NOT NULL REFERENCES abteilung (Nummer)
);

INSERT INTO unternehmen (Nummer, Name, Adresse) VALUES
  (1, 'Test.com', 'Alter Hafen 11'),
  (2, 'Party.de', 'Technostraße 3'),
  (3, 'IT.ch', 'Sequelweg 1');

INSERT INTO abteilung (Nummer, Name, Budget, Manager, UntNr) VALUES
  (1, 'Personal_Care', 20000, 'Huber', 1),
  (11, 'Tequilla_Mix', 50000, 'Taylor', 2),
  (21, 'Nerds', 500, 'Gates', 3);

INSERT INTO mitarbeiter (Nummer, Name, Alter, Gehalt, AbtNr) VALUES
  (1, 'Müller', 30, 30000, 1),
  (2, 'Huber', 45, 80000, 1),
  (3, 'Habermeier', 62, 40000, 1),
  (4, 'Leifsson', 27, 50000, 1),
  (5, 'Taylor', 37, 85000, 11),
  (6, 'Smith', 61, 34000, 11),
  (7, 'Pitt', 36, 40000, 11),
  (8, 'Thompson', 54, 52000, 11),
  (9, 'Gates', 69, 15000000, 21),
  (10, 'Zuckerberg', 36, 10000000, 21),
  (11, 'Jobs', 99, 14000000, 21),
  (12, 'Nakamoto', 66, 5000000, 21);
\end{minted}
\index{SQL mit Übungsdatenbank}

\begin{enumerate}

% a)
\item Wie hoch ist das Durchschnittsalter der Abteilung
„Personal Care“ im Unternehmen „Test.com“?

\begin{bAntwort}
\verb|GROUP BY| nicht nötig, \verb|AS| nicht vergessen.

\begin{minted}{sql}
SELECT AVG(m.Alter) AS Durchschnittsalter
FROM Unternehmen u, Abteilung a, Mitarbeiter m
WHERE
  a.Name = 'Personal Care' AND
  u.Name = 'Test.com' AND
  u.Nummer = a.UntNr AND
  m.AbtNr = a.Nummer;
\end{minted}
\end{bAntwort}

%%
% b)
%%

\item Geben Sie für jedes Unternehmen das Durchschnittsalter
der Mitarbeiter an!

\begin{bAntwort}

Statt \verb|a.UntNr| kann \verb|u.Nummer| verwendet werden.
\verb|a.UntNr| nur deshalb, weil man dann eventuell den Join über die
Unternehmenstabelle sparen kann.

Alles was ausgegeben werden soll, muss auch in \verb|GROUP BY| enthalten
sein.

\begin{minted}{sql}
SELECT a.UntNr, u.Name, AVG(m.Alter) as Durchschnittsalter
FROM Unternehmen u, Abteilung a, Mitarbeiter m
WHERE
  u.Nummer = a.UntNr AND
  m.AbtNr = a.Nummer
GROUP BY a.UntNr, u.Name;
\end{minted}
\end{bAntwort}

%%
% c)
%%

\item Wie viele Mitarbeiter im Unternehmen „Test.com“ sind
älter als ihr Chef? (D.h. sind älter als der Manager der Abteilung, in
der sie arbeiten.)

\begin{bAntwort}
\begin{minted}{sql}
SELECT COUNT(*)
FROM Mitarbeiter m, Abteilung a, Unternehmen u
WHERE
  m.AbtNr = a.Nummer AND
  a.UntNr = u.Nummer AND
  u.Name = 'Test.com'
AND m.Alter > (
  SELECT ma.Alter
  FROM Mitarbeiter ma, Abteilung ab
  WHERE
    ma.Nummer = ab.Manager AND
    a.Nummer = ab.Nummer
);
\end{minted}

oder einfacher:

\begin{minted}{sql}
SELECT COUNT(*)
FROM Mitarbeiter m, Abteilung a, Unternehmen u
WHERE
  m.AbtNr = a.Nummer AND
  a.UntNr = u.Nummer AND
  u.Name = 'Test.com'
AND m.Alter > (
  SELECT ma.Alter
  FROM Mitarbeiter ma
  WHERE ma.Nummer = a.Manager
);
\end{minted}

Alternativ Lösung ohne Unterabfragem, mit Self join:

\begin{minted}{sql}
SELECT COUNT(*)
FROM Mitarbeiter m, Abteilung a, Unternehmen u, Mitarbeiter m2
WHERE
  m.AbtNr = a.Nummer AND
  a.UntNr = u.Nummer AND
  u.Name = 'Test.com' AND
  a.Manager = m2.Nummer AND
  m.Alter > m2.Atler;
\end{minted}
\end{bAntwort}

%%
% (d)
%%

\item Welche Abteilungen haben ein geringeres Budget als die
Summe der Gehälter der Mitarbeiter, die in der Abteilung arbeiten?

\begin{bAntwort}
\begin{minted}{sql}
SELECT a.Name, a.Nummer
FROM Abteilung a
WHERE a.Budget < (
  SELECT SUM(m.Gehalt)
  FROM Mitarbeiter m
  WHERE a.Nummer = m.AbtNr
);
\end{minted}

Ohne Unterabfrage

\begin{minted}{sql}
SELECT a.Name, a.Nummer
FROM Abteilung a, Mitarbeiter m
WHERE a.Nummer = m.AbtNr
GROUP BY a.Nummer, a.Name, a.Budget
HAVING a.Budget < SUM(m.Gehalt);
\end{minted}
\end{bAntwort}

%%
% (e)
%%

\item Versetzen Sie den Mitarbeiter „Wagner“ in die Abteilung „Personal
Care“!

\begin{bAntwort}
\begin{minted}{sql}
UPDATE Mitarbeiter m
SET AbtNr = (
  SELECT a.Nummer FROM
  Abteilung a
  WHERE a.Name = 'Personal Care'
)
WHERE m.Name = 'Wagner';
\end{minted}
\end{bAntwort}

%%
% (f)
%%

\item Löschen Sie die Abteilung „Personal Care“ mit allen ihren
Mitarbeitern!

\begin{bAntwort}
\begin{minted}{sql}
DELETE FROM Mitarbeiter
WHERE AbtNr = (
  SELECT a.Nummer
  FROM Abteilung a
  WHERE a.Name = 'Personal Care'
);

DELETE FROM Abteilung
WHERE Name = 'Personal Care';
\end{minted}
\end{bAntwort}

%%
% (g)
%%

\item Geben Sie den Managern aller Abteilungen, die ihr Budget nicht
überziehen, eine 10 Prozent Gehaltserhöhung. (Das Budget ist überzogen,
wenn die Gehälter der Mitarbeiter höher sind als das Budget der
Abteilung.) Zusatzfrage: Was passiert mit Mitarbeitern, die Manager von
mehreren Abteilungen sind?

\begin{bAntwort}
\begin{minted}{sql}
CREATE VIEW LowBudget AS (
  SELECT Nummer
  FROM Abteilung
  WHERE Nummer NOT IN (
    SELECT a.Nummer
    FROM Abteilung a
    WHERE
      a.Budget < (
        SELECT SUM(Gehalt)
        FROM Mitarbeiter m Abteilung A
        WHERE m.AbtNr = A.Nummer AND
        a.Nummer = A.Nummer
    )
  )
)

UPDATE Mitarbeiter
SET Gehalt = 1.1 * Gehalt
WHERE Nummer IN (
  SELECT Manager
  FROM LowBudget, Abteilung
  WHERE LowBudget.Manager = Abteilung.Nummer
)
\end{minted}
\end{bAntwort}
\end{enumerate}

\end{document}
