\documentclass{bschlangaul-aufgabe}
\bLadePakete{graph}
\begin{document}
\bAufgabenMetadaten{
  Titel = {Aufgabe 6 (Graphrepräsentation)},
  Thematik = {Adjazenzmatrix und Adjazenzliste},
  Referenz = 46114-2006-F.T2-A6,
  RelativerPfad = Staatsexamen/46114/2006/03/Thema-2/Aufgabe-6.tex,
  ZitatSchluessel = 46114:2006:03,
  BearbeitungsStand = mit Lösung,
  Korrektheit = unbekannt,
  Ueberprueft = {unbekannt},
  Stichwoerter = {Graphen, Adjazenzmatrix},
  EinzelpruefungsNr = 46114,
  Jahr = 2006,
  Monat = 03,
  ThemaNr = 2,
  AufgabeNr = 6,
}

\section{Aufgabe 6 (Graphrepräsentation)
\index{Graphen}
\footcite{46114:2006:03}}

Repräsentieren Sie den folgenden Graphen sowohl mit einer
Adjazenzmatrix\index{Adjazenzmatrix} als auch mit einer
Adjazenzliste\index{Adjazenzmatrix}.

\begin{bGraphenFormat}
A: 1 1
B: 1 -1
C: 2 1
D: 2 -1
E: 0 0
F: 3 0

A -> E
B -> A
E -> B
D -> A
A -> C
C -> D
D -> F
F -> C
\end{bGraphenFormat}

\begin{center}
\begin{tikzpicture}[li graph]
\node (A) at (1,1) {A};
\node (B) at (1,-1) {B};
\node (C) at (2,1) {C};
\node (D) at (2,-1) {D};
\node (E) at (0,0) {E};
\node (F) at (3,0) {F};

\path[->] (A) edge node {} (C);
\path[->] (A) edge node {} (E);
\path[->] (B) edge node {} (A);
\path[->] (C) edge node {} (D);
\path[->] (D) edge node {} (A);
\path[->] (D) edge node {} (F);
\path[->] (E) edge node {} (B);
\path[->] (F) edge node {} (C);
\end{tikzpicture}
\end{center}

\begin{bAntwort}
\[
\begin{blockarray}{ccccccc}
   & A & B & C & D & E & F \\
\begin{block}{c(cccccc)}
 A & * & - & 1 & - & 1 & - \\
 B & 1 & * & - & - & - & - \\
 C & - & - & * & 1 & - & - \\
 D & 1 & - & - & * & - & 1 \\
 E & - & 1 & - & - & * & - \\
 F & - & - & 1 & - & - & * \\
\end{block}
\end{blockarray}
\]

\begin{tabular}{lll}
A & $\rightarrow$ C & $\rightarrow$ E \\
B & $\rightarrow$ A \\
C & $\rightarrow$ D \\
D & $\rightarrow$ A & $\rightarrow$ F \\
E & $\rightarrow$ B \\
F & $\rightarrow$ C \\
\end{tabular}
\end{bAntwort}

\end{document}
