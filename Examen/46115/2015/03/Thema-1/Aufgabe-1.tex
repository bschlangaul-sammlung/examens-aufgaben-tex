\documentclass{bschlangaul-aufgabe}
\bLadePakete{automaten}
\begin{document}
\bAufgabenMetadaten{
  Titel = {Aufgabe 1},
  Thematik = {Sprache abc},
  Referenz = 46115-2015-F.T1-A1,
  RelativerPfad = Staatsexamen/46115/2015/03/Thema-1/Aufgabe-1.tex,
  ZitatSchluessel = examen:46115:2015:03,
  BearbeitungsStand = mit Lösung,
  Korrektheit = unbekannt,
  Ueberprueft = {unbekannt},
  Stichwoerter = {Reguläre Sprache},
  EinzelpruefungsNr = 46115,
  Jahr = 2015,
  Monat = 03,
  ThemaNr = 1,
  AufgabeNr = 1,
}

\section{Aufgabe 1
\index{Reguläre Sprache}
\footcite{examen:46115:2015:03}}

Gegeben sei die Sprache $L$.
$L$ besteht aus der Menge aller Worte über dem Alphabet $\{a, b, c\}$,
die mit $a$ beginnen und mit $b$ enden
und die nie zwei aufeinander folgende $c$’s enthalten.
\begin{enumerate}

%%
% (a)
%%

\item Geben Sie einen regulären Ausdruck für $L$ an.

\begin{bAntwort}
\texttt{a(c?[ab]+)*(cb|b)}

\texttt{a(c(a|b)|(a|b))*(cb|b)}
\end{bAntwort}

%%
% (b)
%%

\item Geben Sie einen vollständigen deterministischen endlichen
Automaten für $L$ an.

\begin{bAntwort}
\begin{center}
\begin{tikzpicture}[->,node distance=2cm]
\node[state,initial] (0) {$z_0$};
\node[state,right of=0] (1) {$z_1$};
\node[state,above right of=1] (2) {$z_2$};
\node[state,right of=1,accepting] (3) {$z_3$};

\path (0) edge[above] node{a} (1);
\path (1) edge[below] node{b} (3);
\path (1) edge[below,loop below] node{a,b} (1);
\path (1) edge[below] node{c} (2);
\path (2) edge[above left,bend right] node{a,b} (1);
\path (2) edge[right] node{b} (3);
\end{tikzpicture}
\end{center}
\end{bAntwort}
\end{enumerate}
\end{document}
