\documentclass{bschlangaul-aufgabe}
\bLadePakete{syntax}
\begin{document}
\bAufgabenMetadaten{
  Titel = {Aufgabe 2},
  Thematik = {Minimum und Maximum},
  Referenz = 46115-2021-F.T1-TA2-A2,
  RelativerPfad = Staatsexamen/46115/2021/03/Thema-1/Teilaufgabe-2/Aufgabe-2.tex,
  IdentischeAufgabe = Staatsexamen/66115/2021/03/Thema-1/Teilaufgabe-2/Aufgabe-2.tex,
  ZitatSchluessel = examen:46115:2021:03,
  BearbeitungsStand = mit Lösung,
  Korrektheit = unbekannt,
  Ueberprueft = {unbekannt},
  Stichwoerter = {Lineare Suche},
  EinzelpruefungsNr = 46115,
  Jahr = 2021,
  Monat = 03,
  ThemaNr = 1,
  TeilaufgabeNr = 2,
  AufgabeNr = 2,
}

\begin{enumerate}

%%
% a)
%%

\item Argumentieren Sie, warum man das Maximum von $n$ Zahlen nicht mit
weniger als $n - 1$ Vergleichen bestimmen kann.\index{Lineare Suche}
\footcite{examen:46115:2021:03}

\begin{bAntwort}
Wenn die $n$ Zahlen in einem unsortierten Zustand vorliegen, müssen wir
alle Zahlen betrachten, um das Maximum zu finden. Wir brauchen dazu $n -
1$ und nicht $n$ Vergleiche, da wir die erste Zahl zu Beginn des
Algorithmus als Maximum definieren und anschließend die verbleibenden
Zahlen $n - 1$ mit dem aktuellen Maximum vergleichen.
\end{bAntwort}

%%
% b)
%%

\item Geben Sie einen Algorithmus im Pseudocode an, der das Maximum
eines Feldes der Länge $n$ mit genau $n - 1$ Vergleichen bestimmt.

\begin{bAntwort}
\bJavaExamen[firstline=5,lastline=13]{66115}{2021}{03}{MinimumMaximum.java}
\end{bAntwort}

%%
% c)
%%

\item Wenn man das Minimum und das Maximum von $n$ Zahlen bestimmen
will, dann kann das natürlich mit $2n - 2$ Vergleichen erfolgen. Zeigen
Sie, dass man bei jedem beliebigen Feld mit deutlich weniger Vergleichen
auskommt, wenn man die beiden Werte statt in zwei separaten Durchläufen
in einem Durchlauf geschickt bestimmt.

\begin{bAntwort}
\bJavaExamen[firstline=15]{66115}{2021}{03}{MinimumMaximum.java}
\end{bAntwort}

\end{enumerate}
\end{document}
