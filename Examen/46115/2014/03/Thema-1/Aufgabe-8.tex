\documentclass{bschlangaul-aufgabe}
\bLadePakete{graph}
\begin{document}
\bAufgabenMetadaten{
  Titel = {Frühjahr 2014 (46115) - Thema 1 Aufgabe 8},
  Thematik = {Miminaler Spannbaum im Graph A-H},
  Referenz = 46115-2014-F.T1-A8,
  RelativerPfad = Staatsexamen/46115/2014/03/Thema-1/Aufgabe-8.tex,
  ZitatSchluessel = aud:ab:6,
  ZitatBeschreibung = {Seite 1-2, Aufgabe 2},
  BearbeitungsStand = unbekannt,
  Korrektheit = unbekannt,
  Ueberprueft = {unbekannt},
  Stichwoerter = {Minimaler Spannbaum, Algorithmus von Kruskal},
  EinzelpruefungsNr = 46115,
  Jahr = 2014,
  Monat = 03,
  ThemaNr = 1,
  AufgabeNr = 8,
}

\section{Frühjahr 2014 (46115) - Thema 1 Aufgabe 8
\index{Minimaler Spannbaum}
\index{Algorithmus von Kruskal}
\footcite[Seite 1-2, Aufgabe 2]{aud:ab:6}}

Bestimmen Sie einen minimalen Spannbaum für einen ungerichteten Graphen,
der durch die nachfolgende Entfernungsmatrix gegeben ist! Die Matrix ist
symmetrisch und $\infty$ bedeutet, dass es keine Kante gibt. Zeichnen
Sie den Graphen und geben Sie die Spannbaumkanten ein
\footcite[Seite 5 (PDF 4)]{examen:46115:2014:03}!

\[
\begin{blockarray}{ccccccccc}
  & A      & B      & C      & D      & E      & F      & G      & H      \\
\begin{block}{c(cccccccc)}
A & 0      & 8      & -1     & \infty & 8      & \infty & 7      & \infty \\
B & 8      & 0      & \infty & 2      & \infty & \infty & \infty & 9      \\
C & -1     & \infty & 0      & 5      & 8      & 1      & 7      & \infty \\
D & \infty & 2      & 5      & 0      & 6      & 6      & \infty & \infty \\
E & 8      & \infty & 8      & 6      & 0      & 6      & 3      & \infty \\
F & \infty & \infty & 1      & 6      & 6      & 0      & 11     & 4      \\
G & 7      & \infty & 7      & \infty & 3      & 11     & 0      & 5      \\
H & \infty & 9      & \infty & \infty & \infty & 4      & 5      & 0      \\
\end{block}
\end{blockarray}
\]

% http://graphonline.ru/en/?graph=JACsZrMiExxkFxBK

% 0,8,-1,0,8,0,7,0
% 8,0,0,2,0,0,0,9
% -1,0,0,5,8,1,7,0
% 0,2,5,0,6,6,0,0
% 8,0,8,6,0,6,3,0
% 0,0,1,6,6,0,11,4
% 7,0,7,0,3,11,0,5
% 0,9,0,0,0,4,5,0

\begin{bGraphenFormat}
A: 2 5
B: 5 6
C: 5 1
D: 7 0
E: 2 0
F: 5 3
G: 0 1
H: 3 4.5
A -- B: 8
A -- E: 8
A -- G: 7
B -- D*: 2
B -- H: 9
C -- D*: 5
C -- E: 8
C -- F*
C -- G: 7
D -- E: 6
D -- F: 6
E -- F: 6
E -- G*: 3
F -- G: 11
F -- H*: 4
G -- H*: 5
\end{bGraphenFormat}

\begin{bAntwort}
\begin{minipage}{8cm}
\begin{tikzpicture}[li graph]
\node (A) at (2,5) {A};
\node (B) at (5,6) {B};
\node (C) at (5,1) {C};
\node (D) at (7,0) {D};
\node (E) at (2,0) {E};
\node (F) at (5,3) {F};
\node (G) at (0,1) {G};
\node (H) at (3,4.5) {H};

\path (A) edge node {8} (B);
\path (A) edge node {8} (E);
\path (A) edge node {7} (G);
\path[li markierung] (B) edge node {2} (D);
\path (B) edge node {9} (H);
\path[li markierung] (C) edge node {5} (D);
\path (C) edge node {8} (E);
\path[li markierung] (C) edge node {1} (F);
\path (C) edge node {7} (G);
\path (D) edge node {6} (E);
\path (D) edge node {6} (F);
\path (E) edge node {6} (F);
\path[li markierung] (E) edge node {3} (G);
\path (F) edge node {11} (G);
\path[li markierung] (F) edge node {4} (H);
\path[li markierung] (G) edge node {5} (H);
\end{tikzpicture}

\end{minipage}
\begin{minipage}{4cm}
\begin{tabular}{lr}
Kante & Gewicht \\
\hline
AC & -1 \\
BD & 2 \\
CF & 1 \\
EG & 3 \\
FH & 4 \\
GH & 5 \\
CD & 5 \\\hline
& \textbf{19} \\
\end{tabular}
\end{minipage}

Nach dem Algorithmus von Kruskal wählt man aus den noch nicht gewählten
Kanten immer die kürzeste, die keinen Kreis mit den bisher gewählten
Kanten bildet.
\end{bAntwort}
\end{document}
