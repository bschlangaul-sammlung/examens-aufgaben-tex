\documentclass{bschlangaul-aufgabe}
\bLadePakete{mathe,java}
\begin{document}
\bAufgabenMetadaten{
  Titel = {Aufgabe 4},
  Thematik = {Binomialkoeffizient},
  Referenz = 46115-2014-F.T2-A4,
  RelativerPfad = Staatsexamen/46115/2014/03/Thema-2/Aufgabe-4.tex,
  ZitatSchluessel = examen:46115:2014:03,
  ZitatBeschreibung = {Thema 2 Aufgabe 4},
  BearbeitungsStand = mit Lösung,
  Korrektheit = unbekannt,
  Ueberprueft = {unbekannt},
  Stichwoerter = {Rekursion, Implementierung in Java, Iterative Realisation},
  EinzelpruefungsNr = 46115,
  Jahr = 2014,
  Monat = 03,
  ThemaNr = 2,
  AufgabeNr = 4,
}

\section{Aufgabe 4
\index{Rekursion}
\footcite[Thema 2 Aufgabe 4]{examen:46115:2014:03}
}

Für Binomialkoeffizienten
$\binom{n}{k}$
gelten neben den grundlegenden Beziehungen
$\binom{n}{0} = 1$
und
$\binom{n}{n} = 1$
auch folgende Formeln:

\begin{bExkurs}[Binomialkoeffizient]
Der Binomialkoeffizient ist eine mathematische Funktion, mit der sich
eine der Grundaufgaben der Kombinatorik lösen lässt. Er gibt an, auf wie
viele verschiedene Arten man $k$ bestimmte Objekte aus einer Menge von
$n$ verschiedenen Objekten auswählen kann (ohne Zurücklegen, ohne
Beachtung der Reihenfolge). Der Binomialkoeffizient ist also die Anzahl
der $k$-elementigen Teilmengen einer $n$-elementigen Menge.
\bFussnoteUrl{https://de.wikipedia.org/wiki/Binomialkoeffizient}
\end{bExkurs}

\begin{description}
\item[A]
$
\binom{n+1}{k}
=
\binom{n}{k-1}
+
\binom{n}{k}
$
\item[B]
$
\binom{n}{k}
=
\binom{n-1}{k-1}
\cdot
\frac{n}{k}
$

\end{description}
\begin{enumerate}

%%
% (a)
%%

\item Implementieren\index{Implementierung in Java} Sie unter Verwendung
von Beziehung (A) eine rekursive Methode \bJavaCode{binRek(n, k)} zur
Berechnung des Binomialkoeffizienten in einer objektorientierten
Programmiersprache oder entsprechendem Pseudocode!
\footcite[(entnommen aus Algorithmen und Datenstrukturen, Übungsblatt 2, Universität Würzburg), Aufgabe 6]{aud:pu:7}

\begin{bAntwort}
Zuerst verwandeln wir die Beziehung (A) geringfügig um, indem wir $n$
durch $n-1$ ersetzen:

$
\binom{n}{k}
=
\binom{n-1}{k-1}
+
\binom{n-1}{k}
$

\bJavaExamen[firstline=18,lastline=24]{46115}{2014}{03}{Binomialkoeffizient}
\end{bAntwort}

%%
% (b)
%%

\item Implementieren Sie unter Verwendung von Beziehung (B) eine
iterative Methode\index{Iterative Realisation} \bJavaCode{binIt(n, k)} zur
Berechnung des Binomialkoeffizienten in einer objektorientierten
Programmiersprache oder entsprechendem Pseudocode!

\begin{bAntwort}
\bJavaExamen[firstline=35,lastline=47]{46115}{2014}{03}{Binomialkoeffizient}
\end{bAntwort}

%%
% (c)
%%

\item Geben Sie die Laufzeitkomplexität der Methoden
\bJavaCode{binRek(n, k)} und \bJavaCode{binIt(n, k)} aus den
vorhergehenden beiden Teilaufgaben in O-Notation an!

\end{enumerate}

\bPseudoUeberschrift{Komplette Java-Klasse}

\bJavaExamen{46115}{2014}{03}{Binomialkoeffizient}

\bPseudoUeberschrift{Test}

\bJavaTestDatei{examen/examen_46115/jahr_2014/fruehjahr/BinomialkoeffizientTest}

\end{document}
