\documentclass{bschlangaul-aufgabe}
\bLadePakete{java}
\begin{document}
\bAufgabenMetadaten{
  Titel = {Aufgabe 3},
  Thematik = {Primzahl},
  Referenz = 46115-2017-H.T2-A3,
  RelativerPfad = Staatsexamen/46115/2017/09/Thema-2/Aufgabe-3.tex,
  ZitatSchluessel = examen:46115:2017:09,
  BearbeitungsStand = mit Lösung,
  Korrektheit = unbekannt,
  Ueberprueft = {unbekannt},
  Stichwoerter = {Dynamische Programmierung},
  EinzelpruefungsNr = 46115,
  Jahr = 2017,
  Monat = 09,
  ThemaNr = 2,
  AufgabeNr = 3,
}

Die Methode \bJavaCode{pKR} berechnet die $n$-te Primzahl ($n \geq 1$)
kaskadenartig rekursiv und äußerst ineffizient:
\index{Dynamische Programmierung}
\footcite{examen:46115:2017:09}

\bJavaExamen[firstline=32,lastline=44]{46115}{2017}{09}{PrimzahlDP}

\noindent
Überführen Sie \bJavaCode{pKR} mittels \emph{dynamischer
Programmierung} (hier also \emph{Memoization}) und mit möglichst
\emph{wenigen Änderungen} so in die \emph{linear} rekursive Methode
\bJavaCode{pLR}, dass \bJavaCode{pLR(n, new long[n + 1])} ebenfalls
die $n$-te Primzahl ermittelt:

\begin{minted}{java}
private long pLR(int n, long[] ps) {
  ps[1] = 2;
  // ...
}
\end{minted}

\begin{bAntwort}
\begin{bExkurs}[Kaskadenartig rekursiv]
Kaskadenförmige Rekursion bezeichnet den Fall, in dem mehrere rekursive
Aufrufe nebeneinander stehen.
\end{bExkurs}

\begin{bExkurs}[Linear rekursiv]
Die häufigste Rekursionsform ist die lineare Rekursion, bei der in jedem
Fall der rekursiven Definition höchstens ein rekursiver Aufruf vorkommen
darf.
\end{bExkurs}

\bJavaExamen[firstline=55,lastline=74]{46115}{2017}{09}{PrimzahlDP}

\bPseudoUeberschrift{Der komplette Quellcode}

\bJavaExamen{46115}{2017}{09}{PrimzahlDP}
\end{bAntwort}

\end{document}
