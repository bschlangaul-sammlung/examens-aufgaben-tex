\documentclass{bschlangaul-aufgabe}

\begin{document}
\bAufgabenMetadaten{
  Titel = {Thema Nr. 2},
  Thematik = {Modellierung, Programmierung, Fundamentale Ideen, Werkzeuge, Algorithmen},
  Referenz = 66118-2021-F.A2,
  RelativerPfad = Staatsexamen/66118/2021/03/Aufgabe-2.tex,
  ZitatSchluessel = examen:66118:2021:03,
  BearbeitungsStand = mit Lösung,
  Korrektheit = unbekannt,
  Ueberprueft = {unbekannt},
  Stichwoerter = {DDI},
  EinzelpruefungsNr = 66118,
  Jahr = 2021,
  Monat = 03,
  AufgabeNr = 2,
}
\index{DDI}
\footcite{examen:66118:2021:03}

Im LehrplanPLUS finden Sie für den „Lernbereich 3: Grundlagen der
objektorientierten Modellierung und Programmierung (ca. 26 Stunden)“ für
die 9. Jahrgangsstufe folgenden Text:

\begin{liKasten}

\bPseudoUeberschrift{Kompetenzerwartungen:}

Die Schülerinnen und Schüler...

\begin{itemize}

\item analysieren Objekte aus ihrer Erfahrungswelt (z. B. Fahrzeuge,
Personen) hinsichtlich ihrer Eigenschaften (Attribute) und Fähigkeiten
(Methoden) und abstrahieren sie zu Klassen. Sie stellen Objekte und
Klassen als Grundlage einer möglichen Implementierung grafisch dar.

\item deklarieren eine Klasse sowie die zugehörigen Attribute und
Methoden in einer objektorientierten Programmiersprache.

\item verwenden bei der Implementierung Wertzuweisungen, um
Attributwerte zu ändern, und interpretieren diese als Zustandsänderung
des zugehörigen Objekts.

\item formulieren unter Verwendung der Kontrollstrukturen Algorithmen zu
geeigneten Problemstellungen, u. a. durch grafische Darstellungen.

\item implementieren Methoden auf der Grundlage gegebener Algorithmen
objektorientiert, wobei sie sich des Unterschiedes zwischen
Methodendefinition und Methodenaufruf bewusst sind. Dabei nutzen sie
ggf. auch Methoden anderer Klassen.

\item analysieren, interpretieren und modifizieren Algorithmen, wodurch
sie die Fähigkeit erlangen, fremde Programme flexibel einzusetzen und
kritisch zu bewerten.

\item modellieren durch Klassendiagramme einfache
Generalisierungshierarchien zu geeigneten Strukturen aus ihrer
Erfahrungswelt.

\item implementieren mithilfe einer objektorientierten Sprache einfache
Generalisierungshierarchien; dabei nutzen sie das Konzept der Vererbung
sowie die Möglichkeit, Methoden zu überschreiben
\end{itemize}

\bPseudoUeberschrift{Inhalte zu den Kompetenzen:}

\begin{itemize}
\item objektorientierte Konzepte, u. a. Objekt, Klasse, Attribut,
Attributwert, Methode

\item Variablenkonzept; Arten von Variablen: Parameter, lokale Variable
und Attribute; Übergabewert

\item Wertzuweisung zur Änderung von Variablenwerten

\item Methoden: Methodenkopf, Methodenrumpf, Methodendefinition,
Methodenaufruf, Übergabewert, Rückgabewert; Konstruktor als spezielle
Methode; Standardmethoden zum Geben und Setzen von Attributwerten

\item Algorithmus: Strukturelemente, grafische Darstellung, Pseudocode

\item Datentypen: ganze Zahlen, Gleitkommazahlen, Wahrheitswerte,
Zeichen, Zeichenketten

\item Generalisierung und Spezialisierung: Ober- und Unterklasse,
Vererbung von Attributen und Methoden an Unterklassen, Überschreiben von
Methoden

\item Fachbegriffe: Parameter, Übergabewert, Rückgabewert, lokale
Variable, Wertzuweisung, Konstruktor, Methodenkopf, Methodenrumpf,
Vererbung, Generalisierung, Spezialisierung, Oberklasse, Unterklasse
\end{itemize}

\end{liKasten}
Aufgabe
\begin{enumerate}

%-----------------------------------------------------------------------
% 1.
%-----------------------------------------------------------------------

\item In obigem Lehrplanabschnitt wird mehrfach „Modellierung“ und
„Programmierung“ genannt.

\begin{enumerate}
%%
% a)
%%

\item Grenzen Sie die beiden Begriffe gegeneinander ab und begründen
Sie, weshalb man beides im Informatikunterricht benötigt.

%%
% b)
%%

\item Erläutern Sie das Konzept der „Fundamentalen Ideen“ nach Schwill!
Nehmen Sie dabei Bezug auf „Modellierung“ und „Programmierung“.
\end{enumerate}

%-----------------------------------------------------------------------
% 2.
%-----------------------------------------------------------------------

\item Im Informatikunterricht ist sowohl der Einsatz einer
blockbasierten als auch einer textbasierten Sprache denkbar.
\begin{enumerate}

%%
% a)
%%

\item Gehen Sie jeweils auf Vor- und Nachteile der beiden Möglichkeiten
anhand konkreter Beispiele ein.

%%
% b)
%%

\item Entscheiden Sie dann begründet, welche Wahl Sie in der 9.
Jahrgangsstufe treffen würden.

\end{enumerate}

%-----------------------------------------------------------------------
% 3.
%-----------------------------------------------------------------------

\item Im zitierten Lehrplanabschnitt ist die Verwendung von
Kontrollstrukturen zur Formulierung von Algorithmen vorgesehen. Es gibt
aber keine konkrete Auflistung, welche Kontrollstrukturen besprochen
werden sollen.
\begin{enumerate}

%%
% a)
%%

\item Geben Sie einen kurzen Überblick über die Kontrollstrukturen
imperativer bzw. objektorientierter Programmiersprachen an, die hier
fachlich in Frage kommen könnten.

%%
% b)
%%

\item Im Hinblick darauf, dass für den gesamten Lehrplanabschnitt 26
Unterrichtsstunden zur Verfügung stehen, kann es notwendig sein, sich
auf wenige Kontrollstrukturen beschränken zu müssen. Entscheiden Sie,
welche Kontrollstrukturen Sie wählen würden und erklären Sie, in welcher
Reihenfolge Sie diese in der 9. Jahrgangsstufe einführen würden!
Begründen Sie Ihre Ausführungen.

\end{enumerate}

%%
% 4.
%%

\item Erstellen Sie für den zitierten Lehrplanabschnitt einen
Sequenzplan mit 13 Doppelstunden! Geben Sie für jede dieser
Doppelstunden eine kurze, nachvollziehbare Beschreibung der jeweiligen
Zielsetzung an.

%%
% 5.
%%

\item Erstellen Sie eine Feinplanung für eine Doppelstunde zur bedingten
Wiederholung unter Verwendung der von Ihnen in Aufgabe 2b) gewählten Art
der Programmiersprache.
\begin{enumerate}
%%
% a)
%%

\item Legen Sie zunächst drei beobachtbare Lernziele fest.

%%
% b)
%%

\item Skizzieren Sie eine Einführungsaufgabe, die die Schülerinnen und
Schüler zu diesem Thema bearbeiten sollen.

%%
% c)
%%

\item Erläutern Sie anschließend den Unterrichtsfortgang nachvollziehbar
(textuelle Beschreibung). Begründen Sie dabei Ihre fachdidaktischen
Entscheidungen und gruppieren Sie Ihren Text nach Unterrichtsphasen.

\end{enumerate}
\end{enumerate}
\end{document}
