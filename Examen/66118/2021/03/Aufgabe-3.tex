\documentclass{bschlangaul-aufgabe}

\begin{document}
\bAufgabenMetadaten{
  Titel = {Thema Nr. 3},
  Thematik = {Klassensatz programmierbare Roboter},
  Referenz = 66118-2021-F.A3,
  RelativerPfad = Staatsexamen/66118/2021/03/Aufgabe-3.tex,
  ZitatSchluessel = examen:66118:2021:03,
  BearbeitungsStand = mit Lösung,
  Korrektheit = unbekannt,
  Ueberprueft = {unbekannt},
  Stichwoerter = {7. Jahrgangsstufe},
  EinzelpruefungsNr = 66118,
  Jahr = 2021,
  Monat = 03,
  AufgabeNr = 3,
}

Der LehrplanPLUS der 7. Jahrgangsstufe (Natur und Technik) des
neunjährigen Gymnasiums enthält den folgenden Lehrplanpunkt:
\index{7. Jahrgangsstufe}
\footcite{examen:66118:2021:03}

\begin{liKasten}
NT7 2.3 Beschreibung von Abläufen durch Algorithmen (ca. 11 Std.)

\bPseudoUeberschrift{Kompetenzerwartungen}

Die Schülerinnen und Schüler ...

\begin{itemize}
\item analysieren und strukturieren geeignete Problemstellungen u. a.
aus ihrer Erfahrungswelt (z. B. Bedienung eines Geräts), entwickeln
Algorithmen zu deren Lösung und beschreiben diese unter effizienter
Verwendung von Kontrollstrukturen.

\item setzen unter sinnvoller Nutzung algorithmischer Bausteine einfache
Algorithmen mithilfe geeigneter Programmierwerkzeuge um.
\end{itemize}

\bPseudoUeberschrift{Inhalte zu den Kompetenzen:}

\begin{itemize}
\item Algorithmus: Definition des Begriffs, Strukturelemente (Anweisung,
Sequenz, ein- und zweiseitig bedingte Anweisung, Wiederholung mit fester
Anzahl, Wiederholung mit Bedingung)

\item Fachbegriffe: Algorithmus, Anweisung, Sequenz, ein- und zweiseitig
bedingte Anweisung, Wiederholung mit fester Anzahl, Wiederholung mit
Bedingung
\end{itemize}
\end{liKasten}

\bPseudoUeberschrift{Aufgabe}

\noindent
Gehen Sie bei den folgenden Aufgaben von folgendem Szenario aus:

\begin{quote}
Ihrer Schule wurde eine Sachspende über einen Klassensatz
programmierbare Roboter angeboten. Dabei handelt es sich um fertig
montierte, fahrbare Kleinroboter mit je zwei Abstandssensoren (siehe
Abbildung).

Der Roboter verfügt über zwei unabhängige Motoren, deren Geschwindigkeit
gesteuert werden kann. Kurven können gefahren werden, indem die Motoren
unterschiedlich schnell laufen. Zur Programmierung steht eine einfache
Umgebung zur Verfügung.
\end{quote}

\begin{enumerate}

%%
% 1.
%%

\item Der oben zitierte Lehrplanpunkt wird oftmals anhand einer Software
mit einem simulierten, steuerbaren Roboter — z.B. Robot Karol —
unterrichtet. Diskutieren Sie Unterschiede zwischen dieser Simulation
und dem Einsatz der beschriebenen realen Roboter aus fachdidaktischer
Sicht. Beschreiben Sie, welchen Ansprüchen die Programmierumgebung
genügen sollte, damit ein Unterrichtseinsatz in Natur und Technik zu
obigem Lehrplanpunkt möglich und sinnvoll ist.

\begin{bAntwort}
Der größte Unterschied ist sicherlich, dass der hier beschriebene
Roboter real, also anfassbar ist. SuS können unmittelbar in der Realität
erfahren, was ihre Programmierung auslöst. Dies ist vor allem für solche
SuS vorteilhaft, die einen Computer und dessen Programme als eine Art
Blackbox verstehen, d.h. die z.B. Robot Karol als Spielzeug ansehen, das
keinen Bezug zur Lebenswelt hat, da er nur im Rechner existiert.
Zusätzlich sehe ich den Roboter aus oben genanntem Grund als
genderneutraler als Robot Karol, der wohl eher die computerbegeisterten
Jungs in der Klasse ansprechen wird.

Darüber hinaus kann man anhand des Roboters Fragen aus den Bereich
Technik aufnehmen, wie z.B. „Wie funktioniert eine Lenkung im Auto?“ im
Gegensatz zur Frage „Wie kann man mit einer starren Achse wie z.B. bei
einem Rollstuhl lenken?“. Solche Querverweise sind bei Robot Karol nicht
möglich.

Für Robot Karol spricht hingegen, dass die Möglichkeiten des
Programmierens vielfältiger sind. So kann der reale Roboter z.B. keine
Steine „hinlegen“. Somit erscheint mir auf den ersten Blick Robot Karol
abwechslungsreicher zu sein.

Ein weiterer Unterschied könnte darin liegen, dass Robot Karol eine
textbasierte Programmierung erfordert, wohingegen der reale Roboter
blockbasiert funktionieren könnte (je nach Hersteller). Der große
Vorteil blockbasierter Programmiersprachen wie z.B. Snap! liegt beim
Einstieg in das Thema „Programmierung“ darin, dass sich die SuS auf die
Art und Weise, wie programmiert wird, konzentrieren können und sich
keine Gedanken darüber machen müssen, wie nun der richtige Befehl
lautet. Sie können sich also auf die Strukturen des Programmierens
konzentrieren.

\bPseudoUeberschrift{Ansprüche an die Programmierumgebung}

Der Roboter sollte eine einfache, intuitiv strukturierte
Benutzeroberfläche aufweisen. Vorteilhaft wäre hierfür eine App für
Tablets, die sich über das W-LAN der Schule mit den Robotern verbinden
kann. Die App sollte blockbasiert sein, wobei es für die einzelnen
Fähigkeiten des Roboters vorgefertigte Bricks geben muss. Darüber hinaus
müssen die im Lehrplan geforderten Sequenzen, bedingte Anweisungen und
Wiederholungen möglich sein. Die erstellten Programme sollten für jede
SuS speicherbar sein, so dass in der Folgestunde weitergearbeitet werden
kann.
\end{bAntwort}

%%
% 2.
%%

\item Geben Sie eine Grobplanung der Unterrichtssequenz für oben
zitierten Lehrplanpunkt unter Nutzung der realen Roboter an. Gehen Sie
dabei von insgesamt sechs Doppelstunden aus! Nennen Sie für jede Stunde
ein beobachtbares Feinziel.

\begin{bAntwort}
Anmerkung: Jede Doppelstunde besteht nach der in folgenden beschriebenen Wissensvermittlung per Lehrerinput aus ausreichend Aufgaben, die die SuS selbstständig lösen sollen. Die Schüleraktivierungsphasen werden nicht explizit in der folgenden Grobübersicht erwähnt, sind aber wesentlicher Bestandteil der Doppelstunde.

\begin{description}
\item[1.DS: Einführung in die Bedienung des Roboters]

SuS erhalten einen Grobüberblick über die Funktionsmöglichkeiten des
Roboters und die einzelnen Bricks zur Programmierung in der zugehörigen
App -> Klärung der Fachbegriffe Anweisung und Sequenz Feinziel: SuS sind
am Ende der Stunde in der Lage, einfache Bewegungen des Roboters zu
programmieren und auszuführen.

\item[2.DS: Wiederholung mit fester Anzahl]

SuS lernen, mit Wiederholungen Programmsequenzen mehrfach hintereinander
auszuführen. Auch eine Schachtelung von Wiederholungen werden gelernt
und deren Sinn erarbeitet -> Klärung des Fachbegriffs Wiederholung mit
fester Anzahl Feinziel: SuS sind am Ende der Stunde in der Lage sein,
den Roboter mit Hilfe von Wiederholungen z.B. Quadrate und Rechtecke
automatisch fahren zu lassen.

\item[3.DS: Wiederholung mit Bedingung]

SuS lernen mögliche Bedingungen kennen und die Art, wie man mit diesen
programmieren kann. SuS erkennen den Vorteil gegenüber der Wiederholung
mit fester Anzahl erarbeitet. SuS lernen den Begriff Algorithmus im
Zusammenhang mit ihren Programmsequenzen kenn -> Klärung der
Fachbegriffe Wiederholung mit Bedingung und Algorithmus Feinziel: SuS
sind am Ende der Stunde in der Lage sein, den Roboter mit Hilfe von
Wiederholungen mit Bedingungen zu programmieren.

\item[4.DS: Aufgabenstunde]

SuS erhalten Aufgaben (Schwierigkeit variabel), die sie mit den
bisherigen Unterrichtsinhalten lösen können. Jede SuS wählt
Anfangsschwierigkeit selbst aus. Feinziel: SuS sind am Ende der Stunde
in der Lage sein, vorgefertigte Probleme mit einem eigenen Programm
lösen zu können.

\item[5.DS: bedingte Anweisung]

SuS lernen die Möglichkeit kennen, dem Roboter mögliche Alternativen im
Programmablauf zu programmieren erarbeitet -> Klärung des Fachbegriffs
ein- und zweiseitig bedingte Anweisung Feinziel: SuS sind am Ende der
Stunde in der Lage sein, zweiseitig bedingte Anweisungen programmieren
zu können

\item[6.DS: Zusammenfassende Aufgabenstunde]
Vgl. 4. DS

\end{description}
\end{bAntwort}

%%
% 3.
%%

\item Entwerfen Sie einen schriftlichen Leistungsnachweis mit
Lösungsskizze für eine Bearbeitungszeit von 20 Minuten. Geben Sie an, an
welcher Stelle der Unterrichtssequenz Sie ihn einsetzen würden.

\begin{bAntwort}
Test zu Beginn der 6.DS zur Überprüfung der Programmierfähigkeiten der SuS

\begin{description}
\item[Aufgabe 1:] Erkläre den Unterschied zwischen einer einseitig und
einer zweiseitig bedingten Anweisung!

\item[Aufgabe 2:] Schreibe in Anlehnung an den Unterricht die
Hilfsprogramme „Linksdrehen“ und „Umdrehen“ auf!

\item[Aufgabe 3:] Unser Roboter soll das Ende eines verwinkelten Ganges
finden. Der Gang ist seitlich durch Mauern begrenzt und kann sich nur um
90° nach links bzw. nach rechts biegen (keine Verzweigungen!!!!!). Das
Ende ist durch eine Sackgasse gegeben.
Mögliches Beispiel:

Schreibe ein Programm, mit dem der Roboter für alle möglichen Gänge
dieser Art den Endpunkt findet. Verwende hierbei sowohl Wiederholungen
mit Bedingungen als auch zweiseitig bedingte Anweisungen!
\end{description}

Lösungsskizze:

\begin{description}
\item[Zu 2:] Linksdrehen soll den Roboter um 90° nach links drehen.
Dabei bewegt sich das rechte Rad vorwärts, das linke rückwärts. Umdrehen
ist ein zweimaliges Linksdrehen!

\item[Zu 3:] Die SuS können auch die aus dem Programm bekannten Bricks
darstellen!
\begin{verbatim}
Wiederhole solange NichtIstWand
  vorwärts
endeWiederhole
linksdrehen
Wenn IstWand
  Dann Umdrehen
     Wenn IstWand
        Dann fertig
        Sonst vorwärts
      endeWenn
  Sonst vorwärts
endeWenn
\end{verbatim}
\end{description}
\end{bAntwort}

%%
%  4.
%%

\item Geben Sie für den Einstieg in die Unterrichtssequenz (erste
Doppelstunde) eine Feinplanung an. Beschreiben Sie dabei den geplanten
Ablauf detailliert und skizzieren Sie Tafelbilder, Hefteinträge,
Arbeitsblätter o. Ä. (ggf. mit einer Musterlösung).

\begin{bAntwort}
\begin{description}
\item[Einstieg/Motivation (5Min)]
Zeigen eines Videos zum Thema „Roboter in der Wirtschaft“ (z.B. in der
Autoproduktion) Vorführen des schuleigenen Roboters

\item[Kennenlernphase (15Min)]
Benutzeroberfläche der App wird per Beamer an die Wand projiziert und
die einzelnen grundlegenden Bestandteile der App wird im Lehrervortrag
erklärt. Arbeitsblatt mit Screenshot der App und einzelnen Bestandteile
zum Ausfüllen durch die Schüler (hier nicht möglich, den Screenshot
anzugeben) Die Begriffe Anweisung und Sequenz werden erklärt ->
Tafelbild und Hefteintrag

\item[Ausprobierphase (25Min)]
Die SuS erhalten einen Roboter, verbinden diesen mit der
Bedienoberfläche wie in der Kennenlernphase beschrieben und führen erste
Anweisungen aus (von SuS frei wählbar). Arbeitsauftrag: Wie kann sich
der Roboter rechtsdrehen? -> Gruppen- bzw. Einzelarbeit (nach Anzahl der
Roboter)

\item[Erarbeitungsphase (30Min)]
SuS sollen Sequenzen zu folgenden Aufgaben schreiben:
Mein Roboter fährt einen Kreis mit 70 cm Radius.
Mein Roboter fährt ein Rechteck mit Kantenlängen 40 und 60 cm.
Wettbewerb: Welche Gruppe setzt die Vorgaben am besten um?

\item[Wissenssicherungs- und Aufräumphase (10Min)]
SuS geben folgende Inhalte wiederholend in eigenen Worten wieder:
Anweisung – Sequenz- grobe Steuerelemente des Roboters Aufräumen der
Roboter

\end{description}
\end{bAntwort}

%%
% 5.
%%

\item Geben Sie weitere Stellen des bayerischen Lehrplans für das Fach
Informatik am neunjährigen Gymnasium (NTG) an, an denen Roboter dieser
oder ähnlicher Art sinnvoll eingesetzt werden könnten. Bewerten Sie in
jedem Fall knapp, ob die eingangs beschriebenen Roboter dafür geeignet
wären und nennen Sie ggf. Alternativen.

\begin{bAntwort}
Der Einsatz eines Roboters wäre vielleicht noch in der 9. Jahrgangsstufe
bei der Einführung in die objektorientierte Programmierung möglich.
Dabei kann anhand des Roboters folgende Lehrplaninhalte Eingeführt
werden:

\begin{description}
\item analysieren Objekte aus ihrer Erfahrungswelt (z. B. Fahrzeuge,
Personen) hinsichtlich ihrer Eigenschaften (Attribute) und Fähigkeiten
(Methoden) und abstrahieren sie zu Klassen. Sie stellen Objekte und
Klassen als Grundlage einer möglichen Implementierung grafisch dar.

\item deklarieren eine Klasse sowie die zugehörigen Attribute und
Methoden in einer objektorientierten Programmiersprache.

\item verwenden bei der Implementierung Wertzuweisungen, um
Attributwerte zu ändern, und interpretieren diese als Zustandsänderung
des zugehörigen Objekts.
\end{description}

Um kompliziertere Vorgänge beschreiben zu können, erscheint der in der
Aufgabe beschriebene Roboter zu limitiert zu sein. So haben z.B. Lego
Mindstorm-Roboter aufgrund ihrer vielfältigen Sensoren viel mehr
Möglichkeiten, weshalb diese wohl viel ansprechender für SuS der 9.
Klasse sein werden.
\end{bAntwort}

\end{enumerate}
\end{document}
