\documentclass{bschlangaul-aufgabe}
\bLadePakete{baum}
\begin{document}
\bAufgabenMetadaten{
  Titel = {Aufgabe 6: B-Baum},
  Thematik = {B-Baum der Ordnung 3},
  Referenz = 66116-2015-H.T2-TA1-A3,
  RelativerPfad = Staatsexamen/66116/2015/09/Thema-2/Teilaufgabe-1/Aufgabe-3.tex,
  ZitatSchluessel = examen:66116:2015:09,
  BearbeitungsStand = mit Lösung,
  Korrektheit = unbekannt,
  Ueberprueft = {unbekannt},
  Stichwoerter = {B-Baum},
  EinzelpruefungsNr = 66116,
  Jahr = 2015,
  Monat = 09,
  ThemaNr = 2,
  TeilaufgabeNr = 1,
  AufgabeNr = 3,
}

Gegeben\index{B-Baum}\footcite{examen:66116:2015:09} ist der folgende
B-Baum der Ordnung 3 (max. drei Kindknoten, max. zwei Schlüssel pro
Knoten):
\footcite[Seite 3 und 4, (Einfügen und Löschen-Operation a)]{aud:ab:7}

\begin{center}
\begin{tikzpicture}[
  b bbaum,
  level 1/.style={level distance=10mm,sibling distance=50mm},
  level 2/.style={level distance=10mm,sibling distance=20mm}
]
\node {50} [->]
  child {node {10 \nodepart{two} 30}
    child {node {5 \nodepart{two} 8}}
    child {node {15 \nodepart{two} 20}}
    child {node {31 \nodepart{two} 33}}
  }
  child {node {70}
    child {node {60}}
    child {node {80}}
  };
\end{tikzpicture}
\end{center}

\noindent
Fügen Sie die Werte 9 und 45 ein. Löschen Sie anschließend die Werte 30
und 70. Zeichnen Sie den Baum nach jeder Einfüge- bzw. Lösch-Operation.

\begin{bAntwort}
\bPseudoUeberschrift{9 einfügen}

\begin{tikzpicture}[
  scale=0.8,
  transform shape,
  b bbaum,
  level 1/.style={level distance=10mm,sibling distance=35mm},
  level 2/.style={level distance=10mm,sibling distance=20mm},
]
\node {10 \nodepart{two} 50} [->]
  child {node {8}
    child {node {5}}
    child {node[thick,font=\bfseries] {9}}
  }
  child {node {30}
    child {node {15 \nodepart{two} 20}}
    child {node {31 \nodepart{two} 33}}
  }
  child {node {70}
    child {node {60}}
    child {node {80}}
  }
;
\end{tikzpicture}

\bPseudoUeberschrift{45 einfügen}

\begin{tikzpicture}[
  scale=0.8,
  transform shape,
  b bbaum,
  level 1/.style={level distance=10mm,sibling distance=45mm},
  level 2/.style={level distance=10mm,sibling distance=20mm},
]
\node {10 \nodepart{two} 50} [->]
  child {node {8}
    child {node {5}}
    child {node {9}}
  }
  child {node {30 \nodepart{two} 33}
    child {node {15 \nodepart{two} 20}}
    child {node {31}}
    child {node[thick,font=\bfseries]{45}}
  }
  child {node {70}
    child {node {60}}
    child {node {80}}
  }
;
\end{tikzpicture}

\bPseudoUeberschrift{30 löschen}

\begin{tikzpicture}[
  scale=0.8,
  transform shape,
  b bbaum,
  level 1/.style={level distance=10mm,sibling distance=42mm},
  level 2/.style={level distance=10mm,sibling distance=20mm},
]
\node {10 \nodepart{two} 50} [->]
  child {node {8}
    child {node {5}}
    child {node {9}}
  }
  child {node {20 \nodepart{two} 33}
    child {node {15}}
    child {node {31}}
    child {node {45}}
  }
  child {node {70}
    child {node {60}}
    child {node {80}}
  }
;
\end{tikzpicture}

\bPseudoUeberschrift{70 löschen}

\begin{tikzpicture}[
  scale=0.8,
  transform shape,
  b bbaum,
  level 1/.style={level distance=10mm,sibling distance=32mm},
  level 2/.style={level distance=10mm,sibling distance=20mm},
]
\node {10 \nodepart{two} 33} [->]
  child {node {8}
    child {node {5}}
    child {node {9}}
  }
  child {node {20}
    child {node {15}}
    child {node {31}}
  }
  child {node {50}
    child {node {45}}
    child {node {60 \nodepart{two} 80}}
  }
;
\end{tikzpicture}
\end{bAntwort}

\end{document}
