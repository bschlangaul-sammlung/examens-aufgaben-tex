\documentclass{bschlangaul-aufgabe}
\bLadePakete{normalformen}
\begin{document}
\bAufgabenMetadaten{
  Titel = {2. Normalformen},
  Thematik = {Relation A-H},
  Referenz = 66116-2015-H.T1-TA1-A2,
  RelativerPfad = Staatsexamen/66116/2015/09/Thema-1/Teilaufgabe-1/Aufgabe-2.tex,
  ZitatSchluessel = examen:66116:2015:09,
  BearbeitungsStand = mit Lösung,
  Korrektheit = unbekannt,
  Ueberprueft = {unbekannt},
  Stichwoerter = {Normalformen},
  EinzelpruefungsNr = 66116,
  Jahr = 2015,
  Monat = 09,
  ThemaNr = 1,
  TeilaufgabeNr = 1,
  AufgabeNr = 2,
}

Gegeben sei folgendes verallgemeinerte Relationenschema in 1. Normalform:
\index{Normalformen}
\footcite{examen:66116:2015:09}

\begin{center}
\bRelation{A,B,C,D,E,F,G,H}
\end{center}

Für R soll die folgende Menge FD von funktionalen Abhängigkeiten gelten:

\bFunktionaleAbhaengigkeiten{
  F -> E;
  A -> B, D;
  A, E -> D;
  A -> E,F;
  A, G -> H;
}

Bearbeiten Sie mit diesen Informationen folgende Teilaufgaben. Vergessen
Sie dabei nicht Ihr Vorgehen stichpunktartig zu dokumentieren und zu
begründen.

\begin{enumerate}

%%
% a)
%%

\item Bestimmen Sie alle Schlüsselkandidaten von R. Begründen Sie
stichpunktartig, warum es außer den von Ihnen gefundenen
Schlüsselkandidaten keine weiteren geben kann.

%%
% b)
%%

\item Ist R in 2NF, 3NF?

%%
% c)
%%

\item Berechnen Sie eine kanonische Überdeckung von FD. Es genügt, wenn
Sie für jeden der vier Einzelschritte die Menge der funktionalen
Abhängigkeiten als Zwischenergebnis angeben.

%%
% d)
%%

\item Bestimmen Sie eine Zerlegung von R in 3NF. Wenden Sie hierfür den
Synthesealgorithmus an.

\end{enumerate}

\end{document}
