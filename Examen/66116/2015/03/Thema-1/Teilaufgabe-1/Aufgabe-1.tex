\documentclass{bschlangaul-aufgabe}

\begin{document}
\bAufgabenMetadaten{
  Titel = {Aufgabe},
  Thematik = {Musik-Datenbank},
  Referenz = 66116-2015-F.T1-TA1-A1,
  RelativerPfad = Staatsexamen/66116/2015/03/Thema-1/Teilaufgabe-1/Aufgabe-1.tex,
  ZitatSchluessel = examen:66116:2015:03,
  BearbeitungsStand = mit Lösung,
  Korrektheit = unbekannt,
  Ueberprueft = {unbekannt},
  Stichwoerter = {Entity-Relation-Modell},
  EinzelpruefungsNr = 66116,
  Jahr = 2015,
  Monat = 03,
  ThemaNr = 1,
  TeilaufgabeNr = 1,
  AufgabeNr = 1,
}

In einer Musik-Datenbank sollen folgende Informationen zu Interpreten und deren CDs modelliert
werden:

\begin{itemize}
\item Zu einem Interpreten soll eine eindeutige ID, der Name, das Jahr
seines Bühnenstarts, seine Geschäftsadresse sowie sein Musikgenre
angegeben sein. Das Musikgenre kann mehrere Werte umfassen (mehrwertiges
Attribut).

\item Eine CD hat eine eindeutige ID, einen Namen (Titel), einen
Interpreten, ein Erscheinungsdatum und bis zu 20 Positionen
(Musikstücke). An jeder Position steht ein Musikstück. Für dieses ist
der Titel und die Länge in Sekunden angegeben.

\item Eine CD kann Auszeichnungen - \zB vom Typ goldene Schallplatte
oder Emmy bekommen. Ebenso kann auch ein einzelnes Musikstück
Auszeichnungen bekommen.
\index{Entity-Relation-Modell}
\footcite{examen:66116:2015:03}
\end{itemize}
\begin{enumerate}

%%
% a)
%%

\item  Modellieren Sie das oben dargestellte Szenario möglichst
vollständig in einem ER-Modell. Verwenden Sie, wann immer möglich,
(binäre oder auch höherstellige) Relationships. Modellieren Sie
Musikstücke in einem schwachen Entity-Typen.

%%
% b)
%%

\item Übertragen Sie Ihr ER-Modell - bis auf die Typen zu den
Auszeichnungen - ins relationale Datenmodell. Erstellen Sie dazu
Tabellen mit Hilfe von CREATE TABLE-Statements in Sov. Berücksichtigen
Sie die Fremdschlüsselbeziehungen.

%%
% c)
%%

\item Es soll die Integritätsbedingung eingehalten werden, so dass die
Anzahl der Positionen auf einer CD höchstens 20 ist. Schreiben Sie ein
SELECT-Statement, das diese Integritätsbedin. gung überprüft, indem es
die verletzenden CDs ausgibt.

%%
% d)
%%

\item Geben Sie geeignete INSERT-Statements an, die in alle beteiligten
Tabellen jeweils mindestens ein Tupel einfügen, so dass alle
Integritätsbedingungen erfüllt sind, nachdem alle Einfü- gungen
ausgeführt wurden. Lediglich zu den Auszeichnungen müssen keine Tupel
eingefügt werden.
\end{enumerate}

\end{document}
