\documentclass{bschlangaul-aufgabe}
\bLadePakete{java,kontrollflussgraph}
\begin{document}
\bAufgabenMetadaten{
  Titel = {Aufgabe 3},
  Thematik = {Sort-Methode und datenflussorientierte Überdeckungskritierien},
  Referenz = 66116-2016-H.T1-TA2-A3,
  RelativerPfad = Staatsexamen/66116/2016/09/Thema-1/Teilaufgabe-2/Aufgabe-3.tex,
  ZitatSchluessel = sosy:ab:7,
  BearbeitungsStand = mit Lösung,
  Korrektheit = unbekannt,
  Ueberprueft = {unbekannt},
  Stichwoerter = {Datenflussorientiertes Testen, Bubblesort, Datenfluss-annotierter Kontrollflussgraph, Zyklomatische Komplexität nach Mc-Cabe, C1-Test Zweigüberdeckung (Branch Coverage), all uses},
  EinzelpruefungsNr = 66116,
  Jahr = 2016,
  Monat = 09,
  ThemaNr = 1,
  TeilaufgabeNr = 2,
  AufgabeNr = 3,
}

\let\c=\bKontrollCode
\let\w=\bBedingungWahr
\let\f=\bBedingungFalsch

Gegeben\index{Datenflussorientiertes Testen}
\footcite{sosy:ab:7} Sei folgende Java-Methode \bJavaCode{sort} zum Sortieren eines Feldes
ganzer Zahlen:
\footcite[Thema 1 Teilaufgabe 2 Aufgabe 3]{examen:66116:2016:09}

\bJavaExamen[firstline=4]{66116}{2016}{09}{BubbleSort}
\index{Bubblesort}

\begin{enumerate}

%%
% (a)
%%

\item Konstruieren Sie den
Kontrollflussgraphen\index{Datenfluss-annotierter Kontrollflussgraph}
des obigen Code-Fragments und annotieren Sie an den Knoten und Kanten
die zugehörigen Datenflussinformationen (Definitionen bzw. berechnende
oder prädikative Verwendung von Variablen).

\begin{bAntwort}
\begin{bKontrollflussgraph}[xscale=1.5,yscale=-1.2]
\node[
  pin={\c{boolean swapped; int swapTmp; int[] = (int[]) array.clone()}}
] at (0,1) (1) {1};

\node[
  pin={180:\c{do; swapped = false; int index = 0;}}
] at (0,2) (2) {2};

\node[
  pin={[pin distance=1.5cm]\c{for}}
] at (1,3) (3) {3};

\node[
  pin={\c{if (newArray[index] > newArray[index + 1])}}
] at (2,4) (4) {4};

\node[
  pin={\c{swapTmp = newArray[]; ... swapped = true;)}}
] at (2,5) (5) {5};

\node[
  pin={\c{i++}}
] at (1,6) (6) {6};

\node[
  pin={\c{while (swapped))}}
] at (0,7) (7) {7};

\node[
  pin={\c{return newArray;)}}
] at (0,8) (8) {8};

\path (1) -- (2);
\path (2) -- (3);
\path[wahr] (3) -- node[draw=none,name=34]{} (4);
\path[falsch] (3) -- node[draw=none,name=37]{} (7);
\path[wahr] (4) -- node[draw=none,name=45]{} (5);
\path[falsch] (4) -- node[draw=none,name=46]{} (6);
\path (5) -- (6);
\path (6) -- (3);
\path (6) -- (7);
\path[wahr] (7) -- node[draw=none,name=72]{} (2);
\path[falsch] (7) -- node[draw=none,name=78]{} (8);

\node[usebox] at (0,0) {
  def(array, swapped, swapTmp)\\
  c-use(array)\\
  def(newArray)
} edge[dashed] (1);

\node[usebox] at (-3,1) {
  def(swapped, index)
} edge[dashed] (2);

\node[usebox] at (1,2) {p-use(index, newArray)} edge[dashed] (34) edge[dashed] (37);

\node[usebox] at (3,4.5) {p-use(newArray, index)} edge[dashed] (45) edge[dashed] (46);

\node[usebox] at (2,7) {
  c-use(index,newArray)\\
  def(swapTmp)\\
  c-use(index,newArray)\\
  def(newArray)\\
  c-use(swapTmp,index)\\
  def(newArray,swapped)\\
  c-use(index)\\
  def(index)
} edge[dashed] (5);

\node[usebox] at (-2,6) {p-use(swapped)} edge[dashed] (72) edge[dashed] (78);

\node[usebox] at (0,9) {c-use(newArray)} edge[dashed] (8);

\end{bKontrollflussgraph}
\end{bAntwort}

%%
% (b)
%%

\item Nennen Sie die maximale Anzahl linear unabhängiger Programmpfade,
also die zyklomatische Komplexität nach McCabe.\index{Zyklomatische
Komplexität nach Mc-Cabe}

\begin{bAntwort}
Der Graph hat 8 Knoten und 10 Kanten. Daher ist die zyklomatische
Komplexität nach McCabe gegeben durch 10 - 8 + 2 = 4.
\end{bAntwort}

%%
% (c)
%%

\item Geben Sie einen möglichst kleinen Testdatensatz an, der eine
100\%-ige Verzweigungsüberdeckung dieses Moduls erzielt.\index{C1-Test
Zweigüberdeckung (Branch Coverage)}

\begin{bAntwort}
Die Eingabe muss mindestens ein Feld der Länge 3 sein. Ansonsten wäre
das Feld schon sortiert bzw. bräuchte nur eine Vertauschung und die
innere if-Bedingung wäre nicht zu 100\% überdeckt. Daher wählt man
beispielsweise array = [1,3,2].
\end{bAntwort}

%%
% (d)
%%

\item Beschreiben Sie kurz, welche Eigenschaften eine Testfallmenge
allgemein haben muss, damit das datenflussorientierte
Überdeckungskriterium „all-uses“\index{all uses} erfüllt.

\begin{bAntwort}
Das Kriterium all-uses ist das Hauptkriterium des datenflussorientierten
Testens, denn es testet den kompletten prädikativen und berechnenden
Datenfluss. Konkret: von jedem Knoten mit einem globalen def(x) einer
Variable x existiert ein definitions-freier Pfad bzgl. x (def-clear(x))
zu jedem erreichbaren Knoten mit einem c-use(x) oder p-use(x)).
\end{bAntwort}

\end{enumerate}
\end{document}
