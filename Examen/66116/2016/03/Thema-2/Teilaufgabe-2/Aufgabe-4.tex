\documentclass{bschlangaul-aufgabe}
\bLadePakete{tabelle}
\begin{document}
\bAufgabenMetadaten{
  Titel = {Projektmanagement},
  Thematik = {Projektmanagement},
  Referenz = 66116-2016-F.T2-TA2-A4,
  RelativerPfad = Staatsexamen/66116/2016/03/Thema-2/Teilaufgabe-2/Aufgabe-4.tex,
  ZitatSchluessel = examen:66116:2016:03,
  BearbeitungsStand = nur Angabe,
  Korrektheit = unbekannt,
  Ueberprueft = {unbekannt},
  Stichwoerter = {Gantt-Diagramm},
  EinzelpruefungsNr = 66116,
  Jahr = 2016,
  Monat = 03,
  ThemaNr = 2,
  TeilaufgabeNr = 2,
  AufgabeNr = 4,
}

\begin{enumerate}

%%
% (a)
%%

\item Erklären Sie in maximal zwei Sätzen den Unterschied zwischen
Knoten- und Kantennetzwerken im Kontext des Projektmanagements.
\index{Gantt-Diagramm}
\footcite{examen:66116:2016:03}

%%
% (b)
%%

\item Gegeben ist die folgende Tabelle zur Grobplanung eines
hypothetischen Softwareprojekts:

\begin{tabularx}{\linewidth}{llX}
Aktivität & Minimale Dauer & Einschränkungen\\

Anforderungsanalyse &
2 Monate &
Endet frühestens einen Monat nach dem
Start der Entwurfsphase. \\

Entwurf & 4 Monate &
Startet frühestens zwei Monate nach dem Start der
Anforderungsanalyse. \\

Implementierung & 5 Monate&

Endet frühestens drei Monate nach dem Ende der
Entwurfsphase. Darf erst starten, nachdem die
Anforderungsanalyse abgeschlossen ist.\\
\end{tabularx}

Geben Sie ein CPM-Netzwerkan, das die Aktivitäten und Abhängigkeit des
obigen Projektplans beschreibt. Gehen Sie von der Zeiteinheit „Monate“
aus. Das Projekt hat einen Start- und einen Endknoten.

Jede Aktivität wird auf einen Start- und einen Endknoten abgebildet. Die
Dauer der Aktivitäten sowie Abhängigkeiten sollen durch Kanten
dargestellt werden. Der Start jeder Aktivität hängt vom Projektstart ab,
das Projektende hängt vom Ende aller Aktivitäten ab. Modellieren Sie
diese Abhängigkeiten durch Pseudoaktivitäten mit Dauer null.

%%
% (c)
%%

\item Berechnen Sie für jedes Ereignis (\dh für jeden Knoten) die
früheste Zeit sowie die späteste Zeit. Beachten Sie, dass die
Berechnungsreihenfolge einer topologischen Sortierung des Netzwerks
entsprechensollte.

%%
% (d)
%%

\item Geben Sie einen kritischen Pfad durch das CPM-Netzwerk an.
Möglicherweise sind hierfür weitere Vorberechnungen vonnöten. Welche
Aktivität sollte sich demnach auf keinen Fall verzögern?

%%
% (e)
%%

\item Geben Sie ein Gantt-Diagramm an, das den Projektplan visualisiert.
Gehen Sie davon aus, dass jede Aktivität zur frühesten Zeitihres
Startknotens beginnt und zur spätesten Zeit ihres Endknotens endet(s.
jeweils Teilaufgabe (c)). Geben Sie die minimale Dauer jeder Aktivität,
sowie die Pufferzeit (in Klammern) an. Beispiel: 4 (+2). Notieren Sie
alle Einschränkungen mit Hilfe geeigneter Abhängigkeitsbeziehungen.
Geben Sie eine absolute Zeitskala in Monaten an.

%%
% (f)
%%

\item Nennen Sie zwei weitere Aktivitäten, die in der obigen Tabelle
fehlen, jedoch typischerweise in Softwareentwicklungs-Prozessmodellen
wie etwa dem Wasserfallmodell vorkommen.
\end{enumerate}

\end{document}
