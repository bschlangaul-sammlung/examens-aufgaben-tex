\documentclass{bschlangaul-aufgabe}
\bLadePakete{entwurfsmuster}
\begin{document}
\bAufgabenMetadaten{
  Titel = {Aufgabe 1},
  Thematik = {Grafik: Kreis, Quadrat, Dreieck},
  Referenz = 66116-2019-F.T1-TA2-A1,
  RelativerPfad = Staatsexamen/66116/2019/03/Thema-1/Teilaufgabe-2/Aufgabe-1.tex,
  ZitatSchluessel = examen:66116:2019:03,
  BearbeitungsStand = mit Lösung,
  Korrektheit = unbekannt,
  Ueberprueft = {unbekannt},
  Stichwoerter = {Kompositum (Composite)},
  EinzelpruefungsNr = 66116,
  Jahr = 2019,
  Monat = 03,
  ThemaNr = 1,
  TeilaufgabeNr = 2,
  AufgabeNr = 1,
}

Gegeben sei folgender Sachverhalt: Eine Grafik ist entweder ein Kreis,
ein Quadrat oder ein Dreieck. Eine Grafik kann zudem auch eine
Kombination aus diesen Elementen sein. Des Weiteren können Sie aus
mehreren Grafiken auch neue Grafiken zusammenbauen. Sie denken sich: Ich
möchte eine Menge von Grafiken genauso wie eine einzelne Grafik
behandeln können.\index{Kompositum (Composite)}
\footcite{examen:66116:2019:03}

\begin{enumerate}

%%
% a)
%%

\item Welches Entwurfsmuster sollten Sie zur Modellierung verwenden?

\begin{bAntwort}
Kompositum
\end{bAntwort}

%%
% b)
%%

\item Zeichnen Sie das entsprechende Klassendiagramm. Es reicht, nur die
Klassennamen mit ihren Assoziationen und Vererbungsbeziehungen
anzugeben; \dh ohne Methoden und Attribute.

\begin{bExkurs}[Kompositum]
\bEntwurfsKompositumUml
\end{bExkurs}

\begin{bAntwort}

\begin{tikzpicture}
\umlsimpleclass[x=2.5,y=3,type=abstract]{Grafik}
\umlsimpleclass[x=-2]{Kreis}
\umlsimpleclass[x=0]{Quadrat}
\umlsimpleclass[x=2]{Dreieck}
\umlsimpleclass[x=5]{ZusammengesetzteGrafik}
\umlVHVinherit{Kreis}{Grafik}
\umlVHVinherit{Quadrat}{Grafik}
\umlVHVinherit{Dreieck}{Grafik}
\umlVHVinherit{ZusammengesetzteGrafik}{Grafik}

\umlHVHaggreg[arm1=3cm]{ZusammengesetzteGrafik}{Grafik}

\end{tikzpicture}
\end{bAntwort}

\end{enumerate}
\end{document}
