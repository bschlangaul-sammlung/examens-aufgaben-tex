\documentclass{bschlangaul-aufgabe}
\bLadePakete{java,kontrollflussgraph}
\begin{document}
\bAufgabenMetadaten{
  Titel = {Aufgabe 3},
  Thematik = {White-Box-Tests},
  Referenz = 66116-2019-H.T1-TA1-A3,
  RelativerPfad = Staatsexamen/66116/2019/09/Thema-1/Teilaufgabe-1/Aufgabe-3.tex,
  ZitatSchluessel = examen:66116:2019:09,
  BearbeitungsStand = mit Lösung,
  Korrektheit = unbekannt,
  Ueberprueft = {unbekannt},
  Stichwoerter = {White-Box-Testing},
  EinzelpruefungsNr = 66116,
  Jahr = 2019,
  Monat = 09,
  ThemaNr = 1,
  TeilaufgabeNr = 1,
  AufgabeNr = 3,
}

\let\b=\bBedingung
\let\c=\bKontrollCode
\let\f=\bBedingungFalsch
\let\k=\bKontrollTextzeileKnoten
\let\p=\bKontrollKnotenPfad
\let\w=\bBedingungWahr

Eine Dezimalzahl hat ein optionales Vorzeichen, dem eine nichtleere
Sequenz von Dezimalziffern folgt. Der anschließende gebrochene Anteil
ist optional und besteht aus einem Dezimalpunkt, gefolgt von einer
nichtleeren Sequenz von Dezimalziffern.
\index{White-Box-Testing}
\footcite{examen:66116:2019:09}

Die folgende Java-Methode erkennt, ob eine Zeichenfolge eine Dezimalzahl
ist:

% \bJavaExamen{66116}{2019}{09}{Dezimalzahl}

\begin{bJavaAngabe}
  public static boolean istDezimalzahl(char[] zeichen) { // 1
    boolean resultat; int laenge = zeichen.length; // 2
    if (laenge == 0) // 3
      resultat = false; // 4
    else { // 5
      int i = 0; // 6
      if (zeichen[i] == '+' || zeichen[i] == '-') // 7
        i++; // 8
      int j = i; // 9
      while (i < laenge && '0' <= zeichen[i] && zeichen[i] <= '9') // 10
        i++; // 11
      if (i == j) // 12
        resultat = false; // 13
      else { // 14
        if (i < laenge && zeichen[i] == '.') // 15
          do // 16
            i++; // 17
          while (i < laenge && '0' <= zeichen[i] && zeichen[i] <= '9'); // 18
        resultat = i == laenge && '0' <= zeichen[i - 1] && zeichen[i - 1] <= '9'; // 19
      } // 20
    } // 21
    return resultat; // 22
  }
\end{bJavaAngabe}

\begin{enumerate}

%%
% a)
%%

\item Konstruieren Sie zu dieser Methode einen Kontrollflußgraphen.
Markieren Sie dessen Knoten mit Zeilennummern des Quelltexts.

\begin{bAntwort}
\begin{bKontrollflussgraph}[xscale=1,yscale=-0.8]
% 1
\node[knoten] at
  (0,2) (1) {1};

% 2
\node[pin=\c{boolean resultat; int laenge = zeichen.length;}] at
  (0,3) (2) {2};

% 3
\node[pin=\c{if (laenge == 0)}] at
  (0,4) (3) {3};

% 4
\node[pin=180:\c{resultat = false;}] at
  (-1,5) (4) {4};

% 5

% 6
\node[pin=\c{int i = 0;}] at
  (0,6) (6) {6};

% 7
\node[pin=\c{if (zeichen[i] == '+' || zeichen[i] == '-')}] at
  (0,7) (7) {7};

% 8
\node[pin=\c{i++; }] at
  (1,8) (8) {8};

% 9
\node[pin=\c{int j = i;}] at
  (0,8.8) (9) {9};

% 10
\node[pin=\c{while (i < laenge \&\& '0' <= ...}] at
  (0,10) (10) {10};

% 11
\node[pin=\c{i++;}] at
  (1,11) (11) {11};

% 12
\node[pin=\c{if (i == j)}] at
  (0,12) (12) {12};

% 13
\node[pin=\c{resultat = false;}] at
  (2,13) (13) {13};

% 14

% 15
\node[pin={[pin distance=2cm]\c{if (i < laenge \&\& zeichen[i] == '.')}}] at
  (0,14) (15) {15};

% 16
\node[pin={[pin distance=1cm]\c{i++;}}] at
  (1,15) (17) {17};

% 17

% 18
\node[pin={[pin distance=1cm]\c{while (i < laenge \&\& '0' <= ...}}] at
  (1,17) (18) {18};

% 19
\node[pin={[pin distance=2cm]\c{resultat = i == laenge \&\& '0' <= ...}}] at
  (0,18) (19) {19};

% 20

% 21

% 22
\node[pin=180:\c{return resultat;}] at (0,20) (22) {22};

\path (1) -- (2);
\path (2) -- (3);
\path (3) -- (4);
\path (3) edge[falsch] (6);
\path (4) -- (-1,19) -- (22);
\path (7) -- (8);
\path (7) edge[falsch] (9);
\path (6) -- (7);
\path (9) -- (10);
\path (8) -- (9);
\path (10) edge[bend left] (11);
\path (11) edge[bend left] (10);
\path (10) edge[falsch] (12);
\path (12) -- (13);
\path (13) -- (2,19) -- (22);
\path (12) edge[falsch] (15);
\path (15) -- (17);
\path (15) edge[falsch] (19);
\path (17) -- (18);
\path (18) edge[bend left] (17);
\path (18) edge[falsch] (19);
\path (19) -- (22);
\end{bKontrollflussgraph}
\end{bAntwort}

%%
% b)
%%

\item Geben Sie eine minimale Testmenge an, die das Kriterium der
Knotenüberdeckung erfüllt. Geben Sie für jeden Testfall den
durchlaufenen Pfad in der Notation 1 $\rightarrow$ 2 $\rightarrow$
$\dots$ an.

\begin{bAntwort}
\begin{itemize}
%  1-2-3-4-6-7-8-9-10-11-12-13-15-17-18-19-22
\item „“: \\
%  1-2-3-4-6-7-8-9-10-11-12-13-15-17-18-19-22
\p{1-2-3-4-                                22}

\item „1“:\\
%  1-2-3-4-6-7-8-9-10-11-   12-13-15-17-18-19-22
\p{1-2-3-  6-7-  9-10-11-10-12-   15-      19-22}

\item „+1.0“:\\
%  1-2-3-4-6-7-8-9-10-11-   12-13-15-17-18-19-22
\p{1-2-3-  6-7-8-9-10-11-10-12-   15-17-18-19-22}

\item „x“:\\
%  1-2-3-4-6-7-8-9-10-11-12-13-15-17-18-19-22
\p{1-2-3-  6-7-  9-10-   12-13-            22}
\end{itemize}
\end{bAntwort}

%%
% c)
%%

\item Verfahren Sie wie in b) für das Kriterium der Kantenüberdeckung.

\begin{bAntwort}
\bMetaNochKeineLoesung
\end{bAntwort}

%%
% d)
%%

\item Wie stehen die Kriterien der Knoten- und Kantenüberdeckung
zueinander in Beziehung? Begründen Sie Ihre Anwort.

Hinweis: Eine Testmenge ist minimal, wenn es keine andere Testmenge mit
einer kleineren Zahl von Testfällen gibt. Die Minimalität braucht nicht
bewiesen zu werden.

\begin{bAntwort}
Die Kantenüberdeckung fordert, dass jede \emph{Kante} des
Kontrollflussgraphen von mindestens einem Testfall durchlaufen werden
muss. Um das Kriterium zu erfüllen, müssen die Testfälle so gewählt
werden, dass jede Verzweigungsbedingung mindestens \emph{einmal wahr}
und mindestens \emph{einmal falsch} wird. Da hierdurch alle Knoten
ebenfalls mindestens einmal besucht werden müssen, ist die
\emph{Anweisungsüberdeckung} in der Zweigüberdeckung \emph{vollständig
enthalten}.\footcite[Seite 209]{hoffmann:software}
\end{bAntwort}

\end{enumerate}
\end{document}
