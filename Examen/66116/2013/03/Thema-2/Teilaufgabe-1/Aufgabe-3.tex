\documentclass{bschlangaul-aufgabe}
\bLadePakete{baum}
\begin{document}
\bAufgabenMetadaten{
  Titel = {Aufgabe 3},
  Thematik = {B-Baum k=2},
  Referenz = 66116-2013-F.T2-TA1-A3,
  RelativerPfad = Staatsexamen/66116/2013/03/Thema-2/Teilaufgabe-1/Aufgabe-3.tex,
  ZitatSchluessel = aud:ab:5,
  ZitatBeschreibung = {Aufgabe 4},
  BearbeitungsStand = mit Lösung,
  Korrektheit = unbekannt,
  Ueberprueft = {unbekannt},
  Stichwoerter = {B-Baum},
  EinzelpruefungsNr = 66116,
  Jahr = 2013,
  Monat = 03,
  ThemaNr = 2,
  TeilaufgabeNr = 1,
  AufgabeNr = 3,
}

Gegeben sei der folgende B-Baum:\footcite[Aufgabe 4]{aud:ab:5}
\index{B-Baum}
\footcite{examen:66116:2013:03}

\begin{center}
\begin{tikzpicture}[
  b bbaum,
  level 1/.append style={level distance=15mm,sibling distance=65mm},
  level 2/.append style={level distance=15mm,sibling distance=22mm}
]
\node{12}[->]
  child{node{6 \nodepart{two} 9}
    child{node {1 \nodepart{two} 2 \nodepart{three} 4 \nodepart{four} 5}}
    child{node {7 \nodepart{two} 8}}
    child{node {10 \nodepart{two} 11}}
  }
  child{node{15 \nodepart{two} 19}
    child{node {13 \nodepart{two} 14}}
    child{node {16 \nodepart{two} 17 \nodepart{three} 18}}
    child{node {20 \nodepart{two} 21}}
  }
;
\end{tikzpicture}
\end{center}

\begin{enumerate}

%%
% 1.
%%

\item Was bedeutet $k$ bei einem B-Baum mit Grad $k$? Geben Sie $k$ für
den obigen B-Baum an.

\begin{bAntwort}
Jeder Knoten außer der Wurzel hat mindestens $k$ und höchstens $2k$
Einträge. Die Wurzel hat zwischen einem und $2k$ Einträgen. Die Einträge
werden in allen Knoten sortiert gehalten. Alle Knoten mit $n$ Einträgen,
außer den Blättern, haben $n + 1$ Kinder.
\footcite[Seite 225]{kemper}

Für den gegebenen Baum kann die Ordnung $k = 2$ angegeben werden.
\end{bAntwort}

%%
% 2.
%%

\item Was sind die Vorteile von B-Bäumen im Vergleich zu binären Baumen?

\begin{bAntwort}
B-Bäume sind immer höhenbalanciert. B-Bäume haben eine geringere Höhe,
wodurch eine schnellere Suche möglich wird, da weniger Aufrufe nötig
sind.\bFussnoteUrl{http://wwwbayer.in.tum.de/lehre/WS2001/HSEM-bayer/BTreesAusarbeitung.pdf}
\end{bAntwort}

%%
% 3.
%%

\item Wozu werden B-Bäume in der Regel verwendet und wieso?

\begin{bAntwort}
B-Bäume werden für Hintergrundspeicherung (z. B. von Datenbanksystemen,
Dateisystem) verwendet. Die Knotengrößen werden auf die
Seitenkapazitäten abgestimmt.\footcite[Seite 223]{kemper}

B-Bäume sind eine daten- und Indexstruktur, die häufig in Datenbanken und
Daeisystemen eingesetzt werden. Da ein B-Baum immer vollständig
balanciert ist und die Schlüssel sortiert gespeichert werden, ist ein
schnelles Auffinden von Inhalten gegeben.
\end{bAntwort}

%%
% 4.
%%

\item Fügen Sie den Wert $3$ in den B-Baum ein, und zeichnen Sie den
vollständigen B-Baum nach dem Einfügen und möglichen darauf folgenden
Operationen.

\begin{bAntwort}
\begin{center}
\begin{tikzpicture}[
  b bbaum,
  level 1/.append style={level distance=15mm,sibling distance=70mm},
  level 2/.append style={level distance=15mm,sibling distance=20mm}
]
\node{12}[->]
  child{node{\textbf{3} \nodepart{two} 6 \nodepart{three} 9}
    child{node {\textbf{1} \nodepart{two} \textbf{2}}}
    child{node {\textbf{4} \nodepart{two} \textbf{5}}}
    child{node {7 \nodepart{two} 8}}
    child{node {10 \nodepart{two} 11}}
  }
  child{node{15 \nodepart{two} 19}
    child{node {13 \nodepart{two} 14}}
    child{node {16 \nodepart{two} 17 \nodepart{three} 18}}
    child{node {20 \nodepart{two} 21}}
  }
;
\end{tikzpicture}
\end{center}
\end{bAntwort}

%%
% 5.
%%

\item Entfernen Sie aus dem ursprünglichen B-Baum den Wert $19$. Zeichnen
Sie das vollständige Ergebnis nach dem Löschen und möglichen darauf
folgenden Operationen. Sollte es mehrere richtige Lösungen geben, reicht
es eine Lösung zu zeichnen.

\begin{bAntwort}
\begin{center}
\begin{tikzpicture}[
  b bbaum,
  level 1/.append style={level distance=15mm,sibling distance=65mm},
  level 2/.append style={level distance=15mm,sibling distance=22mm}
]
\node{12}[->]
  child{node{6 \nodepart{two} 9}
    child{node {1 \nodepart{two} 2 \nodepart{three} 4 \nodepart{four} 5}}
    child{node {7 \nodepart{two} 8}}
    child{node {10 \nodepart{two} 11}}
  }
  child{node{15 \nodepart{two} \textbf{18}}
    child{node {13 \nodepart{two} 14}}
    child{node {16 \nodepart{two} 17}}
    child{node {20 \nodepart{two} 21}}
  }
;
\end{tikzpicture}
\end{center}
\end{bAntwort}
\end{enumerate}

\end{document}
