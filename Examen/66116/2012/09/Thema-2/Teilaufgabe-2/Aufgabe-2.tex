\documentclass{bschlangaul-aufgabe}
\bLadePakete{cpm,mathe,gantt}
\begin{document}
\bAufgabenMetadaten{
  Titel = {Aufgabe 2},
  Thematik = {Gantt und CPM},
  Referenz = 66116-2012-H.T2-TA2-A2,
  RelativerPfad = Staatsexamen/66116/2012/09/Thema-2/Teilaufgabe-2/Aufgabe-2.tex,
  ZitatSchluessel = examen:66116:2012:09,
  BearbeitungsStand = mit Lösung,
  Korrektheit = unbekannt,
  Ueberprueft = {unbekannt},
  Stichwoerter = {CPM-Netzplantechnik, Gantt-Diagramm},
  EinzelpruefungsNr = 66116,
  Jahr = 2012,
  Monat = 09,
  ThemaNr = 2,
  TeilaufgabeNr = 2,
  AufgabeNr = 2,
}

\let\f=\footnotesize
\let\FZ=\bCpmFruehI
\let\SZ=\bCpmSpaetI
\let\v=\bCpmVon
\let\vz=\bCpmVonZu
\let\z=\bCpmZu

Die unten stehende Abbildung stellt ein CPM-Netzwerk dar. Die Ereignisse
sind fortlaufend nummeriert (Nummer im Inneren der Kreise) und tragen
keine Namen. Gestrichelte Linien stellen Pseudo-Aktivitäten mit einer
Dauer von 0 dar.\index{CPM-Netzplantechnik}
\footcite{examen:66116:2012:09}

\begin{center}
\begin{tikzpicture}
\bCpmEreignis{1}{0}{2}
\bCpmEreignis{2}{1}{4}
\bCpmEreignis{3}{1}{0}
\bCpmEreignis{4}{3}{4}
\bCpmEreignis{5}{3}{2}
\bCpmEreignis{6}{3}{0}
\bCpmEreignis{7}{5}{4}
\bCpmEreignis{8}{5}{2}
\bCpmEreignis{9}{5}{0}
\bCpmEreignis{10}{7}{2}

\bCpmVorgang{1}{2}{10}
\bCpmVorgang{1}{3}{22}
\bCpmVorgang{1}{5}{6}
\bCpmVorgang{1}{6}{5}
\bCpmVorgang{2}{4}{8}
\bCpmVorgang{2}{5}{5}
\bCpmVorgang{3}{6}{8}
\bCpmVorgang{4}{5}{1}
\bCpmVorgang{4}{7}{12}
\bCpmVorgang{6}{9}{11}
\bCpmVorgang{7}{10}{6}
\bCpmVorgang{7}{8}{3}
\bCpmVorgang{8}{10}{7}
\bCpmVorgang{9}{10}{9}

\bCpmVorgang[schein]{5}{6}{}
\bCpmVorgang[schein]{5}{8}{}
\end{tikzpicture}
\end{center}

\begin{enumerate}

%%
% 1.
%%

\item Berechnen Sie die früheste Zeit für jedes Ereignis, wobei
angenommen wird, dass das Projekt zum Zeitpunkt 0 startet!

\begin{bAntwort}
\bCpmFruehErklaerung
\begin{tabular}{|l|l|r|}
\hline
$i$ & Nebenrechnung             & \FZ  \\\hline
1   &                           & $0$  \\
2   &                           & $10$ \\
3   &                           & $22$ \\
4   &                           & $18$ \\
5   & $\max(15_2,6_1,19_4)$     & $19$ \\
6   & $\max(5_1, 30_6, 19_5)$   & $30$ \\
7   &                           & $30$ \\
8   & $\max(33_7, 19_5)$        & $33$ \\
9   &                           & $41$ \\
10  & $\max(36_7, 40_8, 50_9)$  & $50$ \\\hline
\end{tabular}
\end{bAntwort}

%%
% 2.
%%

\item Setzen Sie anschließend beim letzten Ereignis die späteste Zeit
gleich der frühesten Zeit und berechnen Sie die spätesten Zeiten!

\begin{bAntwort}
\bCpmSpaetErklaerung
\begin{tabular}{|l|l|r|}
\hline
$i$ & Nebenrechnung                 & \SZ \\\hline
10  & siehe \FZ[10]                 & $50$ \\
9   &                               & $41$ \\
8   &                               & $43$ \\
7   & $\min(44_{10}, 40_8)$         & $40$ \\
6   &                               & $30$ \\
5   & $\min(30_6, 43_8)$            & $30$ \\
4   & $\min(29_5, 28_7)$            & $28$ \\
3   &                               & $22$ \\
2   & $\min(20_4, 25_5)$            & $20$ \\
1   & $\min(10_2, 24_5, 0_3, 25_6)$ & $0$ \\\hline

\end{tabular}
\end{bAntwort}

%%
% 3.
%%

\item Berechnen Sie nun für jedes Ereignis die Pufferzeiten!

\begin{bAntwort}
\begin{tabular}{|l||l|l|l|l|l|l|l|l|l|l|}
\hline
i             & 1 & 2  & 3   & 4  & 5  & 6  & 7  & 8  & 9  & 10 \\\hline\hline
\FZ & 0 & 10 & 22  & 18 & 19 & 30 & 30 & 33 & 41 & 50 \\\hline
\SZ & 0 & 20 & 22  & 28 & 30 & 30 & 40 & 43 & 41 & 50 \\\hline
GP            & 0 & 10 & 0   & 10 & 11 & 0  & 10 & 10 & 0  & 0 \\\hline
\end{tabular}
\end{bAntwort}

%%
% 4.
%%

\item Bestimmen Sie den kritischen Pfad!

\begin{bAntwort}
$1 \rightarrow 3 \rightarrow 6 \rightarrow 9 \rightarrow 10$

\begin{center}
\begin{tikzpicture}[scale=0.8,transform shape]
\bCpmEreignis{1}{0}{2}
\bCpmEreignis{2}{1}{4}
\bCpmEreignis{3}{1}{0}
\bCpmEreignis{4}{3}{4}
\bCpmEreignis{5}{3}{2}
\bCpmEreignis{6}{3}{0}
\bCpmEreignis{7}{5}{4}
\bCpmEreignis{8}{5}{2}
\bCpmEreignis{9}{5}{0}
\bCpmEreignis{10}{7}{2}

\bCpmVorgang{1}{2}{10}
\bCpmVorgang[kritisch]{1}{3}{22}
\bCpmVorgang{1}{5}{6}
\bCpmVorgang{1}{6}{5}
\bCpmVorgang[kritisch]{3}{6}{8}
\bCpmVorgang{2}{5}{5}
\bCpmVorgang{2}{4}{8}
\bCpmVorgang{4}{7}{12}
\bCpmVorgang{7}{8}{3}
\bCpmVorgang{7}{10}{6}
\bCpmVorgang[kritisch]{9}{10}{9}
\bCpmVorgang[kritisch]{6}{9}{11}
\bCpmVorgang{8}{10}{7}
\bCpmVorgang{4}{5}{1}

\bCpmVorgang[schein]{5}{6}{}
\bCpmVorgang[schein]{5}{8}{}
\end{tikzpicture}
\end{center}
\end{bAntwort}

%%
% 5.
%%

\item Konvertieren Sie das Gantt-Diagramm\index{Gantt-Diagramm} aus
Abbildung 3 in ein CPM-Netzwerk!

\begin{center}
\begin{ganttchart}[x unit=0.75cm, y unit chart=0.8cm]{0}{11}
\gantttitlelist{0,...,11}{1} \\
\ganttbar[name=1]{1}{0}{1} \\
\ganttbar[name=2]{2}{2}{4} \\
\ganttbar[name=3]{3}{3}{3} \\
\ganttbar[name=4]{4}{6}{7} \\
\ganttbar[name=5]{5}{7}{11}

\node at (1) {2};
\node at (2) {3};
\node at (3) {1};
\node at (4) {2};
\node at (5) {5};

\ganttlink[link type=f-f]{3}{2}
\ganttlink[link type=f-s]{1}{2}
\ganttlink[link type=f-s]{1}{3}
\ganttlink[link type=f-s]{2}{4}
\ganttlink[link type=s-s]{4}{5}
\end{ganttchart}
\end{center}

\begin{bAntwort}
\begin{center}
\begin{tikzpicture}
\bCpmEreignis{SP}{-6}{0}
\bCpmEreignis{S1}{-5}{1.5}
\bCpmEreignis{E1}{-4.5}{0}
\bCpmEreignis{S2}{-3}{1.5}
\bCpmEreignis{E2}{-1.5}{1.5}
\bCpmEreignis{S3}{-3}{0}
\bCpmEreignis{E3}{-1.75}{0}
\bCpmEreignis{S4}{0}{1.5}
\bCpmEreignis{E4}{1.5}{1.5}
\bCpmEreignis{S5}{0.5}{0}
\bCpmEreignis{E5}{2}{0}
\bCpmEreignis{EP}{3.5}{0}

\bCpmVorgang{S1}{E1}{2}
\bCpmVorgang{S2}{E2}{3}
\bCpmVorgang{S3}{E3}{1}
\bCpmVorgang{S4}{E4}{2}
\bCpmVorgang{S5}{E5}{5}

\bCpmVorgang[schein]{SP}{S1}{}
\bCpmVorgang[schein]{E5}{EP}{}
\bCpmVorgang[schein]{E1}{S2}{}
\bCpmVorgang{E1}{S3}{1}
\bCpmVorgang{E3}{E2}{1}
\bCpmVorgang{E2}{S4}{1}
\bCpmVorgang{S4}{S5}{1}
\bCpmVorgang[schein]{E4}{EP}{}
\end{tikzpicture}
\end{center}
\end{bAntwort}
\end{enumerate}

\end{document}
