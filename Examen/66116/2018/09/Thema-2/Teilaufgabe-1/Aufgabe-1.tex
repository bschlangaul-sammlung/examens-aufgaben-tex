\documentclass{bschlangaul-aufgabe}
\bLadePakete{gantt,cpm}
\begin{document}
\bAufgabenMetadaten{
  Titel = {Aufgabe 3},
  Thematik = {Gantt und zwei Softwareentwickler},
  Referenz = 66116-2018-H.T2-TA1-A1,
  RelativerPfad = Staatsexamen/66116/2018/09/Thema-2/Teilaufgabe-1/Aufgabe-1.tex,
  ZitatSchluessel = sosy:ab:5,
  ZitatBeschreibung = {erweiterte Aufgabenstellung, Seite
2},
  BearbeitungsStand = mit Lösung,
  Korrektheit = unbekannt,
  Ueberprueft = {unbekannt},
  Stichwoerter = {Gantt-Diagramm, CPM-Netzplantechnik},
  EinzelpruefungsNr = 66116,
  Jahr = 2018,
  Monat = 09,
  ThemaNr = 2,
  TeilaufgabeNr = 1,
  AufgabeNr = 1,
}

\let\f=\footnotesize
\let\FZ=\bCpmFruehI
\let\SZ=\bCpmSpaetI
\let\v=\bCpmVon
\let\vz=\bCpmVonZu
\let\z=\bCpmZu

Ein\index{Gantt-Diagramm} \footcite[erweiterte Aufgabenstellung, Seite
2]{sosy:ab:5} Team von zwei Softwareentwicklern soll ein Projekt
umsetzen, das in sechs Arbeitspakete unterteilt ist. Die Dauer der
Arbeitspakete und ihre Abhängigkeiten können Sie aus folgender Tabelle
entnehmen:
\footcite[Thema 2 Teilaufgabe 1 Aufgabe 1]{examen:66116:2018:09}

\begin{center}
\begin{tabular}{|l|l|l|}
\hline
Name & Dauer in Wochen & Abhängig von\\\hline\hline
A1 & 2 & - \\\hline
A2 & 5 & - \\\hline
A3 & 2 & A1 \\\hline
A4 & 5 & A1, A2 \\\hline
A5 & 7 & A3 \\\hline
A6 & 4 & A4 \\\hline
\end{tabular}
\end{center}

\begin{enumerate}

%%
% (a)
%%

\item Zeichnen Sie ein Gantt-Diagramm, das eine kürzestmögliche
Projektabwicklung beinhaltet.

\begin{bAntwort}
\begin{center}
\begin{ganttchart}[x unit=0.7cm, y unit chart=0.6cm]{1}{14}
\gantttitlelist{1,...,14}{1} \\
\ganttbar[name=A1]{A1}{3}{5} \\
\ganttbar[name=A2]{A2}{1}{5} \\
\ganttbar[name=A3]{A3}{6}{7} \\
\ganttbar[name=A4]{A4}{6}{10} \\
\ganttbar[name=A5]{A5}{8}{14} \\
\ganttbar[name=A6]{A6}{11}{14}
\ganttlink{A1}{A3}
\ganttlink{A1}{A4}
\ganttlink[link type=f-s]{A2}{A4}
\ganttlink[link type=f-s]{A3}{A5}
\ganttlink[link type=f-s]{A4}{A6}
\end{ganttchart}
\end{center}
\end{bAntwort}

%%
% (b)
%%

\item Bestimmen Sie die Länge des kritischen Pfades und geben Sie an,
welche Arbeitspakete an ihm beteiligt sind.

\begin{bAntwort}
Auf dem kritischen Pfad befinden die Arbeitspakete A2, A4 und A6. Die
Länge des kritischen Pfades ist 14.
\end{bAntwort}

%%
% (c*)
%%

\item Wandeln Sie das Gantt-Diagramm in ein
CPM-Netzplan\index{CPM-Netzplantechnik} um.

\begin{bAntwort}
\begin{tikzpicture}[font=\scriptsize,scale=0.95]
\bCpmEreignis{SP}{0}{1.5}

\bCpmEreignis{1A}{1.5}{3}
\bCpmEreignis{1E}{3}{3}
\bCpmEreignis{2A}{1.5}{0}
\bCpmEreignis{2E}{3}{0}
\bCpmEreignis{3A}{4.5}{3}
\bCpmEreignis{3E}{6}{3}
\bCpmEreignis{4A}{4.5}{0}
\bCpmEreignis{4E}{6}{0}
\bCpmEreignis{5A}{7.5}{3}
\bCpmEreignis{5E}{9}{3}
\bCpmEreignis{6A}{7.5}{0}
\bCpmEreignis{6E}{9}{0}

\bCpmEreignis{EP}{10.5}{1.5}

\bCpmVorgang{1A}{1E}{2}
\bCpmVorgang{2A}{2E}{5}
\bCpmVorgang{3A}{3E}{2}
\bCpmVorgang{4A}{4E}{5}
\bCpmVorgang{5A}{5E}{7}
\bCpmVorgang{6A}{6E}{4}

\bCpmVorgang[schein]{1E}{3A}{}
\bCpmVorgang[schein]{2E}{4A}{}
\bCpmVorgang[schein]{1E}{4A}{}
\bCpmVorgang[schein]{3E}{5A}{}
\bCpmVorgang[schein]{4E}{6A}{}

\bCpmVorgang[schein]{SP}{1A}{}
\bCpmVorgang[schein]{SP}{2A}{}

\bCpmVorgang[schein]{5E}{EP}{}
\bCpmVorgang[schein]{6E}{EP}{}
\end{tikzpicture}
\end{bAntwort}

%%
% (d*)
%%

\item Berechnen Sie für jedes Ereignis den \emph{frühesten Termin} und
den \emph{spätesten Termin} sowie die \emph{Gesamtpufferzeiten}.

\begin{bAntwort}
{\scriptsize
\setlength{\tabcolsep}{5pt}
\begin{tabular}{|l||c|c|c|c|c|c|c|c|c|c|c|c|c|c|}
\hline
$i$   & SP & 1A & 1E & 2A & 2E & 3A & 3E & 4A & 4E & 5A & 5E & 6A & 6E & EP\\\hline\hline
\FZ   & 0  & 0  & 2  & 0  & 5  & 2  & 4  & 5  & 10 & 4  & 11 & 10 & 14 & 14\\\hline
\SZ   & 0  & 3  & 5  & 0  & 5  & 5  & 7  & 5  & 10 & 7  & 14 & 10 & 14 & 14\\\hline
$GP$  & 0  & 3  & 3  & 0  & 0  & 3  & 3  & 0  & 0  & 3  & 3  & 0  & 0  & 0 \\\hline
\end{tabular}
}
\end{bAntwort}
\end{enumerate}

\end{document}
