\documentclass{bschlangaul-aufgabe}
\bLadePakete{normalformen,synthese-algorithmus}
\begin{document}
\bAufgabenMetadaten{
  Titel = {Aufgabe 4},
  Thematik = {Relation A-F},
  Referenz = 66116-2020-F.T1-TA2-A4,
  RelativerPfad = Staatsexamen/66116/2020/03/Thema-1/Teilaufgabe-2/Aufgabe-4.tex,
  ZitatSchluessel = examen:66116:2020:03,
  BearbeitungsStand = mit Lösung,
  Korrektheit = unbekannt,
  Ueberprueft = {unbekannt},
  Stichwoerter = {Synthese-Algorithmus},
  EinzelpruefungsNr = 66116,
  Jahr = 2020,
  Monat = 03,
  ThemaNr = 1,
  TeilaufgabeNr = 2,
  AufgabeNr = 4,
}

\let\ah=\bAttributHuelle
\let\ahl=\bLinksReduktionInline
\let\ahr=\bRechtsReduktionInline
\let\FA=\bFunktionaleAbhaengigkeiten
\let\fa=\bFunktionaleAbhaengigkeit
\let\m=\bAttributMenge
\let\r=\bRelation
\let\schrittE=\bSyntheseUeberErklaerung
\let\u=\underline

Gegeben sei die Relation
\index{Synthese-Algorithmus}
\footcite{examen:66116:2020:03}

\begin{center}
\bRelation{A, B, C, D, E, F}
\end{center}

\noindent
mit den FDs

\bigskip

% https://normalizer.db.in.tum.de/index.py
% A->BCF
% B->ABF
% CD->EF

\FA{
  A -> B, C, F;
  B -> A, B, F;
  C, D -> E, F;
}

\begin{enumerate}

%%
% a)
%%

\item Geben Sie alle Kandidatenschlüssel an.

\begin{bAntwort}
\begin{itemize}
\item \m{A, D}
\item \m{B, D}
\end{itemize}
\end{bAntwort}

%%
% b)
%%

\item Überführen Sie die Relation mittels Synthesealgorithmus in die 3.
NF. Geben Sie alle Relationen in der 3. NF an und \textbf{unterstreichen
Sie in jeder einen Kandidatenschlüssel.} — Falls Sie Zwischenschritte
notieren, machen Sie das Endergebnis \textbf{klar kenntlich.}

\begin{bAntwort}
\begin{enumerate}
\item \schrittE{1}
\begin{enumerate}
\item \schrittE{1-1}

\bPseudoUeberschrift{\fa{C, D -> E, F}}

$\m{E, F} \notin$ \ahl{C, D}{D}{C}\\
$\m{E, F} \notin$ \ahl{C, D}{C}{D}

\FA{
  A -> B, C, F;
  B -> A, B, F;
  C, D -> E, F;
}

\item \schrittE{1-2}

\bPseudoUeberschrift{F}

$F \in$ \ahr{A -> B, C, F}{A -> B, C}{A}{A, B, C, \textbf{F}}

\FA{
  A -> B, C;
  B -> A, B, F;
  C, D -> E, F;
}

$F \notin$ \ahr{B -> A, B, F}{B -> A, B}{B}{A, B, C}\\
$F \notin$ \ahr{C, D -> E, F}{C, D -> E}{C, D}{C, D, E}

\bPseudoUeberschrift{B}

$B \notin$ \ahr{A -> B, C}{A -> C}{A}{A, C}\\
$B \in$ \ahr{B -> A, B, F}{B -> A, F}{B}{A, \textbf{B}, F}

\FA{
  A -> B, C;
  B -> A, F;
  C, D -> E, F;
}

\item \schrittE{1-3}

\bNichtsZuTun

\item \schrittE{1-4}

\bNichtsZuTun

\end{enumerate}
\item \schrittE{2}

\r[R1]{\u{A, B}, C}\\
\r[R2]{\u{A, B}, F}\\
\r[R3]{\u{C, D}, E, F}\\

\item \schrittE{3}

\r[R1]{\u{A, B}, C}\\
\r[R2]{\u{A, B}, F}\\
\r[R3]{\u{C, D}, E, F}\\
\r[R4]{\u{A, D}}\\

\item \schrittE{4}

\bNichtsZuTun
\end{enumerate}
\end{bAntwort}

\end{enumerate}
\end{document}
