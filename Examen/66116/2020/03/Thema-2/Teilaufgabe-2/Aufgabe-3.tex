\documentclass{bschlangaul-aufgabe}
\bLadePakete{normalformen}
\begin{document}
\bAufgabenMetadaten{
  Titel = {Aufgabe 3},
  Thematik = {Relation A-F},
  Referenz = 66116-2020-F.T2-TA2-A3,
  RelativerPfad = Staatsexamen/66116/2020/03/Thema-2/Teilaufgabe-2/Aufgabe-3.tex,
  ZitatSchluessel = examen:66116:2020:03,
  BearbeitungsStand = mit Lösung,
  Korrektheit = unbekannt,
  Ueberprueft = {unbekannt},
  Stichwoerter = {Normalformen},
  EinzelpruefungsNr = 66116,
  Jahr = 2020,
  Monat = 03,
  ThemaNr = 2,
  TeilaufgabeNr = 2,
  AufgabeNr = 3,
}

\let\FA=\bFunktionaleAbhaengigkeiten

Gegeben sei folgendes relationales Schema R in erster Normalform:
\index{Normalformen}
\footcite{examen:66116:2020:03}

\begin{center}
R:{[A,B,C,D,E,F]}
\end{center}

\noindent
Für $R$ gelte folgende Menge FD funktionaler Abhängigkeiten:

\FA{
  A -> F;
  C, E, F -> A, B;
  A, E -> B;
  B, C -> D;
  A, F -> C;
}

\begin{enumerate}

%%
% 1.
%%

\item Bestimmen Sie alle Kandidatenschlüssel/Schlüsselkandidaten von $R$
mit FD. Begründen Sie Ihre Antwort. Begründen Sie zudem, warum es keine
weiteren Kandidatenschlüssel/Schlüsselkandidaten gibt.

\emph{Hinweis: Die Angabe von Attributmengen, die keine
Kandidatenschlüssel sind, führt zu Abzügen.}

\begin{bAntwort}
E muss in allen Superschlüsseln enthalten sein, denn es steht nicht auf
der rechten Seite von FD (*).

D kann in keinem Schlüsselkandidaten vorkommen, denn es steht nur auf
der rechten Seite von FD (**).

E allein ist kein Schlüsselkandidat (***).

AE führt über FD zu B, A zu F, AF zu C und BC zu D, also ist AE ein
Superschlüssel und damit wegen (*) und (***) ein Schlüsselkandidat.
Wegen (*) enthält jeder Superschlüssel, der A enthält, AE. Also ist kein
weiterer Superschlüssel, der A enthält, ein Schlüsselkandidat (****).

BE, CE und EF sind keine Superschlüssel, also auch keine
Schlüsselkandidaten.

BCE ist kein Superschlüssel, da A und F nicht erreicht werden können.

BEF ist kein Superschlüssel, da A, D und F nicht erreicht werden können.

CEF führt über FD zu AB, BC führt dann zu D, also ist CEF ein
Superschlüssel. Wegen (*), (**) und weil CE und EF keine Superschlüssel
sind, ist CEF ein Schlüsselkandidat.

Das waren alle dreielementigen Buchstabenkombinationen, die (*), (**)
und (****) genügen. Vierelementig ist nur BCEF und das enthält CEF, ist
also kein Schlüsselkandidat.

Die einzigen Schlüsselkandidaten sind folglich AE und CEF.
\end{bAntwort}

%%
% 2.
%%

\item Prüfen Sie, ob R mit FD in 2NF bzw. 3NF ist.

\begin{bAntwort}
R mit FD ist nicht in 2NF, denn bei Wahl des Schlüsselkandidaten AE
hängt F von A, also nur einem Teil des Schlüssels, ab. Also ist AE → F
nicht voll funktional. Damit ist R mit FD auch nicht in 3NF, denn 3NF
$\subseteq$ 2NF.
\end{bAntwort}

%%
% 3.
%%

\item Bestimmen Sie mit folgenden Schritten eine kanonische Überdeckung
FDc von FD. Begründen Sie jede Ihrer Entscheidungen:

\begin{enumerate}

%%
% a)
%%

\item Führen Sie eine Linksreduktion von FD durch. Geben Sie die Menge
funktionaler Abhängigkeiten nach der Linksreduktion an (FD;).

\begin{bAntwort}
\FA{
  A -> F;
  C, E, F -> A, B;
  A, E -> B;
  B, C -> D;
  A -> C;
}
\end{bAntwort}

%%
% b)
%%

\item Führen Sie eine Rechtsreduktion des Ergebnisses der Linksreduktion
(FD;) durch. Geben Sie die Menge funktionaler Abhängigkeiten nach der
Rechtsreduktion an (FD).

\begin{bAntwort}
\FA{
  A -> F;
  C, E, F -> A;
  A, E -> B;
  B, C -> D;
  A -> C;
}
\end{bAntwort}

%%
% c)
%%

\item Bestimmen Sie eine kanonische Überdeckung FD. von FD auf Basis des
Ergebnisses der Rechtsreduktion (FD).

\begin{bAntwort}
\FA{
  A -> F, C;
  C, E, F -> A;
  A, E -> B;
  B, C -> D;
}
\end{bAntwort}

\end{enumerate}
%%
% 4.
%%

\item Zerlegen Sie R mit FDc mithilfe des Synthesealgorithmus in 3NF.
Geben Sie zudem alle funktionalen Abhängigkeiten der erzeugten
Relationenschemata an.

%%
% 5.
%%

\item Prüfen Sie für alle Relationen der Zerlegung aus 4., ob sie
jeweils in BCNF sind.

\end{enumerate}
\end{document}
