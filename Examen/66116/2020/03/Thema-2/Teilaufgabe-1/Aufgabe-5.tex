\documentclass{bschlangaul-aufgabe}
\bLadePakete{java}
\begin{document}
\bAufgabenMetadaten{
  Titel = {Aufgabe 5},
  Thematik = {Object2D},
  Referenz = 66116-2020-F.T2-TA1-A5,
  RelativerPfad = Staatsexamen/66116/2020/03/Thema-2/Teilaufgabe-1/Aufgabe-5.tex,
  ZitatSchluessel = examen:66116:2020:03,
  BearbeitungsStand = mit Lösung,
  Korrektheit = unbekannt,
  Ueberprueft = {unbekannt},
  Stichwoerter = {Implementierung in Java},
  EinzelpruefungsNr = 66116,
  Jahr = 2020,
  Monat = 03,
  ThemaNr = 2,
  TeilaufgabeNr = 1,
  AufgabeNr = 5,
}

\let\j=\bJavaCode

\begin{enumerate}

%%
% a)
%%

\item Implementieren Sie ein Programm in einer objektorientierten
Programmiersprache, z. B. Java, für das folgende UML-Klassendiagramm.
\index{Implementierung in Java}
\footcite{examen:66116:2020:03}

Die \j{shift}-Methode soll die \j{x}-Postion eines Objektes um
\j{xShift} verändern und die \j{y}-Position um \j{yShift}. Die
\j{draw}-Methode soll die Werte der Attribute der Klasse auf der Konsole
ausgeben (- dies kann in Java mit \j{System.out.printin ("...")}
erfolgen).

% abstract .
% Object2D interface
% Drawable
% + shift(int xShift, int yShift):
% void + draw(): void
% Point Square
% xPos : int topLeft : Point
% yPos : int bottomRight : Point |
% + Point (int x, int y) | + Square(int top, int left,
% + shift(int xShift, int yShift): int bottom, int right)
% void + shift(int xShift, int yShift):
% + draw(): void void
% er + draw (): void

\begin{bAntwort}
\bJavaExamen{66116}{2020}{03}{object2d/Drawable}
\bJavaExamen{66116}{2020}{03}{object2d/Object2D}
\bJavaExamen{66116}{2020}{03}{object2d/Point}
\bJavaExamen{66116}{2020}{03}{object2d/Square}
\end{bAntwort}

%%
% b)
%%

\item Schreiben Sie eine Methode, die ein zweidimensionales Array aus
ganzen Zahlen (Datentyp \j{int}) als Parameter bekommt und ein
eindimensionales Array (bestehend aus ganzen Zahlen (Datentyp \j{int}))
zurückgibt, dessen Elemente jeweils der Summe der Einträge in der
entsprechenden Zeile des zweidimensionalen Arrays entsprechen.

Achtung: Die Zeilen des zweidimensionalen Arrays können unterschiedlich
lang sein.

Zur Vereinfachung sei die Signatur der Methode gegeben: \j{public int[]
computeSum(int[][] input)}

\begin{bAntwort}
\bJavaExamen{66116}{2020}{03}{ComputeSum}
\end{bAntwort}

%%
% c)
%%

\item Implementieren Sie eine einfach verkettete Liste in einer Klasse
List (z. B. in Java), in der in jedem Listenelement ein String
gespeichert wird. Die Klasse soll folgende Methoden bereitstellen:

\begin{itemize}
\item \j{void addFirst (String element)}: Diese Methode fügt ein Element
am Anfang einer Liste ein.

\item \j{void addLast (String element)}: Diese Methode hängt ein Element
an das Ende der Liste an.

\item \j{boolean exists(String element)}: Diese Methode gibt true
zurück, wenn die Liste ein Element mit dem Inhalt element beinhaltet,
andernfalls gibt sie false zurück.
\end{itemize}

Hinweis: Zwei \j{String}-Objekte können mittels der Funktion
\j{equals(..)} verglichen werden.

\begin{bAntwort}
\bJavaExamen{66116}{2020}{03}{List}
\end{bAntwort}

\end{enumerate}

\end{document}
