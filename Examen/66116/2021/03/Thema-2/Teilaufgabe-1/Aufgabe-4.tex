\documentclass{bschlangaul-aufgabe}

\begin{document}
\bAufgabenMetadaten{
  Titel = {Aufgabe 4},
  Thematik = {Wasserstandsmesser},
  Referenz = 66116-2021-F.T2-TA1-A4,
  RelativerPfad = Staatsexamen/66116/2021/03/Thema-2/Teilaufgabe-1/Aufgabe-4.tex,
  ZitatSchluessel = examen:66116:2021:03,
  BearbeitungsStand = nur Angabe,
  Korrektheit = unbekannt,
  Ueberprueft = {unbekannt},
  Stichwoerter = {Entwurfsmuster},
  EinzelpruefungsNr = 66116,
  Jahr = 2021,
  Monat = 03,
  ThemaNr = 2,
  TeilaufgabeNr = 1,
  AufgabeNr = 4,
}

\begin{enumerate}

%%
% a)
%%

\item Nennen und erklären Sie kurz die drei Kategorien der Organisation
von klassischen Entwurfsmustern. Geben Sie zu jeder Kategorie ein
Beispiel-Pattern an.
\index{Entwurfsmuster}
\footcite{examen:66116:2021:03}

%%
% b)
%%

\item Erstellen Sie ein Klassendiagramm zu den Komponenten einer
Beobachtungsstation, welche aus einem einzigen Wasserstandsmesser
besteht. Dabei sollen bei Änderungen des Wasserstandes der aktuelle
Wasserstand sowohl auf der Konsole als auch in eine Log-Datei
geschrieben werden. Der momentane Wasserstand soll dabei mittels einer
Variablen des Datentyps double dargestellt werden. Die verschiedenen
Anzeigearten (Konsolenanzeige, Logger) sollen als verschiedene Klassen
modelliert werden und enthalten jeweils nur eine Methode zum Anzeigen
bzw. Schreiben des aktuellen Wasserstandes. Verwenden Sie für die
Realisierung dieses Klassendiagramms die passenden Entwurfsmuster.

Hinweise:

\begin{itemize}
\item Getter und Setter müssen nicht eingezeichnet werden.

\item Fügen Sie auch Methoden ein, welche durch die einzelnen Klassen
implementiert werden.

\end{itemize}
\end{enumerate}
\end{document}
