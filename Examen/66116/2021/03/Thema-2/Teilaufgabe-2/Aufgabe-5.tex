\documentclass{bschlangaul-aufgabe}
\bLadePakete{rmodell,relationale-algebra}
\begin{document}
\bAufgabenMetadaten{
  Titel = {Aufgabe 5},
  Thematik = {Mitarbeiter einer Abteilung},
  Referenz = 66116-2021-F.T2-TA2-A5,
  RelativerPfad = Staatsexamen/66116/2021/03/Thema-2/Teilaufgabe-2/Aufgabe-5.tex,
  ZitatSchluessel = examen:66116:2021:03,
  BearbeitungsStand = mit Lösung,
  Korrektheit = unbekannt,
  Ueberprueft = {unbekannt},
  Stichwoerter = {RelaX - relational algebra calculator, Relationale Algebra},
  EinzelpruefungsNr = 66116,
  Jahr = 2021,
  Monat = 03,
  ThemaNr = 2,
  TeilaufgabeNr = 2,
  AufgabeNr = 5,
}

Formulieren Sie basierend auf den in der letzten Aufgabe gegebenen
Relationen die geforderten Anfragen in der Relationenalgebra.
\index{RelaX - relational algebra calculator}
\index{Relationale Algebra}
\footcite{examen:66116:2021:03}

\begin{bRelationenModell}
Mitarbeiter (\bPrimaer{MitarbeiterID}, Vorname, Nachname, Adresse,
Gehalt, \bFremd{Vorgesetzter [Mitarbeiter]} NOT NULL, \bFremd{AbteilungsID[Abteilung]})

\bigskip

Abteilung (\bPrimaer{AbteilungsID}, Bezeichnung UNIQUE NOT NULL)
\end{bRelationenModell}

\begin{enumerate}

%%
% a)
%%

% https://dbis-uibk.github.io/relax/calc/local/uibk/local/0

% Mitarbeiter = {
% 	Vorname Nachname AbteilungsID
% 	Sergej Puschkin 1
% 	Eduard Hans NULL
% }

% Abteilung = {
% 	AbteilungsID Bezeichnung
% 	1 Buchhaltung
% }

% π Vorname, Nachname (σ Bezeichnung = 'Buchhaltung' (Mitarbeiter ⨝ Abteilung))

\item Formulieren Sie eine Anfrage, welche die Vornamen und Nachnamen
der Mitarbeiter ausgibt, die in der Buchhaltung arbeiten.

\begin{bAntwort}
\begin{multline*}
\pi_{\text{Vorname, Nachname}}(\\
  \sigma_{\text{Bezeichnung = 'Buchhaltung'}}(\\
    \text{Mitarbeiter}
    \bowtie_{\text{Mitarbeiter.AbteilungsID = Abteilung.AbteilungsID}}
    \text{Abteilung}\\
  )\\
)
\end{multline*}
\end{bAntwort}

%%
% b)
%%

\item Formulieren Sie eine Anfrage, welche die Vornamen und Nachnamen
der Mitarbeiter ausgibt, die in keiner Abteilung arbeiten.

% σ AbteilungsID = NULL (Mitarbeiter)

\begin{bAntwort}
\begin{multline*}
\pi_{\text{Vorname, Nachname}}(\text{Mitarbeiter})
-\\
\pi_{\text{Vorname, Nachname}}(\text{Mitarbeiter} \bowtie \text{Abteilung})
\end{multline*}
\end{bAntwort}
\end{enumerate}
\end{document}
