\documentclass{bschlangaul-aufgabe}
\bLadePakete{relationale-algebra}
\begin{document}
\bAufgabenMetadaten{
  Titel = {Aufgabe 3},
  Thematik = {Relationen R1 und R2},
  Referenz = 66116-2021-F.T1-TA2-A3,
  RelativerPfad = Staatsexamen/66116/2021/03/Thema-1/Teilaufgabe-2/Aufgabe-3.tex,
  ZitatSchluessel = examen:66116:2021:03,
  BearbeitungsStand = mit Lösung,
  Korrektheit = unbekannt,
  Ueberprueft = {unbekannt},
  Stichwoerter = {Relationale Algebra},
  EinzelpruefungsNr = 66116,
  Jahr = 2021,
  Monat = 03,
  ThemaNr = 1,
  TeilaufgabeNr = 2,
  AufgabeNr = 3,
}

\begin{enumerate}

%%
% a)
%%

\item Gegeben seien die folgenden beiden Relationen R1 und R2:
\index{Relationale Algebra}
\footcite{examen:66116:2021:03}

\bPseudoUeberschrift{R1}

\begin{tabular}{|l|l|l|}
\hline
P  & Q       & S \\ \hline
10 & einfach & 5 \\ \hline
15 & b       & 8 \\ \hline
13 & einfach & 6 \\ \hline
\end{tabular}

\bPseudoUeberschrift{R2}

\begin{tabular}{|l|l|l|}
\hline
A  & B & C \\ \hline
10 & b & 6 \\ \hline
13 & c & 3 \\ \hline
10 & b & 5 \\ \hline
\end{tabular}

Geben Sie das Ergebnis des folgenden relationalen Ausdrucks an:

\begin{center}
$R_1 \bowtie_ {R_1.P = R_2.A} R_2$ (Equi-Join)
\end{center}

\begin{bAntwort}
\begin{tabular}{|l|l|l|l|l|l|}
\hline
P  & Q       & S & A  & B & C \\ \hline
10 & einfach & 5 & 10 & b & 6 \\ \hline
10 & einfach & 5 & 10 & b & 5 \\ \hline
13 & einfach & 6 & 13 & c & 3 \\ \hline
\end{tabular}
\end{bAntwort}

%%
% a)?
%%

\item Zeichnen Sie den Operatorbaum zu folgender Abfrage in relationaler Algebra:

\begin{bAntwort}
\begin{tikzpicture}
\node
  (pi) {$\pi_{\text{NAME, BERUF, PNAME}}$};

% joins
\node[below=of pi]
  (join person abt proj) {$\bowtie$}
  edge (pi);

\node[below right=of join person abt proj]
  (join abt proj) {$\bowtie$}
  edge (join person abt proj);

% person
\node[below left=of join person abt proj]
  (pi person) {$\pi_{\,\text{ANR, NAME, BERUF}}$}
  edge (join person abt proj);

\node[below=of pi person]
  (person) {PERSON}
  edge (pi person);

% abt
\node[below left=of join abt proj]
  (pi abt) {$\pi_{\,\text{ANR}}$}
  edge (join abt proj);

\node[below=of pi abt]
  (sigma abt) {$\sigma_{\,\text{EIN} > 1 000 000}$}
  edge (pi abt);

\node[below=of sigma abt]
  (abt) {ABT}
  edge (sigma abt);

% proj
\node[below right=of join abt proj]
  (pi proj) {$\pi_{\,\text{ANR, PNAME}}$}
  edge (join abt proj);

\node[below=of pi proj]
  (sigma proj) {$\sigma_{\,\text{ORT = DD}}$}
  edge (pi proj);

\node[below=of sigma proj]
  (proj) {PROJ}
  edge (sigma proj);
\end{tikzpicture}
\end{bAntwort}

%%
% b)
%%

\item Ist der linke (bzw. rechte) Verbundoperator (Left- bzw. Right-Outer Join) assoziativ? Falls ja,
beweisen Sie die Aussage, falls nein, geben Sie ein Gegenbeispiel an.

\begin{bAntwort}
Nein. Beleg durch Gegenbeispiel:

\bPseudoUeberschrift{R1}

\begin{tabular}{|l|l|}
\hline
A & B  \\ \hline
1 & 2  \\ \hline
2 & 15 \\ \hline
\end{tabular}

\bPseudoUeberschrift{R2}

\begin{tabular}{|l|l|}
\hline
A  & C  \\ \hline
1  & 35 \\ \hline
2  & 12 \\ \hline
13 & 5  \\ \hline
\end{tabular}

\bPseudoUeberschrift{R3}

\begin{tabular}{|l|l|}
\hline
B   & C  \\ \hline
2   & 35 \\ \hline
100 & 35 \\ \hline
\end{tabular}

\bPseudoUeberschrift{(R1 LEFT OUTER JOIN R2) LEFT OUTER JOIN R3}

\begin{tabular}{|l|l|l|l|l|l|}
\hline
R1.A & R1.B & R2.A & R2.C & R3.B & R3.C \\ \hline
1    & 2    & 1    & 35   & 2    & 35   \\ \hline
2    & 15   & 2    & 12   & NULL & NULL \\ \hline
\end{tabular}

\bPseudoUeberschrift{R1 LEFT OUTER JOIN (R2 LEFT OUTER JOIN R3)}

\begin{tabular}{|l|l|l|l|l|l|}
\hline
R1.A & R1.B & R2.A & R2.C & R3.B & R3.C \\ \hline
1    & 2    & 1    & 35   & 2    & 35   \\ \hline
2    & 15   & NULL & NULL & NULL & NULL \\ \hline
\end{tabular}

(Nur wenn beide Tabellen leer sind, wären auch die Outer Joins assoziativ).
\end{bAntwort}
\end{enumerate}
\end{document}
