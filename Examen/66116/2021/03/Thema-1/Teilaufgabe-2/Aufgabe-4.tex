\documentclass{bschlangaul-aufgabe}
\bLadePakete{normalformen,synthese-algorithmus}
\begin{document}
\bAufgabenMetadaten{
  Titel = {Aufgabe 4},
  Thematik = {Normalisierung},
  Referenz = 66116-2021-F.T1-TA2-A4,
  RelativerPfad = Staatsexamen/66116/2021/03/Thema-1/Teilaufgabe-2/Aufgabe-4.tex,
  ZitatSchluessel = examen:66116:2021:03,
  BearbeitungsStand = mit Lösung,
  Korrektheit = unbekannt,
  Ueberprueft = {unbekannt},
  Stichwoerter = {Normalformen},
  EinzelpruefungsNr = 66116,
  Jahr = 2021,
  Monat = 03,
  ThemaNr = 1,
  TeilaufgabeNr = 2,
  AufgabeNr = 4,
}

\let\FA=\bFunktionaleAbhaengigkeiten
\let\schrittE=\bSyntheseUeberErklaerung

Gegeben ist das folgende Relationenschema in erster Normalform,
bestehend aus zwei Relationen:
\index{Normalformen}
\footcite{examen:66116:2021:03}

\begin{center}
\texttt{Relation1(A, B, C, D, E)\\
Relation2(F, G, H, A, E)}
\end{center}

\noindent
In diesem Schema gelten die folgenden funktionalen Abhängigkeiten:

\bigskip

\FA{
  A, B -> C;
  A, B, C -> E;
  A -> D;
  F, G -> H, A;
  G, H -> E;
}

\begin{enumerate}

%%
% a)
%%

\item Nennen Sie die Bedingungen, damit ein Schema in erster Normalform
ist.

\begin{bAntwort}
Ein Schema ist in erster Normalform, wenn es ausschließlich atomare
Attributwerte aufweist.
\end{bAntwort}

%%
% b)
%%

\item Überprüfen Sie, ob das Schema in zweiter Normalform ist.

\begin{bAntwort}
Eine Relation ist in 2NF, wenn sie in 1NF ist und jedes
Nichtschlüsselattribut von jedem Schlüsselkandidaten voll funktional
abhängig ist.
 
Der Schlüsselkandidat ist (A, B) in Relation 1 sowie (F, G) in Relation
2.

Das Nichtschlüsselartribut D in Relation 1 ist nicht voll funktional
abhängig von (A, B), sondern nur von A. Somit ist das Schema nicht in
2NF. Alle anderen Nichtschlüsselattribute sind voll funktional abhängig.
\end{bAntwort}

%%
% c)
%%

\item Wenden Sie den Synthesealgorithmus an, um das Schema in ein Schema
in dritter Normalform zu überführen.

\begin{bAntwort}
\begin{enumerate}
\item \schrittE{1}
\begin{enumerate}
\item \schrittE{1-1}

\FA{
  A, B -> C;
  A, B -> E;
  A -> D;
  F, G -> H, A;
  G, H -> E;
}

\item \schrittE{1-2}

nichts zu tun

\item \schrittE{1-3}

nichts zu tun

\item \schrittE{1-4}

\FA{
  A, B -> C, E;
  A -> D;
  F, G -> H, A;
  G, H -> E;
}

\end{enumerate}
\item \schrittE{2}

R1 (A, B, C, E)
R2 (A, D)
R3 (F, G, H, A)
R4 (G, H, E)

\item \schrittE{3}

R1 (A, B, C, E)
R2 (A, D)
R3 (F, G, H, A)
R4 (G, H, E)
R5 (B, F, G)

als Verbindung von R1 bis R4
(Attributhülle erhält alle Attribute, ist daher Schlüsselkandidat)
\item \schrittE{4}

nichts zu tun
\end{enumerate}
\end{bAntwort}

%%
% d)
%%

\item Sei nun das Relationenschema R(A,B,C,D) in erster Normalform
gegeben. In R gelten die folgenden funktionalen Abhängigkeiten:

\FA{
  A, B -> D;
  B -> C;
  C -> B;
}

Welches ist die höchste Normalform, in der sich das Schema R befindet?
Begründen Sie Ihre Entscheidung.

\end{enumerate}
\end{document}
