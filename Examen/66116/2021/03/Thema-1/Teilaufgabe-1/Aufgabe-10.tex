\documentclass{bschlangaul-aufgabe}
\begin{document}
\bAufgabenMetadaten{
  Titel = {Aufgabe 10},
  Thematik = {AJAX},
  Referenz = 66116-2021-F.T1-TA1-A10,
  RelativerPfad = Staatsexamen/66116/2021/03/Thema-1/Teilaufgabe-1/Aufgabe-10.tex,
  ZitatSchluessel = examen:66116:2021:03,
  BearbeitungsStand = mit Lösung,
  Korrektheit = unbekannt,
  Ueberprueft = {unbekannt},
  Stichwoerter = {Client-Server-Modell},
  EinzelpruefungsNr = 66116,
  Jahr = 2021,
  Monat = 03,
  ThemaNr = 1,
  TeilaufgabeNr = 1,
  AufgabeNr = 10,
}

\begin{enumerate}

%%
% a)
%%

\item Was bedeutet die Abkürzung AJAX?
\index{Client-Server-Modell}
\footcite{examen:66116:2021:03}

\begin{bAntwort}
Asynchronous JavaScript and XML
\end{bAntwort}

%%
% b)
%%

\item Erklären Sie in max. drei Sätzen die grundlegende Funktion von AJAX.

\begin{bAntwort}
Konzept der asynchronen Datenübertragung zwischen einem Browser und dem
Server. Dieses ermöglicht es, HTTP-Anfragen durchzuführen, während eine
HTML-Seite angezeigt wird, und die Seite zu verändern, ohne sie komplett
neu zu laden.
\bFussnoteUrl{https://de.wikipedia.org/wiki/Ajax_(Programmierung)}
\end{bAntwort}
\end{enumerate}
\end{document}
