\documentclass{bschlangaul-aufgabe}
\bLadePakete{mathe}
\begin{document}
\bAufgabenMetadaten{
  Titel = {Aufgabe 2},
  Thematik = {Beweisen von Aussagen},
  Referenz = 66115-2020-H.T2-TA2-A2,
  RelativerPfad = Staatsexamen/66115/2020/09/Thema-2/Teilaufgabe-2/Aufgabe-2.tex,
  ZitatSchluessel = examen:66115:2020:09,
  BearbeitungsStand = nur Angabe,
  Korrektheit = unbekannt,
  Ueberprueft = {unbekannt},
  Stichwoerter = {Komplexitätstheorie},
  EinzelpruefungsNr = 66115,
  Jahr = 2020,
  Monat = 09,
  ThemaNr = 2,
  TeilaufgabeNr = 2,
  AufgabeNr = 2,
}

Beweisen Sie die folgenden Aussagen:
\index{Komplexitätstheorie}
\footcite{examen:66115:2020:09}

\begin{enumerate}

%%
% a)
%%

\item Sei fn)=2-.n?+3-n?+4- (log,n) + 7. Dann gilt f € O(n?).
%%
% b)
%%

\item Sei f(n) = 4”. Dann gilt nicht f € O(2").

\begin{bAntwort}
\bMetaNochKeineLoesung
\end{bAntwort}

%%
% c)
%%

\item Sei fn) = (n+1)! (d. h. die Fakultät von n +1). Dann gilt f € O(n”).

\begin{bAntwort}
z. B. $5 \cdot 4 \cdot 3 \cdot  2 \cdot  1$ $5^5$
\end{bAntwort}
%%
% d)
%%

\item Sei $f \colon \mathbb{N} \leftarrow \mathbb{N}$ definiert durch
die folgende Rekursionsgleichung:

\begin{equation*}
f(n)=
\begin{cases}
3,& \text{für }n = 1\\

(n - 1)^2 + f(n - 1),
& \text{für }n > 1
\end{cases}
\end{equation*}

Dann gilt $f \in \mathcal{O}(n^3)$

\begin{bAntwort}

\begin{align*}
f(n)
&=(n - 1)^2 + f(n - 1) + \dots + f(1)\\
&=(n - 1)^2 + (n - 1)^2 + f(n - 2) + \dots + f(1)\\
&=\underbrace{(n - 1)^2 + \dots + (n - 1)^2 + 3}_{n}\\
&=\underbrace{(n - 1)^2 + \dots + (n - 1)^2}_{n - 1} + 3\\
&= (n - 1)^2 \cdot (n - 1) + 3 \\
&= (n - 1)^3 + 3
\end{align*}
\end{bAntwort}

\end{enumerate}
\end{document}
