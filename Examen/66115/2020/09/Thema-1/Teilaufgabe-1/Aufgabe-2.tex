\documentclass{bschlangaul-aufgabe}
\bLadePakete{formale-sprachen,automaten}
\begin{document}
\bAufgabenMetadaten{
  Titel = {Aufgabe 2},
  Thematik = {Reguläre Sprache xyz},
  Referenz = 66115-2020-H.T1-TA1-A2,
  RelativerPfad = Staatsexamen/66115/2020/09/Thema-1/Teilaufgabe-1/Aufgabe-2.tex,
  ZitatSchluessel = examen:66115:2020:09,
  BearbeitungsStand = mit Lösung,
  Korrektheit = unbekannt,
  Ueberprueft = {unbekannt},
  Stichwoerter = {Reguläre Sprache},
  EinzelpruefungsNr = 66115,
  Jahr = 2020,
  Monat = 09,
  ThemaNr = 1,
  TeilaufgabeNr = 1,
  AufgabeNr = 2,
}

Sei \bAlphabet{x, y, z}. Sei $L = (x^* y z x^*)^* \subseteq \Sigma^*$.
\index{Reguläre Sprache}
\footcite{examen:66115:2020:09}

\begin{enumerate}

%%
% a)
%%

\item Geben Sie einen endlichen (deterministischen oder
nichtdeterministischen) Automaten $A$ an, der $L$ erkennt bzw. akzeptiert.

\begin{bAntwort}
\bPseudoUeberschrift{Nichtdeterministischer endlicher Automat}

\bAutomat{
  nea,
  alphabet={x,y,z},
  zustaende={z_0, z_1, z_2, z_3, z_T},
  ende={z_0, z_2}
}

\begin{center}
\begin{tikzpicture}[li automat]
  \node[state] (z1) at (3.14cm,-2.29cm) {$z_1$};
  \node[state] (z2) at (5.29cm,-2.29cm) {$z_2$};
  \node[state] (z3) at (7.57cm,-2.29cm) {$z_3$};
  \node[state,initial,accepting] (z0) at (3cm,-4.43cm) {$z_0$};

  \path (z1) edge[auto] node{$y$} (z2);
  \path (z1) edge[auto,loop above] node{$x$} (z1);
  \path (z2) edge[auto] node{$z$} (z3);
  \path (z3) edge[auto,bend left] node{$\varepsilon$} (z0);
  \path (z3) edge[auto,loop above] node{$x$} (z3);
  \path (z0) edge[auto] node{$\varepsilon$} (z1);
\end{tikzpicture}
\end{center}
\bFlaci{Ajpmxqvh9}

\bPseudoUeberschrift{Deterministischer endlicher Automat}

\bAutomat{
  dea,
  alphabet={x,y,z},
  zustaende={z_0, z_1, z_2, z_3, z_T},
  ende={z_0, z_2}
}

\begin{center}
\begin{tikzpicture}[li automat]
  \node[state,initial,accepting] (z0) at (2.14cm,-2.14cm) {$z_0$};
  \node[state] (z1) at (3.86cm,-4.43cm) {$z_1$};
  \node[state] (zT) at (2.57cm,-7.43cm) {$z_T$};
  \node[state,accepting] (z2) at (7.71cm,-8.43cm) {$z_2$};
  \node[state] (z3) at (7.86cm,-3.71cm) {$z_3$};

  \path (z0) edge[auto] node{$x$} (z1);
  \path (z0) edge[auto] node{$y$} (z3);
  \path (z0) edge[auto] node{$z$} (zT);
  \path (z1) edge[auto] node{$y$} (z3);
  \path (z1) edge[auto,loop above] node{$x$} (z1);
  \path (z1) edge[auto] node{$z$} (zT);
  \path (zT) edge[auto,loop left] node{$x,y,z$} (zT);
  \path (z2) edge[auto,loop below] node{$x$} (z2);
  \path (z2) edge[auto,bend left] node{$y$} (z3);
  \path (z2) edge[auto] node{$z$} (zT);
  \path (z3) edge[auto,bend left] node{$z$} (z2);
  \path (z3) edge[auto] node{$x,y$} (zT);
\end{tikzpicture}
\end{center}
\bFlaci{A5xo470g9}
\end{bAntwort}

%%
% b)
%%

\item Geben Sie eine reguläre und eindeutige Grammatik $G$ an, die $L$
erzeugt.

\begin{bAntwort}
\begin{bProduktionsRegeln}
Z_0 -> x Z_1 | y Z_3 | EPSILON,
Z_1 -> y Z_3 | x Z_1,
Z_2 -> x Z_2 | x | y Z_3,
Z_3 -> z Z_2 | z,
\end{bProduktionsRegeln}
\bFlaci{Gjfc3c2d2}
\end{bAntwort}

\end{enumerate}
\end{document}
