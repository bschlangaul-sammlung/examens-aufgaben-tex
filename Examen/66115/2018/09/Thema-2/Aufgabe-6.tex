\documentclass{bschlangaul-aufgabe}
\bLadePakete{mathe,java}
\begin{document}
\bAufgabenMetadaten{
  Titel = {Aufgabe 6},
  Thematik = {Rucksackproblem},
  Referenz = 66115-2018-H.T2-A6,
  RelativerPfad = Staatsexamen/66115/2018/09/Thema-2/Aufgabe-6.tex,
  ZitatSchluessel = examen:66115:2018:09,
  ZitatBeschreibung = {Seite 10},
  BearbeitungsStand = mit Lösung,
  Korrektheit = unbekannt,
  Ueberprueft = {unbekannt},
  Stichwoerter = {Backtracking},
  EinzelpruefungsNr = 66115,
  Jahr = 2018,
  Monat = 09,
  ThemaNr = 2,
  AufgabeNr = 6,
}

Ein sehr bekanntes Optimierungsproblem ist das sogenannte
Rucksackproblem: Gegeben ist ein Rucksack mit der Tragfähigkeit $B$.
Weiterhin ist eine endliche Menge von Gegenständen mit Werten und
Gewichten gegeben. Nun soll eine Teilmenge der Gegenstände so ausgewählt
werden, dass ihr Gesamtwert maximal ist, aber ihr Gesamtgewicht die
Tragfähigkeit des Rucksacks nicht überschreitet.
\index{Backtracking}
\footcite[Seite 10]{examen:66115:2018:09}

Mathematisch exakt kann das Rucksackproblem wie folgt formuliert werden:

Gegeben ist eine endliche Menge von Objekten $U$. Durch eine
Gewichtsfunktion $w \colon U \rightarrow \mathbb{R}^+$ wird den Objekten
ein Gewicht und durch eine Nutzenfunktion $v \colon U \rightarrow
\mathbb{R}^+$ ein festgelegter Nutzwert zugeordnet.

Des Weiteren gibt es eine vorgegebene Gewichtsschranke $B \in
\mathbb{R}^+$. Gesucht ist eine Teilmenge $K \subseteq U$, die die
Bedingung $\sum_{u \in K} w(u) \leq B$ einhält und die Zielfunktion
$\sum_{u \in K} v(u)$ maximiert.

Das Rucksackproblem ist NP-vollständig (Problemgröße ist die Anzahl der
Objekte), sodass es an dieser Stelle wenig Sinn macht, über eine
effiziente Lösung nachzudenken. Lösen Sie das Rucksackproblem daher
mittels Backtracking und formulieren Sie einen entsprechenden
Algorithmus. Gehen Sie davon aus, dass die Gewichtsschranke $B$ sowie
die Anzahl an Objekten $N$ beliebig, aber fest vorgegeben sind.

\bigskip

\noindent
Das Programm soll folgende Ausgaben liefern:

\begin{enumerate}
\item Maximaler Nutzwert, der durch eine Objektauswahl unter Einhaltung
der Gewichtsschranke $B$ erreicht werden kann.

\item Das durch die maximierende Objektmenge erreichte Gesamtgewicht.

\item Diejenigen Objekte (Objektnummern) aus $U$, die zur Maximierung
des Nutzwerts beigetragen haben.

\end{enumerate}

\begin{bAntwort}
\bJavaExamen{66115}{2018}{09}{Rucksack}
\end{bAntwort}
\end{document}
