\documentclass{bschlangaul-aufgabe}
\bLadePakete{java,mathe}
\begin{document}
\bAufgabenMetadaten{
  Titel = {Aufgaben 3},
  Thematik = {Fibonacci},
  Referenz = 66115-2017-F.T1-A3,
  RelativerPfad = Staatsexamen/66115/2017/03/Thema-1/Aufgabe-3.tex,
  ZitatSchluessel = examen:66115:2017:03,
  ZitatBeschreibung = {Thema 1 Aufgabe 3 Seite 5},
  BearbeitungsStand = mit Lösung,
  Korrektheit = unbekannt,
  Ueberprueft = {unbekannt},
  EinzelpruefungsNr = 66115,
  Jahr = 2017,
  Monat = 03,
  ThemaNr = 1,
  AufgabeNr = 3,
}

Gegeben\footcite[Thema 1 Aufgabe 3 Seite 5]{examen:66115:2017:03} seien
die folgenden Formeln zur Berechnung der \emph{ersten}
Fibonacci-Zahlen\footcite{wiki:fibonacci-folge}:

\begin{equation*}
\text{fib}_n =
\begin{cases}
1 & \text{falls}\ n \leq 2\\
\text{fib}_{n-1} + \text{fib}_{n-2} & \text{sonst}
\end{cases}
\end{equation*}

\noindent
sowie der Partialsumme der Fibonacci-Quadrate:
\footcite[Seite 1, Aufgabe 1]{aud:pu:3}

\begin{equation*}
\text{sos}_n =
\begin{cases}
\text{fib}_n & \text{falls}\ n = 1\\
\text{fib}_n^2 + \text{sos}_{n-1} & \text{sonst}
\end{cases}
\end{equation*}

\noindent
Sie dürfen im Folgenden annehmen, dass die Methoden nur mit $1 \leq n
\leq 46$ aufgerufen werden, so dass der Datentyp \bJavaCode{long} zur
Darstellung aller Werte ausreicht.

\begin{bExkurs}[Fibonacci-Folge]
Die Fibonacci-Folge beginnt zweimal mit der Zahl 1. Im Anschluss ergibt
jeweils die Summe zweier aufeinanderfolgender Zahlen die unmittelbar
danach folgende Zahl: 1, 1, 2, 3, 5, 8,
13\footcite{wiki:fibonacci-folge}
\end{bExkurs}

\begin{bExkurs}[Partialsumme]
Unter der n-ten Partialsumme $s_n$ einer Zahlenfolge $a_n$ versteht man
die Summe der Folgenglieder von $a_1$ bis $a_n$. Die immer
weiter fortgesetzte Partialsumme einer (unendlichen) Zahlenfolge nennt
man eine (unendliche) Reihe.
\footnote{\url{https://www.lernhelfer.de/schuelerlexikon/mathematik/artikel/folgen-partialsummen}}
Partialsummen sind das Bindeglied zwischen Summen und Reihen.
Gegeben sei die Reihe $\sum_{k = 1}^{\infty} a_k$.
Die $n$-te Partialsumme dieser Reihe lautet: $\sum_{k = 1}^{n} a_k$.
\dh wir summieren unsere Reihe nur bis zum Endindex $n$.
\footnote{\url{https://www.massmatics.de/merkzettel/index.php\#!164:Partialsummen}}
\end{bExkurs}

\texttt{sos} steht für \emph{Summe of Squares}

\begin{tabular}{|l|l|l|l|l|}
\hline
n &
$\text{fib}_n$ &
$\text{fib}_n^2$ &
&
$\sum_{k = 1}^{n} \text{fib}^k$\\\hline\hline

1 &
1 &
1 &
1 &
1\\\hline

2 &
1 &
1 &
$1+1$ &
2\\\hline

3 &
2 &
4 &
$1+1+4$ &
6\\\hline

4 &
3 &
9 &
$1+1+4+9$ &
15\\\hline

5 &
5 &
25 &
$1+1+4+9+25$ &
40\\\hline

6 &
8 &
64 &
$1+1+4+9+25+64$ &
104\\\hline

7 &
13 &
169 &
$1+1+4+9+25+64+169$ &
273\\\hline

8 &
21 &
441 &
$1+1+4+9+25+64+169+441$ &
714\\\hline

9 &
34 &
1156 &
$1+1+4+9+25+64+169+441+1156$ &
1870\\\hline

10 &
55 &
3025 &
$1+1+4+9+25+64+169+441+1156+3025$ &
4895 \\\hline

\end{tabular}

\begin{enumerate}

%%
% a)
%%

\item Implementieren Sie die obigen Formeln zunächst rekursiv (ohne
Schleifenkonstrukte wie \mintinline{java}|for| oder
\mintinline{java}|while|) und ohne weitere Optimierungen („naiv”) in
Java als:

\hspace{1cm}\mintinline{java}|long fibNaive (int n) {|

bzw.

\hspace{1cm}\mintinline{java}|long sosNaive (int n) {|

\begin{bAntwort}
\bJavaExamen[firstline=16,lastline=28]{66115}{2017}{03}{Fibonacci}
\end{bAntwort}

%%
% b)
%%

\item Offensichtlich ist die naive Umsetzung extrem ineffizient, da
viele Zwischenergebnisse wiederholt rekursiv ausgewertet werden müssen.
Die Dynamische Programmierung (DP) erlaubt es Ihnen, die Laufzeit auf
Kosten des Speicherbedarfs zu reduzieren, indem Sie alle einmal
berechneten Zwischenergebnisse speichern und bei erneutem Bedarf „direkt
abrufen“. Implementieren Sie obige Formeln nun rekursiv aber mittels DP
in Java als:

\hspace{1cm}\mintinline{java}|long fibDP (int n) {|

bzw.

\hspace{1cm}\mintinline{java}|long sosDP (int n) {|

\begin{bAntwort}
\bJavaExamen[firstline=30,lastline=57]{66115}{2017}{03}{Fibonacci}
\end{bAntwort}

%%
% c)
%%

\item Am „einfachsten“ und bzgl. Laufzeit [in $\mathcal{O}(n)$] sowie
Speicherbedarf [in $\mathcal{O}(1)$] am effizientesten ist sicherlich
eine iterative Implementierung der beiden Formeln. Geben Sie eine solche
in Java an als:

\hspace{1cm}\mintinline{java}|long fibIter (int n) {|

bzw.

\hspace{1cm}\mintinline{java}|long sosIter (int n) {|

\begin{bAntwort}
\bJavaExamen[firstline=59,lastline=81]{66115}{2017}{03}{Fibonacci}
\end{bAntwort}
\end{enumerate}

\begin{bAdditum}
\bPseudoUeberschrift{Kompletter Code}

\bJavaExamen{66115}{2017}{03}{Fibonacci}

\bPseudoUeberschrift{Test}

\bJavaTestDatei{examen/examen_66115/jahr_2017/fruehjahr/FibonacciTest}
\end{bAdditum}

\end{document}
