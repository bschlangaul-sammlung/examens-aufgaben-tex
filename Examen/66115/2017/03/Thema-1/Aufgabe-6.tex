\documentclass{bschlangaul-aufgabe}
\bLadePakete{mathe}
\begin{document}
\bAufgabenMetadaten{
  Titel = {Aufgabe 6},
  Thematik = {Turingmaschin mit mindestens 1000 Schritten},
  Referenz = 66115-2017-F.T1-A6,
  RelativerPfad = Staatsexamen/66115/2017/03/Thema-1/Aufgabe-6.tex,
  ZitatSchluessel = examen:66115:2017:03,
  BearbeitungsStand = nur Angabe,
  Korrektheit = unbekannt,
  Ueberprueft = {unbekannt},
  Stichwoerter = {Entscheidbarkeit},
  EinzelpruefungsNr = 66115,
  Jahr = 2017,
  Monat = 03,
  ThemaNr = 1,
  AufgabeNr = 6,
}

Es sei $E$ die Menge aller (geeignet codierten) Turingmaschinen $M$ mit
folgender Eigenschaft: Es gibt eine Eingabe $w$, so dass $M$ gestartet
auf $w$ mindestens 1000 Schritte rechnet und dann irgendwann hält.
\index{Entscheidbarkeit}
\footcite{examen:66115:2017:03}

Das Halteproblem auf leerer Eingabe $H_0$ ist definiert als die Menge
aller Turingmaschinen, die auf leerer Eingabe gestartet, irgendwann
halten.

\begin{enumerate}

%%
% a)
%%

\item Zeigen Sie, dass $E$ unentscheidbar ist (etwa durch Reduktion vom
Halteproblem $H_0$).

% Info_2021-05-07-2021-05-07_13.30.16.mp4
% 1h34min - 1h47min
\begin{bAntwort}
zu zeigen: $L_H \leq L \rightarrow L$ ist genauso unentscheidbar wie $L_H$

Eingabeinstanzen von $L_H (TM(M), u)$ durch Funktion umbauen in
Eingabeinstanzen von $L (TM (M^\prime))$.

Idee: Turingmaschine so modifizieren, dass sie zunächst
1000 Schritte macht und dann $M$ auf $u$ startet.
\footcite[Seite 53]{theo:fs:4}

Dazu definieren wir die Funktion $f : \Sigma^* \rightarrow \Sigma^*$ wie
folgt:

\begin{equation*}
f(u) =
\begin{cases}
c(M^\prime) &
\text{falls }u = c(M^\prime)w\text{ ist für eine Turingmaschine }M\text{ und Eingabe }w\\

0 & \text{sonst}
\end{cases}
\end{equation*}

Dabei sei M’ eine Turingmaschine, die sich wie folgt verhält:
\begin{enumerate}
\item Geht 1000 Schritte nach rechts
\item Schreibt festes Wort w (für M’ ist w demnach fest!)
\item Startet M
\end{enumerate}

\begin{description}
\item [total:]

ja

\item [berechenbar:]

Syntaxcheck, 1000 Schritte über 1000 weitere Zustände realisierbar

\item [Korrektheit:]

$u \in L_{halt} \Leftrightarrow u = c(M)w$ für TM $M$, die auf $w$ hält
$\Leftrightarrow f(u) = c(M^\prime)$, wobei $M^\prime$ $1000$ Schritte
macht und dann hält $\Leftrightarrow f(u) \in L$
\end{description}
\end{bAntwort}
\footcite[Seite 54]{theo:fs:4}

%%
% b)
%%

\item Begründen Sie, dass $E$ partiell entscheidbar ist.

%%
% c)
%%

\item Geben Sie ein Problem an, welches nicht einmal partiell
entscheidbar ist.

\end{enumerate}
\end{document}
