\documentclass{bschlangaul-aufgabe}
\bLadePakete{java}
\begin{document}
\bAufgabenMetadaten{
  Titel = {Aufgabe 2},
  Thematik = {Binäre Bäume},
  Referenz = 66115-2014-F.T1-A2,
  RelativerPfad = Staatsexamen/66115/2014/03/Thema-1/Aufgabe-2.tex,
  ZitatSchluessel = examen:66115:2014:03,
  BearbeitungsStand = mit Lösung,
  Korrektheit = unbekannt,
  Ueberprueft = {unbekannt},
  Stichwoerter = {Binärbaum},
  EinzelpruefungsNr = 66115,
  Jahr = 2014,
  Monat = 03,
  ThemaNr = 1,
  AufgabeNr = 2,
}

\let\j=\bJavaCode

Implementieren Sie in einer objekt-orientierten Sprache Ihrer Wahl eine
Klasse namens \j{BinBaum}, deren Instanzen binäre Bäume mit ganzzahligen
Datenknoten darstellen, nach folgender Spezifikation:\index{Binärbaum}
\footcite{examen:66115:2014:03}

\begin{enumerate}

%%
% a)
%%

\item Beginnen Sie zunächst mit der Datenstruktur selbst:

\begin{itemize}
\item Mit Hilfe des Konstruktors soll ein neuer Baum erstellt werden,
der aus einem einzelnen Knoten besteht, in dem der dem Konstruktor als
Parameter übergebene Wert (ganzzahlig, in Java z.B. \j{int}) gespeichert
ist.

\begin{bAntwort}
\bJavaExamen[firstline=14,lastline=22]{66115}{2014}{03}{BinBaum}
\end{bAntwort}

\item Die Methoden \j{setLeft(int value)} bzw. \j{setRight(int value)}
sollen den linken bzw. rechten Teilbaum des aktuellen Knotens durch
einen jeweils neuen Teilbaum ersetzen, der seinerseits aus einem
einzelnen Knoten besteht, in dem der übergebene Wert \j{value}
gespeichert ist. Sie haben keinen Rückgabewert.

\begin{bAntwort}
\bJavaExamen[firstline=24,lastline=30]{66115}{2014}{03}{BinBaum}
\end{bAntwort}

\item Die Methoden \j{getLeft()} bzw. \j{getRight()} sollen den linken
bzw. rechten Teilbaum zurückgeben (bzw. \j{null}, wenn keiner vorhanden
ist)

\begin{bAntwort}
\bJavaExamen[firstline=32,lastline=38]{66115}{2014}{03}{BinBaum}
\end{bAntwort}

\item Die Methode \j{int getValue()} soll den Wert zurückgeben, der im
aktuellen Wurzelknoten gespeichert ist.

\begin{bAntwort}
\bJavaExamen[firstline=40,lastline=42]{66115}{2014}{03}{BinBaum}
\end{bAntwort}
\end{itemize}

%%
% b)
%%

\item Implementieren Sie nun die Methoden \j{preOrder()} bzw.
\j{postOrder()}. Sie sollen die Knoten des Baumes mit Tiefensuche
traversieren, die Werte dabei in pre-order bzw. post-order Reihenfolge
in eine Liste (z.B. \j{List<Integer>}) speichern und diese Ergebnisliste
zurückgeben. Die Tiefensuche soll dabei zuerst in den linken und dann in
den rechten Teilbaum absteigen.

\begin{bAntwort}
\bJavaExamen[firstline=45,lastline=71]{66115}{2014}{03}{BinBaum}
\end{bAntwort}

\item Ergänzen Sie schließlich die Methode \j{isSearchTree ()}. Sie soll
überprüfen, ob der Binärbaum die Eigenschaften eines binären Suchbaums
erfüllt. Beachten Sie, dass die Laufzeit-Komplexität Ihrer
Implementierung linear zur Anzahl der Knoten im Baum sein muss.

\begin{bAntwort}
\bJavaExamen[firstline=73,lastline=87]{66115}{2014}{03}{BinBaum}
\end{bAntwort}

\end{enumerate}

\begin{bAdditum}
\bJavaExamen{66115}{2014}{03}{BinBaum}
\end{bAdditum}
\end{document}
