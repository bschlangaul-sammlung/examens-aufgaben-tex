\documentclass{bschlangaul-aufgabe}
\bLadePakete{komplexitaetstheorie}
\begin{document}
\bAufgabenMetadaten{
  Titel = {Aufgabe 4},
  Thematik = {CLIQUE - ALMOST CLIQUE},
  Referenz = 66115-2021-F.T2-TA1-A4,
  RelativerPfad = Staatsexamen/66115/2021/03/Thema-2/Teilaufgabe-1/Aufgabe-4.tex,
  ZitatSchluessel = examen:66115:2021:03,
  BearbeitungsStand = mit Lösung,
  Korrektheit = unbekannt,
  Ueberprueft = {unbekannt},
  Stichwoerter = {Polynomialzeitreduktion},
  EinzelpruefungsNr = 66115,
  Jahr = 2021,
  Monat = 03,
  ThemaNr = 2,
  TeilaufgabeNr = 1,
  AufgabeNr = 4,
}

\let\n=\bProblemName

Betrachten Sie die folgenden Probleme:
\index{Polynomialzeitreduktion}
\footcite{examen:66115:2021:03}

\bProblemBeschreibung
{Clique}
{Ein ungerichteter Graph $G = (V, E)$, eine Zahl $k \in \mathcal{N}$}
{Gibt es eine Menge $S \subseteq V$ mit $|S| = k$,
sodass für alle Knoten $u \neq v \in V$ gilt,
dass $\{ u, v \}$ eine Kante in $E$ ist?}

\bProblemBeschreibung
{Almost Clique}
{Ein ungerichteter Graph $G = (V, E)$, eine Zahl $k \in \mathcal{N}$}
{Gibt es eine Menge $S \subseteq V$ mit $|S| = k$,
sodass die Anzahl der Kanten zwischen
Knoten in $S$ genau $\frac{k(k - 1)}{2} - 1$ ist?}

\noindent
Zeigen Sie, dass das Problem \n{Almost Clique} NP-vollständig ist.
Nutzen Sie dafür die NP-Vollständigkeit von \n{Clique}.

Hinweis: Die Anzahl der Kanten einer $k$-Clique sind $\frac{k(k -
1)}{2}$.

\begin{bExkurs}[Cliquenproblem]
\bProblemClique
\end{bExkurs}

\begin{bExkurs}[\n{Almost Clique}]
Eine Gruppe von Knoten wird \n{Almost Clique} genannt, wenn nur eine
Kante ergänzt werden muss, damit sie zu einer Clique wird.
\end{bExkurs}

\begin{bAntwort}
You can reduce to this from $CLIQUE$.

Given a graph $G=(V,E)$ and $t$, construct a new graph $G^*$ by adding
two new vertices $\{v_{n+1},v_{n +2}\}$ and connecting them with all of
$G$'s vertices but removing the edge $\{v_{n+1},v_{n+2}\}$, i.e. they
are not neighbors in $G^*$. return $G^*$ and $t+2$.

If $G$ has a $t$ sized clique by adding it to the two vertices we get an
$t+2$ almost clique in $G^*$ (by adding $\{v_{n+1},v_{n+2}\}$).

If $G^*$ has a $t+2$ almost clique we can look at three cases:

1) It contains the two vertices $\{v_{n+1},v_{n+2}\}$, then the missing
edge must be $\{v_{n+1},v_{n+2}\}$ and this implies that the other $t$
vertices form a $t$ clique in $G$.

2) It contains one of the vertices $\{v_{n+1},v_{n+2}\}$, say w.l.o.g.
$v_{n+1}$, then the missing edge must be inside $G$, say $e=\{u,v\}\in
G$. If we remove $u$ and $v_{n+1}$ then the other $t$ vertices, which
are in $G$ must form a clique of size $t$.

3) It does not contain any of the vertices $\{v_{n+1},v_{n+2}\}$, then
it is clear that this group is in $G$ and must contain a clique of size
$t$.

It is also clear that the reduction is in polynomial time, actually in
linear time, log-space.
\bFussnoteUrl{https://cs.stackexchange.com/a/76627}
\end{bAntwort}
\end{document}
