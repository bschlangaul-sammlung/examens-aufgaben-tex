\documentclass{bschlangaul-aufgabe}
\bLadePakete{baum}
\begin{document}
\bAufgabenMetadaten{
  Titel = {Aufgabe 7},
  Thematik = {Heap und binärer Suchbaum},
  Referenz = 66115-2013-H.T2-A7,
  RelativerPfad = Staatsexamen/66115/2013/09/Thema-2/Aufgabe-7.tex,
  ZitatSchluessel = examen:66115:2013:09,
  BearbeitungsStand = mit Lösung,
  Korrektheit = unbekannt,
  Ueberprueft = {unbekannt},
  Stichwoerter = {Binärbaum, Halde (Heap)},
  EinzelpruefungsNr = 66115,
  Jahr = 2013,
  Monat = 09,
  ThemaNr = 2,
  AufgabeNr = 7,
}

\begin{enumerate}

\item \strut

\begin{enumerate}

%%
% i)
%%

\item Fügen Sie nacheinander die Zahlen $7$, $1$, $12$, $8$, $10$, $3$,
$5$ in einen leeren binären Suchbaum ein und zeichnen Sie den Suchbaum
nach „$8$“ und nach „$3$“.\index{Binärbaum}\index{Halde (Heap)}
\footcite{examen:66115:2013:09}

\begin{bAntwort}
\begin{bBaum}{Nach dem Einfügen von „8“}
\begin{tikzpicture}[b binaer baum]
\Tree
[.7
  [.1 ]
  [.12
    [.8 ]
    \edge[blank]; \node[blank]{};
  ]
]
\end{tikzpicture}
\end{bBaum}

\begin{bBaum}{Nach dem Einfügen von „3“}
\begin{tikzpicture}[b binaer baum]
\Tree
[.7
  [.1
    \edge[blank]; \node[blank]{};
    [.3 ]
  ]
  [.12
    [.8
      \edge[blank]; \node[blank]{};
      [.10 ]
    ]
    \edge[blank]; \node[blank]{};
  ]
]
\end{tikzpicture}
\end{bBaum}

\begin{bBaum}{Nach dem Einfügen von „5“}
\begin{tikzpicture}[b binaer baum]
\Tree
[.7
  [.1
    \edge[blank]; \node[blank]{};
    [.3
      \edge[blank]; \node[blank]{};
      [.5 ]
    ]
  ]
  [.12
    [.8
      \edge[blank]; \node[blank]{};
      [.10 ]
    ]
    \edge[blank]; \node[blank]{};
  ]
]
\end{tikzpicture}
\end{bBaum}
\end{bAntwort}

%%
% ii)
%%

\item Löschen Sie die „$1$“ aus dem in (i) erstellten Suchbaum und
zeichnen Sie den Suchbaum.

\begin{bAntwort}
\begin{bBaum}{Nach dem Löschen von „1“}
\begin{tikzpicture}[b binaer baum]
\Tree
[.7
  [.3
    \edge[blank]; \node[blank]{};
    [.5 ]
  ]
  [.12
    [.8
      \edge[blank]; \node[blank]{};
      [.10 ]
    ]
    \edge[blank]; \node[blank]{};
  ]
]
\end{tikzpicture}
\end{bBaum}
\end{bAntwort}

%%
% iii)
%%

\item Fügen Sie 7, 1, 12, 8, 10, 3, 5 in einen leeren MIN-Heap ein,
der bzgl. „$\leq$“ angeordnet ist.
Geben Sie den Heap nach jedem Element an.

\begin{bAntwort}
\begin{bBaum}{Nach dem Einfügen von „7“}
\begin{tabular}{l}
\bf{0} \\
\hline
7      \\
\end{tabular}

\begin{tikzpicture}[b binaer baum]
\Tree
[.7 ]
\end{tikzpicture}
\end{bBaum}

\begin{bBaum}{Nach dem Einfügen von „1“}
\begin{tabular}{ll}
\bf{0} & \bf{1} \\
\hline
7      & 1      \\
\end{tabular}

\begin{tikzpicture}[b binaer baum]
\Tree
[.7
  [.1 ]
  \edge[blank]; \node[blank]{};
]
\end{tikzpicture}
\end{bBaum}

\begin{bBaum}{Nach dem Vertauschen von „1“ und „7“}
\begin{tabular}{ll}
\bf{0} & \bf{1} \\
\hline
1      & 7      \\
\end{tabular}

\begin{tikzpicture}[b binaer baum]
\Tree
[.1
  [.7 ]
  \edge[blank]; \node[blank]{};
]
\end{tikzpicture}
\end{bBaum}

\begin{bBaum}{Nach dem Einfügen von „12“}
\begin{tabular}{lll}
\bf{0} & \bf{1} & \bf{2} \\
\hline
1      & 7      & 12     \\
\end{tabular}

\begin{tikzpicture}[b binaer baum]
\Tree
[.1
  [.7 ]
  [.12 ]
]
\end{tikzpicture}
\end{bBaum}

\begin{bBaum}{Nach dem Einfügen von „8“}
\begin{tabular}{llll}
\bf{0} & \bf{1} & \bf{2} & \bf{3} \\
\hline
1      & 7      & 12     & 8      \\
\end{tabular}

\begin{tikzpicture}[b binaer baum]
\Tree
[.1
  [.7
    [.8 ]
    \edge[blank]; \node[blank]{};
  ]
  [.12 ]
]
\end{tikzpicture}
\end{bBaum}

\begin{bBaum}{Nach dem Einfügen von „10“}
\begin{tabular}{lllll}
\bf{0} & \bf{1} & \bf{2} & \bf{3} & \bf{4} \\
\hline
1      & 7      & 12     & 8      & 10     \\
\end{tabular}

\begin{tikzpicture}[b binaer baum]
\Tree
[.1
  [.7
    [.8 ]
    [.10 ]
  ]
  [.12 ]
]
\end{tikzpicture}
\end{bBaum}

\begin{bBaum}{Nach dem Einfügen von „3“}
\begin{tabular}{llllll}
\bf{0} & \bf{1} & \bf{2} & \bf{3} & \bf{4} & \bf{5} \\
\hline
1      & 7      & 12     & 8      & 10     & 3      \\
\end{tabular}

\begin{tikzpicture}[b binaer baum]
\Tree
[.1
  [.7
    [.8 ]
    [.10 ]
  ]
  [.12
    [.3 ]
    \edge[blank]; \node[blank]{};
  ]
]
\end{tikzpicture}
\end{bBaum}

\begin{bBaum}{Nach dem Vertauschen von „3“ und „12“}
\begin{tabular}{llllll}
\bf{0} & \bf{1} & \bf{2} & \bf{3} & \bf{4} & \bf{5} \\
\hline
1      & 7      & 3      & 8      & 10     & 12     \\
\end{tabular}

\begin{tikzpicture}[b binaer baum]
\Tree
[.1
  [.7
    [.8 ]
    [.10 ]
  ]
  [.3
    [.12 ]
    \edge[blank]; \node[blank]{};
  ]
]
\end{tikzpicture}
\end{bBaum}

\begin{bBaum}{Nach dem Einfügen von „5“}
\begin{tabular}{lllllll}
\bf{0} & \bf{1} & \bf{2} & \bf{3} & \bf{4} & \bf{5} & \bf{6} \\
\hline
1      & 7      & 3      & 8      & 10     & 12     & 5      \\
\end{tabular}

\begin{tikzpicture}[b binaer baum]
\Tree
[.1
  [.7
    [.8 ]
    [.10 ]
  ]
  [.3
    [.12 ]
    [.5 ]
  ]
]
\end{tikzpicture}
\end{bBaum}
\end{bAntwort}
\end{enumerate}

%%
% b)
%%

\item Was ist die worst-case Laufzeit in O-Notation
für das Einfügen eines Elements in einen Heap der Größe $n$?
Begründen Sie ihre Antwort.

\begin{bAntwort}
Die worst-case Laufzeit berechnet sich aus dem Aufwand für das
Durchsickern eines eingefügten Elementes. Da das Durchsickern entlang
eines Pfades im Baum erfolgt, entspricht der Aufwand im ungünstigsten
Fall der Höhe des Baumes, \dh $\log_2 n$. Insgesamt ergibt sich somit
eine worst-case Laufzeit von $\mathcal{O}(\log n)$.
\footcite[Seite 413]{saake}
\end{bAntwort}
\end{enumerate}

\end{document}
