\documentclass{bschlangaul-aufgabe}
\bLadePakete{graph}
\begin{document}
\bAufgabenMetadaten{
  Titel = {Aufgabe 8},
  Thematik = {Graph a-h},
  Referenz = 66115-2019-H.T2-A8,
  RelativerPfad = Staatsexamen/66115/2019/09/Thema-2/Aufgabe-8.tex,
  ZitatSchluessel = examen:66115:2019:09,
  BearbeitungsStand = nur Angabe,
  Korrektheit = unbekannt,
  Ueberprueft = {unbekannt},
  Stichwoerter = {Minimaler Spannbaum},
  EinzelpruefungsNr = 66115,
  Jahr = 2019,
  Monat = 09,
  ThemaNr = 2,
  AufgabeNr = 8,
}

Gegeben Sei der folgende ungerichtete Graph mit Kantengewichten.
\index{Minimaler Spannbaum}
\footcite{examen:66115:2019:09}

\begin{bGraphenFormat}
a: 0 0
b: 0 1
c: 0 2
d: 1 0
e: 1 2
f: 1.5 1
g: 2 0
h: 2 2
a -- b: 7
a -- d: 1
b -- c: 6
b -- d: 4
b -- e: 9
c -- e: 8
d -- e: 13
d -- f: 12
d -- g: 5
e -- f: 10
e -- h: 2
f -- g: 11
g -- h: 3
\end{bGraphenFormat}

\begin{center}
\begin{tikzpicture}[li graph,x=3cm,y=2cm]
\node (a) at (0,0) {a};
\node (b) at (0,1) {b};
\node (c) at (0,2) {c};
\node (d) at (1,0) {d};
\node (e) at (1,2) {e};
\node (f) at (1.5,1) {f};
\node (g) at (2,0) {g};
\node (h) at (2,2) {h};

\path (a) edge node {7} (b);
\path (a) edge (d);
\path (b) edge node {6} (c);
\path (b) edge node {4} (d);
\path (b) edge node {9} (e);
\path (c) edge node {8} (e);
\path (d) edge node {13} (e);
\path (d) edge node {12} (f);
\path (d) edge node {5} (g);
\path (e) edge node {10} (f);
\path (e) edge node {2} (h);
\path (f) edge node {11} (g);
\path (g) edge node {3} (h);
\end{tikzpicture}
\end{center}

\begin{enumerate}

%%
% (a)
%%

\item Zeichnen Sie den (hier eindeutigen) minimalen Spannbaum.

%%
% (b)
%%

\item Geben Sie sowohl für den Algorithmus von Jarník-Prim als auch für
den Algorithmus von Kruskal die Reihenfolge an, in der die Kanten
hinzugefügt werden. Starten Sie für den Algorithmus von Jarník-Prim beim
Knoten $a$.

Übernehmen Sie den Graph auf Ihre Bearbeitung und füllen Sie hierzu das
Tupel jeder Kante - aus dem MST in der Form $(n, m)$ aus, wobei die
Kante $e$ vom Algorithmus von Jarník-Prim als $n$’te Kante und vom
Algorithmus von Kruskal als $m$’te Kante hinzugefügt wird. Lassen Sie
andere Tupel unausgefüllt.
\end{enumerate}

\end{document}
