\documentclass{bschlangaul-aufgabe}
\bLadePakete{pseudo,mathe,master-theorem}
\begin{document}
\bAufgabenMetadaten{
  Titel = {Aufgabe 7},
  Thematik = {Sortieren mit Quicksort},
  Referenz = 66115-2016-H.T2-A7,
  RelativerPfad = Staatsexamen/66115/2016/09/Thema-2/Aufgabe-7.tex,
  ZitatSchluessel = examen:66115:2016:09,
  BearbeitungsStand = mit Lösung,
  Korrektheit = unbekannt,
  Ueberprueft = {unbekannt},
  Stichwoerter = {Quicksort},
  EinzelpruefungsNr = 66115,
  Jahr = 2016,
  Monat = 09,
  ThemaNr = 2,
  AufgabeNr = 7,
}

\let\O=\bO
\let\o=\bOmega
\let\T=\bT
\let\t=\bTheta

\begin{enumerate}

%%
% a)
%%

\item Gegeben ist die Ausgabe der Methode \textbf{Partition} (s.
Pseudocode), rekonstruieren Sie die Eingabe.\index{Quicksort}
\footcite{examen:66115:2016:09}

Konkret sollen Sie das Array A = (\_, \_, 1, \_, \_) so
vervollständigen, dass der Aufruf Partition(A, 1, 5) die Zahl 3
zurückgibt und nach dem Aufruf gilt, dass A = (1, 2, 3, 4, 5) ist.

Geben Sie A nach jedem Durchgang der for-Schleife in \textbf{Partition}
an.

\begin{bAntwort}
%  bschlangaul-werkzeug.java sortiere 2 4 1 5 3  -q --rechts
\begin{verbatim}
 2  4  1  5  3  Eingabe
 2  4  1  5  3  zerlege
 2  4  1  5  3* markiere (i 4)
>2< 4  1  5  3  vertausche (i 0<>0)
 2 >4  1< 5  3  vertausche (i 1<>2)
 2  1 >4  5  3< vertausche (i 2<>4)
 2  1           zerlege
 2  1*          markiere (i 1)
>2  1<          vertausche (i 0<>1)
          5  4  zerlege
          5  4* markiere (i 4)
         >5  4< vertausche (i 3<>4)
 1  2  3  4  5  Ausgabe
\end{verbatim}
\end{bAntwort}

%%
% b)
%%

\item Beweisen Sie die Korrektheit von \textbf{Partition} (\zB mittels
einer Schleifeninvarianten)!

%%
% c)
%%

\item Geben Sie für jede natürliche Zahl $n$ eine Instanz $I_n$, der
Länge $n$ an, so dass QuickSort($I_n$) $\Omega(n^2)$ Zeit benötigt.
Begründen Sie Ihre Behauptung.

\begin{bAntwort}
$I_n = 1, 2, 3, \dots, n$

Die Methode \textbf{Partition} wird $n$ mal aufgerufen, weil bei jedem
Aufruf der Methode nur eine Zahl, nämlich die größte Zahl, abgespalten
wird.

\begin{itemize}
\item Partition(A, 1, $n$)
\item Partition(A, 1, $n-1$)
\item Partition(A, 1, $n-2$)
\item Partition(A, 1, $\dots$)
\item Partition(A, 1, $1$)
\end{itemize}

In der For-Schleife der Methode Partition wird bei jeder Wiederholung
ein Vertauschvorgang durchgeführt (Die Zahlen werden mit sich
selbst getauscht.)

\begin{verbatim}
 1  2  3  4  5  6  7  zerlege
 1  2  3  4  5  6  7* markiere (i 6)
>1< 2  3  4  5  6  7  vertausche (i 0<>0)
 1 >2< 3  4  5  6  7  vertausche (i 1<>1)
 1  2 >3< 4  5  6  7  vertausche (i 2<>2)
 1  2  3 >4< 5  6  7  vertausche (i 3<>3)
 1  2  3  4 >5< 6  7  vertausche (i 4<>4)
 1  2  3  4  5 >6< 7  vertausche (i 5<>5)
 1  2  3  4  5  6 >7< vertausche (i 6<>6)
 1  2  3  4  5  6     zerlege
 1  2  3  4  5  6*    markiere (i 5)
>1< 2  3  4  5  6     vertausche (i 0<>0)
 1 >2< 3  4  5  6     vertausche (i 1<>1)
 1  2 >3< 4  5  6     vertausche (i 2<>2)
 1  2  3 >4< 5  6     vertausche (i 3<>3)
 1  2  3  4 >5< 6     vertausche (i 4<>4)
 1  2  3  4  5 >6<    vertausche (i 5<>5)
 1  2  3  4  5        zerlege
 1  2  3  4  5*       markiere (i 4)
>1< 2  3  4  5        vertausche (i 0<>0)
 1 >2< 3  4  5        vertausche (i 1<>1)
 1  2 >3< 4  5        vertausche (i 2<>2)
 1  2  3 >4< 5        vertausche (i 3<>3)
 1  2  3  4 >5<       vertausche (i 4<>4)
 1  2  3  4           zerlege
 1  2  3  4*          markiere (i 3)
>1< 2  3  4           vertausche (i 0<>0)
 1 >2< 3  4           vertausche (i 1<>1)
 1  2 >3< 4           vertausche (i 2<>2)
 1  2  3 >4<          vertausche (i 3<>3)
 1  2  3              zerlege
 1  2  3*             markiere (i 2)
>1< 2  3              vertausche (i 0<>0)
 1 >2< 3              vertausche (i 1<>1)
 1  2 >3<             vertausche (i 2<>2)
 1  2                 zerlege
 1  2*                markiere (i 1)
>1< 2                 vertausche (i 0<>0)
 1 >2<                vertausche (i 1<>1)
\end{verbatim}

\end{bAntwort}

%%
% d)
%%

\item Was müsste Partition (in Linearzeit) leisten, damit QuickSort
Instanzen der Länge n in $\mathcal{O}(n \cdot \log n)$ Zeit sortiert?
Zeigen Sie, dass Partition mit der von Ihnen geforderten Eigenschaft zur
gewünschten Laufzeit von QuickSort führt.

\bMasterExkurs

\begin{bAntwort}
Die Methode \textbf{Partition} müsste die Instanzen der Länge $n$ in
zwei gleich große Teile spalten ($\frac{n - 1}{2}$).

\bMasterVariablenDeklaration
{2} % a
{2} % b
{n} % f(n)
\bMasterFallRechnung
% 1. Fall
{für $\varepsilon = 4$: \\
$f(n) = n \notin \O{n^{\log_2 {2 - \varepsilon}}}$}
% 2. Fall
{$f(n) = n \in \t{n^{\log_2 {2}}} = \t{n}$}
% 3. Fall
{$f(n) = n \notin \o{n^{\log_2 {2 + \varepsilon}}}$}

$\Rightarrow T(n) \in \t{n^{\log_2 {2}} \cdot \log n} = \t{n \cdot \log n}$

\bMasterWolframLink{T[n]=2T[n/2]\%2Bn}
\end{bAntwort}

\begin{function}
\caption{Quicksort(A, $l = 1$, $r = A.\text{length}$)}

\If{$l < r$}{
  $m = \text{Partition}(A, l, r)$\;
  $\text{Quicksort}(A, l, m - 1)$\;
  $\text{Quicksort}(A, m + 1, r)$\;
}
\end{function}

\begin{function}
% int PartitionVar(int[] A, int l, int r)
\caption{Partition(A, int l, int r)}
$\text{pivot} = A[r]$\;
$i = l$\;
\For{$j = l$ \KwTo $r - 1$}{
  \If{$A[j] \leq \text{pivot}$}{
    $\text{Swap}(A, i, j)$\;
    $i = i + l$\;
  }
}
\end{function}

\begin{function}
\caption{Swap(A, int l, int r)}
temp = A[i]\;
A[i] = A[y]\;
A[j] = temp\;
\end{function}

\end{enumerate}
\end{document}
