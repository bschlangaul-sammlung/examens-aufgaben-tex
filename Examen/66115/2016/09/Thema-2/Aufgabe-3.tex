\documentclass{bschlangaul-aufgabe}
\bLadePakete{mathe}
\begin{document}
\bAufgabenMetadaten{
  Titel = {Aufgabe 3},
  Thematik = {Registermaschinen (RAMs)},
  Referenz = 66115-2016-H.T2-A3,
  RelativerPfad = Staatsexamen/66115/2016/09/Thema-2/Aufgabe-3.tex,
  ZitatSchluessel = examen:66115:2016:09,
  BearbeitungsStand = mit Lösung,
  Korrektheit = unbekannt,
  Ueberprueft = {unbekannt},
  Stichwoerter = {Berechenbarkeit},
  EinzelpruefungsNr = 66115,
  Jahr = 2016,
  Monat = 09,
  ThemaNr = 2,
  AufgabeNr = 3,
}

Sei\index{Berechenbarkeit} \footcite{examen:66115:2016:09} $M_0, M_1,
\dots$ eine Registermaschinen (RAMs). Beantworten Sie folgende Fragen
zur Aufzählbarkeit und Entscheidbarkeit. Beweisen Sie Ihre Antwort.
\footcite[Seite 8, Aufgabe 6]{theo:ab:4}

\begin{bExkurs}[Registermaschinen (RAMs)]
Die Random Access Machine (kurz RAM) ist eine spezielle Art von
Registermaschine. Sie hat die Fähigkeit der indirekten Adressierung der
Register.

Die Random Access Machine besteht aus:

\begin{itemize}
\item einem Programm bestehend aus endlich vielen durchnummerierten
Befehlen (beginnend mit Nummer $1$)

\item einem Befehlszähler $b$

\item einem Akkumulator $c(0)$

\item und einem unendlich großen Speicher aus durchnummerierten
Speicherzellen (Registern) $c(1), c(2), c(3), \dots$
\end{itemize}

Jedes Register (einschließlich $b$ und $c(0)$) speichert eine beliebig
große natürliche Zahl.\footcite{wiki:registermaschine}
\end{bExkurs}

\begin{enumerate}

%%
% a)
%%

\item Ist folgende Menge entscheidbar?

$A = \{ x \in \mathbb{N} \,|\, x = 100 \text{ oder } M_x \text{ hält bei Eingabe
} x \}$

\begin{bAntwort}
Ja, $x \geq 100$ ist entscheidbar und aufgrund des „oder“ ist die 2.
Bedingung nur für $x < 100$ relevant. Da $x < 100$ eine endliche Menge
darstellt, kann eine endliche Liste geführt werden und ein Experte kann
für jeden Fall entscheiden, ob $M_x$ hält oder nicht, somit ist $A$
entscheidbar.
\end{bAntwort}

%%
% b)
%%

\item Ist folgende Menge entscheidbar?

$B = \{
(x, y) \in \mathbb{N} \times \mathbb{N}
\,| \,
M_x
\text{ hält bei Eingabe }
x
\text{ genau dann, wenn }
M_y
\text{ bei Eingabe }
y
\text{ hält}
\}$

\begin{bAntwort}
Nein. Dieses Problem entspricht der parallelen Ausführung des
Halteproblems auf zwei Bändern. Das Halteproblem ist unentscheidbar,
damit ist auch die parallele Ausfürhung des Halteproblems und damit $B$
unentscheidbar.
\end{bAntwort}

%%
% c)
%%

\item Ist folgende Menge aufzählbar?

$C = \{
x \in \mathbb{N}
\,|\,
M_x
\text{ hält bei Eingabe }
0
\text{ mit dem Ergebnis }
1
\}$

\begin{bAntwort}
Ja, die Menge ist aufzählbar, da die Menge aller Turningmaschinen
aufzählbar und über natürliche Zahlen definiert ist (die wiederum
aufzählbar sind).
\end{bAntwort}
\end{enumerate}

\end{document}
