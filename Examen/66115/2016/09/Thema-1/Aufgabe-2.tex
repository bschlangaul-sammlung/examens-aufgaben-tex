\documentclass{bschlangaul-aufgabe}
\bLadePakete{formale-sprachen}
\begin{document}
\bAufgabenMetadaten{
  Titel = {Aufgabe 2},
  Thematik = {Chomsky-Hierarchie},
  Referenz = 66115-2016-H.T1-A2,
  RelativerPfad = Staatsexamen/66115/2016/09/Thema-1/Aufgabe-2.tex,
  ZitatSchluessel = examen:66115:2016:09,
  BearbeitungsStand = mit Lösung,
  Korrektheit = unbekannt,
  Ueberprueft = {unbekannt},
  Stichwoerter = {Formale Sprachen},
  EinzelpruefungsNr = 66115,
  Jahr = 2016,
  Monat = 09,
  ThemaNr = 1,
  AufgabeNr = 2,
}

Ordnen\index{Formale Sprachen} \footcite{examen:66115:2016:09} Sie die
folgenden Sprachen über \bAlphabet{a, b} bestmöglich in die
Chomsky-Hierarchie ein und geben Sie jeweils eine kurze Begründung (1-2
Sätze).\footcite[Aufgabe 8]{theo:ab:5}

\begin{enumerate}

%%
% (a)
%%

\item \bAusdruck[L_1]{a^n b^n}{n \geq 1}

\begin{bAntwort}
Typ-2-Sprache: Die Sprache $L_1$ ist kontextfrei, denn die Sprache
braucht einen Speicher, da sie sich die Anzahl der $a$’s merken muss, um
die gleiche Anzahl an $b$’s produzieren zu können. Dies ist mit einem
Kellerautomaten möglich. Eine Grammtik der Sprache ist
\bGrammatik{variablen={S},produktionen={S -> aSb | EPSILON}}
\end{bAntwort}

%%
% (b)
%%

\item \bAusdruck[L_2]{a^n b^n} {\text{die Turingmaschine mit
Gödelnummer } n \text{ hält auf leerer Eingabe}}

\begin{bAntwort}
Typ-0-Sprache: Die Sprache hat eine Typ-0-Grammatik, da sie
offensichtlich semientscheidbar, aber nicht entscheidbar ist.
\end{bAntwort}

%%
% (c)
%%

\item $L_3 = Σ^* \setminus L_1$

\begin{bAntwort}
Typ-2-Sprache: Die Sprache $L_3$ ist kontextfrei, da ein PDA existiert,
der nicht aktzeptiert, wenn er $L_1$ aktzeptiert. (Ausgänge umgepolt)
\end{bAntwort}

%%
% (d)
%%

\item $L_4 = Σ^* \setminus L_2$

\begin{bAntwort}
Nicht in der Hierachie: Das Komplement einer semi- aber unentscheidbaren
Sprache kann nicht semi-entscheidbar sein, da L sonst entscheidbar wäre.
\end{bAntwort}

%%
% (e)
%%

\item \bAusdruck[L_5]{a^n b^m}
{n + m \text{ ist ein Vielfaches von drei}}

\begin{bAntwort}
Typ-3-Sprache: regulär, 2 Teilautomaten mit je 3 Zuständen (modulo 2
mal)
\end{bAntwort}

%%
% (f)
%%

\item \bAusdruck[L_6]{a^n b^n}{n\text{ Quadratzahl}}

\begin{bAntwort}
nicht regulär, nicht kontextfrei (Pumping-Lemma)
\end{bAntwort}

\end{enumerate}
\end{document}
