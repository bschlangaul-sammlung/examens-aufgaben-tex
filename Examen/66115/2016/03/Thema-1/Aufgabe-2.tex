\documentclass{bschlangaul-aufgabe}
\bLadePakete{formale-sprachen,chomsky-normalform}
\begin{document}
\bAufgabenMetadaten{
  Titel = {Aufgabe 2},
  Thematik = {Nonterminale: SA, Terminale: 012},
  Referenz = 66115-2016-F.T1-A2,
  RelativerPfad = Staatsexamen/66115/2016/03/Thema-1/Aufgabe-2.tex,
  ZitatSchluessel = theo:ab:5,
  ZitatBeschreibung = {Aufgabe 2e)},
  BearbeitungsStand = mit Lösung,
  Korrektheit = unbekannt,
  Ueberprueft = {unbekannt},
  Stichwoerter = {Chomsky-Normalform},
  EinzelpruefungsNr = 66115,
  Jahr = 2016,
  Monat = 03,
  ThemaNr = 1,
  AufgabeNr = 2,
}

\let\schrittE=\bChomskyUeberErklaerung

Betrachten Sie die folgende Grammatik \bGrammatik{variablen={S, A},
alphabet={0, 1, 2}} mit\footcite[Aufgabe 2e)]{theo:ab:5}
\index{Chomsky-Normalform}
\footcite{examen:66115:2016:03}

\begin{bProduktionsRegeln}
S -> 0 S 0 | 1 S 1 | 2 A 2 | 0 | 1 | EPSILON,
A -> A 2
\end{bProduktionsRegeln}
\bFlaci{Gf6scqja9}

\noindent
Konstruieren Sie für die Grammatik $G$ schrittweise eine äquivalente
Grammatik in Chomsky-Normalform. Geben Sie für jeden einzelnen Schritt
des Verfahrens das vollständige Zwischenergebnis an und erklären Sie
kurz, was in dem Schritt getan wurde.

\begin{bAntwort}
Die Regeln \bProduktionen{S -> 2 A 2} und  \bProduktionen{A -> A 2}
können gelöscht werden, da es keine Regel \bProduktionen{A -> EPSILON}
oder \bProduktionen{A -> S} gibt. So erhalten wir:

\begin{bProduktionsRegeln}
S -> 0 S 0 | 1 S 1 | 0 | 1 | EPSILON
\end{bProduktionsRegeln}

\begin{enumerate}

\item \schrittE{1}

falls $S \rightarrow \varepsilon \in P$ neuen Startzustand $S_1$
einführen

\begin{bProduktionsRegeln}
S -> 0 S 0 | 1 S 1 | 0 | 1 | 0 0 | 1 1,
S_1 -> EPSILON | S
\end{bProduktionsRegeln}

\item \schrittE{2}

\bNichtsZuTun

\item \schrittE{3}

N = Null
E = Eins

\begin{bProduktionsRegeln}
S -> N S N | E S E | 0 | 1 | N N | E E,
S_1 -> EPSILON | S,
A -> A Z,
N -> 0,
E -> 1,
\end{bProduktionsRegeln}

\item \schrittE{4}

% S.S -> S | EPSILON
% S   -> T1 S.1 | T2 S.2 | 0 | 1 | T1 T1 | T2 T2
% T1  -> 0
% T2  -> 1
% S.1 -> S T1
% S.2 -> S T2

\begin{bProduktionsRegeln}
S -> N S_N | E S_E | 0 | 1 | N N | E E, % T1 S.1 | T2 S.2 | 0 | 1 | T1 T1 | T2 T2
S_1 -> EPSILON | S, % S.S -> S | EPSILON
S_N -> S N, % S.1 -> S T1
S_E -> S E, % S.2 -> S T2
N -> 0, % T1  -> 0
E -> 1, % T2  -> 1
\end{bProduktionsRegeln}

\end{enumerate}
\end{bAntwort}

\end{document}
