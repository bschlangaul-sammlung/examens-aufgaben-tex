\documentclass{bschlangaul-aufgabe}
\bLadePakete{syntax,automaten}
\begin{document}
\bAufgabenMetadaten{
  Titel = {Aufgabe 4},
  Thematik = {a n b n},
  Referenz = 66115-2016-F.T1-A4,
  RelativerPfad = Staatsexamen/66115/2016/03/Thema-1/Aufgabe-4.tex,
  BearbeitungsStand = mit Lösung,
  Korrektheit = unbekannt,
  Ueberprueft = {unbekannt},
  EinzelpruefungsNr = 66115,
  Jahr = 2016,
  Monat = 03,
  ThemaNr = 1,
  AufgabeNr = 4,
}

\let\l=\bTuringLeerzeichen

\begin{enumerate}

%%
% a)
%%

\item Geben Sie eine deterministische 2-Band Turingmaschine $M$ an, die
die Funktion

\begin{displaymath}
f_M(a^n) = a^n b^n
\end{displaymath}

berechnet. Die Maschine $M$ nimmt somit immer einen String der Form
$a^n$ (ein String, der aus $n$ $a$’s für beliebiges $n \in \mathbb{N}$
besteht) als Eingabe und produziert anschließend auf Band 2 als Ausgabe
den String $a^n b^n$ (ein String aus $n$ $a$’s gefolgt von $n$ $b$’s).

Beschreiben Sie außerdem die Idee hinter Ihrer Konstruktion.

\begin{bAntwort}
\begin{minted}{md}
name: 66115 2016 03 1 4
init: z0
accept: z2

z0, a,_
z0, a,a, >,>

z0, _,_
z1, _,_, <,-

z1, a,_
z1, a,_, <,-

z1, _,_
z2, _,_, >,-

z2, a,_
z2, a,b, >,>
\end{minted}
\bFussnoteUrl{http://turingmachinesimulator.com/shared/lyptczerhe}

\begin{tabular}{c|l}
$z_0$ & $a$’s auf das 2. Band kopieren \\
$z_1$ & Zu Beginn der Eingabe auf dem 1. Band, 2. Band bleibt\\
$z_2$ & Für jedes $a$’s auf dem 1. Band ein $b$ auf dem 2. Band erzeugen\\
\end{tabular}
\end{bAntwort}

%%
% b)
%%
\item Geben Sie die Konfigurationsfolge der Turingmaschine aus (a) für
die Eingabe $aa$ an.

\begin{bAntwort}
% 0
z0 a a,
z0 \l \l

% 1
a z0 a,
a z0 \l

% 2
a a z0 \l,
a a z0 \l

% 3
a z1 a,
a a z1 \l

% 4
z1 a a \l,
a a z1 \l

% 5
z1 \l a a,
a a z1 \l

% 6
z2 a a,
a a z2 \l

% 7
a z2 a,
a a b z2 \l

% 8
a a z2 \l,
a a b b z2 \l
\end{bAntwort}

\end{enumerate}
\end{document}
