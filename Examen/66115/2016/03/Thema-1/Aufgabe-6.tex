\documentclass{bschlangaul-aufgabe}

\begin{document}
\bAufgabenMetadaten{
  Titel = {Aufgabe 6},
  Thematik = {1 45 8 53 9 2 17 10},
  Referenz = 66115-2016-F.T1-A6,
  RelativerPfad = Staatsexamen/66115/2016/03/Thema-1/Aufgabe-6.tex,
  ZitatSchluessel = examen:66115:2016:03,
  BearbeitungsStand = mit Lösung,
  Korrektheit = unbekannt,
  Ueberprueft = {unbekannt},
  Stichwoerter = {Quicksort},
  EinzelpruefungsNr = 66115,
  Jahr = 2016,
  Monat = 03,
  ThemaNr = 1,
  AufgabeNr = 6,
}

Sortieren Sie die Werte
\index{Quicksort}
\footcite{examen:66115:2016:03}

\begin{center}
1 45 8 53 9 2 17 10
\end{center}

mit Quicksort.

% sortiere 1 45 8 53 9 2 17 10 -q
\begin{bAntwort}
Sortieralgorithmus nach Saake
\begin{verbatim}
 1   45  8   53  9   2   17  10  zerlege
 1   45  8   53* 9   2   17  10  markiere (i 3)
 1   45  8  >53  9   2   17  10< vertausche (i 3<>7)
>1<  45  8   10  9   2   17  53  vertausche (i 0<>0)
 1  >45< 8   10  9   2   17  53  vertausche (i 1<>1)
 1   45 >8<  10  9   2   17  53  vertausche (i 2<>2)
 1   45  8  >10< 9   2   17  53  vertausche (i 3<>3)
 1   45  8   10 >9<  2   17  53  vertausche (i 4<>4)
 1   45  8   10  9  >2<  17  53  vertausche (i 5<>5)
 1   45  8   10  9   2  >17< 53  vertausche (i 6<>6)
 1   45  8   10  9   2   17 >53< vertausche (i 7<>7)
 1   45  8   10  9   2   17      zerlege
 1   45  8   10* 9   2   17      markiere (i 3)
 1   45  8  >10  9   2   17<     vertausche (i 3<>6)
>1<  45  8   17  9   2   10      vertausche (i 0<>0)
 1  >45  8<  17  9   2   10      vertausche (i 1<>2)
 1   8  >45  17  9<  2   10      vertausche (i 2<>4)
 1   8   9  >17  45  2<  10      vertausche (i 3<>5)
 1   8   9   2  >45  17  10<     vertausche (i 4<>6)
 1   8   9   2                   zerlege
 1   8*  9   2                   markiere (i 1)
 1  >8   9   2<                  vertausche (i 1<>3)
>1<  2   9   8                   vertausche (i 0<>0)
 1  >2<  9   8                   vertausche (i 1<>1)
 1   2  >9   8<                  vertausche (i 2<>3)
 1   2                           zerlege
 1*  2                           markiere (i 0)
>1   2<                          vertausche (i 0<>1)
>2   1<                          vertausche (i 0<>1)
                     17  45      zerlege
                     17* 45      markiere (i 5)
                    >17  45<     vertausche (i 5<>6)
                    >45  17<     vertausche (i 5<>6)
\end{verbatim}

Sortieralgorithmus nach Horare
% sortiere 1 45 8 53 9 2 17 10 -Q
\begin{verbatim}
 1   45  8   53  9   2   17  10  zerlege
 1   45  8   53* 9   2   17  10  markiere (i 3)
 1   45  8  >53  9   2   17  10< vertausche (i 3<>7)
 1   45  8   10  9   2   17      zerlege
 1   45  8   10* 9   2   17      markiere (i 3)
 1  >45  8   10  9   2<  17      vertausche (i 1<>5)
 1   2   8  >10  9<  45  17      vertausche (i 3<>4)
 1   2   8   9                   zerlege
 1   2*  8   9                   markiere (i 1)
 1   2                           zerlege
 1*  2                           markiere (i 0)
         8   9                   zerlege
         8*  9                   markiere (i 2)
                 10  45  17      zerlege
                 10  45* 17      markiere (i 5)
                 10 >45  17<     vertausche (i 5<>6)
                 10  17          zerlege
                 10* 17          markiere (i 4)
\end{verbatim}
\end{bAntwort}
\end{document}
