\documentclass{bschlangaul-aufgabe}
\bLadePakete{komplexitaetstheorie}
\begin{document}
\bAufgabenMetadaten{
  Titel = {Komplexität},
  Thematik = {k-COL},
  Referenz = 66115-2016-F.T1-A5,
  RelativerPfad = Staatsexamen/66115/2016/03/Thema-1/Aufgabe-5.tex,
  ZitatSchluessel = examen:66115:2016:03,
  BearbeitungsStand = mit Lösung,
  Korrektheit = unbekannt,
  Ueberprueft = {unbekannt},
  Stichwoerter = {Polynomialzeitreduktion},
  EinzelpruefungsNr = 66115,
  Jahr = 2016,
  Monat = 03,
  ThemaNr = 1,
  AufgabeNr = 5,
}

\let\n=\bProblemName
\let\r=\bPolynomiellReduzierbar

% Info_2021-06-11-2021-06-11_09.38.05.mp4
% 45min

Das Problem k-COL ist wie folgt definiert:
\index{Polynomialzeitreduktion}
\footcite{examen:66115:2016:03}

\bProblemBeschreibung
{k-Col}
{Ein ungerichteter Graph $G = (V, E)$.}
{Kann man jedem Knoten $v$ in $V$ eine Zahl $z(v) \in \{1, \dots ,k\}$
zuordnen, so dass für alle Kanten $(u_1,u_2) \in E$ gilt: $z(u_1) \neq
z(u_2)?$}

\noindent
Zeigen Sie, dass man \n{3-Col} in polynomieller Zeit auf \n{4-Col}
reduzieren kann. Beschreiben Sie dazu die Reduktion und zeigen Sie
anschließend ihre Korrektheit.

\begin{bAntwort}
Zu Zeigen: \bPolynomiellReduzierbar{3-COL}{4-Col}

\noindent
also \n{4-Col} ist mindestens so schwer wie \n{3-COL} Eingabeinstanz von
\n{3-COL} durch eine Funktion in eine Eingabeinstanz von \n{4-Col}
umbauen so, dass jede JA- bzw. NEIN-Instanz von \n{3-COL} eine JA- bzw.
NEIN-Instanz von \n{4-Col} ist.
\footcite[Seite 67]{theo:fs:4}

Funktion ergänzt einen beliebigen gegebenen Graphen um einen
weiteren Knoten, der mit allen Knoten des ursprünglichen Graphen
durch eine Kante verbunden ist.

\begin{description}
\item[total]

ja

\item[in Polynomialzeit berechenbar]

ja
(Begründung: \zB Adjazenzmatrix $\rightarrow$ neue Spalte)

\item[Korrektheit:]

ja

Färbe den „neuen“ Knoten mit einer Farbe. Da er mit allen anderen Knoten
verbunden ist, bleiben für die übrigen Knoten nur drei Farben.
\footcite[Seite 69]{theo:fs:4}
\end{description}
\end{bAntwort}
\end{document}
