\documentclass{bschlangaul-aufgabe}
\bLadePakete{automaten,formale-sprachen}
\begin{document}
\bAufgabenMetadaten{
  Titel = {Aufgabe 1},
  Thematik = {Reguläre Sprachen},
  Referenz = 66115-2016-F.T1-A1,
  RelativerPfad = Staatsexamen/66115/2016/03/Thema-1/Aufgabe-1.tex,
  ZitatSchluessel = examen:66115:2016:03,
  BearbeitungsStand = mit Lösung,
  Korrektheit = unbekannt,
  Ueberprueft = {unbekannt},
  Stichwoerter = {Reguläre Sprache},
  EinzelpruefungsNr = 66115,
  Jahr = 2016,
  Monat = 03,
  ThemaNr = 1,
  AufgabeNr = 1,
}

\def\Z#1#2{$q_#1z_#2$}
\def\q#1{$q_#1$}
\def\z#1{$z_#1$}
\let\m=\bMenge

\begin{enumerate}

%%
% a)
%%

\item Geben\index{Reguläre Sprache} \footcite{examen:66115:2016:03} Sie
einen möglichst einfachen regulären Ausdruck für die Sprache
\bAusdruck[L_1] {a_1,a_2,\dots,a_n} {n \geq 3, a_i \in \bMenge{a, b}
\text{ für alle } i = 1,\dots,n \text{ und } a_1 \geq a_n} an.

\begin{bAntwort}
\texttt{((a(a|b)+b)|(b(a|b)+a))}
\end{bAntwort}

%%
% b)
%%

\item Geben Sie einen möglichst einfachen regulären Ausdruck für die
Sprache
\bAusdruck[L_2]
{w \in \m{a,b}^*}
{
  w
  \text{ enthält genau ein }
  b
  \text{ und ist von ungerader Länge}
}
an.

\begin{bAntwort}
\texttt{(aa)*(b|aba)(aa)*}
\end{bAntwort}

%%
% c)
%%

\item Beschreiben Sie die Sprache des folgenden Automaten $A_1$,
möglichst einfach und präzise in ihren eigenen Worten.

\begin{center}
\begin{tikzpicture}[li automat]
  \node[state,,initial] (q0) at (2.14cm,-2.14cm) {$q_0$};
  \node[state,] (q1) at (4cm,-2.14cm) {$q_1$};
  \node[state,] (q2) at (6cm,-2.14cm) {$q_2$};
  \node[state,,accepting] (q3) at (7.57cm,-2.14cm) {$q_3$};

  \path (q0) edge[auto,bend left] node{a} (q1);
  \path (q0) edge[auto,loop] node{b} (q0);
  \path (q1) edge[auto] node{b} (q2);
  \path (q1) edge[auto,bend left] node{a} (q0);
  \path (q2) edge[auto] node{a} (q3);
  \path (q2) edge[auto,bend left] node{b} (q0);
  \path (q3) edge[auto,loop] node{a,b} (q3);
\end{tikzpicture}
\end{center}
\bFlaci{Arz003ccg}

\begin{bAntwort}
Die Sprache enthält das Teilwort $aba$
\end{bAntwort}

%%
% d)
%%

\item Betrachten Sie folgenden Automaten $A_2$:

\begin{center}
\begin{tikzpicture}[li automat]
  \node[state,initial] (z0) at (2.14cm,-2.14cm) {$z_0$};
  \node[state,accepting] (z1) at (4.86cm,-2.14cm) {$z_1$};

  \path (z0) edge[auto,bend left] node{a,b} (z1);
  \path (z1) edge[auto,bend left] node{a,b} (z0);
\end{tikzpicture}
\end{center}
\bFlaci{Ap9qbkumc}

Im Original sind die Zustände mit $q_x$ benannt. Damit wir die
Schnittmenge besser bilden können, wird hier $z_x$ verwendet.

Konstruieren Sie einen endlichen Automaten, der die Schnittmenge der
Sprachen $L(A_1)$ und $L(A_2)$ akzeptiert.

\begin{bAntwort}

$A_1$

\begin{tabular}{l|l|l}
  & a & b \\\hline
\q0 & \q1 & \q0 \\
\q1 & \q0 & \q2 \\
\q2 & \q3 & \q0 \\
\q3 & \q3 & \q3 \\
\end{tabular}

$A_2$

\begin{tabular}{l|l|l}
  & a & b \\\hline
\z0 & \z1 & \z1 \\
\z1 & \z0 & \z0 \\
\end{tabular}

Neuer Endzustand:  \Z31

\begin{tabular}{l|l|l}
     & a    & b    \\\hline
 \Z00 &  \Z11 &  \Z01 \\
 \Z10 &  \Z01 &  \Z21 \\
 \Z20 &  \Z31 &  \Z01 \\
 \Z30 &  \Z31 &  \Z31 \\
 \Z01 &  \Z10 &  \Z00 \\
 \Z11 &  \Z00 &  \Z20 \\
 \Z21 &  \Z30 &  \Z00 \\
 \Z31 &  \Z30 &  \Z30 \\
\end{tabular}

\begin{center}
\begin{tikzpicture}[li automat]
  \node[state,initial] (q0z0) at (2.14cm,-2.14cm) {\Z00};
  \node[state] (q1z0) at (2.14cm,-8.43cm) {\Z10};
  \node[state] (q2z0) at (5.29cm,-8.43cm) {\Z20};
  \node[state] (q3z0) at (7.43cm,-2.14cm) {\Z30};
  \node[state] (q0z1) at (2.14cm,-5.29cm) {\Z01};
  \node[state] (q1z1) at (5.43cm,-4.29cm) {\Z11};
  \node[state] (q2z1) at (5.29cm,-2.14cm) {\Z21};
  \node[state,accepting] (q3z1) at (8.86cm,-5.43cm) {\Z31};

  \path (q0z0) edge[auto,bend left=10] node{a} (q1z1);
  \path (q0z0) edge[auto,bend left] node{b} (q0z1);
  \path (q1z0) edge[auto,bend left] node{a} (q0z1);
  \path (q1z0) edge[auto] node{b} (q2z0);
  \path (q2z0) edge[auto] node{a} (q3z1);
  \path (q2z0) edge[auto] node{b} (q0z1);
  \path (q3z0) edge[auto,bend left] node{a,b} (q3z1);
  \path (q0z1) edge[auto,bend left] node{b} (q0z0);
  \path (q0z1) edge[auto,bend left] node{a} (q1z0);
  \path (q1z1) edge[auto,bend left=10] node{a} (q0z0);
  \path (q1z1) edge[auto] node{b} (q2z0);
  \path (q2z1) edge[auto] node{a} (q3z0);
  \path (q2z1) edge[auto] node{b} (q0z0);
  \path (q3z1) edge[auto,bend left] node{a,b} (q3z0);
\end{tikzpicture}
\end{center}

\bFlaci{Ar3pc5rh7}

\end{bAntwort}
\end{enumerate}

\end{document}
