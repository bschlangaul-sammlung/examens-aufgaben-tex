\documentclass{bschlangaul-basis}

\begin{document}
\section{Vorwort}

%-----------------------------------------------------------------------
%
%-----------------------------------------------------------------------

\subsection{Abkürzungen der Modulnamen}

\begin{tabular}{|l|l|}
Abk.  & Modulename                                        \\\hline
MATHE & Mathematische Grundlagen                          \\
DB    & Datenbanksysteme                                  \\
OOMUP & Objektorientierte Modellierung und Programmierung \\
AUD   & Algorithmen und Datenstrukturen                   \\
FUMUP & Funktionale Modellierung und Programmierung       \\
SOSY  & Softwaresysteme und Softwareentwicklungspraktikum \\
TECH  & Technische Informatik                             \\
THEO  & Theoretische Informatik                           \\
DDI   & Didaktik der Informatik                           \\
\end{tabular}

%-----------------------------------------------------------------------
%
%-----------------------------------------------------------------------

\subsection{Repositories}

Das Projekt ist auf mehrere Git-Repositories aufgeteilt. Alle Repos
sind unter der Github-Organisation
\url{https://github.com/bschlangaul-sammlung} zusammengefasst.

\def\TmpRepo#1#2{\item[#1] \strut \par

{\footnotesize \url{https://github.com/bschlangaul-sammlung/#1}}

\par #2

}

\begin{description}
\TmpRepo{examens-aufgaben-tex}
{Die LaTeX-Quelltexte aller Übungs- und Examensaufgaben.
Haupt-Repository des Projekts. Sammlung von Examensaufgaben und
weiteren, zusätzlichen Übungsaufgaben mit Lösungen für das Studium
„Lehramt Informatik“ in Bayern.}

\TmpRepo{examens-aufgaben-pdf}
{Die aus den LaTeX-Quellentexten kompilierten PDFs der Aufgabensammlung.}

\TmpRepo{examen-scans}
{Scans der Staatsexamensaufgaben als PDF- und als Textdateien.}

\TmpRepo{latex-vorlagen}
{LaTeX- bzw. LuaLaTeX-Pakete und -Klassen zum Setzen der Aufgaben.}

\TmpRepo{java-fuer-examens-aufgaben}
{Java-Code zum Einbetten in die LaTeX-Quelltexte der Aufgaben,
Implementation von einigen Datenstrukturen (z. B. AVL-Baum),
Kommandozeilen-Werkzeug}

\TmpRepo{kommandozeilen-werkzeug}
{Ein Kommandozeilen-Tool (Werkzeug) geschrieben in Typescript
(Javascript), um verschiedene administrative Aufgaben, wie z. B. das
Erzeugen von Aufgaben-Sammlungen, TeX-Vorlagen etc. zu erledigen.}

\TmpRepo{logo-grafiken}
{Logo und sonstige Grafiken}
\end{description}

%-----------------------------------------------------------------------
%
%-----------------------------------------------------------------------

\subsection{Lizenzen}

Das Bschlangaul-Projekt steht unter freien Lizenzen:

\begin{description}

\item[CC Attribution-NonCommercial-ShareAlike 4.0 International]

examens-aufgaben-tex,
examens-aufgaben-pdf,
logo-grafiken

\item[GNU General Public License v3.0]

java-fuer-examens-aufgaben,
kommandozeilen-werkzeug

\item[LPPL Version 1.3c]

latex-vorlagen
\end{description}

\end{document}
