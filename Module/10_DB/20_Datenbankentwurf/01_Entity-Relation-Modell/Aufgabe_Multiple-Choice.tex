\documentclass{bschlangaul-aufgabe}

\begin{document}
\bAufgabenMetadaten{
  Titel = {Aufgabe 5: Entity-Relationship-Modell},
  Thematik = {Multiple-Choice},
  Referenz = DB.Datenbankentwurf.Entity-Relation-Modell.Multiple-Choice,
  RelativerPfad = Module/10_DB/20_Datenbankentwurf/01_Entity-Relation-Modell/Aufgabe_Multiple-Choice.tex,
  ZitatSchluessel = db:ab:7,
  ZitatBeschreibung = {Aufgabe 5},
  BearbeitungsStand = nur Angabe,
  Korrektheit = unbekannt,
  Ueberprueft = {unbekannt},
  Stichwoerter = {Entity-Relation-Modell},
}

Multiple-Choice-Online-Tests für Nachqualifizierungskurse sollen
datenbankgestützt erfolgen. Dabei liegt folgende Situation vor:
\index{Entity-Relation-Modell}
\footcite[Aufgabe 5]{db:ab:7}

Ein Kurs (Kursname, verantwortliche Universität), der aus höchstens 30
Teilnehmern (Teilnehmernummer, Name, Passwort) besteht, behandelt
mindestens ein Themenmodul. Ein Themenmodul kann aber in mehreren
verschiedenen Kursen behandelt werden. Ein Teilnehmer besucht genau
einen Kurs. Jeder Kurs wird von genau einem Tutor betreut, vom dem
Personalnummer, Name und Spezialgebiet abgespeichert werden sollen. Ein
Tutor kann maximal zwei Kurse parallel betreuen. Jedes Themenmodul ist
eindeutig durch seine Modulnummer charakterisiert. Von jedem Modul soll
der Modultitel abgespeichert werden. Die Kursteilnehmer können
Multiple-ChoiceTests durchführen. Es soll gespeichert werden, welcher
Test von welchem Teilnehmer unter Angabe des Datums bereits bearbeitet
wurde. Die Bearbeitung eines bestimmten Tests ist aber pro Person nur
einmal möglich. Außerdem will der Tutor wissen, wie viele Tests ein
Teilnehmer bereits durchgeführt hat. Ein Test hat eine Nummer und eine
kurze Beschreibung und gehört zu einem eindeutig bestimmten Themenmodul.
Jeder Test besteht aus einer Menge von Aufgaben. Eine Aufgabe besteht
dabei aus einer Frage und möglichen Antworten. Ein Test kann eine
beliebige Anzahl von Aufgaben haben und jede Aufgabe kann eine beliebige
Zahl von möglichen Antworten haben, von denen aber mindestens eine
richtig sein muss. Bei einer Antwort muss natürlich vermerkt sein, ob
sie richtig oder falsch ist. Jede Aufgabe gehört zu genau einem Test und
jede Antwort genau zu einer Aufgabe. Bei der Bearbeitung eines Tests
soll gespeichert werden, welche Antworten von einem Prüfling angekreuzt
werden, um daraus das Testergebnis berechnen zu können.

Erstellen Sie ein ER-Modell! Verarbeiten Sie dabei nur die unbedingt
notwendigen Informationen. Geben Sie aber an, wie die nicht in Ihrem
ER-Modell auftauchenden Informationen bestimmt werden können. Für im
Text nicht näher spezifizierte Funktionalitäten besteht keine
Einschränkung. Formulieren Sie eventuell auftauchende
Integritätsbedingungen, die nicht in das Modell eingebaut werden
können, extra und geben Sie jeweils eine Möglichkeit zur Gewährleistung
dieser Bedingung an.
\end{document}
