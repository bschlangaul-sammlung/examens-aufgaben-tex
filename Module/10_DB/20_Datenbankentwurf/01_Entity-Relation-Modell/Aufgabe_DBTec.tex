\documentclass{bschlangaul-aufgabe}
\bLadePakete{er}
\begin{document}
\bAufgabenMetadaten{
  Titel = {Aufgabe 3: DBTec},
  Thematik = {DBTec},
  Referenz = DB.Datenbankentwurf.Entity-Relation-Modell.DBTec,
  RelativerPfad = Module/10_DB/20_Datenbankentwurf/01_Entity-Relation-Modell/Aufgabe_DBTec.tex,
  ZitatSchluessel = db:ab:1,
  BearbeitungsStand = mit Lösung,
  Korrektheit = unbekannt,
  Ueberprueft = {unbekannt},
  Stichwoerter = {Entity-Relation-Modell},
}

\let\a=\bErMpAttribute
\let\d=\bErDatenbankName
\let\e=\bErMpEntity
\let\r=\bErMpRelationship

Die Firma \emph{DBTec} fertigt verschiedene Geräte. Für die betriebliche
Organisation dieser Firma soll eine relationale Datenbank eingesetzt
werden. Dabei gilt folgendes:\index{Entity-Relation-Modell}
\footcite{db:ab:1}
% Entity: Bauteil
Jedes \e{Bauteil}, das verwendet wird, hat eine eindeutige \a{Nummer}
und eine \a{Bezeichnung}, die allerdings für mehrere verschiedene
Bauteile gleich sein kann. Von jedem Teil werden außerdem der \a{Name
des Herstellers}, der \a{Einkaufspreis} pro Stück und der am Lager
vorhandene \a{Vorrat} gespeichert.

% Entity: Gerät
Jedes herzustellende \e{Gerät} hat eine eindeutige \a{Bezeichnung}. Auch
von jedem schon gefertigten Gerätetyp soll der aktuelle \a{Lagerbestand}
gespeichert werden, ebenso wie der \a{Verkaufspreis} des Gerätes. In
unserem fiktiven Betrieb gilt die Regelung, dass Maschinen, die mehr als
1000,- EUR kosten, unentgeltlich an die Kunden ausgeliefert werden; für
Geräte, die weniger kosten, ist zusätzlich zum Preis eine
gerätespezifische \a{Anliefergebühr} zu entrichten. In der Datenbank ist
ebenfalls zu speichern, welche Bauteile für welche Geräte \r{benötigt}
werden. Es gibt Bauteile, die für mehrere Geräte verwendet werden.

% Entity: Kunde
Von jedem \e{Kunden} werden der \a{Name}, die \a{Adresse} und die
\a{Branche} gespeichert. Es kann verschiedene Kunden mit demselben Namen
oder derselben Adresse geben.
% Relationship: betreutKunde
Außerdem ist zu jedem Kunden vermerkt, wer aus unserer Firma für die
entsprechende \r{Kundenbetreuung zuständig} ist.
% Relationship: beliefert
Natürlich ist auch zu speichern, welche Kunden mit welchen Geräten
\r{beliefert} werden. Es kann sein, dass gewissen Kunden
für bestimmte Geräte \a{Sonderkonditionen} eingeräumt worden
sind, dies soll ggf. ebenfalls in der Datenbank vermerkt werden.

\begin{enumerate}

%%
% (a)
%%

\item Bestimmen Sie die Entity- und die Relationship-Typen mit ihren
Attributen und zeichnen Sie ein mögliches Entity-Relationship-Diagramm!

%%
% (b)
%%

\item Bestimmen Sie zu allen Entity-Typen einen Primärschlüssel und
tragen Sie diese in das Modell ein.

%%
% (c)
%%

\item Bestimmen Sie die Funktionalitäten (1:1, 1:n, n:m) der
Relationship-Typen und tragen Sie diese in das Modell ein.

%%
% (d)
%%

\item In der Firma wird ein neues Betreuungssystem eingeführt. Jeder
\e{Kundenbetreuer} ist für die Kunden eines festgelegten Bezirks
\r{zuständig}. Die \e{Bezirke} sind
\a{durchnummeriert}. Für jeden Bezirk existiert eine
\a{Beschreibung}, die nicht näher festgelegt ist. Erweitern
Sie Ihr ER-Modell aus Teilaufgabe a) entsprechend. Bezirke werden nur
festgelegt, wenn es dazu auch Kunden gibt.

\end{enumerate}

\end{document}
