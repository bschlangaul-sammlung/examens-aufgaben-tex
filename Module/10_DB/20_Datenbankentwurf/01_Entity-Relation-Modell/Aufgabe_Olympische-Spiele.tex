\documentclass{bschlangaul-aufgabe}
\bLadePakete{er}
\begin{document}
\bAufgabenMetadaten{
  Titel = {Aufgabe 3: Olympische Spiele},
  Thematik = {Olympische Spiele},
  Referenz = DB.Datenbankentwurf.Entity-Relation-Modell.Olympische-Spiele,
  RelativerPfad = Module/10_DB/20_Datenbankentwurf/01_Entity-Relation-Modell/Aufgabe_Olympische-Spiele.tex,
  ZitatSchluessel = db:ab:4,
  ZitatBeschreibung = {Aufgabe 3},
  BearbeitungsStand = nur Angabe,
  Korrektheit = unbekannt,
  Ueberprueft = {unbekannt},
  Stichwoerter = {Entity-Relation-Modell},
}

\let\a=\bErMpAttribute
\let\d=\bErDatenbankName
\let\e=\bErMpEntity
\let\r=\bErMpRelationship

Geben Sie ein Entity-Relationship-Diagramm (mit Schlüssel und
Funktionalität) für folgendes Problem an:\index{Entity-Relation-Modell}
\footcite[Aufgabe 3]{db:ab:4}

\bigskip

Eine europäische Fachzeitschrift für Leichtathletik möchte in einer
relationalen Datenbank die folgenden Informationen zu den Olympischen
Sommerspielen speichern:

Für jeden \e{Sportler}, der bei den Olympischen Spielen teilgenommen
hat, sollen der \a{Name}, die \a{Nationalität}, die \a{Größe} und das
\a{Gewicht} bekannt sein. Zusätzlich soll abgefragt werden können, ob
der betreffenden Sportler \a{aus Europa} stammt.

Zu den \e{Olympischen Spielen} sollen das \a{Jahr}, der
\a{Austragungsort} und die \a{Teilnehmerzahl} gespeichert werden.

Außerdem soll bekannt sein, welcher
Sportler bei welchen Spielen welchen Rekord \r{erzielt} hat.

Zu jedem \e{Rekord} sollen die \a{Disziplin} und die \a{Art} des Rekords
(Weltrekord, Europarekord,...) gespeichert werden. Darüber soll ohne
Umweg abrufbar sein, ob es sich um einen Rekord in der
\a{Leichtathletik} handelt.

Es sei vorausgesetzt, dass jeder Sportler eindeutig durch seinen Namen
identifizierbar ist und dass Olympische Spiele in verschiedenen Jahren
in derselben Stadt stattfinden können. Ein Rekord soll eindeutig durch
die Disziplin und die Art des Rekords gekennzeichnet sein.

\begin{bAntwort}
Die Frage, ob ein Sportler Europäer ist, und die Anzahl der Teilnehmer
kann über SQL-Anfragen gelöst werden. Diese Informationen brauchen
nicht explizit gespeichert zu werden. Die Feststellung, ob ein Sportler
\emph{Europäer} ist konnte man auch ohne spezielles Attribut Europäer
finden, indem man beispielsweise folgende Anfrage stellt: Gib mir alle
Sportler, für deren Nationalität gilt: Nationalität = deutsch ODER
Nationalität = französisch. Auch die Anzahl der Teilnehmer kann man ohne
eigenes Attribut Teilnehmerzahl mit Hilfe des Relationsship-Typen
\emph{nimmt teil} herausfinden. Über die Angaben bei der(min, max)
Notation kann man natürlich diskutieren. Hier wurde angenommen: Ein
Sportler kann bei Olympischen Spielen Rekorde aufstellen, muss aber
nicht. Bei einer Olympiade kann ein Rekord von mehreren Sportlern
aufgestellt werden, ein Sportler muss aber beteiligt sein. Bei
Olympischen Spielen können viele Rekorde aufgestellt werden, es muss
aber keiner aufgestellt werden. Ein Sportler kann bei mehreren
Olympischen Spielen teilnehmen. An den Olympischen Spielen nehmen
mehrere Sportler teil.
\end{bAntwort}

\end{document}
