\documentclass{bschlangaul-aufgabe}
\bLadePakete{normalformen,synthese-algorithmus}
\begin{document}
\bAufgabenMetadaten{
  Titel = {Kanonische Überdeckung (kleines Beispiel aus Kemper)},
  Thematik = {Kanonische Überdeckung (Kemper)},
  Referenz = DB.Relationale-Entwurfstheorie.Normalformen.Kanonische-Ueberdeckung-Kemper,
  RelativerPfad = Module/10_DB/50_Relationale-Entwurfstheorie/30_Normalformen/Aufgabe_Kanonische-Ueberdeckung-Kemper.tex,
  ZitatSchluessel = kemper,
  ZitatBeschreibung = {Seite 186},
  BearbeitungsStand = mit Lösung,
  Korrektheit = unbekannt,
  Ueberprueft = {unbekannt},
  Stichwoerter = {Kanonische Überdeckung},
}

\let\FA=\bFunktionaleAbhaengigkeiten
\let\fa=\bFunktionaleAbhaengigkeit
\let\m=\bAttributMenge
\let\ah=\bAttributHuelle
\let\schrittE=\bSyntheseUeberErklaerung

\FA{%
  A -> B;
  B -> C;
  A, B -> C;
}
\index{Kanonische Überdeckung}
\footcite[Seite 186]{kemper}

\begin{bAntwort}

\begin{enumerate}
\item \schrittE{1-1}

$\ah{F, \m{A, B} - \m{B}} = \m{A, B, C}$

$\ah{F, \m{A, B} - \m{A}} = \m{C}$

\FA{%
  A -> B;
  B -> C;
  A -> C;
}

\item \schrittE{1-2}

\FA{%
  A -> B;
  B -> C;
  A -> NICHTS;
}

\item \schrittE{1-3}

\FA{%
  A -> B;
  B -> C;
}

\item \schrittE{1-4}

Nichts zu tun
\end{enumerate}

\end{bAntwort}
\end{document}
