\documentclass{bschlangaul-aufgabe}
\bLadePakete{normalformen}
\begin{document}
\bAufgabenMetadaten{
  Titel = {Normalformen (Mietwagenfirma)},
  Thematik = {Mietwagenfirma},
  Referenz = DB.Relationale-Entwurfstheorie.Normalformen.Mietwagenfirma,
  RelativerPfad = Module/10_DB/50_Relationale-Entwurfstheorie/30_Normalformen/Aufgabe_Mietwagenfirma.tex,
  ZitatSchluessel = db:pu:4,
  ZitatBeschreibung = {Seite 2 Normalformen, Aufgabe 3},
  BearbeitungsStand = mit Lösung,
  Korrektheit = unbekannt,
  Ueberprueft = {unbekannt},
  Stichwoerter = {Zweite Normalform, Delete-Anomalie, Update-Anomalie, Insert-Anomalie, Dritte Normalform},
}

\let\fa=\bFunktionaleAbhaengigkeit

Gegeben sei die folgende relationale Datenbank der Mietwagenfirma
„MobilRent“, Station „München-Mitte“.\footcite[Seite 2 Normalformen, Aufgabe 3]{db:pu:4}

\begin{center}
\tiny
\begin{tabular}{lllrrllccll}
\underline{KdNr} & Name & Wohnort & \underline{Buchungsdatum} & Aktion & Fahrzeug & Typ &
Tarif & Tage & Rueckgabestation & Stationsleiter
\\\hline

123 & Chomsky & Nürnberg & 23.01.2004 & 0 & Cala & Klein &
1 & 2 & Nürnberg-Nord & Backus \\

123 & Chomsky & Nürnberg & 07.10.2003 & -25\% & Vanny & Transp &
5 & 1 & Nürnberg-Nord & Hoare \\

220 & Neumann & München & 02.04.2004 & 0 & Baro & Klein &
1 & 2 & München-Mitte & Zuse \\

710 & Turing & München & 20.02.2004 & -10\% & Cala & Klein &
1 & 2 & München-Mitte & Zuse \\

888 & Neumann & Passau & 07.10.2003 & -25\% & Lux & Mittelkl &
3 & 3 & München-Mitte & Zuse \\
\end{tabular}
\end{center}

\noindent
$KdNr$ steht für die Kundennummer der Kunden. An bestimmten Tagen
gewährt die Firma Rabatt. Folgende funktionale Abhängigkeiten seien
vorgegeben.

\bigskip

\bFunktionaleAbhaengigkeiten{
  KdNr -> Name, Wohnort;
  KdNr, Buchungsdatum -> Fahrzeug, Typ, Tarif, Tage, Rueckgabe-Station;
  Buchungsdatum -> Aktion;
  Fahrzeug -> Typ, Tarif;
  Typ -> Tarif;
  Rueckgabestation -> Stationsleiter;
}

\bigskip

\noindent
Der Primärschlüssel besteht aus den Attributen \emph{KdNr} und
\emph{Buchungsdatum}.

\begin{enumerate}

%%
% (a)
%%

\item Begründe, dass diese Tabelle in 1. Normalform vorliegt.

\begin{bAntwort}
Es gibt nur atomare Werte. Alle Attributewerte sind atomar.
\end{bAntwort}

%%
% (b)
%%

\item Erläutere, warum nur Tabellen mit zusammengesetzem Primärschlüssel
die zweite Normalform verletzen können.
\index{Zweite Normalform}

\begin{bAntwort}
Nur bei Tabellen mit zusammengesetzten Primärschlüssel kann es
vorkommen, dass ein Nichtschlüssel-Attribut von einer echten Teilmenge
des Primärschlüssel voll funktional abhängt.
\end{bAntwort}

%%
% (c)
%%

\item Zeige mögliche Anomalien auf, die hier auftreten können.

\begin{bAntwort}
\begin{description}
\item[Delete-Anomalie]

\zB Kunde \emph{888} wird gelöscht, gleichzeitig gehen alle
Informationen über das Fahrzeug \emph{A3} verloren.
\index{Delete-Anomalie}

\item[Update-Anomalie]

\zB \emph{Chomsky} zieht um, Wohnort wird nicht in allen Tupeln geändert.
\index{Update-Anomalie}

\item[Insert-Anomalie]

\zB Eine neue Rückgabestation kann erst eröffnet werden, wenn ein
Kunde ein bereits gebuchtes Auto auch an dieser Station zurückgeben
will.
\index{Insert-Anomalie}
\end{description}
\end{bAntwort}

%%
% (d)
%%

\item Überführe das Schema in die 2. Normalform.

\begin{bAntwort}
Der \emph{Name} ist voll funktional abhängig von \emph{KdNr}, nicht aber
von \emph{Buchungsdatum}. Deswegen verletzt die FD \fa{KdNr -> Name} die
2. Normalform (und also auch die 3. Normalform). \emph{Stationsleiter}
ist nur transitiv von \emph{KdNr}, \emph{Buchungsdatum} abhängig (über
\emph{Rückgabestation}). Dies verletzt die 3. Normalform. Überführung in
2NF: Entfernung der vom Primärschlüssel nicht voll funktional abhängigen
Attribute.

MobilRent(\underline{KdNr, Buchungsdatum}, Fahrzeug, Typ, Tarif, Tage,
Rückgabestation, Stationsleiter)

Kunden(\underline{Kdnr}, Name, Wohnort)

Aktion(\underline{Buchungsdatum}, Aktion)
\end{bAntwort}

%%
% (3)
%%

\item Überführe das Schema in die 3. Normalform.
\index{Dritte Normalform}

\begin{bAntwort}
Für die 3. NF: Entfernung der transitiven Abhängigkeiten. Die Tabellen
\emph{„Kunde“} und \emph{„Sonderaktionen“} bleiben erhalten.

Kunden(\underline{Kdnr}, Name, Wohnort)

Aktion(\underline{Buchungsdatum}, Aktion)

Fahrzeugtypen(\underline{Fahrzeug}, Typ)

Tarife(\underline{Typ}, Tarif)

Stationsleiter(\underline{Rückgabestation}, Stationsleiter)

MobilRent(\underline{KdNr, Buchungsdatum}, Fahrzeug, Tage, Rückgabestation)
\end{bAntwort}
\end{enumerate}
\end{document}
