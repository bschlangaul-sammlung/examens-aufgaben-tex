\documentclass{bschlangaul-aufgabe}
\bLadePakete{normalformen,synthese-algorithmus}
\begin{document}
\bAufgabenMetadaten{
  Titel = {Synthesealgorithmus},
  Thematik = {Relation A-H},
  Referenz = DB.Relationale-Entwurfstheorie.Normalformen.Synthesealgorithmus.Relation-A-H,
  RelativerPfad = Module/10_DB/50_Relationale-Entwurfstheorie/30_Normalformen/10_Synthesealgorithmus/Aufgabe_Relation-A-H.tex,
  ZitatSchluessel = db:pu:4,
  ZitatBeschreibung = {Seite 1: Synthesealgorithmus, Aufgabe 2},
  BearbeitungsStand = mit Lösung,
  Korrektheit = unbekannt,
  Ueberprueft = {unbekannt},
  Stichwoerter = {Synthese-Algorithmus},
}

\let\ah=\bAttributHuelle
\let\ahl=\bLinksReduktionInline
\let\ahr=\bRechtsReduktionInline
\let\fa=\bFunktionaleAbhaengigkeit
\let\FA=\bFunktionaleAbhaengigkeiten
\let\m=\bAttributMenge
\let\r=\bRelation
\let\schrittE=\bSyntheseUeberErklaerung
\let\u=\underline

Überführen Sie das Relationenschema mit Hilfe des Synthesealgorithmus in
die 3. Normalform!
\index{Synthese-Algorithmus}
\footcite[Seite 1: Synthesealgorithmus, Aufgabe 2]{db:pu:4}

\begin{center}
\r[R]{A, B, C, D, E, F, G, H}
\end{center}

\FA{
  F -> E;
  A -> B, D;
  A, E -> D;
  A -> E, F;
  A, G -> H;
}

\begin{bAntwort}

\begin{enumerate}

\item Kanonische Überdeckung

\begin{enumerate}

%%
%
%%

\item \schrittE{1-1}

Wir betrachten nur die zusammengesetzten Attribute:

\bPseudoUeberschrift{\fa{A, E -> D}}

$D \in$ \ahl{A, E}{E}{A, E, F, B, \textbf{D}} \\
$D \notin$ \ahl{A, E}{A}{E}

\bPseudoUeberschrift{\fa{A, G -> H}}

$H \notin$ \ahl{A, G}{G}{A, E, F, B, D} \\
$H \notin$ \ahl{A, G}{A}{G}

\FA{
  F -> E;
  A -> B, D;
  A -> D;
  A -> E, F;
  A, G -> H;
}

%%
%
%%

\item \schrittE{1-2}

Nur die Attribute betrachten, die rechts doppelt vorkommen:

\bPseudoUeberschrift{E}

\ahr{F -> E}{}{F}{F} \\
\ahr{A -> E, F}{A -> E}{A}{A, B, D, F, \textbf{E}}

\bPseudoUeberschrift{D}

\ahr{A -> D}{}{A}{A, B, \textbf{D}, F, E}

\fa{A -> D} kann wegen der Armstrongschen Dekompositionsregel
weggelassen werden. Wenn gilt \fa{A -> B, D}, dann gilt auch \fa{A -> B}
und \fa{A -> D}

\FA{
  F -> E;
  A -> B, D;
  A -> NICHTS;
  A -> F;
  A, G -> H;
}

\item \schrittE{1-3}

\FA{
  F -> E;
  A -> B, D;
  A -> F;
  A, G -> H;
}

\item \schrittE{1-4}

\FA{
  F -> E;
  A -> B, D, F;
  A, G -> H;
}

\end{enumerate}

\item \schrittE{2}

\r[R1]{\u{F}, E}\\
\r[R2]{\u{A}, B, D, F}\\
\r[R3]{\u{A, G}, H}

\item \schrittE{3}

\r[R1]{\u{F}, E}\\
\r[R2]{\u{A}, B, D, F}\\
\r[R3]{\u{A, G}, H}\\
\r[R4]{\u{A, C, G}}

\item \schrittE{4}

\bNichtsZuTun

\end{enumerate}
\end{bAntwort}

\end{document}
