\documentclass{bschlangaul-aufgabe}
\bLadePakete{normalformen}
\begin{document}
\bAufgabenMetadaten{
  Titel = {Abstaktes R},
  Thematik = {Abstraktes R},
  Referenz = DB.Relationale-Entwurfstheorie.Normalformen.Abstraktes-R,
  RelativerPfad = Module/10_DB/50_Relationale-Entwurfstheorie/30_Normalformen/Aufgabe_Abstraktes-R.tex,
  ZitatSchluessel = db:ab:6,
  ZitatBeschreibung = {Seite 1, Aufgabe 1},
  BearbeitungsStand = mit Lösung,
  Korrektheit = unbekannt,
  Ueberprueft = {unbekannt},
  Stichwoerter = {Schlüsselkandidat, Zweite Normalform, Kanonische Überdeckung},
}

\let\ah=\bAttributHuelle
\let\fa=\bFunktionaleAbhaengigkeit
\let\FA=\bFunktionaleAbhaengigkeiten
\let\m=\bAttributMenge

Gegeben sei das Relationenschema \bRelation{A, B, C, D, E, G} mit
\footcite[Seite 1, Aufgabe 1]{db:ab:6}

\FA[F]{
  E -> D;
  C -> B;
  C, E -> G;
  B -> A;
}

\begin{enumerate}

%%
% (a)
%%

\item Zeigen Sie: ${C, E}$ ist der einzige Schlüsselkandidat von $R$.
\index{Schlüsselkandidat}

\begin{bAntwort}
$C$ und $E$ kommen auf keiner rechten Seite der Funktionalen
Abhängigkeiten aus $F$ vor, \dh $C$ und $E$ müssen Teil jedes
Schlüsselkandidaten sein.

Außerdem gilt: $\ah{F, \m{C, E}} = \m{A, B, C, D, E, G} = R$

\m{C, E} ist somit Superschlüssel von $R$. Zudem ist \m{C, E}
minimal, da beide Attribute Teil jedes Schlüsselkandidaten sein müssen.

$\Rightarrow$ $\m{C, E}$ ist damit der einzige Schlüsselkandidat von $R$
(da kein Schlüssel ohne $C$ und $E$ möglich ist).

\bPseudoUeberschrift{Anmerkung:}

\begin{itemize}
\item Man könnte hier auch einen Algorithmus zur Bestimmung der
Schlüsselkandidaten verwenden, dessen einziges Ergebnis wäre dann $\{C,
E\}$. In diesem Fall lässt sich die Schlüsselkandidateneigenschaft
jedoch einfacher zeigen, sodass man den Algorithmus und somit Zeit
sparen kann.

\item Achtung! \m{C, E} ist zwar der einzige Schlüsselkandidat, aber
nicht der einzige Superschlüssel, auch \m{A, B, C, D, E, G} wäre ein
Superschlüssel!
\end{itemize}

\end{bAntwort}

%%
% (b)
%%

\item Ist $R$ in 2NF?
\index{Zweite Normalform}

\begin{bAntwort}
$R$ ist nicht in 2NF, denn:

Betrachte \fa{E -> D}: $D$ ist ein Nicht-Schlüsselattribut und $E$
ist echt Teilmenge des Schlüsselkandidaten $\m{C, E}$.
%
Ebenso ist $B$ nicht voll funktional abhängig vom Schlüsselkandidaten,
sondern nur von einer echten Teilmenge des Schlüsselkandidaten, nämlich
$C$.

\bPseudoUeberschrift{Anmerkung:}

\begin{itemize}
\item Ob alle Attributwerte atomar sind, können wir in einem abstrakten
Schema wie diesem nicht wirklich sagen, daher kann dies Annahme in der
Regel nicht getroffen werden.

\item Dass $A$ von $B$ abhängig ist, spielt bei der Entscheidung über
die 2. NF keine Rolle, da $B$ selbst (genauso wie $A$) ein
Nicht-Schlüsselattribut ist. Wichtig ist nur, ob es Abhängigkeiten
zwischen einem Teil der Schlüsselkandidaten (also einem
Schlüsselattribut) und einem Nicht-Schlüsselattribut gibt.

\item Um der 2NF zu genügen, müsste in folgenden Relationen aufgeteilt
werden:

\bRelation[R1]{C, E, G}
\bRelation[R2]{C, B, A}
\bRelation[R2]{E, D}
\end{itemize}

\end{bAntwort}

%%
% (c)
%%

\item Ist $F$ minimal?
\index{Kanonische Überdeckung}

\FA{
  E -> D;
  C -> B;
  CE -> G;
  B -> A;
}

\begin{bAntwort}
Kanonische Überdeckung

\begin{enumerate}
\item Linksreduktion

$AttrHul\m{F, \m{C}} = \m{C, B}$ $\rightarrow G$ nicht enthalten\\
$AttrHul\m{F, \m{E}} = \m{E, D}$ $\rightarrow G$ nicht enthalten

\item Rechtsreduktion

Kein Attribut auf einer rechten Seite ist redundant: Da das einzelne
Attribut, das die rechte Seite einer FD aus F bildet, bei keiner anderen
FD auf der rechten Seite auftritt, kann die rechte Seite einer FD nicht
unter ausschließlicher Verwendung der restlichen FD aus der
entsprechenden linken Seite abgeleitet werden.
\end{enumerate}

\end{bAntwort}

\begin{bAntwort}
Vorgehen: Entsprechen die hier abgebildeten Funktionalen Abhängigkeiten
bereits einer kanonischen Überdeckung von F oder nicht?

\begin{itemize}
\item Eliminierung redundanter Attribute auf der linken Seite: Die
Attributmenge auf den linken Seiten der FDs sind bereits bis auf
\fa{C, E -> G} einelementig. Bei \fa{C, E -> G} ist \m{CE} der
Schlüsselkandidat, also kann kein redundantes Attribut vorliegen.

\item Eliminierung redundanter Attribute auf der rechten Seite (hier
müssen auch alle einelementigen FA’s betrachtet werden)

\begin{itemize}
\item \fa{E -> D}: $\ah{F - \m{E \rightarrow D}, \m{E}} = \m{E}$, \dh
$D \notin \ah{F - \m{E \rightarrow D}, \m{E}}$

\item \fa{C -> B}: $\ah{F - \m{C \rightarrow B}, \m{C}} = \m{C}$, \dh
$B \notin \ah{F - \m{C \rightarrow B}, \m{E}}$

\item \fa{CE -> G}: $\ah{F - \m{CE \rightarrow G}, \m{C, E}} = \m{A, B, C, D, E}$, \dh
$G \notin \ah{F - \m{CE \rightarrow G}, \m{E}}$
$\Rightarrow$ $CE \rightarrow G$ ist nicht redundant

\item \fa{B -> A}: $\ah{F - \fa{B -> A}, \m{B}} = \m{B}$, \dh

$A \notin \ah{F - \m{B \rightarrow A}, \m{E}}$
$\Rightarrow$ $B \rightarrow A$ ist nicht redundant
\end{itemize}

F ist bereits minimal.
\end{itemize}
\end{bAntwort}
\end{enumerate}
\end{document}
