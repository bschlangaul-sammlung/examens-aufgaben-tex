\documentclass{bschlangaul-aufgabe}

\begin{document}
\bAufgabenMetadaten{
  Titel = {Aufgabe 2: ddi.cs.fau.de},
  Thematik = {Speicherung-Dateisystem},
  Referenz = DB.Uebersicht.Speicherung-Dateisystem,
  RelativerPfad = Module/10_DB/10_Uebersicht/Aufgabe_Speicherung-Dateisystem.tex,
  ZitatSchluessel = db:ab:1,
  ZitatBeschreibung = {Seite 1},
  BearbeitungsStand = mit Lösung,
  Korrektheit = unbekannt,
  Ueberprueft = {unbekannt},
  Stichwoerter = {Datenbank-Übersicht},
}

Zur\index{Datenbank-Übersicht}
\footcite[Seite 1]{db:ab:1} Speicherung von Daten kann ein Dateisystem verwendet werden.
Betrachten Sie die Informationsseiten der Didaktik der Informatik an der
FAU im Internet (https: //ddi.cs.fau.de/). Gehen Sie davon aus, dass für
jede Seite, die Sie betrachten, eine Datei existiert, in der die auf der
Seite sichtbaren Informationen gespeichert sind.
\footcite[Seite 19-20]{winter}

\begin{enumerate}

%%
% (a)
%%

\item Warum ist diese Art der Datenabspeicherung nicht besonders
günstig?

\begin{bAntwort}
\begin{description}
\item[Redundanz:]

In dieser Art zu speichern sind viele Redundanzen enthalten. So muss
beispielsweise das Menü mit den HTML-Links auf jeder Seite gespeichert
sein, um durch die Seite navigieren zu können. Die gleiche Information
kommt auf sehr vielen verschiedenen Seiten vor. Damit muss dieselbe
Information in unterschiedlichen Dateien abgespeichert werden, \dh ein
Großteil der Information ist redundant gespeichert.

\item[Beschränkte Zugriffsmöglichkeit:]

Die Daten können nur schlecht maschinell abgefragt werden. Sie können
nur einzeln im Browser aufgerufen werden.

\item[Beschränkte Zugriffskontrolle:]

Die HTML-Seiten können entweder als ganze Seite unter einen
Passwortschutz gestellt werden oder vollkommen öffentlich ins Netz
gestellt werden. So können einzelne Informationen auf den Seite, wie
zum Beispiel die Raumnummer nicht einzeln in ihrer Sichtbarkeit
beeinflusst werden.
\end{description}
\end{bAntwort}

%%
% (b)
%%

\item Es wird angenommen, dass folgende Aktualisierungen durchgeführt
werden müssen:

\begin{itemize}
\item Die Mitarbeiter haben sich geändert.
\item Das Projekt „AMI – Agile Methoden im Informatikunterricht“ ist
abgeschlossen. Alle diesbezüglichen Informationen sollen deshalb
gelöscht werden.
\end{itemize}

\item Zu welchem Problemen kann es dabei (aufgrund der Datenspeicherung)
kommen?

\begin{bAntwort}
Nicht aus alle Seiten werden die Links zur der Unterseite Projekt „AMI“
entfernt. Es kann zur Inkonsistenz kommen.

Mitarbeiterinformationen wie auch die Daten des Kurses sind redundant
gespeichert. Änderungen oder Löschungen müssen deshalb in allen Dateien
erfolgen, in denen die entsprechenden Informationen gespeichert sind.
Bei vielen Dateien hat man aber in der Regel keinen Überblick, wo eine
bestimmte Information überall abgespeichert ist. Dies kann sehr leicht
zu einem inkonsistenten, d. h. nicht stimmigen, Datenbestand führen.
Konkret können beispielsweise folgende Probleme auftreten:

\begin{itemize}
\item Die Änderung der Mitarbeiter wird an einer Stelle vergessen, d. h.
dass \zB Dr. X noch Anfragen und E-Mails an die Adresse X@fau.de
bekommt, obwohl er schon längst ausgeschieden ist und die o. g.
E-Mail-Adresse nicht mehr existiert.

\item Zum Löschen der AMI-Seite müssen die zentrale Projektseite und
alle Links gelöscht werden. Die Löschung eines Link auf die zentrale
Projektseite wird vergessen, d. h. man bekommt einen „toten“ Link.
\end{itemize}
\end{bAntwort}
\end{enumerate}
\end{document}
