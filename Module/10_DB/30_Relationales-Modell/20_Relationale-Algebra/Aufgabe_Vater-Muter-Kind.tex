\documentclass{bschlangaul-aufgabe}

\begin{document}
\bAufgabenMetadaten{
  Titel = {Division},
  Thematik = {Vater-Muter-Kind},
  Referenz = DB.Relationales-Modell.Relationale-Algebra.Vater-Muter-Kind,
  RelativerPfad = Module/10_DB/30_Relationales-Modell/20_Relationale-Algebra/Aufgabe_Vater-Muter-Kind.tex,
  BearbeitungsStand = mit Lösung,
  Korrektheit = unbekannt,
  Ueberprueft = {unbekannt},
  Stichwoerter = {Division},
}

\bPseudoUeberschrift{$R$}
\index{Division}
\bFussnoteUrl{https://de.wikibooks.org/wiki/Relationenalgebra_und_SQL:_Division}

\begin{tabular}{llll}
Vater  & Mutter    & Kind    & Alter \\
Hans   & Helga     & Harald  & 5     \\
Hans   & Helga     & Maria   & 4     \\
Hans   & Ursula    & Sabine  & 2     \\
Martin & Melanie   & Gertrud & 7     \\
Martin & Melanie   & Maria   & 4     \\
Martin & Melanie   & Sabine  & 2     \\
Peter  & Christina & Robert  & 9
\end{tabular}

\bPseudoUeberschrift{$S$}

\begin{tabular}{ll}
Kind   & Alter \\
Maria  & 4     \\
Sabine & 2
\end{tabular}

\bPseudoUeberschrift{$R \div S$}

\begin{tabular}{ll}
Kind   & Alter \\
Maria  & 4     \\
Sabine & 2
\end{tabular}

\end{document}
