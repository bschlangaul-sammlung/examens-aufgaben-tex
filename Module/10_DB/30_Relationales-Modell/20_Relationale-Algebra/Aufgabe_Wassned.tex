\documentclass{bschlangaul-aufgabe}
\bLadePakete{syntax,mathe}
\begin{document}
\bAufgabenMetadaten{
  Titel = {Aufgabe 4: Relationale Algebra Einstieg},
  Thematik = {Wassned},
  Referenz = DB.Relationales-Modell.Relationale-Algebra.Wassned,
  RelativerPfad = Module/10_DB/30_Relationales-Modell/20_Relationale-Algebra/Aufgabe_Wassned.tex,
  ZitatSchluessel = db:pu:2,
  BearbeitungsStand = mit Lösung,
  Korrektheit = unbekannt,
  Ueberprueft = {unbekannt},
  Stichwoerter = {Relationale Algebra},
}

Gegeben ist folgende Datenbank-Anfrage:
\index{Relationale Algebra}
\footcite{db:pu:2}

\begin{math}
\pi_{\text{Bezeichnung}}(
  \sigma_{\text{SWS} = 2 \land\neg(\text{Name} = \mlq \text{Wassned} \mrq)}(
    \text{Vorlesung} \bowtie \text{Professor}
  )
)
\end{math}

\begin{enumerate}

%%
% (a)
%%

\item Geben Sie eine umgangssprachliche Formulierung der Anfrage an!

\begin{bAntwort}
Eine Liste mit den Bezeichnungen der Vorlesungen, die 2 Semester
Wochenstunden dauern und die nicht vom dem Professor „Wassned“ gelesen
werden.
\end{bAntwort}

%%
% (b)
%%

\item Geben Sie die Ergebnistabelle an!

\begin{bAntwort}

\begin{tabular}{|l|}
\hline
\textbf{Bezeichnung}\\
Japanische Malerei\\
Chinesische Schrift\\
\hline
\end{tabular}
\end{bAntwort}
\end{enumerate}
\end{document}
