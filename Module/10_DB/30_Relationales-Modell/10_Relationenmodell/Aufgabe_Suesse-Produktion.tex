\documentclass{bschlangaul-aufgabe}

\begin{document}
\bAufgabenMetadaten{
  Titel = {Aufgabe 8: Süße Produktion?},
  Thematik = {Süße Produktion},
  Referenz = DB.Relationales-Modell.Relationenmodell.Suesse-Produktion,
  RelativerPfad = Module/10_DB/30_Relationales-Modell/10_Relationenmodell/Aufgabe_Suesse-Produktion.tex,
  ZitatSchluessel = db:ab:klausurvorbereitung,
  BearbeitungsStand = mit Lösung,
  Korrektheit = unbekannt,
  Ueberprueft = {unbekannt},
  Stichwoerter = {Relationenmodell, Kartesisches Produkt},
}

Vorgegeben sind die Domänen Abteilung = Verwaltung, Lager, Produktion
und Mitarbeiter = Fent, Süß, Dobler.\index{Relationenmodell}
\footcite{db:ab:klausurvorbereitung}

\begin{enumerate}

%%
% (a)
%%

\item Bestimmen Sie Mitarbeiter x Abteilung! Welche Aussage liefert
dieses kartesische Produkt?\index{Kartesisches Produkt}

%%
% (b)
%%

\item Geben Sie – orientiert an Aufgabe a) - eine mögliche
Interpretation der Menge (Fent,Produktion), (Süß, Lager) an.

\end{enumerate}
\end{document}
