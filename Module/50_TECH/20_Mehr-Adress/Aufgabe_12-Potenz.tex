\documentclass{bschlangaul-aufgabe}
\bLadePakete{java,mathe}
\begin{document}
\bAufgabenMetadaten{
  Titel = {Potenzberechnung},
  Thematik = {Potenzberechnung},
  Referenz = TECH.Mehr-Adress.12-Potenz,
  RelativerPfad = Module/50_TECH/20_Mehr-Adress/Aufgabe_12-Potenz.tex,
  BearbeitungsStand = mit Lösung,
  Korrektheit = unbekannt,
  Ueberprueft = {unbekannt},
  Stichwoerter = {Mehr-Adress-Befehl-Assembler},
}

Erstelle ein rekursives Assemblerprogramm, das seine beiden Parameter
über zwei Variablen $a$ und $n$ aus dem Speicher übernimmt und den Wert
$\text{power}(a, n)$ berechnet. Das Ergebnis soll in $R0$ liegen. Dabei
soll die Rekursion gelten:\index{Mehr-Adress-Befehl-Assembler}

\begin{displaymath}
\text{power}(a, n) = a \cdot \text{power}(a, n − 1)
\end{displaymath}

\noindent
Die Lösung der Berechnung soll zum Schluss in $R5$ liegen.

\begin{bAntwort}
\bAssemblerDatei{Aufgabe_12-Potenz.mi}

\bJavaDatei[firstline=3]{aufgaben/tech_info/assembler/mehr_adress/Power}
\end{bAntwort}
\end{document}
