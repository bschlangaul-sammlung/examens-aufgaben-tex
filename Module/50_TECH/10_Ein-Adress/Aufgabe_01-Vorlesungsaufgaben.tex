\documentclass{bschlangaul-aufgabe}
\bLadePakete{syntax,spalten}
\begin{document}
\bAufgabenMetadaten{
  Titel = {Vorlesungsaufgaben},
  Thematik = {Vorlesungsaufgaben},
  Referenz = TECH.Ein-Adress.01-Vorlesungsaufgaben,
  RelativerPfad = Module/50_TECH/10_Ein-Adress/Aufgabe_01-Vorlesungsaufgaben.tex,
  BearbeitungsStand = mit Lösung,
  Korrektheit = unbekannt,
  Ueberprueft = {unbekannt},
  Stichwoerter = {Ein-Adress-Befehl-Assembler},
}

Geben Sie die Lösungen zu den Aufgaben aus der Assembler-Vorlesung
ab. Bearbeiten Sie erst danach die folgenden Aufgaben auf diesem
Übungsblatt.
\index{Ein-Adress-Befehl-Assembler}

\begin{enumerate}

%%
% (a)
%%

\item Folie 28/2: Berechnung der Potenz $a^n$.

\begin{multicols}{2}
\bPseudoUeberschrift{Assembler}

\bAssemblerDatei{Aufgabe_01-Vorlesungsaufgaben-Potenz.mia}

\bSpaltenUmbruch
\bPseudoUeberschrift{Minisprache}

\bMinispracheDatei{Aufgabe_01-Vorlesungsaufgaben-Potenz.mis}
\end{multicols}

%%
% (b)
%%

\newpage

\item Folie 28/3: Größten gemeinsamen Teiler zweier Zahlen

\begin{multicols}{2}
\bPseudoUeberschrift{Assembler}

\bAssemblerDatei{Aufgabe_01-Vorlesungsaufgaben-GGT.mia}

\bSpaltenUmbruch
\bPseudoUeberschrift{Minisprache}

\bMinispracheDatei{Aufgabe_01-Vorlesungsaufgaben-GGT.mis}
\end{multicols}
\end{enumerate}

\end{document}
