\documentclass{bschlangaul-aufgabe}
\bLadePakete{syntax,spalten}
\begin{document}
\bAufgabenMetadaten{
  Titel = {Abitur 2018 IV Aufgabe 3},
  Thematik = {funkyFunction},
  Referenz = TECH.Ein-Adress.Abitur-2018-IV,
  RelativerPfad = Module/50_TECH/10_Ein-Adress/Aufgabe_Abitur-2018-IV.tex,
  BearbeitungsStand = mit Lösung,
  Korrektheit = unbekannt,
  Ueberprueft = {unbekannt},
  Stichwoerter = {Ein-Adress-Befehl-Assembler},
}

% Info_2020-11-20-2020-11-20_09.49.27.mp4 1h

\begin{enumerate}

%%
%
%%

\item Betrachten Sie folgendes Struktogramm einer Methode funkyFunction:

Beschreiben Sie kurz, was diese Methode berechnet.
Schreiben Sie ein Programm für die gegebene Registermaschine, das den
Algorithmus der Methode funkyFunction umsetzt. Geben Sie an, in welcher
Speicherzelle der Rückgabewert steht.
\index{Ein-Adress-Befehl-Assembler}

\begin{multicols}{2}
\bPseudoUeberschrift{Assembler}

\bAssemblerDatei{Aufgabe_Abitur-2018-IV.mia}

\bSpaltenUmbruch
\bPseudoUeberschrift{Minisprache}

\bMinispracheDatei{Aufgabe_Abitur-2018-IV.mis}
\end{multicols}
\end{enumerate}

\end{document}
