\documentclass{bschlangaul-aufgabe}
\bLadePakete{formale-sprachen,mathe,automaten,syntax}
\begin{document}
\bAufgabenMetadaten{
  Titel = {Komplement der Binärzahl},
  Thematik = {Vorlesungsaufgaben Komplement der Binärzahl},
  Referenz = THEO.Formale-Sprachen.Typ-1_Kontextsensitiv.Turing-Maschine.Vorlesungsaufgaben-Komplement-Binaerzahl,
  RelativerPfad = Module/70_THEO/10_Formale-Sprachen/30_Typ-1_Kontextsensitiv/Turing-Maschine/Aufgabe_Vorlesungsaufgaben-Komplement-Binaerzahl.tex,
  ZitatSchluessel = theo:fs:3,
  ZitatBeschreibung = {Seite 22},
  BearbeitungsStand = mit Lösung,
  Korrektheit = unbekannt,
  Ueberprueft = {unbekannt},
  Stichwoerter = {Kontextsensitive Sprache},
}

% Info_2021-04-23-2021-04-23_09.35.52 2h25min

Gegeben ist eine Binärzahl auf dem Band einer Turingmaschine.
\index{Kontextsensitive Sprache}
\footcite[Seite 22]{theo:fs:3}

\begin{enumerate}
\item Definieren Sie vollständig eine TM, die das Komplement der
Binärzahl ($0110 \rightarrow 1001$) berechnet. Die Überführungsfunktion
kann als Tabelle oder als Graph angegeben werden.

\begin{bAntwort}
\begin{center}
\begin{tikzpicture}[li turingmaschine]
  \node[state,initial,accepting] (z0) at (4.43cm,-5.71cm) {$z_0$};

  \bTuringKante[above,loop above]{z0}{z0}{
    0, 1, R;
    1, 0, R;
  }
\end{tikzpicture}
\end{center}
\bFlaci{Ap9qtjgg7}

\begin{minted}{md}
name: Komplement Binärzahl
init: z0
accept: z0

z0,0
z0,1,>

z0,1
z0,0,>
\end{minted}
\bFussnoteUrl{http://turingmachinesimulator.com/shared/lqpsawxdeh}
\end{bAntwort}

\item Erweiteren Sie Ihre Maschine aus Aufgabe a) so, dass der
Schreib-/Lesekopf auf dem ersten Zeichen der Eingabe terminiert.

\begin{bAntwort}
\begin{center}
\begin{tikzpicture}[li turingmaschine]
  \node[state,initial] (z0) at (4.29cm,-6.14cm) {$z_0$};
  \node[state] (z1) at (6.86cm,-6.14cm) {$z_1$};
  \node[state,accepting] (z2) at (9.43cm,-6.14cm) {$z_2$};

  \bTuringKante[above,loop above]{z0}{z0}{
    0, 1, R;
    1, 0, R;
  }

  \bTuringKante[above]{z0}{z1}{
    LEER, LEER, L;
  }

  \bTuringKante[above]{z1}{z2}{
    LEER, LEER, R;
  }

  \bTuringKante[above,loop above]{z1}{z1}{
    1, 1, L;
    0, 0, L;
  }
\end{tikzpicture}
\end{center}
\bFlaci{A5o7tug5r}
\end{bAntwort}
\end{enumerate}
\end{document}
