\documentclass{bschlangaul-aufgabe}
\bLadePakete{formale-sprachen,automaten}
\begin{document}
\bAufgabenMetadaten{
  Titel = {Binärzahl dekrementieren},
  Thematik = {Binärzahl dekrementieren},
  Referenz = THEO.Formale-Sprachen.Typ-1_Kontextsensitiv.Turing-Maschine.Binaerzahl-dekrementieren,
  RelativerPfad = Module/70_THEO/10_Formale-Sprachen/30_Typ-1_Kontextsensitiv/Turing-Maschine/Aufgabe_Binaerzahl-dekrementieren.tex,
  ZitatSchluessel = theo:ab:3,
  ZitatBeschreibung = {Aufgabe 3},
  BearbeitungsStand = mit Lösung,
  Korrektheit = unbekannt,
  Ueberprueft = {unbekannt},
  Stichwoerter = {Turing-Maschine},
}

\let\t=\bTuringUebergangZelle

Sei \bAlphabet{0, 1} und $\Gamma = \{ 0, 1, \bTuringLeerzeichen \}$.
Konstruiere eine Turingmaschine $M$, die eine in Binärform gegebene,
natürliche Zahl $(\neq 0)$ um $1$ dekrementiert (und wieder in Binärform
ausgibt). Der Schreib-/Lesekopf steht zu Beginn der Berechnung auf dem
ersten Leerzeichen links von der Eingabe und soll auch am Ende wieder
dort stehen. Beachte, dass führende Nullen in der Eingabe/Ausgabe nicht
vorkommen dürfen.
\index{Turing-Maschine}
\footcite[Aufgabe 3]{theo:ab:3}

\begin{bAntwort}
\begin{center}
\begin{tabular}{r|l}
\textbf{dezimal} & \textbf{binär} \\\hline
0 & 0 \\
1 & 1 \\
2 & 10 \\
3 & 11 \\
4 & 100 \\
5 & 101 \\
6 & 110 \\
7 & 111 \\
8 & 1000 \\
9 & 1001 \\
10 & 1010 \\
11 & 1011 \\
12 & 1100 \\
13 & 1101 \\
14 & 1110 \\
15 & 1111 \\
16 & 10000 \\
\end{tabular}
\end{center}

\bigskip

\noindent
Die Maschine geht zunächst ans rechte Ende des Wortes, dann invertiert
sie alle $0$ Bits, bis sie auf eine $1$ trifft. Diese wird durch $0$
ersetzt. Damit ist der Dekrementierungsvorgang beendet. Nun sucht Sie
das linke Ende des Wortes und löscht eventuell entstandene führende
Nullen. Trifft Sie dabei auf das Leerzeichen, so war die Ausgabe die
Zahl $0$ und diese wird wieder aufs Band geschrieben. Insgesamt ergibt
sich

\begin{center}
\bTuringMaschine{zustaende={z_0, z_1, z_2, z_3, z_4, z_5}, start={z_0}, ende={z_5}}
\end{center}

mit unten angegebener Übergangsfunktion:

\begin{tabular}{lllll}
δ &
0 &
1 &
d &
Kommentar\\\hline

z0 &
∅ &
∅ &
\t{z1, LEER, R} &
Gehe auf erstes Zeichen des Wortes\\\hline

z1 &
\t{z1, 0, R} &
\t{z1, 1, R} &
\t{z2, LEER, L} &
Gehe ans rechte Ende des Wortes\\\hline

z2 &
\t{z2, 1, L} &
\t{z3, 0, L} &
∅ &
Flippe alle 0 Bits bis die erste 1 erreicht wird, setzte diese auf 0\\\hline

z3 &
\t{z3, 0, L} &
\t{z3, 1, L} &
\t{z4, LEER, R} &
suche linkes Ende des Wortes\\\hline

z4 &
\t{z4, LEER, R} &
\t{z5, 1, L} &
\t{z5, 0, L} &
lösche führende Nullen , schreibe evtl. 0 aufs Band\\\hline

z5 &
∅ &
∅ &
∅ &
Endzustand\\\hline
\end{tabular}

\begin{center}
\begin{tikzpicture}[li turingmaschine]
  \node[state,initial] (z0) at (2.86cm,-2.29cm) {$z_0$};
  \node[state] (z1) at (6.43cm,-2.29cm) {$z_1$};
  \node[state] (z2) at (9.57cm,-2.29cm) {$z_2$};
  \node[state] (z3) at (2.57cm,-4.86cm) {$z_3$};
  \node[state] (z4) at (6.14cm,-6.29cm) {$z_4$};
  \node[state,accepting] (z5) at (9.57cm,-6.29cm) {$z_5$};

  \bTuringKante[above]{z0}{z1}{
    1, 1, R;
    0, 0, R;
    LEER, LEER, R;
  }

  \bTuringKante[above]{z1}{z2}{
    LEER, LEER, L;
  }

  \bTuringKante[above,loop above]{z1}{z1}{
    1, 1, R;
    0, 0, R;
  }

  \bTuringKante[above,loop above]{z2}{z2}{
    0, 1, L;
  }

  \bTuringKante[above]{z2}{z3}{
    1, 0, L;
  }

  \bTuringKante[above,loop above]{z3}{z3}{
    1, 1, L;
    0, 0, R;
  }

  \bTuringKante[above]{z3}{z4}{
    LEER, LEER, R;
  }

  \bTuringKante[above,loop above]{z4}{z4}{
    0, LEER, R;
  }

  \bTuringKante[above]{z4}{z5}{
    1, 1, L;
    LEER, 0, R;
  }
\end{tikzpicture}
\end{center}
\bFlaci{Ahifz611c}
\end{bAntwort}

\end{document}
