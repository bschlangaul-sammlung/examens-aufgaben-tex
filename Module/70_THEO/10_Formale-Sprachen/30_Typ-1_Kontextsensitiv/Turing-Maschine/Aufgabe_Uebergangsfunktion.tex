\documentclass{bschlangaul-aufgabe}
\bLadePakete{automaten,formale-sprachen}
\begin{document}
\bAufgabenMetadaten{
  Titel = {Turingmaschine mit folgender Übergangsfunktion},
  Thematik = {Übergangsfunktion},
  Referenz = THEO.Formale-Sprachen.Typ-1_Kontextsensitiv.Turing-Maschine.Uebergangsfunktion,
  RelativerPfad = Module/70_THEO/10_Formale-Sprachen/30_Typ-1_Kontextsensitiv/Turing-Maschine/Aufgabe_Uebergangsfunktion.tex,
  ZitatSchluessel = theo:ab:3,
  ZitatBeschreibung = {Aufgabe 2},
  BearbeitungsStand = mit Lösung,
  Korrektheit = unbekannt,
  Ueberprueft = {unbekannt},
  Stichwoerter = {Turing-Maschine},
}

\def\l{\,\bTuringLeerzeichen\,}
\let\t=\bTuringUebergangZelle
\def\z#1{\ifmmode \,z_#1\,\else$z_#1$\fi}

Gegeben sei eine TM mit folgender Übergangsfunktion:
\index{Turing-Maschine}
\footcite[Aufgabe 2]{theo:ab:3}

\begin{center}
\begin{tabular}{c|ccccc}
& \z1 & \z2 & \z3 & \z4 & \z5 \\\hline

0 &
\t{z2, LEER, R} &
\t{z3, X, R} &
\t{z4, 0, R} &
\t{z3, X, R} &
\t{z5, 0, L} \\

X &
- &
\t{z2, X, R} &
\t{z3, X, R} &
\t{z4, X, R} &
\t{z5, X, L} \\

\l &
- &
\t{z_f, LEER, R} &
\t{z5, LEER, L} &
- &
\t{z2, LEER, R} \\
\end{tabular}
\end{center}

\begin{center}
\begin{tikzpicture}[li turingmaschine]
  \node[state,initial] (z1) at (2.43cm,-4.14cm) {$z_1$};
  \node[state] (z2) at (4.43cm,-4.14cm) {$z_2$};
  \node[state] (z3) at (8.71cm,-4.14cm) {$z_3$};
  \node[state] (z4) at (8.71cm,-6.43cm) {$z_4$};
  \node[state] (z5) at (7cm,-1.43cm) {$z_5$};
  \node[state,accepting] (zf) at (4.43cm,-5.71cm) {$z_f$};

  \bTuringKante[above]{z1}{z2}{
    0, X, R;
  }

  \bTuringKante[above]{z2}{z3}{
    0, X, R;
  }

  \bTuringKante[above,loop above]{z2}{z2}{
    X, X, R;
  }

  \bTuringKante[right]{z2}{zf}{
    LEER, LEER, R;
  }

  \bTuringKante[right,bend left]{z3}{z4}{
    0, 0, R;
  }

  \bTuringKante[above,loop right]{z3}{z3}{
    X, X, R;
  }

  \bTuringKante[right,bend right]{z3}{z5}{
    LEER, LEER, R;
  }

  \bTuringKante[left,bend left]{z4}{z3}{
    0, X, R;
  }

  \bTuringKante[above,loop below]{z4}{z4}{
    X, X, R;
  }

  \bTuringKante[above,loop above]{z5}{z5}{
    0, 0, L;
    X, X, L;
  }

  \bTuringKante[above left,pos=0.3]{z5}{z2}{
    LEER, LEER, R;
  }
\end{tikzpicture}
\end{center}
\bFlaci{Apew8cea2}

\noindent
Erreicht die TM den Zustand $z_f$ (final), so hält sie an und bearbeitet
keine weitere Eingabe. Zu Beginn der Berechnung soll die TM auf dem
ersten Symbol der Eingabe (links) stehen.
\begin{enumerate}

%%
% (a)
%%

\item Gebe für die folgenden Eingaben die Konfigurationsfolgen der
Berechnung an:

\def\z#1{\,z_#1\,}
\def\p{&\rightarrow}

\begin{itemize}
\item 00000

\begin{bAntwort}
Der Zustand der TM steht vor dem nächsten gelesenen Zeichen

\begin{align*}
\z1 00000 \p \\
\p \l \z2 0000 \\
\p \l X\z3 000 \\
\p \l X0\z4 00 \\
\p \l X0X\z3 0 \\
\p \l X0X0\z4
\end{align*}

\end{bAntwort}

\item 000000

\begin{bAntwort}
Der Zustand der TM steht vor dem nächsten gelesenen Zeichen

\begin{align*}
\z1 000000 \p \\
\p \l \z2 00000 \\
\p \l X\z3 0000 \\
\p \l X0\z4 000 \\
\p \l X0X\z3 00 \\
\p \l X0X0\z4 0 \\
\p \l X0X0X\z3 \l \\
\p \l X0X0\z5 X\l \\
\p \l X0X\z5 0X\l \\
\p \l X0\z5 X0X\l \\
\p \l X\z5 0X0X\l \\
\p \l \z5 X0X0X\l \\
\p \z5 \l X0X0X\l \\
\p \l \z2 X0X0X\l \\
\p \l X\z2 0X0X\l \\
\p \l XX\z3 X0X\l \\
\p \l XXX\z3 0X\l \\
\p \l XXX0\z4 X\l \\
\p \l XXX0X\z4 \l
\end{align*}

\end{bAntwort}

\item 0000
\begin{bAntwort}
Der Zustand der TM steht vor dem nächsten gelesenen Zeichen

\begin{align*}
\z1 000 \p \\
\p \l \z1 000 \\
\p \l X\z3 00 \\
\p \l X0\z4 0 \\
\p \l X0X\z3 \l \\
\p \l X0\z5 X\l \\
\p \l X\z5 0X\l \\
\p \l \z5 X0X\l \\
\p \z5 \l X0X\l \\
\p \l \z2 X0X\l \\
\p \l X\z2 0X\l \\
\p \l XX\z3 X\l \\
\p \l XXX\z3 \l \\
\p \l XX\z5 X\l \\
\p \l X\z5 XX\l \\
\p \l \z5 XXX\l \\
\p \z5 \l XXX\l \\
\p \l \z2 XXX\l \\
\p \l X\z2 XX\l \\
\p \l XX\z2 X\l \\
\p \l XXX\z2 \l \\
\p \l XXX\l \z f
\end{align*}
\end{bAntwort}

\end{itemize}

%%
% (b)
%%

\item Gebe zwei andere Wörter über der Sprache $L \subset \{ \, 0^* \,
\}$ an, für die TM im Zustand $z_f$ endet.

\begin{bAntwort}
\zB $0$ oder $00$
\end{bAntwort}

%%
% (c)
%%

\item Für welche Sprache ist die TM ein Akzeptor?

\begin{bAntwort}
Die TM erkennt alle Wörter mit der Eigenschaft, dass die Anzahl der
Nullen eine 2er-Potenzen ist.
\end{bAntwort}
\end{enumerate}
\end{document}
