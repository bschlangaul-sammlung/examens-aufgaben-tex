\documentclass{bschlangaul-aufgabe}
\bLadePakete{formale-sprachen}
\begin{document}
\bAufgabenMetadaten{
  Titel = {Kontextsensitive Grammatik},
  Thematik = {Kontextsensitive Grammatik},
  Referenz = THEO.Formale-Sprachen.Typ-1_Kontextsensitiv.Kontextsensitive-Grammatik.Kontextsensitive-Grammatik,
  RelativerPfad = Module/70_THEO/10_Formale-Sprachen/30_Typ-1_Kontextsensitiv/Kontextsensitive-Grammatik/Aufgabe_Kontextsensitive-Grammatik.tex,
  ZitatSchluessel = theo:ab:3,
  ZitatBeschreibung = {Aufgabe 1},
  BearbeitungsStand = mit Lösung,
  Korrektheit = unbekannt,
  Ueberprueft = {unbekannt},
  Stichwoerter = {Kontextsensitive Grammatik},
}

Sei \bAusdruck{a^n b^m c^n}{n, m \geq 0, m \leq n}.
\index{Kontextsensitive Grammatik}
\footcite[Aufgabe 1]{theo:ab:3}

\begin{enumerate}

%%
% (a)
%%

\item Geben Sie eine kontextsensitive Grammatik für $L$ an.

\begin{bAntwort}
\bGrammatik{variablen={S, A, B, C}, alphabet={a, b, c}} mit
Produktionen $P$ wie folgt:

\begin{bProduktionsRegeln}
S -> EPSILON | a S c | a S B C,
C B -> B C,
b C -> b c,
a B -> a b,
b B -> b b,
c C -> c c
\end{bProduktionsRegeln}
\end{bAntwort}

%%
% (b)
%%

\item Geben Sie eine Ableitung des Wortes $a^3 b^2 c^3$ mittels der in
Teilaufgabe a) erstellten Grammatik an.

\begin{bAntwort}
\bAbleitung{
S
-> aSc
-> aaSBCc
-> aaaSBCBCc
-> aaaBCBCc
-> aaaBBCCc
-> aaabBCCc
-> aaabbCCc
-> aaabbcCc
-> aaabbccc}
\end{bAntwort}
\end{enumerate}

\end{document}
