\documentclass{bschlangaul-aufgabe}
\bLadePakete{formale-sprachen}
\begin{document}
\bAufgabenMetadaten{
  Titel = {Ableitung einer Phrasenstrukturgrammatik},
  Thematik = {a-2-hoch-n},
  Referenz = THEO.Formale-Sprachen.Typ-0_Phrasenstruktur.a-2-hoch-n,
  RelativerPfad = Module/70_THEO/10_Formale-Sprachen/40_Typ-0_Phrasenstruktur/Aufgabe_a-2-hoch-n.tex,
  ZitatSchluessel = thoe:fs:3,
  ZitatBeschreibung = {Seite 15-16},
  BearbeitungsStand = mit Lösung,
  Korrektheit = unbekannt,
  Ueberprueft = {unbekannt},
  Stichwoerter = {Unbeschränkte Sprache},
}

Die folgende Grammatik erzeugt die Sprache \bAusdruck{a^{2^n}}{n \in
\mathbb{N}}, nicht kontextsensitiv ist:\index{Unbeschränkte Sprache}
\footcite[Seite 15-16]{thoe:fs:3}

\bGrammatik{variablen={S, L, R, D}, alphabet={a}}
\begin{bProduktionsRegeln}
S -> L a R,
L -> L D | EPSILON,
D a -> a a D,
D R -> R,
R -> EPSILON,
\end{bProduktionsRegeln}

\noindent
Leiten Sie das Wort $a^8$ ab.

\begin{bAntwort}
\bAbleitung{LaR
-> LDaR
-> LDDaR
-> LDDDaR
-> DDDaR
-> DDaaDR
-> DaaDaDR
-> aaDaDaDR
-> aaaaDDaDR
-> aaaaaaaaDDDR
-> aaaaaaaaDDR
-> aaaaaaaaR
-> aaaaaaa}
\end{bAntwort}

\end{document}
