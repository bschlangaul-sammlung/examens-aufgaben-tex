\documentclass{bschlangaul-aufgabe}
\bLadePakete{formale-sprachen,syntaxbaum}
\begin{document}
\bAufgabenMetadaten{
  Titel = {Ableitungen},
  Thematik = {Ableitungen},
  Referenz = THEO.Formale-Sprachen.Typ-2_Kontextfrei.Ableitung-Ableitungsbaum.Ableitungen,
  RelativerPfad = Module/70_THEO/10_Formale-Sprachen/20_Typ-2_Kontextfrei/Ableitung-Ableitungsbaum/Aufgabe_Ableitungen.tex,
  ZitatSchluessel = theo:ab:2,
  BearbeitungsStand = mit Lösung,
  Korrektheit = unbekannt,
  Ueberprueft = {unbekannt},
  Stichwoerter = {Ableitung (Kontextfreie Sprache), Ableitungsbaum},
}

Bestimmen Sie für die folgende Grammatik jeweils für die angegebenen
Wörter $w_i$ mehrere Ableitungen sowie Parsebäume, wenn dies nicht
eindeutig möglich ist.
\index{Ableitung (Kontextfreie Sprache)}
\footcite{theo:ab:2}
\index{Ableitungsbaum}

\begin{bProduktionsRegeln}
S -> AB,
A -> aAb | epsilon,
B -> cBc | c
\end{bProduktionsRegeln}

\begin{enumerate}

%%
% (a)
%%

\item $w_1 = abccc$

\begin{bAntwort}
\begin{center}
\begin{tikzpicture}[b syntaxbaum,level distance=1.2cm]
\Tree
[.S [.A a [.A $\varepsilon$ ] b ] [.B c [.B c ] c ] ]
\end{tikzpicture}
\end{center}
\end{bAntwort}

%%
% (b)
%%

\item $w_2 = ccc$

\begin{bAntwort}
\begin{center}
\begin{tikzpicture}[b syntaxbaum,level distance=1.2cm]
\Tree
[.S [.A $\varepsilon$ ] [.B c [.B c ] c ] ]
\end{tikzpicture}
\end{center}
\end{bAntwort}

%%
% (c)
%%

\item $w_3 = aabbc$

\begin{bAntwort}
\begin{center}
\begin{tikzpicture}[b syntaxbaum,level distance=1.2cm]
\Tree
[.S [. A a [.A a [.A $\varepsilon$ ] b ] b ] [.B c ] ]
\end{tikzpicture}
\end{center}
\end{bAntwort}

\end{enumerate}
Welche Sprache wird durch die Grammatik erzeugt?

\begin{bAntwort}
Die Sprache beinhaltet alle Wörter, die gleich viele $a$’s gefolgt von
gleich vielen $b$’s und auf ungeradzahlig viele $c$’s endet.
\end{bAntwort}

\end{document}
