\documentclass{bschlangaul-aufgabe}
\bLadePakete{formale-sprachen,cyk-algorithmus}
\begin{document}
\bAufgabenMetadaten{
  Titel = {CYK-Algorithmus},
  Thematik = {Youtube-Video Karsten-Morisse},
  Referenz = THEO.Formale-Sprachen.Typ-2_Kontextfrei.CYK-Algorithmus.Youtube-Video-Karsten-Morisse,
  RelativerPfad = Module/70_THEO/10_Formale-Sprachen/20_Typ-2_Kontextfrei/CYK-Algorithmus/Aufgabe_Youtube-Video-Karsten-Morisse.tex,
  BearbeitungsStand = mit Lösung,
  Korrektheit = unbekannt,
  Ueberprueft = {unbekannt},
  Stichwoerter = {CYK-Algorithmus},
}

\let\l=\bKurzeTabellenLinie

\bGrammatik{variablen={S, A, B, C}, alphabet={a, b}}
\index{CYK-Algorithmus}
\bFussnoteUrl{https://www.youtube.com/watch?v=Q5TvCyu4RUo}

\begin{bProduktionsRegeln}
S -> A B | B C,
A -> B A | a,
B -> C C | b,
C -> A B | a
\end{bProduktionsRegeln}

\begin{bAntwort}
\begin{tabular}{|c|c|c|c|c|}
a   & b   & a   & a   & b \\\hline\hline

A,C & B   & A,C & A,C & B \l5
S,C & A,S & B   & S,C \l4
B   & -   & B \l3
S,A & - \l2
S,S,C \l1
\end{tabular}

\bWortInSprache{abaab}
\end{bAntwort}

\end{document}
