\documentclass{bschlangaul-aufgabe}
\bLadePakete{formale-sprachen}
\begin{document}
\bAufgabenMetadaten{
  Titel = {Mindestens eine Eins},
  Thematik = {0-1},
  Referenz = THEO.Formale-Sprachen.Typ-2_Kontextfrei.Grammatik.0-1,
  RelativerPfad = Module/70_THEO/10_Formale-Sprachen/20_Typ-2_Kontextfrei/Grammatik/Aufgabe_0-1.tex,
  ZitatSchluessel = theo:ab:5,
  ZitatBeschreibung = {Aufgabe 2a 1.},
  BearbeitungsStand = mit Lösung,
  Korrektheit = unbekannt,
  Ueberprueft = {unbekannt},
  Stichwoerter = {Kontextfreie Grammatik},
}

Geben Sie für die folgenden Sprachen eine kontextfreie Grammatik an:
\index{Kontextfreie Grammatik}
\footcite[Aufgabe 2a 1.]{theo:ab:5}

Die Wörter bestehen aus beliebig vielen Nullen, einer 1 sowie weiteren
beliebig vielen Nullen oder Einsen.

\begin{bAntwort}
\begin{bProduktionsRegeln}
S -> A 1 A,
A -> 0 A | 1 A | EPSILON,
\end{bProduktionsRegeln}
\bFlaci{G539t1rgc}
\end{bAntwort}

\end{document}
