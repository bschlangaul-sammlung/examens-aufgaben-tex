\documentclass{bschlangaul-aufgabe}
\bLadePakete{formale-sprachen,automaten}
\begin{document}
\bAufgabenMetadaten{
  Titel = {Vorlesungsaufgabe Kellerautomaten},
  Thematik = {Nonterminal: P, Terminale: 01},
  Referenz = THEO.Formale-Sprachen.Typ-2_Kontextfrei.Kellerautomat.Vorlesungsaufgabe-2,
  RelativerPfad = Module/70_THEO/10_Formale-Sprachen/20_Typ-2_Kontextfrei/Kellerautomat/Aufgabe_Vorlesungsaufgabe-2.tex,
  ZitatSchluessel = theo:fs:2,
  ZitatBeschreibung = {Seite 27},
  BearbeitungsStand = mit Lösung,
  Korrektheit = korrekt,
  Ueberprueft = {unbekannt},
  Stichwoerter = {Kellerautomat},
}

\let\m=\bMenge
\let\z=\bZustandsnameTiefgestellt

Erstellen Sie einen Kellerautomaten zu der Grammatik
\bGrammatik{variablen={S}, alphabet={0, 1}} mit den folgenden
Produktionsregeln\footcite[Seite 27]{theo:fs:2}
\index{Kellerautomat}

\begin{enumerate}

%%
%
%%

\item

\begin{bProduktionsRegeln}
S -> EPSILON | 0 | 1 | 0 S 0 | 1 S 1,
\end{bProduktionsRegeln}

\begin{bAntwort}
\bKellerAutomat{
  zustaende={\z0, \z1},
  alphabet={0, 1},
  kelleralphabet={\#, N, E},
  ende={z_1},
}

N = Null

E = Eins

\begin{center}
\begin{tikzpicture}[li kellerautomat]
  \node[state,initial] (z0) at (3.57cm,-3.57cm) {$z_0$};
  \node[state,accepting] (z1) at (6.86cm,-3.57cm) {$z_1$};

  \bKellerKante[above,loop]{z0}{z0}{
    0, KELLERBODEN, N KELLERBODEN;
    0, N, N N;
    0, E, N E;
    1, KELLERBODEN, E KELLERBODEN;
    1, E, E E;
    1, N, N E;
  }

  \bKellerKante[above]{z0}{z1}{
    EPSILON, KELLERBODEN, EPSILON;
    EPSILON, N, N;
    EPSILON, E, E;
    1, N, N;
    1, E, E;
    0, E, E;
    0, N, N;
  }

  \bKellerKante[above,loop]{z1}{z1}{
    0, N, EPSILON;
    1, E, EPSILON;
  }
\end{tikzpicture}
\end{center}

\bFlaci{Ahij8jnn7}
\end{bAntwort}

%%
%
%%

\item

\begin{bProduktionsRegeln}
S -> A 1 B,
A -> 0 A | EPSILON,
B -> 0 B | 1 B | EPSILON,
\end{bProduktionsRegeln}
\end{enumerate}

\begin{bAntwort}
\bKellerAutomat{
  zustaende={\z0, \z1},
  alphabet={0, 1},
  kelleralphabet={\#, E},
  ende={z_1},
}

\noindent
E = Eins ist gesetzt

\begin{center}
\begin{tikzpicture}[li kellerautomat]
  \node[state,initial] (z0) at (4cm,-8cm) {$z_0$};
  \node[state,accepting] (z1) at (7.86cm,-8cm) {$z_1$};

  \bKellerKante[above,loop]{z0}{z0}{
    0, KELLERBODEN, KELLERBODEN;
    1, KELLERBODEN, E KELLERBODEN;
  }

  \bKellerKante[above]{z0}{z1}{
    0, E, EPSILON;
    1, E, EPSILON;
    EPSILON, E, EPSILON;
  }

  \bKellerKante[above,loop]{z1}{z1}{
    0, KELLERBODEN, KELLERBODEN;
    1, KELLERBODEN, KELLERBODEN;
    EPSILON, KELLERBODEN, EPSILON;
  }
\end{tikzpicture}
\end{center}

\bFlaci{Ar3imp8a7}
\end{bAntwort}

\end{document}
