\documentclass{bschlangaul-aufgabe}
\bLadePakete{automaten,mathe,formale-sprachen}
\begin{document}
\bAufgabenMetadaten{
  Titel = {PDA},
  Thematik = {0n 1n, gleich viele ab, kein Präfix mehr Einsen},
  Referenz = THEO.Formale-Sprachen.Typ-2_Kontextfrei.Kellerautomat.PDA,
  RelativerPfad = Module/70_THEO/10_Formale-Sprachen/20_Typ-2_Kontextfrei/Kellerautomat/Aufgabe_PDA.tex,
  ZitatSchluessel = theo:ab:2,
  ZitatBeschreibung = {Aufgabe 7: PDA aufstellen},
  BearbeitungsStand = mit Lösung,
  Korrektheit = unbekannt,
  Ueberprueft = {unbekannt},
  Stichwoerter = {Kellerautomat},
}

\let\m=\bMengeOhneMathe
\def\z#1{$z_#1$}

Erstellen Sie jeweils einen PDA, der die angegebenen Sprachen akzeptiert.
\index{Kellerautomat}
\footcite[Aufgabe 7: PDA aufstellen]{theo:ab:2}

\begin{enumerate}

%%
% (a)
%%

\item \bAusdruck{0^n 1^n}{n \in \mathbb{N}}

\begin{bAntwort}
\bKellerAutomat{
  zustaende={z_0, z_1, z_2, z_3},
  alphabet={0, 1},
  kelleralphabet={\#, N},
  ende={z_3},
}

N = Null

\begin{center}
\begin{tikzpicture}[li kellerautomat]
  \node[state,initial] (z0) at (4.43cm,-4cm) {$z_0$};
  \node[state] (z1) at (5.86cm,-2.29cm) {$z_1$};
  \node[state] (z2) at (8.71cm,-2.29cm) {$z_2$};
  \node[state,accepting] (z3) at (10.29cm,-4.14cm) {$z_3$};

  \bKellerKante[left]{z0}{z1}{
    0, KELLERBODEN, N KELLERBODEN;
  }

  \bKellerKante[above]{z1}{z2}{
    1, N, EPSILON;
  }

  \bKellerKante[above,loop]{z1}{z1}{
    0, N, N N;
  }

  \bKellerKante[right]{z2}{z3}{
    EPSILON, KELLERBODEN, EPSILON;
  }

  \bKellerKante[above,loop]{z2}{z2}{
    1, N, EPSILON;
  }
\end{tikzpicture}
\end{center}

\bFlaci{Aji1r8mf7}
\end{bAntwort}

%%
% (b)
%%

\item Alle Wörter, die gleich viele $a$ wie $b$ enthalten.

\begin{bAntwort}
\bKellerAutomat{
  zustaende={z_0, z_1, z_2, z_3},
  alphabet={a, b},
  kelleralphabet={\#, A, B},
  ende={z_3},
}

\begin{center}
\begin{tikzpicture}[li kellerautomat]
  \node[state,initial] (z0) at (2.29cm,-4cm) {$z_0$};
  \node[state] (z1) at (6cm,-1.86cm) {$z_1$};
  \node[state] (z2) at (7.29cm,-5.71cm) {$z_2$};
  \node[state,accepting] (z3) at (1cm,-1.43cm) {$z_3$};

  \bKellerKante[left,bend left]{z0}{z1}{
    a, KELLERBODEN, A KELLERBODEN;
  }

  \bKellerKante[above right,bend left=10]{z0}{z2}{
    b, KELLERBODEN, B KELLERBODEN;
  }

  \bKellerKante[left]{z0}{z3}{
    EPSILON, KELLERBODEN, EPSILON;
  }

  \bKellerKante[right,bend left]{z1}{z0}{
    b, A, EPSILON;
  }

  \bKellerKante[above,loop]{z1}{z1}{
    a, A, A A;
    b, A, EPSILON;
  }

  \bKellerKante[above,loop]{z2}{z2}{
    b, B, B B;
    a, B, EPSILON;
  }

  \bKellerKante[below left,bend left=10]{z2}{z0}{
    a, B, EPSILON;
  }
\end{tikzpicture}
\end{center}

\bFlaci{Arar6fos7}
\end{bAntwort}

%%
% (c)
%%

\item Alle Wörter, bei denen kein Präfix mehr Einsen wie Nullen hat.

\begin{bAntwort}
\bKellerAutomat{
  zustaende={z_0, z_1, z_2},
  alphabet={0, 1},
  kelleralphabet={\#, N},
  ende={z_1},
}

akzeptiert:

\begin{itemize}
\item 000
\item 00101
\item 01
\end{itemize}

nicht akzeptiert:

\begin{itemize}
\item 011
\item 111
\item 1
\end{itemize}

\begin{center}
\begin{tikzpicture}[li kellerautomat]
  \node[state,initial] (z0) at (2.43cm,-6.43cm) {$z_0$};
  \node[state] (z2) at (7.43cm,-6.43cm) {$z_2$};
  \node[state,accepting] (z1) at (5.57cm,-4.14cm) {$z_1$};

  \bKellerKante[above,bend left]{z0}{z2}{
    0, KELLERBODEN, N KELLERBODEN;
  }

  \bKellerKante[above left]{z0}{z1}{
    0, KELLERBODEN, EPSILON;
    EPSILON, KELLERBODEN, EPSILON;
  }

  \bKellerKante[above,loop below]{z0}{z0}{
    EPSILON, N, EPSILON;
  }

  \bKellerKante[above,loop]{z2}{z2}{
    1, N, EPSILON;
    0, N, N N;
  }

  \bKellerKante[below,bend left]{z2}{z0}{
    0, N, N;
    EPSILON, KELLERBODEN, KELLERBODEN;
    0, KELLERBODEN, EPSILON;
    EPSILON, N, EPSILON;
  }
\end{tikzpicture}
\end{center}

\bFlaci{Af7rfyqqg}
\end{bAntwort}
\end{enumerate}
\end{document}
