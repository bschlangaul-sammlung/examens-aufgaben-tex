\documentclass{bschlangaul-aufgabe}
\bLadePakete{mathe,formale-sprachen}
\begin{document}
\bAufgabenMetadaten{
  Titel = {Pumping-Lemma},
  Thematik = {„wn2“ „an bm cn“},
  Referenz = THEO.Formale-Sprachen.Typ-3_Regulaer.Pumping-Lemma.Pumping-Lemma,
  RelativerPfad = Module/70_THEO/10_Formale-Sprachen/10_Typ-3_Regulaer/Pumping-Lemma/Aufgabe_Pumping-Lemma.tex,
  ZitatSchluessel = theo:ab:1,
  BearbeitungsStand = mit Lösung,
  Korrektheit = unbekannt,
  Ueberprueft = {unbekannt},
  Stichwoerter = {Pumping-Lemma (Reguläre Sprache)},
}

\let\m=\bMenge

Begründe jeweils, ob die folgenden Sprachen regulär sind oder nicht.
\index{Pumping-Lemma (Reguläre Sprache)}
\footcite{theo:ab:1}
\bFussnoteUrl{https://www.uni-muenster.de/Informatik/u/lammers//EDU/ws08/AutomatenFormaleSprachen/Loesungen/Loesung05.pdf}

\begin{enumerate}

%%
% a)
%%

\item \bAusdruck[L_1]
{w \in \m{a, b}^* }
{\text{ auf ein } a \text{ folgt immer ein } b}

\begin{bAntwort}
$L_1 = L(b^* (ab)^* b^*)$ und damit regulär.
\end{bAntwort}

%%
% b)
%%

\item \bAusdruck[L_2]
{w \in \{ 1 \}^*}
{\exists n \in \mathbb{N} \text{ mit } |w| = n^2}

\begin{bAntwort}
$L_2$ ist nicht regulär.

\bPseudoUeberschrift{Pumping-Lemma:}

$j$ sei eine Quadratzahl: Somit ist $1^j \in L_2$. Es gilt $|uv| \leq j$
und $|v| \geq 1$. Daraus folgt, dass in $v$ mindestens eine $1$
existiert. Somit wird immer ein $i \in \mathbb{N}$ existieren, sodass
$uv^iw \notin L$, weil die Quadratzahlen nicht linear darstellbar sind.

\bPseudoUeberschrift{Begründung über die Zahlentheorie:}

Angenommen, $L_2$ sei regulär, sei $m$ die kleinste Zahl mit $m^2 > j$.
Dann ist $x = 1^{m^2} \in L_2$. Für eine Zerlegung $x = uvw$ nach dem
Pumping-Lemma muss dann ein $k$ existieren mit $v = 1^k$ und $m^2 - l +
k^l$ ist eine Quadratzahl für jedes $l \geq 0$. Das kann offenbar
zahlentheoretisch nicht sein, und somit haben wir einen Widerspruch zur
Annahme.
\end{bAntwort}

%%
% c)
%%

\item \bAusdruck[L_3]{a^n b^m c^n}{m, n \in \mathbb{N}_0}

\begin{bAntwort}
\bAusdruck[L_3]{a^n b^m c^n}{m, n \in \mathbb{N}_0} ist nicht regulär.

$a^j b^j c^j \in L_3$:

$|uv| \leq j$ und $|v| \geq 1$

\begin{itemize}
\item[$\rightarrow$]
in $uv$ sind nur $a$’s und in $v$ ist mindestens ein $a$

\item[$\rightarrow$]
$u v^2 w \notin L_3$, weil dann mehr $a$’s als $c$’s in diesem Wort
vorkommen
\end{itemize}
\end{bAntwort}

%%
% d)
%%

\item \bAusdruck[L_4]{w \in \{a\}^*}{\mod_3(|w|) = 0}

\begin{bAntwort}
$L_4 = ((aaa)^*)$ und damit regulär.
\end{bAntwort}

\end{enumerate}
\end{document}
