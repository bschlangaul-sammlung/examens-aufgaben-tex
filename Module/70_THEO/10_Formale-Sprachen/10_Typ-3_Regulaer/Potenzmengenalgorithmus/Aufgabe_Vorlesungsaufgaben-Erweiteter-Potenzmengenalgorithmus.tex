\documentclass{bschlangaul-aufgabe}
\bLadePakete{automaten,formale-sprachen}
\begin{document}
\bAufgabenMetadaten{
  Titel = {Erweiterter Potenzmengenalgorithmus e-NEA zum DEA},
  Thematik = {NEA: z01234, Alphabet: ab},
  Referenz = THEO.Formale-Sprachen.Typ-3_Regulaer.Potenzmengenalgorithmus.Vorlesungsaufgaben-Erweiteter-Potenzmengenalgorithmus,
  RelativerPfad = Module/70_THEO/10_Formale-Sprachen/10_Typ-3_Regulaer/Potenzmengenalgorithmus/Aufgabe_Vorlesungsaufgaben-Erweiteter-Potenzmengenalgorithmus.tex,
  ZitatSchluessel = theo:fs:1,
  ZitatBeschreibung = {Seite 50},
  BearbeitungsStand = mit Lösung,
  Korrektheit = unbekannt,
  Ueberprueft = {unbekannt},
  Stichwoerter = {Erweiteter Potenzmengenalgorithmus},
}

Gegeben ist der folgende $\varepsilon$-NEA:
\index{Erweiteter Potenzmengenalgorithmus}
\footcite[Seite 50]{theo:fs:1}

\begin{center}
\begin{tikzpicture}[->,node distance=3cm]
\node[state,initial]              (0) {$z_0$};
\node[state,below right of=0]     (1) {$z_1$};
\node[state,right of=0]           (2) {$z_2$};
\node[state,below of=0]           (3) {$z_3$};
\node[state,right of=2,accepting] (4) {$z_4$};

\path (0) edge[above] node{$\varepsilon$} (2);
\path (0) edge[right] node{a} (1);
\path (0) edge[left] node{b} (3);
\path (1) edge[right] node{$\varepsilon$} (2);
\path (2) edge[above,loop] node{c} (2);
\path (2) edge[above] node{$\varepsilon$} (4);
\path (3) edge[above] node{$\varepsilon$} (1);
\end{tikzpicture}
\end{center}

Wandlen Sie den gegebenen Automaten in eine $\varepsilon$-freien DEA um.

\begin{bAntwort}
\let\p=\bPotenzmenge

\begin{tabular}{lllll}
Name & Zustandsmenge & Eingabe a & Eingabe b & Eingabe c \\\hline\hline
$z'_0$ &
\p{z_0, z_2, z_4} &
\p{z_1, z_2, z_4} &
\p{z_1, z_2, z_3, z_4} &
\p{z_2, z_4} \\

$z'_1$ &
\p{z_1, z_2, z_4} &
\p{} &
\p{} &
\p{z_2, z_4} \\

$z'_2$ &
\p{z_1, z_2, z_3, z_4} &
\p{} &
\p{} &
\p{z_2, z_4} \\

$z'_3$ &
\p{z_2, z_4} &
\p{} &
\p{} &
\p{z_2, z_4} \\

$z'_t$ &
\p{} &
\p{} &
\p{} &
\p{} \\
\end{tabular}

\begin{center}
\begin{tikzpicture}[->,node distance=3cm]
\node[state,initial,accepting,label=south:\p{z_0, z_2, z_4}]
(0) {$z'_0$};

\node[state,above of=0,accepting,label=west:\p{z_1, z_2, z_4}]
(1) {$z'_1$};

\node[state,above right=2.5cm of 0,accepting,label=100:\p{z_{1234}}]
(2) {$z'_2$};

\node[state,right=4cm of 0,accepting,label=south:\p{z_2, z_4}]
(3) {$z'_3$};

\node[state,above right=3cm of 1,label=east:\p{}]
(t) {$z'_t$};

\path (0) edge[right] node{a} (1);
\path (0) edge[above] node{b} (2);
\path (0) edge[above] node{c} (3);

\path (1) edge[left] node{a,b} (t);
\path (1) edge[above] node{} (3);

\path (2) edge[right] node{a,b} (t);
\path (2) edge[above] node{c} (3);

\path (3) edge[right] node{a,b} (t);
\path (3) edge[above,loop right] node{c} (3);

\path (t) edge[above,loop] node{a,b,c} (t);
\end{tikzpicture}
\end{center}
\end{bAntwort}

\end{document}
