\documentclass{bschlangaul-aufgabe}
\bLadePakete{automaten,formale-sprachen}

% Info_2021-02-26-2021-02-26_13.01.25.mp4 3h02m30
\begin{document}
\bAufgabenMetadaten{
  Titel = {Erweiterter Potenzmengenalgorithmus epsilon-NEA zum DEA},
  Thematik = {NEA: z012, Alphabet: abc},
  Referenz = THEO.Formale-Sprachen.Typ-3_Regulaer.Potenzmengenalgorithmus.Erweiteter-Potenzmengenalgorithmus,
  RelativerPfad = Module/70_THEO/10_Formale-Sprachen/10_Typ-3_Regulaer/Potenzmengenalgorithmus/Aufgabe_Erweiteter-Potenzmengenalgorithmus.tex,
  ZitatSchluessel = theo:fs:1,
  ZitatBeschreibung = {Seite 47-49},
  BearbeitungsStand = mit Lösung,
  Korrektheit = unbekannt,
  Ueberprueft = {unbekannt},
  Stichwoerter = {Erweiteter Potenzmengenalgorithmus},
}

\begin{center}
\begin{tikzpicture}[->,node distance=3cm]
\node[state,initial] (0) {$z0$};
\node[state,right of=0] (1) {$z1$};
\node[state,right of=1,accepting] (2) {$z2$};

\path (0) edge[above] node{$\varepsilon$} (1);
\path (1) edge[above] node{$\varepsilon$} (2);
\path (0) edge[above,loop] node{a} (0);
\path (1) edge[above,loop] node{b} (1);
\path (2) edge[above,loop] node{c} (3);
\end{tikzpicture}
\end{center}

\begin{enumerate}

%-----------------------------------------------------------------------
%
%-----------------------------------------------------------------------

\item Welche Sprache akzeptiert dieser Automat? Beschreiben Sie in
Worten und stellen Sie einen regulären Ausdruck sowie eine Grammatik
hierfür auf.
\index{Erweiteter Potenzmengenalgorithmus}
\footcite[Seite 47-49]{theo:fs:1}

\begin{bAntwort}
\begin{description}

%%
%
%%

\item[in Worten]

Das Alphabet besteht aus $a$, $b$, $c$. Am Anfang stehen $0$ oder
beliebig viele $a$’s, dann kommen $0$ oder beliebig viele $b$’s und dann
$0$ oder beliebig viele $c$’s.

%%
%
%%

\item[Regulärer Ausdruck]

$a^*b^*c^*$

%%
%
%%

\item[Grammatik] \strut

\begin{bProduktionsRegeln}
S -> a S | b A | c B | EPSILON,
A -> b A | c B | EPSILON,
B -> c B | EPSILON
\end{bProduktionsRegeln}
\end{description}
\end{bAntwort}

%-----------------------------------------------------------------------
%
%-----------------------------------------------------------------------

\item Wandeln Sie den $\varepsilon$-NEA zum einem DEA mit Hilfe des
erweiterter Potenzmengenalgorithmus um.

\begin{bAntwort}
\let\p=\bPotenzmenge
\let\s=\bZustandsnameGross

\begin{tabular}{l|l|l|l|l}
Name & Zustandsmenge & Eingabe a & Eingabe b & Eingabe c\\\hline\hline
\s{0} &
\p{z0, z1, z2} &
\p{z0, z1, z2} &
\p{z1, z2} &
\p{z2} \\

\s{1} &
\p{z1, z2} &
\p{} &
\p{z1, z2} &
\p{z2} \\

\s{2} &
\p{z2} &
\p{} &
\p{} &
\p{z2} \\
\end{tabular}

Trap-Übergänge werden aus Übersichtsgründen weg gelassen.

\begin{center}
\begin{tikzpicture}[->,node distance=3cm]
\node[state,initial,accepting,label=south:\p{z0, z1, z2}] (0) {$z'_0$};
\node[state,above right of=0,accepting,label=south:\p{z1, z2}] (1) {$z'_1$};
\node[state,below right of=1,accepting,label=south:\p{z2}] (2) {$z'_2$};

\path (0) edge[above,loop] node{a} (0);
\path (0) edge[above] node{b} (1);
\path (0) edge[above] node{c} (2);
\path (1) edge[above,loop] node{b} (1);
\path (1) edge[above] node{c} (2);
\path (2) edge[above,loop] node{c} (2);
\end{tikzpicture}
\end{center}
\end{bAntwort}

\end{enumerate}

\end{document}
