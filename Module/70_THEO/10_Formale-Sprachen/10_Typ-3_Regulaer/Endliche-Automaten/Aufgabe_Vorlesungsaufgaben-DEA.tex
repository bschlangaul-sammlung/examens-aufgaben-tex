\documentclass{bschlangaul-aufgabe}
\bLadePakete{mathe,automaten,formale-sprachen}
\begin{document}
\bAufgabenMetadaten{
  Titel = {Deterministisch endlicher Automat},
  Thematik = {Vorlesungsaufgaben},
  Referenz = THEO.Formale-Sprachen.Typ-3_Regulaer.Endliche-Automaten.Vorlesungsaufgaben-DEA,
  RelativerPfad = Module/70_THEO/10_Formale-Sprachen/10_Typ-3_Regulaer/Endliche-Automaten/Aufgabe_Vorlesungsaufgaben-DEA.tex,
  ZitatSchluessel = theo:fs:1,
  ZitatBeschreibung = {Seite 28},
  BearbeitungsStand = mit Lösung,
  Korrektheit = unbekannt,
  Ueberprueft = {unbekannt},
  Stichwoerter = {Deterministisch endlicher Automat (DEA)},
}

Stellen Sie einen Automaten zu den folgenden Sprachen (\bAlphabet{a,b})
auf:\index{Deterministisch endlicher Automat (DEA)}\footcite[Seite 28]{theo:fs:1}

\begin{enumerate}

%%
%
%%

% https://flaci.com/A5415d28w
\item \bAusdruck[L_1]{x }{ x \text{ beinhaltet eine gerade Anzahl von } a}

\begin{bAntwort}
\begin{center}
\begin{tikzpicture}[->,node distance=2cm]
\node[state,initial,accepting] (0) {$z_0$};
\node[state,right of=0] (1) {$z_1$};

\path (0) edge[above,bend left] node{a} (1);
\path (1) edge[above,bend left] node{a} (0);
\path (0) edge[above,loop] node{b} (0);
\path (1) edge[above,loop] node{b} (1);
\end{tikzpicture}
\end{center}
\end{bAntwort}

%%
%
%%

% https://flaci.com/Arozb1437
\item \bAusdruck[L_2]{x }{ x \text{ beinhaltet eine ungerade Anzahl von } b}

\begin{bAntwort}
\begin{center}
\begin{tikzpicture}[->,node distance=2cm]
\node[state,initial] (0) {$z_0$};
\node[state,right of=0,accepting] (1) {$z_1$};

\path (0) edge[above,bend left] node{b} (1);
\path (1) edge[above,bend left] node{b} (0);
\path (0) edge[above,loop] node{a} (0);
\path (1) edge[above,loop] node{a} (1);
\end{tikzpicture}
\end{center}
\end{bAntwort}

%%
%
%%

% https://flaci.com/Ajiq3o8ma
\item Geben Sie einen DEA an, der eine syntaktisch gültige
E-Mail-Adresse erkennt. (mindestens 1 Zeichen (Groß-/Kleinbuchstabe oder
Zahl) vor dem \texttt{@}; mindestens 1 Zeichen (Groß-/Kleinbuchstabe
oder Zahl) nach dem \texttt{@}; alle E-Mail-Adressen sollen auf
\texttt{.de} oder \texttt{.com} enden.

\begin{bAntwort}
Es müsste noch der Trap-Zustand hinzugefügt werden.
\begin{center}

\begin{tikzpicture}[->,node distance=1.8cm]
\node[state,initial] (0) {$z_0$};
\node[state,right of=0] (1) {$z_1$};
\node[state,right of=1] (2) {$z_2$};
\node[state,right of=2] (3) {$z_3$};
\node[state,right of=3,node distance=3cm,accepting] (4) {$z_4$};
\path (0) edge[above] node{\string\w} (1);
\path (1) edge[above] node{@} (2);
\path (1) edge[above,loop] node{\string\w} (1);
\path (2) edge[above] node{\string\w} (3);
\path (3) edge[above,loop] node{\string\w} (3);
\path (3) edge[above] node{.de,.com} (4);
\end{tikzpicture}
\end{center}
\end{bAntwort}

\end{enumerate}
\end{document}
