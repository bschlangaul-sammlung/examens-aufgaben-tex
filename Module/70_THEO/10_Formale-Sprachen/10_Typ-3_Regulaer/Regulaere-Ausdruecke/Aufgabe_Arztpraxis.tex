\documentclass{bschlangaul-aufgabe}
\bLadePakete{java}
\begin{document}
\bAufgabenMetadaten{
  Titel = {Aufgabe 4: Reguläre Ausdrücke in JAVA:},
  Thematik = {Arztpraxis und Autohauskette},
  Referenz = THEO.Formale-Sprachen.Typ-3_Regulaer.Regulaere-Ausdruecke.Arztpraxis,
  RelativerPfad = Module/70_THEO/10_Formale-Sprachen/10_Typ-3_Regulaer/Regulaere-Ausdruecke/Aufgabe_Arztpraxis.tex,
  ZitatSchluessel = theo:ab:1,
  BearbeitungsStand = mit Lösung,
  Korrektheit = unbekannt,
  Ueberprueft = {unbekannt},
  Stichwoerter = {Reguläre Ausdrücke},
}

Abitur 2018 Inf2. IV. 1. und Abitur 2015 Inf2. III. 2.

\begin{enumerate}

%%
% (a)
%%

\item Die Software einer Arztpraxis ermöglicht unter anderem die
Erstellung von Rechnungen für die Patienten. Dabei muss eine
Rechnungskennzahl angegeben werden, die wie folgt aufgebaut ist:
\index{Reguläre Ausdrücke}
\footcite{theo:ab:1}

\begin{itemize}
\item zwei Buchstaben für die Initialen des Patienten (erster Buchstabe
des Nachnamens gefolgt vom ersten Buchstaben des Vornamens)

\item Bindestrich

\item Patientennummer beliebiger Länge, die aus den Ziffern 0 bis 9
besteht, aber nicht mit 0 beginnt

\item Versicherungsart: P bei Privatpatienten, G bei gesetzlich
Versicherten

\item nur bei gesetzlich Versicherten: zweistellige
Versicherungskennzahl (Nummern im Bereich von 01 bis 12)

\item Bindestrich

\item fortlaufende Rechnungsnummer, die aus beliebig vielen Ziffern von
0 bis 9 besteht, aber nicht mit 0 beginnt

\end{itemize}

Beispiele:

Privatpatient Ingo Matik mit der Patientennummer 32 erhält seine 9.
Rechnung. Die zugehörige Rechnungskennzahl ist: MI-32P-9.

Seine Frau Martha Matik mit der Patientennummer 1234, die gesetzlich bei
einer Versicherung mit der Kennzahl 07 versichert ist, erhält ihre 12.
Rechnung. Die zugehörige Rechnungskennzahl ist: MM-1234G07-12. Für die
Darstellung der Rechnungskennzahl stehen das lateinische Alphabet der
Großbuchstaben, die Ziffern 0 bis 9 und der Bindestrich zur Verfügung.

Stelle einen regulären Ausdruck für die Rechnungskennzahl in
Java-Schreibweise auf.

\begin{bAntwort}
\bJavaDatei[firstline=3]{aufgaben/theo_inf/regulaere_ausdruecke/KrankenversicherungsNummer}
\end{bAntwort}

%%
% (b)
%%

\item Um in die zentrale Personalabteilung der Autohauskette zu
gelangen, muss man vor der Sicherheitstür ein aus genau drei Zeichen
bestehendes Passwort eingeben. Dieses wird jährlich gemäß den
folgenden Vorgaben der Firma festgelegt:

\begin{itemize}
\item Das Passwort muss mindestens einen Kleinbuchstaben und eine Ziffer
enthalten;

\item es darf nicht mit einem der 32 Sonderzeichen (\zB*, §, ...)
beginnen;

\item Großbuchstaben sind an keiner Stelle zugelassen.

Stelle einen regulären Ausdruck für die Rechnungskennzahl in
Java-Schreibweise auf.
\end{itemize}

\begin{bAntwort}
\bJavaDatei[firstline=3]{aufgaben/theo_inf/regulaere_ausdruecke/PasswortTuer}
\end{bAntwort}

\end{enumerate}
\end{document}
