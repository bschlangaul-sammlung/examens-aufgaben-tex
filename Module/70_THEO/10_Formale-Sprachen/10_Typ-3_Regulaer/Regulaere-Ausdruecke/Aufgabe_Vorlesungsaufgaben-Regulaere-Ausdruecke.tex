\documentclass{bschlangaul-aufgabe}
\bLadePakete{java,mathe,formale-sprachen}
\begin{document}
\bAufgabenMetadaten{
  Titel = {Übungen zu regulären Ausdrücken},
  Thematik = {Vorlesungsaufgaben},
  Referenz = THEO.Formale-Sprachen.Typ-3_Regulaer.Regulaere-Ausdruecke.Vorlesungsaufgaben-Regulaere-Ausdruecke,
  RelativerPfad = Module/70_THEO/10_Formale-Sprachen/10_Typ-3_Regulaer/Regulaere-Ausdruecke/Aufgabe_Vorlesungsaufgaben-Regulaere-Ausdruecke.tex,
  ZitatSchluessel = theo:fs:1,
  ZitatBeschreibung = {Seite 21},
  BearbeitungsStand = mit Lösung,
  Korrektheit = unbekannt,
  Ueberprueft = {unbekannt},
  Stichwoerter = {Reguläre Ausdrücke},
}

\begin{enumerate}

%%
%
%%

\item Gegeben ist eine Sprache $L \subset \Sigma^*$ mit
\bAlphabet{a,b}. Zu der Sprache $L$ gehören alle Wörter, die die
Zeichenfolge \texttt{abba} beinhalten.
\index{Reguläre Ausdrücke}\footcite[Seite 21]{theo:fs:1}

Geben Sie einen regulären Ausdruck für diese Sprache an („klassischer“
regulärer Ausdruck).

\begin{bAntwort}
$(a|b)^*abba(a|b)^*$

\bJavaDatei[firstline=5,lastline=5]{aufgaben/theo_inf/regulaere_ausdruecke/TestRegularExpressions}
\end{bAntwort}

%%
%
%%

\item Gebe möglichst einfache reguläre Ausdrücke für die folgenden
Sprachen $L_x \subset \Sigma^*$ mit \bAlphabet{a,b} und $x \in \{1, 2, 3\}$ („klassischer“ regulärer Ausdruck).

\begin{description}

%%
%
%%

\item[$L_1$] = $\{ x | x \text{ beinhaltet eine gerade Anzahl von } a \}$

\begin{bAntwort}
$b^*(ab^*ab^*)^*$

\bJavaDatei[firstline=6,lastline=6]{aufgaben/theo_inf/regulaere_ausdruecke/TestRegularExpressions}
\end{bAntwort}

%%
%
%%

\item[$L_2$] = $\{ x | x \text{ beinhaltet eine ungerade Anzahl von } b \}$

\begin{bAntwort}
$a^*ba^*(ba^*ba^*)^*$

\bJavaDatei[firstline=7,lastline=7]{aufgaben/theo_inf/regulaere_ausdruecke/TestRegularExpressions}
\end{bAntwort}

%%
%
%%

\item[$L_3$] = $\{ x | x \text{ beinhaltet an seinen geradzahligen Positionen ausschließlich } a \}$

\begin{bAntwort}
$((a|b)a)^*(a^*|b)$

\bJavaDatei[firstline=8,lastline=8]{aufgaben/theo_inf/regulaere_ausdruecke/TestRegularExpressions}
\end{bAntwort}
\end{description}

%%
%
%%

\item Geben Sie einen regulären Ausdruck an, der eine syntaktisch
gültige E-Mail-Adresse erkennt. (mindestens 1 Zeichen
(Groß-/Kleinbuchstabe oder Zahl) vor dem \texttt{@}; mindestens 1
Zeichen (Groß-/Kleinbuchstabe oder Zahl) nach dem \texttt{@}; alle
E-Mail-Adressen sollen auf \texttt{.de} oder \texttt{.com} enden.

\begin{bAntwort}
\bJavaDatei[firstline=9,lastline=9]{aufgaben/theo_inf/regulaere_ausdruecke/TestRegularExpressions}
\end{bAntwort}
\end{enumerate}

\begin{bAntwort}
\bJavaDatei[firstline=3]{aufgaben/theo_inf/regulaere_ausdruecke/TestRegularExpressions}
\end{bAntwort}

\end{document}
