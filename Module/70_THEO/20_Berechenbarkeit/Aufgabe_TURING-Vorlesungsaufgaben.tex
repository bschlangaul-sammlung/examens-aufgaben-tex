\documentclass{bschlangaul-aufgabe}
\bLadePakete{syntax}
\begin{document}
\bAufgabenMetadaten{
  Titel = {Vorlesungsaufgaben},
  Thematik = {Vorlesungsaufgaben},
  Referenz = THEO.Berechenbarkeit.TURING-Vorlesungsaufgaben,
  RelativerPfad = Module/70_THEO/20_Berechenbarkeit/Aufgabe_TURING-Vorlesungsaufgaben.tex,
  ZitatSchluessel = theo:fs:4,
  ZitatBeschreibung = {Seite 29},
  BearbeitungsStand = nur Angabe,
  Korrektheit = unbekannt,
  Ueberprueft = {unbekannt},
  Stichwoerter = {Berechenbarkeit},
}

\begin{enumerate}
\item Zeige, dass es nur abzählbar viele Turingmaschinen gibt.\index{Berechenbarkeit}
\footcite[Seite 29]{theo:fs:4}

\item Turing-berechenbar

\begin{enumerate}
\item Definiere eine berechenbare Funktion $f: N \rightarrow N$ mit
entscheidbarem

\item Definitionsbereich und unentscheidbarem Wertebereich. Untersuche
folgende Aussagen

\begin{enumerate}

\item Jede berechenbare Funktion $h: N \rightarrow N$ mit endlichem
Wertebereich besitzt einen entscheidbaren Definitionsbereich.

\item Jede berechenbare Funktion $g: N \rightarrow N$ mit endlichem
Definitionsbereich besitzt einen entscheidbaren Wertebereich.
\end{enumerate}
\end{enumerate}
\end{enumerate}
\end{document}
