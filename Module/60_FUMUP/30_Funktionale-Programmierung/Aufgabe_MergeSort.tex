\documentclass{bschlangaul-aufgabe}
\bLadePakete{syntax}
\begin{document}
\bAufgabenMetadaten{
  Titel = {MergeSort},
  Thematik = {MergeSort},
  Referenz = FUMUP.Funktionale-Programmierung.MergeSort,
  RelativerPfad = Module/60_FUMUP/30_Funktionale-Programmierung/Aufgabe_MergeSort.tex,
  ZitatSchluessel = fumup:ab:3,
  BearbeitungsStand = mit Lösung,
  Korrektheit = unbekannt,
  Ueberprueft = {unbekannt},
  Stichwoerter = {Funktionale Programmierung mit Haskell},
}

Implementiere in der Datei Sortierverfahren.hs die nachfolgend genannten
Funktionen, die den Merge-Sort-Algorithmus umsetzen. Der
Sortieralgorithmus folgt dem Teile-und-Herrsche-Prinzip (divide and
conquer). Er erhält eine unsortierte Liste mit Zahlen, sortiert diese in
aufsteigender Reihenfolge und gibt die sortierte Liste zurück. Die
Signatur der Sortierfunktion lautet: \bHaskellCode{mergeSort :: [Int]
-> [Int]}\index{Funktionale Programmierung mit Haskell}
\footcite{fumup:ab:3}

\begin{enumerate}

%%
% (a)
%%

\item \bHaskellCode{divide :: [Int] -> ([Int], [Int])}: Nimmt eine
Liste entgegen und spaltet diese in der Mitte in zwei Teillisten auf.
(Tipp: Verwende zur Implementierung die Listen-Funktionen
\bHaskellCode{div}, \bHaskellCode{length}, \bHaskellCode{take} und
\bHaskellCode{drop}.)

%%
% (b)
%%

\item \bHaskellCode{conquer :: ([Int], [Int]) -> ([Int], [Int])}: Nimmt
als Parameter ein durch \bHaskellCode{divide} erzeugtes Listenpaar
entgegen und ruft für jede Teilliste \bHaskellCode{mergeSort}
(Rekursion!) auf. Das Ergebnis ist ein Listenpaar mit den sortierten
Teillisten.

%%
% (c)
%%

\item \bHaskellCode{merge :: ([Int], [Int]) -> [Int]}: Verschmilzt die
beiden als Listenpaar übergebenen Listen zu einer Ergebnisliste, die
alle Elemente beider Teillisten in aufsteigender Reihenfolge sortiert
enthält.

%%
% (d)
%%

\item \bHaskellCode{mergeSort :: [Int] -> [Int]}: Falls die übergebene
Liste weniger als 2 Elemente enthält wird sie unverändert zurückgegeben.
In allen anderen Fällen wird die übergebene Liste mittels der Funktion
\bHaskellCode{divide} zerteilt, deren Ergebnis an die Funktion
\bHaskellCode{conquer} übergeben und abschließend deren Ergebnis an die
Funktion \bHaskellCode{merge} übergeben. Das Ergebnis von
\bHaskellCode{mergeSort} ist dann eine vollständig sortierte Liste.

\end{enumerate}
\end{document}
