\documentclass{bschlangaul-aufgabe}

\begin{document}
\bAufgabenMetadaten{
  Titel = {Klasse und Objekt},
  Thematik = {Klasse und Objekt},
  Referenz = OOMUP.Objektorientierung.Klasse-und-Objekt,
  RelativerPfad = Module/20_OOMUP/Objektorientierung/Aufgabe_Klasse-und-Objekt.tex,
  ZitatSchluessel = oomup:ab:1,
  BearbeitungsStand = mit Lösung,
  Korrektheit = unbekannt,
  Ueberprueft = {unbekannt},
  Stichwoerter = {Klasse, Objekt, Objektorientierung},
}

Erlären Sie den Unterschied zwischen Klassen\index{Klasse} und
Objekten\index{Objekt}. Gehen Sie dabei auch auf deren Zusammenhang ein
und veranschaulichen Sie Ihre Antwort anhand eines konkreten Beispiels.
Geben Sie für dieses eine passende Klassen- sowie Objektkarte an und
beschriften Sie diese.\index{Objektorientierung}
\footcite{oomup:ab:1}

\begin{bAntwort}
Eine Klasse ist ein abstraktes Modell, eine Art Bauplan, während ein
Objekt eine konkrete Ausprägung dieser Klasse ist. Die Welt besteht aus
Objekten, es gibt jedoch kein Objekt, das nicht einer Klasse angehört.
Objekte werden während der Laufzeit eines Programms erzeugt und haben
alle die gleichen Attribute (nicht Attributwerte!) und Methoden, welche
zuvor im Rahmen der Klasse definiert wurden. Die Attributewerte
beschreiben den aktuellen Zustand eines Objekts. Objekte haben
eindeutige Namen.

Ein konkretes Beispiel wäre die Klasse \emph{Mensch} mit den Attributen
\emph{Vorname}, \emph{Name}, \emph{Größe}, \emph{Geburtsjahr} und
\emph{Haarfarbe} sowie den Methoden \emph{laufen} und \emph{atmen}.
Diese Attribute können bei Objekt mensch1 folgendermaßen ausgeprägt
sein:
\end{bAntwort}
\end{document}
