\documentclass{bschlangaul-aufgabe}
\bLadePakete{uml}
\begin{document}
\bAufgabenMetadaten{
  Titel = {Universitätsverwaltung},
  Thematik = {Universitätsverwaltung},
  Referenz = OOMUP.Diagramme.Struktur.Klassendiagramm.Universitätsverwaltung,
  RelativerPfad = Module/20_OOMUP/Diagramme/10_Struktur/10_Klassendiagramm/Aufgabe_Universitätsverwaltung.tex,
  ZitatSchluessel = net:pdf:uzh-zuerich:uebung-4,
  ZitatBeschreibung = {Seite 4},
  BearbeitungsStand = mit Lösung,
  Korrektheit = unbekannt,
  Ueberprueft = {unbekannt},
  Stichwoerter = {Klassendiagramm},
}

Gegeben ist der folgende Sachverhalt.
\index{Klassendiagramm}
\footcite[Seite 4]{net:pdf:uzh-zuerich:uebung-4}

Jede \textbf{Person} hat einen \emph{Namen}, eine \emph{Telefonnummer}
und \emph{E-Mail}.

Jede \textbf{Wohnadresse} wird von nur einer Person bewohnt. Es kann
aber sein, dass einige Wohnadressen nichtbewohnt sind. Den Wohnadressen
sind je eine \emph{Strasse}, eine \emph{Stadt}, eine \emph{PLZ} und ein
\emph{Land} zugeteilt. Alle Wohnadressen können bestätigt werden und als
Beschriftung (für Postversand) gedruckt werden.

Es gibt zwei Sorten von \textbf{Personen}: \textbf{Student}, welcher
sich für ein \emph{Modul einschreiben} kann und \textbf{Professor},
welcher einen \emph{Lohn} hat. Der Student besitzt eine
\emph{Matrikelnummer} und eine \emph{Durchschnittsnote}.

Modellieren Sie diesen Sachverhalt mit einem UML Klassendiagramm.

\begin{tikzpicture}
\umlclass[x=2,y=4]{Person}{
  + name: String\\
  + telefonNummer: String\\
  + eMail: String
}{}
\umlclass[]{Professor}{
  + lohn: double
}{}
\umlclass[x=5]{Student}{
  + matrikelNummer: int\\
  + durchschnittsNote: double
}{
  + einschreibenFuerModule()
}
\umlclass[x=10,y=4]{Wohnadresse}{
  + strasse: String \\
  + stadt: String \\
  + plz: int \\
  + land: String \\
}{
  + bestaetigen()\\
  + drucken()
}

\umlVHVinherit{Professor}{Person}
\umlVHVinherit{Student}{Person}
\umlassoc[arg1=adresse,arg2=person,stereo=wohnt,mult1=0..1,mult2=1,pos1=0.15,pos2=0.85,name=Person-Wohnadresse]{Person}{Wohnadresse}
\bUmlLeserichtung[pos=below left,dir=down,distance=0cm]{Person-Wohnadresse}
\end{tikzpicture}
\end{document}
