\documentclass{bschlangaul-aufgabe}
\bLadePakete{uml}
\begin{document}
\bAufgabenMetadaten{
  Titel = {Fußballmeisterschaft},
  Thematik = {Fußballmeisterschaft},
  Referenz = OOMUP.Diagramme.Struktur.Klassendiagramm.Fußballmeisterschaft,
  RelativerPfad = Module/20_OOMUP/Diagramme/10_Struktur/10_Klassendiagramm/Aufgabe_Fußballmeisterschaft.tex,
  ZitatSchluessel = net:html:tu-dortmund:uebung-softwaretechnik,
  ZitatBeschreibung = {Seite 1, Aufgae 2},
  BearbeitungsStand = mit Lösung,
  Korrektheit = unbekannt,
  Ueberprueft = {unbekannt},
  Stichwoerter = {Klassendiagramm},
}

Modellieren Sie die folgende Situation als Klassendiagramm:
\index{Klassendiagramm}
\footcite[Seite 1, Aufgae 2]{net:html:tu-dortmund:uebung-softwaretechnik}

Fußballmannschaften einer Liga bestreiten während einer
Meisterschaftsrunde Spiele gegen andere Mannschaften. Dabei werden in
jeder Mannschaft Spieler für einen bestimmten Zeitraum (in Minuten)
eingesetzt, die dabei eventuell Tore schießen.

Die Modellierung soll es ermöglichen, festzustellen, welcher Spieler in
welchem Spiel wie lange auf dem Feld war und wie viele Tore geschossen
hat. Ebenso soll es möglich sein, für jede Mannschaft festzustellen,
gegen welche Mannschaft welche Ergebnisse erzielt wurden.

\begin{bAntwort}
\begin{center}
\begin{tikzpicture}
\umlsimpleclass[y=5]{Liga}
\umlsimpleclass[y=3]{Mannschaft}
\umlsimpleclass[y=1]{Spieler}
\umlsimpleclass[x=5,y=3]{Spiel}

\umlassoc[mult2=*]{Liga}{Mannschaft}
\umlassoc[mult1=1,mult2=*]{Mannschaft}{Spieler}
\umlassoc[mult1=2,mult2=*,name=mannschaftspiel]{Mannschaft}{Spiel}
\umlHVassoc[mult1=*,mult2=*,pos2=1.7,name=spielerspiel]{Spieler}{Spiel}
\node[below=0cm of spielerspiel-2,font=\scriptsize] {nimmt teil};

\umlassocclass[x=2.5,y=5]{Saison}{mannschaftspiel-1}{- spieljahr}{}
\umlassocclass[x=8,y=1]{Teilnahme}{spielerspiel-2}{- minuten: int\\- tore: int}{}
\end{tikzpicture}
\end{center}
\end{bAntwort}

\end{document}
