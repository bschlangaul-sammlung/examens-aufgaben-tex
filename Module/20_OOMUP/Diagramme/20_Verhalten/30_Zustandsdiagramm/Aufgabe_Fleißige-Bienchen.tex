\documentclass{bschlangaul-aufgabe}

\begin{document}
\bAufgabenMetadaten{
  Titel = {Übung von Wikiversity},
  Thematik = {Restaurant „Fleißige Bienchen“},
  Referenz = OOMUP.Diagramme.Verhalten.Zustandsdiagramm.Fleißige-Bienchen,
  RelativerPfad = Module/20_OOMUP/Diagramme/20_Verhalten/30_Zustandsdiagramm/Aufgabe_Fleißige-Bienchen.tex,
  ZitatSchluessel = net:pdf:wikiversity:zustandsdiagramm,
  BearbeitungsStand = nur Angabe,
  Korrektheit = unbekannt,
  Ueberprueft = {unbekannt},
  Stichwoerter = {Zustandsdiagramm zeichnen},
}

Max und Sarah haben ein Ziel. Jeder Gast, der ihr Restaurant, das
„fleißige Bienchen“ besucht, soll so schnell wie möglich satt werden. Um
das zu erreichen arbeiten sie auf Hochtouren. Wenn ein Gast sein Essen
bestellt, wird das sofort in der Küche gehört. Max fängt direkt an das
Gericht zuzubereiten. Man hat es noch nicht bewiesen, aber
wahrscheinlich ist er der schnellste Koch der Welt und dementsprechend
dauert das Kochen nur wenige Minuten. Ist das Gericht fertig rennt Sarah
zu den Gästen und serviert es. Manchmal verliert sie dabei, weil sie so
schnell rennt, Teile des Essens. Dann muss sie zurück in die Küche, in
der das Gericht neu zubereitet wird. Wenn aber alles gut gegangen ist,
kann der Gast schnell essen. Gäste mit kleinem Hunger sind nach dem
Essen satt. Hungrige Gäste bestellen noch einen Nachtisch der schon
bereit steht, damit keine Zeit verloren geht. Ist dieser aufgegessen
sind auch die hungrigsten Gäste satt. Sarah und Max haben
herausgefunden, dass die Gäste beim Bezahlen der Rechnung viel mehr
Trinkgeld geben, wenn sie innerhalb von einer halben Stunde satt sind.
Aber egal, wie viel Geld sie bekommen. Max und Sarah sind glücklich über
jeden Gast der sie besucht.\index{Zustandsdiagramm zeichnen}
\footcite{net:pdf:wikiversity:zustandsdiagramm}

% https://www.isf.cs.tu-bs.de/cms/teaching/2012w/se1/solution6.pdf
\end{document}
