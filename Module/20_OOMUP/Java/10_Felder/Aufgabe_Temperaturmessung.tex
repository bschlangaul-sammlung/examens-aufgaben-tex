\documentclass{bschlangaul-aufgabe}
\bLadePakete{java}
\begin{document}
\bAufgabenMetadaten{
  Titel = {Temperaturmessung},
  Thematik = {Temperaturmessung},
  Referenz = OOMUP.Java.Felder.Temperaturmessung,
  RelativerPfad = Module/20_OOMUP/Java/10_Felder/Aufgabe_Temperaturmessung.tex,
  ZitatSchluessel = oomup:fs:3,
  ZitatBeschreibung = {Seite 61, Klett, Informatik 3, S. 68},
  BearbeitungsStand = mit Lösung,
  Korrektheit = unbekannt,
  Ueberprueft = {unbekannt},
  Stichwoerter = {Feld (Array), Implementierung in Java},
}

Steffi will ein Jahr lang jeden Tag um 15 Uhr die Temperatur auf ihrem
Balkon messen und die Ergebnisse auswerten. Dazu definiert sie eine
Klasse \bJavaCode{Tempmessung}.
\index{Feld (Array)}
\footcite[Seite 61, Klett, Informatik 3, S. 68]{oomup:fs:3}
\begin{enumerate}

%%
% a)
%%

\item Lege ein Feld \bJavaCode{temperatur} an, welches die reellen Werte für
jeden Tag eines Jahres aufnehmen kann. Definiere eine Methode, um das
Feld mit zufälligen Temperaturwerten zu belegen.

%%
% b)
%%

\item Nach genau einem Jahr sollen mithilfe dreier Methoden der Tag mit
dem höchsten Temperaturwert, die niedrigste gemessene Temperatur und der
Durchschnittswert aller Messwerte bestimmt werden. Implementiere
geeignete Methoden.\index{Implementierung in Java}

\begin{bAntwort}
\bJavaDatei{aufgaben/oomup/pu_3/Tempmessung}
\end{bAntwort}

\end{enumerate}

\end{document}
