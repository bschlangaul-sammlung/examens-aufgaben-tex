\documentclass{bschlangaul-aufgabe}

\begin{document}
\bAufgabenMetadaten{
  Titel = {Quadratisches Sondieren i 3i hoch 2},
  Thematik = {},
  Referenz = AUD.Baeume.Hashing.Sondieren_i-3i-hoch-2,
  RelativerPfad = Module/30_AUD/80_Baeume/60_Hashing/Aufgabe_Sondieren_i-3i-hoch-2.tex,
  BearbeitungsStand = mit Lösung,
  Korrektheit = unbekannt,
  Ueberprueft = {unbekannt},
  Stichwoerter = {Streutabellen (Hashing)},
}
\index{Streutabellen (Hashing)}

\footnote{nach Foliensatz der RWTH Aachen, Seite 19 \url{https://moves.rwth-aachen.de/wp-content/uploads/SS15/dsal/lec13.pdf}}

$h'(k) = k \mod 11$

$h(k, i) = (h'(k) + i + 3i^2) \mod 11$

$h'(17) = 17 \mod 11 = 6$

\bPseudoUeberschrift{Sondierungsfolgen}

$h(17, 0) = (17 + 0 + 3 \cdot 0^2) \mod 11 = 6$

$h(17, 1) = (17 + 1 + 3 \cdot 1^2) \mod 11 = 21 \mod 11 = 10$

$h(17, 2) = (17 + 2 + 3 \cdot 2^2) \mod 11 = 31 \mod 11 = 31 - 2 \cdot 11 = 9$

$h(17, 3) = (17 + 3 + 3 \cdot 3^2) \mod 11 = 47 \mod 11 = 47 - 4 \cdot 11 = 3$

\end{document}
