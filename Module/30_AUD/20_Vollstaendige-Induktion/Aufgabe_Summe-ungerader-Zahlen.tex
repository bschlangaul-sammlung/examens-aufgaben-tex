\documentclass{bschlangaul-aufgabe}
\bLadePakete{vollstaendige-induktion,java}
\begin{document}
\bAufgabenMetadaten{
  Titel = {Summe ungerader Zahlen (Maurolicus 1575)},
  Thematik = {Summe ungerader Zahlen (Maurolicus 1575)},
  Referenz = AUD.Vollstaendige-Induktion.Summe-ungerader-Zahlen,
  RelativerPfad = Module/30_AUD/20_Vollstaendige-Induktion/Aufgabe_Summe-ungerader-Zahlen.tex,
  BearbeitungsStand = mit Lösung,
  Korrektheit = unbekannt,
  Ueberprueft = {unbekannt},
  Stichwoerter = {Vollständige Induktion},
}

\let\m=\bInduktionMarkierung
\let\e=\bInduktionErklaerung

Die schrittweise Berechnung der Summe der ersten $n$ ungeraden Zahlen
legt die Vermutung nahe: Die Summe aller ungeraden Zahlen von $1$ bis
$2n-1$ ist gleich dem Quadrat von $n$:
\index{Vollständige Induktion}

\bigskip

$1 = 1 = 1^2$

$1 + 3 = 4 = 2^2$

$1 + 3 + 5 = 9 = 3^2$

$1 + 3 + 5 + 7 = 16 = 4^2$

\bigskip

\noindent
Folgende Java-Methode berechnet die Summer aller ungeraden Zahlen:

\bJavaDatei[firstline=8,lastline=13]{aufgaben/aud/induktion/Maurolicus}

\bigskip

\noindent
Beweisen Sie mittels vollständiger Induktion, dass der
Methodenaufruf \bJavaCode{oddSum(n)} die Summe aller ungeraden Zahlen von
$1$ bis nur $n$-ten ungeraden Zahl berechnet, wobei gilt:

\begin{displaymath}
\sum\limits^n_{i=1} (2i-1) = n^2
\end{displaymath}.

\begin{bAntwort}

\bInduktionAnfang

\begin{displaymath}
\sum\limits^1_{i=1} (2i-1) = 2 \cdot 1 - 1 = 1 = 1^2
\end{displaymath}

\begin{displaymath}
\texttt{oddSum(1)} = 1 = 1^2
\end{displaymath}

\bInduktionVoraussetzung

\begin{displaymath}
\sum\limits^n_{i=1} (2i-1) = n^2
\end{displaymath}

\begin{displaymath}
\texttt{oddSum(n)} = 2n - 1 + (n - 1)^2
\end{displaymath}

\bInduktionSchritt

\begin{align*}
\texttt{oddSum(n)}
& = 2(n+1) - 1 + ((n+1) - 1)^2
& \e{}\\
%
& = 2(n+1) - 1 + \m{n^2}
& \e{}\\
%
& = \m{2n + 2} + n^2 - 1
& \e{ausmultiplizieren}\\
%
& = 2n \m{+ 1} + n^2
& \e{$2-1 = 1$}\\
%
& = \m{n^2 + 2n + 1}
& \e{Kommutativgesetz} \\
%
& = (\m{n+1})^2
& \e{mit erster Binomischer Formel: $(a+b)^{2}=a^{2}+2ab+b^{2}$} \\
\end{align*}
\end{bAntwort}

\bJavaTestDatei{aufgaben/aud/induktion/MaurolicusTest}

\end{document}
