\documentclass{bschlangaul-aufgabe}
\bLadePakete{graph}
\begin{document}
\bAufgabenMetadaten{
  Titel = {Standardbeispiel TUM},
  Thematik = {TUM},
  Referenz = AUD.Graphen.Spannbaume.TUM,
  RelativerPfad = Module/30_AUD/90_Graphen/20_Spannbaume/Aufgabe_TUM.tex,
  BearbeitungsStand = nur Angabe,
  Korrektheit = unbekannt,
  Ueberprueft = {unbekannt},
  Stichwoerter = {Algorithmus von Prim},
}

Standardbeispiel
\bFussnoteUrl{https://algorithms.discrete.ma.tum.de/graph-algorithms/mst-prim/index_en.html}
\index{Algorithmus von Prim}

\begin{bGraphenFormat}
0: 0 3
1: 2 0
2: 3 7
3: 5 0
4: 5 5
5: 8 3
0 -- 1: 10
0 -- 2: 20
1 -- 3: 50
1 -- 4: 10
2 -- 3: 20
2 -- 4: 33
3 -- 4: 20
3 -- 5: 2
4 -- 5: 1
\end{bGraphenFormat}

\begin{tikzpicture}[li graph]
\node (0) at (0,3) {0};
\node (1) at (2,0) {1};
\node (2) at (3,7) {2};
\node (3) at (5,0) {3};
\node (4) at (5,5) {4};
\node (5) at (8,3) {5};

\path (0) edge node {10} (1);
\path (0) edge node {20} (2);
\path (1) edge node {50} (3);
\path (1) edge node {10} (4);
\path (2) edge node {20} (3);
\path (2) edge node {33} (4);
\path (3) edge node {20} (4);
\path (3) edge node {2} (5);
\path (4) edge (5);
\end{tikzpicture}

\end{document}
