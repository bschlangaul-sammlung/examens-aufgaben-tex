\documentclass{bschlangaul-aufgabe}
\bLadePakete{graph}
\begin{document}
\bAufgabenMetadaten{
  Titel = {Spannbaum},
  Thematik = {Minimaler Spannbaum A-H},
  Referenz = AUD.Graphen.Spannbaume.Minimaler-Spannbaum-A-H,
  RelativerPfad = Module/30_AUD/90_Graphen/20_Spannbaume/Aufgabe_Minimaler-Spannbaum-A-H.tex,
  ZitatSchluessel = aud:e-klausur,
  ZitatBeschreibung = {Aufgabe 6},
  BearbeitungsStand = mit Lösung,
  Korrektheit = unbekannt,
  Ueberprueft = {unbekannt},
  Stichwoerter = {Minimaler Spannbaum, Algorithmus von Prim},
}

Ermitteln Sie einen minimalen Spannbaum des vorliegenden Graphen. Nutzen
Sie den \emph{Knoten A als Startknoten} in ihrem Algorithmus.
\index{Minimaler Spannbaum}
\footcite[Aufgabe 6]{aud:e-klausur}

\begin{bGraphenFormat}
A: 0 5
B: 8 5
C: 4 4
D: 2 3
E: 6 3
F: 4 1
G: 0 0
H: 8 0
A -- B: 5
A -- C: 8
A -- D: 7
A -- G
B -- C: 2
B -- E: 7
B -- H: 2
C -- D
C -- E: 3
C -- F: 6
D -- F: 2
D -- G: 8
E -- H: 6
F -- E: 5
F -- H: 3
G -- F: 5
G -- H: 4
\end{bGraphenFormat}

\begin{center}
\begin{tikzpicture}[li graph]
\node (A) at (0,5) {A};
\node (B) at (8,5) {B};
\node (C) at (4,4) {C};
\node (D) at (2,3) {D};
\node (E) at (6,3) {E};
\node (F) at (4,1) {F};
\node (G) at (0,0) {G};
\node (H) at (8,0) {H};

\path (A) edge node {5} (B);
\path (A) edge node {8} (C);
\path (A) edge node {7} (D);
\path (A) edge node {1} (G);
\path (B) edge node {2} (C);
\path (B) edge node {7} (E);
\path (B) edge node {2} (H);
\path (C) edge node {1} (D);
\path (C) edge node {3} (E);
\path (C) edge node {6} (F);
\path (D) edge node {2} (F);
\path (D) edge node {8} (G);
\path (E) edge node {6} (H);
\path (F) edge node {5} (E);
\path (F) edge node {3} (H);
\path (G) edge node {5} (F);
\path (G) edge node {4} (H);
\end{tikzpicture}
\end{center}

\begin{enumerate}
\item Welches Gewicht hat der Spannbaum insgesamt?

\begin{bAntwort}
Das Kantengewicht des minimalen Spannbaums beträgt $15$.

Wir setzen den Algorithmus von Prim ein. Der Algorithmus läuft folgendermaßen ab:

\begin{verbatim}
Besuche Knoten „A“
Füge Kante (A, B, 5) hinzu
Füge Kante (A, C, 8) hinzu
Füge Kante (A, D, 7) hinzu
Füge Kante (A, G, 1) hinzu

Besuche Knoten „G“
Füge Kante (G, F, 5) hinzu
Füge Kante (G, H, 4) hinzu

Besuche Knoten „H“
Aktualisiere Kante (H, B, 2)
Füge Kante (H, E, 6) hinzu
Aktualisiere Kante (H, F, 3)

Besuche Knoten „B“
Aktualisiere Kante (B, C, 2)

Besuche Knoten „C“
Aktualisiere Kante (C, D, 1)
Aktualisiere Kante (C, E, 3)

Besuche Knoten „D“
Aktualisiere Kante (D, F, 2)

Besuche Knoten „F“

Besuche Knoten „E“
\end{verbatim}

\begin{tabular}{ll}
\bf{schwarze} & \bf{graue} \\
\hline
(A, null, 0) & (B, A, 5); (C, A, 8); (D, A, 7); (G, A, 1);  \\
(G, A, 1) & (B, A, 5); (C, A, 8); (D, A, 7); (F, G, 5); (H, G, 4);  \\
(H, G, 4) & (B, H, 2); (C, A, 8); (D, A, 7); (E, H, 6); (F, H, 3);  \\
(B, H, 2) & (C, B, 2); (D, A, 7); (E, H, 6); (F, H, 3);  \\
(C, B, 2) & (D, C, 1); (E, C, 3); (F, H, 3);  \\
(D, C, 1) & (E, C, 3); (F, D, 2);  \\
(F, D, 2) & (E, C, 3);  \\
(E, C, 3) &   \\
\end{tabular}

\begin{bGraphenFormat}
A: 0 5
B: 8 5
C: 4 4
D: 2 3
E: 6 3
F: 4 1
G: 0 0
H: 8 0
A -- B: 5
A -- C: 8
A -- D: 7
A -- G*
B -- C*: 2
B -- E: 7
B -- H*: 2
C -- D*
C -- E*: 3
C -- F: 6
D -- F*: 2
D -- G: 8
E -- H: 6
F -- E: 5
F -- H: 3
G -- F: 5
G -- H*: 4
\end{bGraphenFormat}

\begin{center}
\begin{tikzpicture}[li graph]
\node (A) at (0,5) {A};
\node (B) at (8,5) {B};
\node (C) at (4,4) {C};
\node (D) at (2,3) {D};
\node (E) at (6,3) {E};
\node (F) at (4,1) {F};
\node (G) at (0,0) {G};
\node (H) at (8,0) {H};

\path (A) edge node {5} (B);
\path (A) edge node {8} (C);
\path (A) edge node {7} (D);
\path[li markierung] (A) edge node {1} (G);
\path[li markierung] (B) edge node {2} (C);
\path (B) edge node {7} (E);
\path[li markierung] (B) edge node {2} (H);
\path[li markierung] (C) edge node {1} (D);
\path[li markierung] (C) edge node {3} (E);
\path (C) edge node {6} (F);
\path[li markierung] (D) edge node {2} (F);
\path (D) edge node {8} (G);
\path (E) edge node {6} (H);
\path (F) edge node {5} (E);
\path (F) edge node {3} (H);
\path (G) edge node {5} (F);
\path[li markierung] (G) edge node {4} (H);
\end{tikzpicture}
\end{center}

\end{bAntwort}

\item Welchen Algorithmus haben Sie zur Ermittlung eingesetzt?
\index{Algorithmus von Prim}
\begin{bAntwort}
Algorithmus von Prim. Dieser Algorithmus benötigt einen Startknoten.

\end{bAntwort}
\end{enumerate}
\end{document}
