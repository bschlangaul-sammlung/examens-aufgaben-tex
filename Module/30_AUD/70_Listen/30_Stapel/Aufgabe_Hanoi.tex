\documentclass{bschlangaul-aufgabe}
\bLadePakete{syntax,hanoi}
\begin{document}
\bAufgabenMetadaten{
  Titel = {Kellerspeicher: Türme von Hanoi},
  Thematik = {Hanoi},
  Referenz = AUD.Listen.Stapel.Hanoi,
  RelativerPfad = Module/30_AUD/70_Listen/30_Stapel/Aufgabe_Hanoi.tex,
  ZitatSchluessel = aud:ab:7,
  ZitatBeschreibung = {Aufgabe 2},
  BearbeitungsStand = unbekannt,
  Korrektheit = unbekannt,
  Ueberprueft = {unbekannt},
  Stichwoerter = {Stapel (Stack), Teile-und-Herrsche (Divide-and-Conquer), Rekursion},
}

Betrachten wir das folgende Spiel (Türme von Hanoi), das aus drei Stäben
$1$, $2$ und $3$ besteht, die senkrecht im Boden befestigt sind. Weiter
gibt es $n$ kreisförmige Scheiben mit einem Loch im Mittelpunkt, so dass
man sie auf die Stäbe stecken kann. Dabei haben die Scheiben
verschiedene Radien, alle sind unterschiedlich groß. Zu Beginn stecken
alle Scheiben auf dem Stab $1$, wobei immer eine kleinere auf einer
größeren liegt. Das Ziel des Spiels ist es nun, die Scheiben so
umzuordnen, dass sie in der gleichen Reihenfolge auf dem Stab $3$
liegen. Dabei darf immer nur eine Scheibe bewegt werden und es darf nie
eine größere auf einer kleineren Scheibe liegen. Stab $2$ darf dabei als
Hilfsstab verwendet werden.\footcite[Aufgabe 2]{aud:ab:7}
\index{Stapel (Stack)}
\index{Teile-und-Herrsche (Divide-and-Conquer)}
\index{Rekursion}

Ein Beispiel für $4$ Scheiben finden Sie in folgendem Bild:

\begin{center}
\bHanoi{4}{4/1,3/1,2/1,1/1}
\end{center}

\begin{center}
\bHanoi{4}{4/3,3/3,2/3,1/3}
\end{center}

\begin{enumerate}

%%
% 1.
%%

\item Ein \bJavaCode{Element} hat immer einen Wert (Integer) und kennt das
Nachfolgende \bJavaCode{Element}, wobei immer nur das jeweilige
\bJavaCode{Element} auf seinen Wert und seinen Nachfolger zugreifen darf.

\begin{bAntwort}
\bJavaDatei[firstline=6]{aufgaben/aud/listen/hanoi/Element}
\end{bAntwort}

%%
% 2.
%%

\item Ein \bJavaCode{Turm} ist einem Stack (Kellerspeicher)
nachempfunden und kennt somit nur das erste Element. Hinweis: Beachten
Sie, dass nur kleinere Elemente auf den bisherigen Stack gelegt werden
können

\begin{bAntwort}
\bJavaDatei[firstline=6]{aufgaben/aud/listen/hanoi/Turm}
\end{bAntwort}

%%
% 3.
%%

\item In der Klasse \bJavaCode{Hanoi} müssen Sie nur die Methode
\bJavaCode{public void hanoi (int n, Turm quelle, Turm ziel, Turm
hilfe)} implementieren. Die anderen Methoden sind zur Veranschaulichung
des Spiels! Entwerfen Sie eine rekursive Methode die einen Turm der Höhe
$n$ vom Stab \bJavaCode{quelle} auf den Stab \bJavaCode{ziel}
transportiert und den Stab \bJavaCode{hilfe} als Hilfsstab verwendet.

\begin{bAntwort}
\bJavaDatei[firstline=6]{aufgaben/aud/listen/hanoi/Hanoi}
\end{bAntwort}
\end{enumerate}

% 1  0  0
% 2  0  0
% 3  0  0
% 4  0  0
% 5  0  0

% 0  0  0
% 2  0  0
% 3  0  0
% 4  0  0
% 5  0  1
% Fertig!

% 0  0  0
% 0  0  0
% 3  0  0
% 4  0  0
% 5  2  1

% 0  0  0
% 0  0  0
% 3  0  0
% 4  1  0
% 5  2  0
% Fertig!

% 0  0  0
% 0  0  0
% 0  0  0
% 4  1  0
% 5  2  3

% 0  0  0
% 0  0  0
% 1  0  0
% 4  0  0
% 5  2  3
% Fertig!

% 0  0  0
% 0  0  0
% 1  0  0
% 4  0  2
% 5  0  3

% 0  0  0
% 0  0  0
% 0  0  1
% 4  0  2
% 5  0  3
% Fertig!

% 0  0  0
% 0  0  0
% 0  0  1
% 0  0  2
% 5  4  3

% 0  0  0
% 0  0  0
% 0  0  0
% 0  1  2
% 5  4  3
% Fertig!

% 0  0  0
% 0  0  0
% 0  0  0
% 2  1  0
% 5  4  3

% 0  0  0
% 0  0  0
% 1  0  0
% 2  0  0
% 5  4  3
% Fertig!

% 0  0  0
% 0  0  0
% 1  0  0
% 2  3  0
% 5  4  0

% 0  0  0
% 0  0  0
% 0  0  0
% 2  3  0
% 5  4  1
% Fertig!

% 0  0  0
% 0  0  0
% 0  2  0
% 0  3  0
% 5  4  1

% 0  0  0
% 0  1  0
% 0  2  0
% 0  3  0
% 5  4  0
% Fertig!

% 0  0  0
% 0  1  0
% 0  2  0
% 0  3  0
% 0  4  5

% 0  0  0
% 0  0  0
% 0  2  0
% 0  3  0
% 1  4  5
% Fertig!

% 0  0  0
% 0  0  0
% 0  0  0
% 0  3  2
% 1  4  5

% 0  0  0
% 0  0  0
% 0  0  1
% 0  3  2
% 0  4  5
% Fertig!

% 0  0  0
% 0  0  0
% 0  0  1
% 0  0  2
% 3  4  5

% 0  0  0
% 0  0  0
% 0  0  0
% 0  1  2
% 3  4  5
% Fertig!

% 0  0  0
% 0  0  0
% 0  0  0
% 2  1  0
% 3  4  5

% 0  0  0
% 0  0  0
% 1  0  0
% 2  0  0
% 3  4  5
% Fertig!

% 0  0  0
% 0  0  0
% 1  0  0
% 2  0  4
% 3  0  5

% 0  0  0
% 0  0  0
% 0  0  1
% 2  0  4
% 3  0  5
% Fertig!

% 0  0  0
% 0  0  0
% 0  0  1
% 0  0  4
% 3  2  5

% 0  0  0
% 0  0  0
% 0  0  0
% 0  1  4
% 3  2  5
% Fertig!

% 0  0  0
% 0  0  0
% 0  0  3
% 0  1  4
% 0  2  5

% 0  0  0
% 0  0  0
% 0  0  3
% 0  0  4
% 1  2  5
% Fertig!

% 0  0  0
% 0  0  2
% 0  0  3
% 0  0  4
% 1  0  5

% 0  0  1
% 0  0  2
% 0  0  3
% 0  0  4
% 0  0  5
% Fertig!
\end{document}
