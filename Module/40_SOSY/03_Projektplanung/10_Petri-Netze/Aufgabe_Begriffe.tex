\documentclass{bschlangaul-aufgabe}
\bLadePakete{petri}
\begin{document}
\bAufgabenMetadaten{
  Titel = {Aufgabe 1: Begriffe},
  Thematik = {Begriffe},
  Referenz = SOSY.Projektplanung.Petri-Netze.Begriffe,
  RelativerPfad = Module/40_SOSY/03_Projektplanung/10_Petri-Netze/Aufgabe_Begriffe.tex,
  ZitatSchluessel = sosy:ab:4,
  ZitatBeschreibung = {Seite 1},
  BearbeitungsStand = TeX-Fehler,
  Korrektheit = unbekannt,
  Ueberprueft = {unbekannt},
  Stichwoerter = {Petri-Netz, Beschränktheit, Lebendigkeit, Verklemmungsfreiheit, Umkehrbarkeit},
}

Begründen Sie, welche der folgenden Petri-Netze
\index{Petri-Netz}
\footcite[Seite 1]{sosy:ab:4}

%%
%
%%

\def\TmpA#1{
  \bPetriSetzeSchluessel%
  \pgfkeys{/petri/.cd,#1}%
  \begin{tikzpicture}[li petri]
  \node at (-0.25,-0.25) {};
  \node at (\TmpX,\TmpY) {};

  \begin{scope}[transform canvas={scale=\TmpScale},x=2cm,y=2cm,]
    \node[place,tokens=\TmpPlaceOne,label=$p_1$] at (0,1) (p1) {};
    \node[place,tokens=\TmpPlaceTwo,label=$p_2$] at (2,2) (p2) {};
    \node[place,tokens=\TmpPlaceThree,label=east:$p_3$] at (2,0) (p3) {};

    \node[transition,label=east:$t_1$,\TmpTransitionOne] at (2,1) {}
      edge[pre] (p2)
      edge[post] (p3);
    \node[transition,label=$t_2$,\TmpTransitionTwo] at (1,1.5) {}
      edge[pre] (p1)
      edge[post] (p2);
    \node[transition,label=$t_3$,\TmpTransitionThree] at (1,0.5) {}
      edge[pre] (p3)
      edge[post] (p1);
    \node[transition,label=$t_4$,\TmpTransitionFour] at (1,1) {}
      edge[pre] (p2)
      edge[pre] (p3)
      edge[post] (p1);
  \end{scope}
  \end{tikzpicture}
}

%%
%
%%

\def\TmpB#1{
  \bPetriSetzeSchluessel%
  \pgfkeys{/petri/.cd,#1}%
  \begin{tikzpicture}[li petri]
  \node at (-0.25,-0.25) {};
  \node at (\TmpX,\TmpY) {};

  \begin{scope}[transform canvas={scale=\TmpScale},x=2cm,y=2cm,]
    \node[place,tokens=\TmpPlaceOne,label=$p_1$] at (0,2) (p1) {};
    \node[place,tokens=\TmpPlaceTwo,label=$p_2$] at (0,0) (p2) {2019};
    \node[place,tokens=\TmpPlaceThree,label=east:$p_3$] at (2,1) (p3) {};

    \node[transition,label=east:$t_1$,\TmpTransitionOne] at (1,1.5) {}
      edge[pre] (p3)
      edge[post] (p1);
    \node[transition,label=$t_2$,\TmpTransitionTwo] at (1,1) {}
      edge[pre] (p1)
      edge[pre] (p2)
      edge[post] (p3);
    \node[transition,label=$t_3$,\TmpTransitionThree] at (1,0.5) {}
      edge[pre] (p2)
      edge[post] (p2)
      edge[post] (p3);
  \end{scope}
  \end{tikzpicture}
}

%%
%
%%

\def\TmpC#1{%
  \bPetriSetzeSchluessel%
  \pgfkeys{/petri/.cd,#1}%
  \begin{tikzpicture}[li petri]
    \node at (-0.25,-0.25) {};
    \node at (\TmpX,\TmpY) {};
    \begin{scope}[transform canvas={scale=\TmpScale},x=2cm,y=2cm,]
    \node[place,tokens=\TmpPlaceOne,label=west:$p_1$] at (0,4) (p1) {};
    \node[place,tokens=\TmpPlaceTwo,label=west:$p_2$] at (0,2) (p2) {};
    \node[place,tokens=\TmpPlaceThree,label=west:$p_3$] at (0,0) (p3) {};
    \node[place,tokens=\TmpPlaceFour,label=$p_4$] at (1,2) (p4) {};
    \node[place,tokens=\TmpPlaceFive,label=east:$p_5$] at (3,0) (p5) {};
    \node[place,tokens=\TmpPlaceSix,label=east:$p_6$] at (3,2) (p6) {};
    \node[place,tokens=\TmpPlaceSeven,label=east:$p_7$] at (3,4) (p7) {};

    \node[transition,label=west:$t_1$,\TmpTransitionOne] at (0,3) {}
      edge[pre] (p1)
      edge[post] node[auto] {2} (p2) edge[post] (p4);

    \node[transition,label=$t_2$,\TmpTransitionTwo] at (2,4) {}
      edge[pre] (p7)
      edge[post] (p1);

    \node[transition,label=south:$t_3$,\TmpTransitionThree] at (2,0) {}
      edge[pre] (p3)
      edge[post] (p5) edge[post] (p6);

    \node[transition,label=$t_4$,\TmpTransitionFour] at (2,2) {}
      edge[pre] node[auto] {2} (p6)
      edge[post] (p4);

    \node[transition,label=west:$t_5$,\TmpTransitionFive] at (0,1) {}
      edge[pre] (p2) edge[pre] (p4)
      edge[post] (p3);

    \node[transition,label=east:$t_6$,\TmpTransitionSix] at (3,1) {}
      edge[pre] (p5)
      edge[post] (p6);

    \node[transition,label=east:$t_7$,\TmpTransitionSeven] at (3,3) {}
      edge[pre] (p6)
      edge[post] (p7) edge[post] (p1);
    \end{scope}
  \end{tikzpicture}%
}

\def\TmpD#1{
  \bPetriSetzeSchluessel%
  \pgfkeys{/petri/.cd,#1}%
  \begin{tikzpicture}[li petri]
    \node at (-0.25,-0.25) {};
    \node at (\TmpX,\TmpY) {};
    \begin{scope}[transform canvas={scale=\TmpScale},x=2cm,y=2cm,]
    \node[place,tokens=\TmpPlaceOne,label=west:$p_1$] at (0,2) (p1) {};
    \node[place,tokens=\TmpPlaceTwo,label=$p_2$] at (2,2) (p2) {};
    \node[place,tokens=\TmpPlaceThree,label=$p_3$] at (4,2) (p3) {};
    \node[place,tokens=\TmpPlaceFour,label=east:$p_4$] at (6,2) (p4) {};
    \node[place,tokens=\TmpPlaceFive,label=east:$p_5$] at (6,0) (p5) {};
    \node[place,tokens=\TmpPlaceSix,label=$p_6$] at (4,0) (p6) {};
    \node[place,tokens=\TmpPlaceSeven,label=$p_7$] at (2,0) (p7) {};
    \node[place,tokens=\TmpPlaceEight,label=west:$p_8$] at (0,0) (p8) {};
    \node[place,tokens=\TmpPlaceNine,label=$p_9$] at (4,1) (p9) {};
    \node[place,tokens=\TmpPlaceTen,label=$p_{10}$] at (2,1) (p10) {};

    \node[transition,label=$t_1$,\TmpTransitionOne] at (1,2) {}
      edge[pre] (p1) edge[pre] (p10)
      edge[post] (p2);
    \node[transition,label=$t_2$,\TmpTransitionTwo] at (3,2) {}
      edge[pre] (p2) edge[pre] (p9)
      edge[post] (p10) edge[post] (p3);
    \node[transition,label=$t_3$,\TmpTransitionThree] at (5,2) {}
      edge[pre] (p3)
      edge[post] (p9) edge[post] (p4);
    \node[transition,label=east:$t_4$,\TmpTransitionFour] at (6,1) {}
      edge[pre] (p4)
      edge[post] (p5);
    \node[transition,label=$t_5$,\TmpTransitionFive] at (5,0) {}
      edge[pre] (p5) edge[pre] (p9)
      edge[post] (p6);
    \node[transition,label=$t_6$,\TmpTransitionSix] at (3,0) {}
      edge[pre] (p6) edge[pre] (p10)
      edge[post] (p7) edge[post] (p9);
    \node[transition,label=$t_7$,\TmpTransitionSeven] at (1,0) {}
      edge[pre] (p7)
      edge[post] (p10) edge[post] (p8);
    \node[transition,label=west:$t_8$,\TmpTransitionEight] at (0,1) {}
      edge[pre] (p8)
      edge[post] (p1);
    \end{scope}
  \end{tikzpicture}
}
\begin{enumerate}
\item beschränkt\index{Beschränktheit}
\item lebendig\index{Lebendigkeit}
\item verklemmungsfrei\index{Verklemmungsfreiheit}
\item umkehrbar\index{Umkehrbarkeit}
\end{enumerate}

\begin{enumerate}
\item \TmpA{x=3.3,y=2.5,scale=0.7,p2=2}
\item \TmpB{x=3.3,y=2.5,scale=0.7}
\item \TmpC{x=3.3,y=5.5,scale=0.7,p1=1}
\item \TmpD{x=3.3,y=3,scale=0.7,p1=2,p5=2,p9=1,p10=1}

\end{enumerate}

\begin{bAntwort}

\bPseudoUeberschrift{(a)}

\tikzset{/petri/.cd,x=2,y=2,scale=0.45}

\noindent
\TmpA{p2=2,t1,}
%
\TmpA{p2=1,p3=1,t1,t3,t4,}
%
\TmpA{p1=1,p3=1,t2,t3,}
%
\TmpA{p1=1,p2=1,t1,t2,}

\begin{description}
\item[beschränkt] ja, $M = 2$.

\item[lebendig] Nein, die Transition $t_4$ kann maximal einmal schalten
(\zB $t_1 \rightarrow t_4$)

\item[verklemmungsfrei] Ja, mit $t_1 \rightarrow t_3 \rightarrow  t_2$
ist ein Zyklus gegeben.

\item[umkehrbar] Nein, nachdem $t_4$ einmal geschaltet hat, wird dem
Petri-Netz eine Markierung entzogen, welche nie wieder erzeugt werden
kann.
\end{description}

\bPseudoUeberschrift{(b)}

\begin{description}
\item[beschränkt] Nein, solange in $p_2$ mindestens eine Markierung ist,
kann $t_3$ beliebig oft schalten und somit die Anzahl der Markierungen
in $p_3$ beliebig erhöhen.

\item[lebendig] Nein, da es nicht verklemmungsfrei ist.

\item[verklemmungsfrei] Nein, nachdem 2019 mal $t_2$ und anschließend
$t_1$ geschaltet haben, befindet sich in $p_2$ keine Marke mehr. Daher
können weder $t_2$ noch $t_3$ schalten.

\item[umkehrbar]
Nein, da es nicht verklemmungsfrei ist.
\end{description}

\bPseudoUeberschrift{(c)}

\begin{description}
\item[beschränkt] Nein, $t_1 \rightarrow t_5 \rightarrow t_3 \rightarrow
t_6 \rightarrow t_7 \rightarrow t_2$ bildet einen Zyklus, der nach jedem
Umlauf die Anzahl der Marken in $p_1$ um eins erhöht.

\item[lebendig] Nein, da es nicht verklemmungsfrei ist.

\item[verklemmungsfrei] Nein, die Schaltfolge $t_1 \rightarrow t_5
\rightarrow t_3 \rightarrow t_6 \rightarrow t_4 \rightarrow t_5
\rightarrow t_3 \rightarrow t_6 \rightarrow t_4$ führt zu einer
Verklemmung.

\item[umkehrbar] Nein, da es nicht verklemmungsfrei ist.

\end{description}

\tikzset{/petri/.cd,x=3.3,y=4.3,scale=0.45}

\noindent
\TmpC{p1=1,t1}
\TmpC{p2=2,p4=1,t5}
\TmpC{p2=1,p3=1,t3}
\TmpC{p2=1,p5=1,p6=1,t4,t7,t6}
\TmpC{p1=1,p2=1,p7=1,p6=1,t1,t2,t4,t5,t7}
\TmpC{p1=2,p2=1,p7=1,t1,t2,t5,t7}

\bPseudoUeberschrift{(d)}

\begin{description}
\item[beschränkt] Ja, mit M = 4.

\item[lebendig] Ja.

\item[verklemmungsfrei] Ja.

\item[umkehrbar] Ja.
\end{description}

\end{bAntwort}

\noindent
sind.
%
Im Falle der Beschränktheit soll ein minimales $M$ gefunden werden,
sodass jede Stelle zu jedem möglichen Zeitpunkt höchstens $M$ Marken
enthält.
\end{document}
