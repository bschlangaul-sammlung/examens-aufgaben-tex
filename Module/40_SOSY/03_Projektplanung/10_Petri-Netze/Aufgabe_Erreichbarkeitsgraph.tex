\documentclass{bschlangaul-aufgabe}
\bLadePakete{petri,mathe}
\begin{document}

% Metadaten ------------------------------------------------------------

\bAufgabenMetadaten{
  Titel = {Aufgabe 4:},
  Thematik = {Erreichbarkeitsgraph},
  Referenz = SOSY.Projektplanung.Petri-Netze.Erreichbarkeitsgraph,
  RelativerPfad = Module/40_SOSY/03_Projektplanung/10_Petri-Netze/Aufgabe_Erreichbarkeitsgraph.tex,
  ZitatSchluessel = sosy:ab:4,
  ZitatBeschreibung = {Seite 3, Aufgabe 5},
  BearbeitungsStand = mit Lösung,
  Korrektheit = unbekannt,
  Ueberprueft = {unbekannt},
  Stichwoerter = {Petri-Netz, Erreichbarkeitsgraph},
}

% Kürzel ---------------------------------------------------------------

\let\t=\bPetriErreichTransition

% Aufgabe _-------------------------------------------------------------

Gegeben ist das folgende Petri-Netz:
\footcite[Seite 3, Aufgabe 5]{sosy:ab:4}
\index{Petri-Netz}

\begin{center}
\begin{tikzpicture}[li petri]
\node[place,tokens=3,label=$p_1$] at (0,2) (p1) {};
\node[place,label=$p_2$] at (4,4) (p2) {};
\node[place,label=east:$p_3$] at (4,0) (p3) {};

\node[transition,label=$t_1$] at (2,4) {}
  edge[pre] (p1)
  edge[post] node[auto]{2} (p2);
\node[transition,label=$t_2$] at (2,2) {}
  edge[pre] node[auto]{2} (p1)
  edge[post] (p2)
  edge[post] node[auto]{3} (p3);
\node[transition,label=$t_3$] at (2,0) {}
  edge[pre] node[auto]{4} (p3)
  edge[post] node[auto]{3} (p1);
\node[transition,label=east:$t_4$] at (4,2) {}
  edge[pre] node[auto]{3} (p2)
  edge[post] (p3);
\end{tikzpicture}
\end{center}

\begin{enumerate}

%%
% (a)
%%

\item Geben Sie die dazugehörige Darstellungsmatrix sowie den
Belegungsvektor an.

\begin{bAntwort}
\begin{displaymath}
A =
\begin{blockarray}{ccccc}
      & t_1 & t_2 & t_3 & t_4 \\
  \begin{block}{c(cccc)}
  p_1 & -1  & -2  & 3   & 0 \\
  p_2 & 2   & 1   & 0   & -3 \\
  p_3 & 0   & 3   & -4  & 1 \\
  \end{block}
\end{blockarray}
\enspace
,
\enspace
v =
\left(
  \begin{array}{c}
  3 \\
  0 \\
  0
  \end{array}
\right)
\end{displaymath}
\end{bAntwort}

%%
% (b)
%%

\item Skizzieren Sie den Erreichbarkeitsgraphen des Petri-Netzes.
\index{Erreichbarkeitsgraph}

\begin{bAntwort}
\begin{center}
\begin{tikzpicture}[li petri,x=1.5cm,y=1.5cm]
\node at (0,0) (300) {(3,0,0)};
\node at (-1,-1) (113) {(1,1,3)};
\node at (0,-2) (033) {(0,3,3)};
\node at (1,-1) (220) {(2,2,0)};
\node at (-1,-3) (004) {(0,0,4)};
\node at (3,-1) (140) {(1,4,0)};
\node at (2,-2) (060) {(0,6,0)};
\node at (4,-2) (111) {(1,1,1)};
\node at (3,-3) (031) {(0,3,1)};
\node at (3,-4) (002) {(0,0,2)};

\t{300}{113}{2}
\t{113}{033}{1}
\t{300}{220}{1}
\t{033}{004}{4}
\t{220}{033}{2}
\t{004}{300}{3}[bend left=90]
\t{220}{140}{1}
\t{140}{060}{1}
\t{060}{031}{4}
\t{031}{002}{4}
\t{111}{031}{1}
\t{140}{111}{4}
\end{tikzpicture}
\end{center}
\end{bAntwort}

%%
% (c)
%%

\item Begründen Sie anhand des Erreichbarkeitsgraphen, ob das Petri-Netz
verklemmungsfrei ist oder nicht.

\begin{bAntwort}
Durch Schalten von $t_1 \rightarrow t_1 \rightarrow t_1 \rightarrow t_4
\rightarrow t_4$ wird beispielsweise eine Verklemmung erreicht. Das
Petri-Netz ist also nicht verklemmungsfrei. Am Erreichbarkeitsgraphen
erkennt man das anhand der Senke im Knoten [0,0,2].
\end{bAntwort}
\end{enumerate}
\end{document}
