\documentclass{bschlangaul-aufgabe}
\bLadePakete{tabelle}
\begin{document}
\bAufgabenMetadaten{
  Titel = {Aufgabe 1},
  Thematik = {Grundwissen},
  Referenz = SOSY.Projektplanung.Entwurfsmuster.Grundwissen,
  RelativerPfad = Module/40_SOSY/03_Projektplanung/Entwurfsmuster/Aufgabe_Grundwissen.tex,
  ZitatSchluessel = sosy:ab:6,
  BearbeitungsStand = mit Lösung,
  Korrektheit = unbekannt,
  Ueberprueft = {unbekannt},
  Stichwoerter = {Entwurfsmuster},
}

\begin{enumerate}

%%
% (a)
%%

\item Erklären Sie, was man bei der Entwicklung von Softwaresystemen
unter einem Entwurfsmuster versteht und gehen Sie dabei auch auf die
Vorteile ein.\index{Entwurfsmuster}
\footcite{sosy:ab:6}

\begin{bAntwort}
Entwurfsmuster sind bewährte Lösungsschablonen für wiederkehrende
Entwurfsprobleme sowohl in der Architektur als auch in der
Softwarearchitektur und -entwicklung. Sie stellen damit eine
wiederverwendbare Vorlage zur Problemlösung dar, die in einem bestimmten
Zusammenhang einsetzbar ist.

Der primäre Nutzen eines Entwurfsmusters liegt in der Beschreibung einer
Lösung für eine bestimmte Klasse von Entwurfsproblemen. Weiterer Nutzen
ergibt sich aus der Tatsache, dass jedes Muster einen Namen hat. Dies
vereinfacht die Diskussion unter Entwicklern, da man abstrakt über eine
Struktur sprechen kann. So sind etwa Software-Entwurfsmuster zunächst
einmal unabhängig von der konkreten Programmiersprache. Wenn der Einsatz
von Entwurfsmustern dokumentiert wird, ergibt sich ein weiterer Nutzen
dadurch, dass durch die Beschreibung des Musters ein Bezug zur dort
vorhandenen Diskussion des Problemkontextes und der Vor- und Nachteile
der Lösung hergestellt wird.
\end{bAntwort}

%%
% (b)
%%

\item Nennen Sie die drei ursprünglichen Typen von Entwurfsmuster,
erklären Sie diese kurz und geben Sie zu jedem Typ drei Beispiele an.

\begin{bAntwort}

\begin{tabularx}{\linewidth}{p{2cm}|X|p{2cm}}
\textbf{Typ} & \textbf{Erlärung} & \textbf{Beispiele} \\\hline\hline

Erzeugungsmuster &
Dienen der Erzeugung von Objekten; diese wird dadurch gekapselt und
ausgelagert, um den Kontext der Objekterzeugung unabhängig von der
konkreten Implementierung zu halten &
abstrakte Fabrik, Singleton, Prototyp \\\hline

Strukturmuster &
Erleichtern den Entwurf von Software durch vorgefertigte Schablonen für
Beziehungen zwischen Klassen. &
Adapter, Kompositum, Stellvertreter \\\hline

Verhaltensmuster  &
Modellieren komplexes Verhalten der Software und erhöhen damit die
Flexibilität der Software hinsichtlich ihres Verhaltens. &
Beobachter, Interpreter, Iterator \\
\end{tabularx}
\end{bAntwort}
\end{enumerate}
\end{document}
