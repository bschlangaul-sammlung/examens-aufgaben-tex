
\documentclass{bschlangaul-aufgabe}
\bLadePakete{wpkalkuel,syntax,java}
\begin{document}
\bAufgabenMetadaten{
  Titel = {Aufgabe},
  Thematik = {Invariante},
  Referenz = SOSY.Testen.Formale-Verifikation.E-Klausur_Invariante,
  RelativerPfad = Module/40_SOSY/05_Testen/10_Formale-Verifikation/Aufgabe_E-Klausur_Invariante.tex,
  BearbeitungsStand = mit Lösung,
  Korrektheit = unbekannt,
  Ueberprueft = {unbekannt},
}

\let\wp=\bWpKalkuelOhneMathe

Gegeben sei die endrekursive Methode \bJavaCode{geoSum}:

\bJavaDatei[firstline=4,lastline=15]{aufgaben/sosy/eklausur/Invariante}

\begin{description}
\item[$I_1$:] $\text{res} = 0 \lor \text{res} = 1 - q^i$

\item[$I_2$:] $\text{res} > 0$

\item[$I_3$:] $\text{res} = 0 \lor \text{res} = 1 - q^{i+1}$

\item[$I_4$:] $\text{res} = 1 - q^{i+1}$

\item[$I_5$:] $n \geq 0$
\end{description}

\noindent
Sie dürfen für die folgenden Antworten annehmen, dass die Precondition
$P$ gilt und die Methode dann

\begin{displaymath}
\text{geoSum}(n, q) = 1 - q^{n+1}
\end{displaymath}

\noindent
berechnet. Entscheiden Sie, ob die folgenden Vorschläge
Schleifeninvarianten darstellen können. (Diese müssen nicht unbedingt
hilfreich für einen Beweis sein.)

\begin{description}
\item[$I_1$:] $\text{res} = 0 \lor \text{res} = 1 - q^i$

\item[$I_2$:] $\text{res} > 0$

\item[$I_3$:] $\text{res} = 0 \lor \text{res} = 1 - q^{i+1}$

\item[$I_4$:] $\text{res} = 1 - q^{i+1}$

\item[$I_5$:] $n \geq 0$
\end{description}

\begin{bAntwort}
\begin{description}
\item[$I_1$:] keine Invariante

\item[$I_2$:] keine Invariante

\item[$I_3$:] Invariante

\item[$I_4$:] keine Invariante

\item[$I_5$:] Invariante
\end{description}

%-----------------------------------------------------------------------
%
%-----------------------------------------------------------------------

\bPseudoUeberschrift{Code vor der Schleife}

\begin{align*}
\wp{Code vor der Schleife}{I_1}
& \equiv \wp{res = 0; i = 0}{\text{res} = 0 \lor \text{res} = 1 - q^i} \\
& \equiv \wp{}{0 = 0 \lor 0 = 1 - q^0} \\
& \equiv 0 = 0 \lor 0 = 1 - q^0 \\
& \equiv 0 = 0 \lor 0 = 1 \\
& \equiv 0 = 0 \\
& \equiv \text{wahr} \\
\end{align*}

\begin{align*}
\wp{Code vor der Schleife}{I_2}
& \equiv \wp{res = 0; i = 0}{\text{res} > 0} \\
& \equiv \wp{i = 0}{0 > 0} \\
& \equiv 0 > 0 \\
& \equiv \text{falsch} \\
\end{align*}

\begin{align*}
\wp{Code vor der Schleife}{I_3}
& \equiv \wp{res = 0; i = 0}{\text{res} = 0 \lor \text{res} = 1 - q^{i+1}} \\
& \equiv \wp{}{0 = 0 \lor 0 = 1 - q^{0+1}} \\
& \equiv 0 = 0 \lor 0 = 1 - q \\
& \equiv 0 = 0 \lor q = 1 \\
& \equiv 0 = 0 \\
& \equiv \text{wahr}\\
\end{align*}

\begin{align*}
\wp{Code vor der Schleife}{I_4}
& \equiv \wp{res = 0; i = 0}{\text{res} = 1 - q^{i+1}} \\
& \equiv \wp{}{0 = 1 - q^{0+1}} \\
& \equiv 0 = 1 - q \\
& \equiv q = 1 \\
& \equiv \text{falsch für alle } q > 0\\
\end{align*}

\begin{align*}
\wp{Code vor der Schleife}{I_5}
& \equiv \wp{res = 0; i = 0}{n \geq 0} \\
& \equiv n \geq 0 \\
& \equiv \text{wahr für alle } n \geq 0\\
\end{align*}

%-----------------------------------------------------------------------
%
%-----------------------------------------------------------------------

\bPseudoUeberschrift{Code in der Schleife}

Wir formulieren den Code in der Schleife etwas um:

\begin{minted}{java}
  double geoSum(int n, double q) {
    // P: n >= 0, q > 0
    double res = 0;
    int i = 0;
    while (i < n) {
      res = res + (1 - q) * Math.pow(q, n);
      i = i + 1;
    }
    return res;
  }
\end{minted}

\begin{align*}
&\wp{Code in der Schleife}{I_1 \land i < n}\\
&\equiv \wp{res = res + (1-q) * Math.pow(q, n); i = i + 1;}
{(\text{res} = 0 \lor \text{res} = 1 - q^i) \land i < n} \\
&\equiv \wp{}
{(\text{res} + (1-q) \cdot q^n = 0 \lor \text{res} + (1-q) \cdot q^n = 1 - q^{i + 1}) \land i + 1 < n} \\
&\equiv
(\text{res} + (1-q) \cdot q^n = 0 \lor \text{res} + (1-q) \cdot q^n = 1 - q^{i + 1}) \land i + 1 < n \\
\end{align*}

\begin{align*}
&\wp{Code in der Schleife}{I_3 \land i < n}\\
& \equiv \wp{res = res + (1-q) * Math.pow(q, n); i = i + 1;}
{(\text{res} = 0 \lor \text{res} = 1 - q^{i+1}) \land i < n} \\
& \equiv \wp{}
{(\text{res} + (1-q) \cdot q^n = 0 \lor \text{res} + (1-q) \cdot q^n  = 1 - q^{i + 1 + 1}) \land i + 1 < n} \\
& \equiv
(\text{res} + (1-q) \cdot q^n = 0 \lor \text{res} + (1-q) \cdot q^n  = 1 - q^{i + 2}) \land i + 1 < n \\
\end{align*}

\begin{align*}
&\wp{Code in der Schleife}{I_5 \land i < n}\\
& \equiv \wp{res = res + (1-q) * Math.pow(q, n); i = i + 1;}{n \geq 0 \land i < n} \\
& \equiv \wp{}{n \geq 0 \land i + 1 < n} \\
& \equiv n \geq 0 \land i + 1 < n \\
& \equiv \text{wahr für }n \geq 0 \land i < n - 1 \\
\end{align*}
\end{bAntwort}
\end{document}
