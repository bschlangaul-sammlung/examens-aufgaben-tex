\documentclass{bschlangaul-aufgabe}

\begin{document}
\bAufgabenMetadaten{
  Titel = {Aufgabe 1: Grundwissen},
  Thematik = {Grundwissen},
  Referenz = SOSY.Testen.Black_White-Box-Test.Grundwissen,
  RelativerPfad = Module/40_SOSY/05_Testen/20_Black_White-Box-Test/Aufgabe_Grundwissen.tex,
  ZitatSchluessel = sosy:ab:7,
  BearbeitungsStand = mit Lösung,
  Korrektheit = unbekannt,
  Ueberprueft = {unbekannt},
  Stichwoerter = {White-Box-Testing, Black-Box-Testing, Funktionalorienteres Testen, V-Modell},
}

\begin{enumerate}

%%
% (a)
%%

\item Nennen Sie die wesentlichen Unterschiede zwischen
\emph{White-Box-Testen} und \emph{Black-Box-Testen} und geben Sie
jeweils zwei Beispiele an.
\index{White-Box-Testing}
\index{Black-Box-Testing}
\footcite{sosy:ab:7}

\begin{bAntwort}
\begin{description}
\item[White-Box-Tests]

sind \emph{strukturorientiert}, \zB
Kontrollflussorientiertes Testen oder Datenflussorientiertes Testen.

\item[Black-Box-Test]

sind \emph{spe\-zifi\-kations- und funktionsorientiert} (nur Ein- und
Ausgabe relevant), \zB Äquivalenzklassenbildung oder Grenzwertanalyse.
\end{description}
\end{bAntwort}

%%
% (b)
%%

\item Geben Sie drei nicht-funktionalorientierte Testarten an.
\index{Funktionalorienteres Testen}

\begin{bAntwort}
\begin{enumerate}
\item Performanztest
\item Lasttest
\item Stresstest
\end{enumerate}
\end{bAntwort}

%%
% (c)
%%

\item Nennen Sie die vier verschiedenen Teststufen aus dem
V-Modell\index{V-Modell} und erläutern Sie deren Ziele.

\begin{bAntwort}
\begin{description}
\item[Komponenten-Test:]
Fehlerzustände in Modulen finden.

\item[Integrations-Test:]
Fehlerzustände in Schnittstellen und Interaktionen finden.

\item[System-Test:]
Abgleich mit Spezifikation.

\item[Abnahme-Test:]
Vertrauen in System und nicht-funktionale Eigenschaften gewinnen.
\footcite[Seite 50, Abbildung 3.2]{schatten}
\end{description}
\end{bAntwort}

%%
% (d)
%%

\item Nennen Sie fünf Aktivitäten des Testprozesses.

\begin{bAntwort}
\begin{enumerate}
\item Testplanung und Steuerung,

\item Testanalyse und Testentwurf,

\item Testrealisierung und Testdurchführung,

\item Bewertung von Endekriterien und Bericht,

\item Abschluss der Testaktivitäten\footcite[Kapitel „5.6.2 Der
traditionelle Testprozess“ Seite 135-138]{schatten}
\end{enumerate}
\end{bAntwort}

\end{enumerate}
\end{document}
