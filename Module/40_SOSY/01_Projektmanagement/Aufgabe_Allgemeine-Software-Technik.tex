\documentclass{bschlangaul-aufgabe}
\bLadePakete{checkbox}
\begin{document}
\bAufgabenMetadaten{
  Titel = {Allgemeine Software-Technologie, Vorgehensmodelle und Requirements Engineering},
  Thematik = {Multiple-Choice Allgemeine Software-Technologie},
  Referenz = SOSY.Projektmanagement.Allgemeine-Software-Technik,
  RelativerPfad = Module/40_SOSY/01_Projektmanagement/Aufgabe_Allgemeine-Software-Technik.tex,
  ZitatSchluessel = sosy:e-klausur,
  BearbeitungsStand = mit Lösung,
  Korrektheit = unbekannt,
  Ueberprueft = {unbekannt},
  Stichwoerter = {EXtreme Programming, V-Modell, Wasserfallmodell, SCRUM, Prototyping, Unit-Test, Anforderungsanalyse},
}

Kreuzen Sie bei der folgenden Multiple-Choice-Frage die richtige(n)
Antwort(en) an. Auf falsch gesetzte Kreuze gibt es je einen Minuspunkt.
Die Aufgabe wird nicht mit weniger als 0 Punkten
gewertet.\footcite{sosy:e-klausur}

\begin{enumerate}

%%
%
%%

\item  Welche Vorgehensmodelle sind für Projekte mit häufigen Änderungen
gedacht?

\begin{itemize}
\bCheckboxLeer EXtreme Programming (XP)\index{EXtreme Programming}
\bCheckboxLeer Das V-Modell 97\index{V-Modell}
\bCheckboxLeer Wasserfallmodell\index{Wasserfallmodell}
\bCheckboxLeer Scrum\index{SCRUM}
\end{itemize}

\begin{bAntwort}
\begin{itemize}
\bCheckboxAngekreuzt EXtreme Programming (XP)\index{EXtreme Programming}
\bCheckboxLeer Das V-Modell 97\index{V-Modell}
\bCheckboxLeer Wasserfallmodell\index{Wasserfallmodell}
\bCheckboxAngekreuzt Scrum\index{SCRUM}
\end{itemize}
\end{bAntwort}

%%
%
%%

\item Welche der folgenden Aussagen ist korrekt?

\begin{itemize}
\bCheckboxLeer Mittels Prototyping\index{Prototyping} versucht man die Anzahl
an nötigen Unit-Tests\index{Unit-Test} zu reduzieren.

\bCheckboxLeer Ein Ziel von Prototyping ist die Erhöhung der Qualität
während der Anforderungsanalyse\index{Anforderungsanalyse}.

\bCheckboxLeer Mit Prototyping versucht man sehr früh Feedback von
Stakeholdern zu erhalten.
\end{itemize}

\begin{bAntwort}
\begin{itemize}
\bCheckboxLeer Mittels Prototyping\index{Prototyping} versucht man die
Anzahl an nötigen Unit-Tests\index{Unit-Test} zu reduzieren.

\bCheckboxAngekreuzt Ein Ziel von Prototyping ist die Erhöhung der
Qualität während der Anforderungsanalyse\index{Anforderungsanalyse}.

\bCheckboxAngekreuzt Mit Prototyping versucht man sehr früh Feedback von
Stakeholdern zu erhalten.
\end{itemize}
\end{bAntwort}

%%
%
%%

\item Welche der folgenden Aussagen ist korrekt?

\begin{itemize}
\bCheckboxLeer Das Wasserfallmodell sollte nur für große Projekte
eingesetzt werden, da der Einarbeitungsaufwand sehr groß ist.

\bCheckboxLeer Eine gute Anforderungsspezifikation muss vor allem für
Ingenieure verständlich sein,  da die Anforderungsspezifikation die
Grundlage der Systementwicklung bildet.

\bCheckboxLeer Verifikation ist der Prozess der Beurteilung eines
Systems mit dem Ziel festzustellen, ob die spezifizierten Anforderungen
erfüllt sind.

\bCheckboxLeer Durch Validierung kann überprüft werden, ob das Produkt
den Erwartungen des Kunden entspricht.

\bCheckboxLeer Mit Hilfe eines Black-Box-Tests kann man die Korrektheit
eines Programmcodes beweisen.
\end{itemize}

\begin{bAntwort}
\begin{itemize}
\bCheckboxLeer Das Wasserfallmodell sollte nur für große Projekte
eingesetzt werden, da der Einarbeitungsaufwand sehr groß ist.

\bCheckboxLeer Eine gute Anforderungsspezifikation muss vor allem für
Ingenieure verständlich sein,  da die Anforderungsspezifikation die
Grundlage der Systementwicklung bildet.

\bCheckboxLeer Verifikation ist der Prozess der Beurteilung eines
Systems mit dem Ziel festzustellen, ob die spezifizierten Anforderungen
erfüllt sind.

\bCheckboxAngekreuzt Durch Validierung kann überprüft werden, ob das
Produkt den Erwartungen des Kunden entspricht.

\bCheckboxLeer Mit Hilfe eines Black-Box-Tests kann man die Korrektheit
eines Programmcodes beweisen.
\end{itemize}
\end{bAntwort}
\end{enumerate}
\end{document}
